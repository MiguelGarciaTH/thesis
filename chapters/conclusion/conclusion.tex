\chapter{Conclusion and Future Research Directions}
\label{chap:conclusion}

\section{Conclusions}
The correctness of \gls{bft} systems is tied to the assumption that replicas fail independently. 
Otherwise, once a replica is intruded, the next $f+1$ replicas could be compromised within virtually the same time.
In this thesis, we addressed the long-standing open problem of many works on \gls{bft} replication research: evaluate, select, and manage the failure independence (through diversity) of the replicas.
%That said, we developed a control plane that maximizes the failure independence of \gls{bft} replicated nodes practically and effectively to make them more resilient to malicious adversaries.


The first step to achieve this objective was to validate the diversity hypothesis.  
To this end, we leveraged on \gls{nvd} data about \glspl{os} vulnerabilities to conduct an analysis of groups of \glspl{os} that could provide dependable configuration sets for replicated systems.
In particular, we presented several strategies to choose the most diverse \glspl{os}.
These strategies comprised different approaches to make decisions based on the data depending on whether one (i) considers all common vulnerabilities as being of equal importance; (ii) place greater emphasis on more recent shared vulnerabilities; or (iii) is primarily interested in the common vulnerabilities being reported less frequently in calendar time.
The experiments shown that it is possible to build sets of \glspl{os} with no or just a few shared vulnerabilities across long periods of time.
All strategies delivered the same best combination of \glspl{os}, and this emphasizes the importance of carefully selecting \glspl{os} that minimize the number of common weaknesses on \gls{bft} replicated systems.
These results were encouraging for diversity selection approaches as a way to achieve replicas' failure independence.



The additional costs of maintaining a system with diversity cannot be ignored. 
Therefore, critical systems are a first major target for deploying such mechanisms.
For example, replicated firewalls stand as a critical component in a network as they often correspond to the main line of defense.
%One of the \gls{bft} systems used in the \system evaluation, is \sieveq. 
We presented \sieveq a multi-layer firewall-like application that was designed to tolerate internal and external attacks with additional resilient mechanisms.
%Firewalls represent one of the most critical roles in infrastructures, as they control which packets go in and out of the network.
The main improvement of the \sieveq multi-layered architecture, when compared with previous systems, is the separation of message filtering in several components that carry on verifications progressively more costly and complex.
This allows the proposed system to be more efficient than the state-of-the-art replicated firewalls under attack.
\sieveq also includes several resilience mechanisms that allow the creation, removal, and recovery of components in a dynamic way, to respond adequately to evolving threats against the system. 
Experimental results show that such resilience mechanisms can significantly reduce the effects of \gls{dos} attacks against the protected network.


Despite the good results that we have achieved in the first contribution, we have identified some problems that lead us to pursue further analysis.
Namely, our and a few other works that used \gls{nvd} as a primary data source were missing essential data. 
In particular, \gls{nvd} does not report all products that are affected by a particular vulnerability.
We addressed this issue by using clustering techniques to group similar vulnerabilities.
Moreover, we added other data sources that bring additional relevant data (that \gls{nvd} does not provide) to enrich the information associated with vulnerabilities.
We devised a new metric that uses the vulnerability attributes provided by \gls{nvd} and \gls{cvss}, and other attributes added from extra \gls{osint} sources. 
This metric is used by an algorithm that builds and reconfigures replica sets of \gls{bft} systems.
The algorithm decides when and which replica should be replaced by estimating the risk of the \gls{bft} system being vulnerable to compromise.
The results show that our algorithm supports better decisions when selecting \glspl{os} to run in the replicated system. 
For example, for our approach the \emph{compromised rate} varies between 2\%-5\% of the total of executions while in a random selection the rate stays between 98\%-99\%.
 

%Besides the lack of proof for supporting diversity, most of the works did not address diversity in a practical manner.
As a final step, we implemented \system to manage \gls{bft} systems according to security data available online, which is analyzed to trigger replicas' recoveries.
We conducted an extensive evaluation of the costs of using real \gls{os} diversity in \gls{bft} systems.
In the experiments, we have found a significant limitation on the virtualization technology, forcing us to limit the evaluation setup by reconfiguring the bare metal machines (e.g., limit the number of CPUs).
Therefore, we manage to make a fair comparison (for some of the cases) between a bare metal configuration with a virtualized (yet limited) solution that accommodates \system.
Although most of the results show a non-negligible overhead, we have shown that for \glspl{os} that did not suffer virtualization limitations the overhead is minimal.
In these cases, the performance of the system managed by \system is 86\%-94\% of the bare metal, which runs a replicated non-virtualized homogenous \gls{os} configuration.



\section{Future Work}
The results of this thesis open many avenues for future work:


\textbf{Integration of prevention techniques:}
In Chapter~\ref{chap:related_work}, we have briefly described a few prevention techniques.
Although we did not address them here, they could be integrated in our control plane solution.
\system replicas have two states where they are dormant, namely when they are in quarantine or before they are deployed in the execution plane.
Thus, this time could be used to apply a sequence of automatic procedures (e.g., vulnerability detectors) that would (1) detect additional weaknesses, which could be reported to \gls{nvd}, and (2) remediate these weaknesses with mechanisms like automatic patching~\cite{Huang:2016}.
Moreover, although limited to open source products, it could use vulnerable code clone detectors, that would indicate potentially shared vulnerabilities~\cite{Kim:2017a,Xu:2017b}.
Such mechanisms would decrease the vulnerability surface, especially if common flaws are detected.
Although we are mostly concerned with more complex attacks (e.g., Advanced Persistent Threats), using automatic attacks~\cite{Hu:2015} could activate some common vulnerabilities before replicas are deployed.
These would trigger the algorithm to adjust the risk associated with such replicas avoiding their selection.


\textbf{Extending the \system source types and sensors:}
\system monitors nine security data feeds on the internet to collect data about vulnerabilities, exploits, and patches for the \glspl{os} it manages.
However, it could be extended to monitor other indicators of compromise (e.g., \gls{ip} blacklists) extracted from a much richer set of sources~\cite{Liao:2016,Sabottke:2015}.
Additionally, \system could use the \glspl{ids} outputs to assess the \gls{bft} system behavior and trigger replica's reconfigurations in case of need.
Finally, it may be interesting to mine black market data sources to discover new threats that may compromise several replicas~\cite{Allodi:2014}.
%However, exploring this path would require natural language processing techniques, and thus which is an entirely different area of research.

\textbf{Virtualization technology:}
Our prototype implements the LTU as a trusted component isolated from the rest of the replica in a \gls{vm}, in the same way as many works on hybrid \gls{bft}~\cite{Veronese:2013,Roeder:2010,Platania:2014,Sousa:2010,Distler:2011}.
However, one of the main limitations of \system is its performance overhead due to the use of virtualization.
In one hand, it allows \system to run 17 \gls{os} versions on top of VirtualBox.
But in the other hand, it brings a significant impact on the performance as it limits the CPU/memory resources available.
Nevertheless, the results also show that less resource-limited OSes have a similar performance.
Thus, the main overhead is due to virtualization and not (so much) to diversity.
We are currently working on a bare metal \gls{ltu} implementation based on Razor~\cite{razor} and developing the \gls{bft} control plane outlined in Section~\ref{sec:implementationdecisions} to integrate \system in decentralized Hyperledger Fabric deployments.



\textbf{Diversity-aware replication:}
In the same line, the evaluation of \textsc{BFT-SMaRt} on top of \system shows that different replica set configurations can impact on the performance of applications, mostly due to the heterogenous overheads of the different \glspl{os}.
It would be interesting to consider protocols in which this heterogeneity is taken into account.
For example, the leader could be allocated in the fastest replica, or weighted-replication protocols such as WHEAT~\cite{Sousa:2015} could be used to assign higher weights to the replicas running in faster replicas.
These approaches would reduce the impact of running \glspl{os} that suffer from resource limitations.


%\textbf{Trusted components:}
%\note{Discutir isto}
%Our prototype implements the LTU as a trusted component isolated from the rest of the replica in a VM, in the same way as many works on hybrid BFT~\cite{Veronese:2013,Roeder:2010,Platania:2014,Sousa:2010,Distler:2011}.
%The recent popularization of trusted computing technologies such as Intel SGX~\cite{sgx}, and its use for implementing efficient BFT replication~\cite{Behl:2017}, open interesting possibilities for using novel hardware to support services like \system on bare metal.
%An alternative solution would be to leverage on the existent rack-server firmware (e.g., Dell's iDRAC~\cite{idrac}) to implement \system in bare metal.
%Nevertheless, the controller firmware will be the local trusted component.
%Such solution would also improve \system performance as it would elimnate the virtualization limitations.





