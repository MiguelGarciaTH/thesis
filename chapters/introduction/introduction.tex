\chapter{Introduction}
\label{chap:introduction}

\section{Context and Motivation}

cite all papers on BFT diversity assumption

cite papers on diversity 

mention real systems which depend on diversity

In the last twenty years of \gls{bft} replication research, few efforts were made to justify or support the assumption that nodes fail independently. 
On contrary, there were great advances on the performance (e.g.,~\cite{Kotla:2010,Aublin:2015,Behl:2015}), use of resources (e.g.,~\cite{Yin:2003,Veronese:2013,Liu:2016,Behl:2017}), and robustness (e.g.,~\cite{Amir:2011,Bessani:2014,Clement:2009b}) of \gls{bft} systems.
These works assume, either implicitly or explicitly, that replicas fail independently, relying on some orthogonal mechanism (e.g.,~\cite{Roeder:2010,Chen:1995}) to remove common weaknesses, or rule out the possibility of malicious failures from their system models.
A few works have implemented and experimented with such mechanisms~\cite{Rodrigues:2001,Roeder:2010,Amir:2011}, but in a very limited way.
Nonetheless, in practice, diversity is a fundamental building block of dependable services in avionics~\cite{Yeh:2004}, military systems~\cite{rhimes}, and even in recent blockchain platforms such as Ethereum\footnote{\url{https://www.reddit.com/r/ethereum/comments/55s085/geth_nodes_under_attack_again_we_are_actively/}} -- three essential applications of \gls{bft}. 

A few works consider the diversity of replicas as a way to achieve such failure independence, however, it is mostly taken for granted.
For example, by using memory randomization techniques~\cite{Roeder:2010} or different \glspl{os}~\cite{Rodrigues:2001,Junqueira:2005} without providing evidence for such independence. 
Moreover, it has been shown that memory randomization does not suffice to impede common failures to occurring~\cite{Snow:2013,Bittau:2014}, and that although diversity promotes fault independence to some extent, it does not avoid utterly different \glspl{os} from sharing vulnerabilities~\cite{Garcia:2014}.

Even considering an initial set of $n$ diverse replicas, supporting the fault independence assumption, long-running services need to be cleaned from possible failures and intrusions.
A few works on the proactive recovery of \gls{bft} systems~\cite{Castro:2002,Sousa:2010,Roeder:2010,Platania:2014,Distler:2011} periodically restart their replicas to clean undetected faulty states introduced by a stealth attacker. 
However, a common limitation of these works is that they assume that these weaknesses will be cleaned after the recovery.
In practice, this will not happen unless the replica changes its software (e.g., via the previously described techniques) after its recovery.



\section{Objectives and Contributions}


\subsection*{Diversity a corner stone of Intrusion tolerance}
The current solutions that vouch for diversity as a way to guarantee failure independence, either lack of supporting evidence, or use evidence that is limited to some extent.
Therefore, some of the results may provide false conclusions.
We identify a need for finding \emph{accurate} sources for supporting sound evidence of the benefits diversity as a dependable mechanism.


The results achieved during the MSc thesis encouraged us to extend the work on \gls{ots} \glspl{os} common vulnerabilities.
In this extension, we devised three manual strategies for selecting diverse software components to minimize the incidence of common vulnerabilities in replicated systems.
Moreover, we observed that using different \gls{os} releases of the same \gls{os} are enough to warrant its adoption as a more straightforward, less complicated, more manageable configuration for replicated systems.

\begin{enumerate}
\item[6.] \emph{Analysis of operating system diversity for intrusion tolerance, Miguel Garcia, Alysson Bessani, Ilir Gashi Nuno Neves, and Rafael Obelheiro, in Software: Practice and Experience, 2014}~\cite{Garcia:2014}.
\end{enumerate}



\subsection*{Managing diversity in a dependable way and Creating mechanisms to support intrusion-tolerance on long-lived systems}
Therefore, some of the results may provide false conclusions.
We identify a need for finding \emph{accurate} sources for supporting sound evidence of the benefits diversity as a dependable mechanism.

A few works use automatic and artificial diversity (e.g.,~\cite{Roeder:2010,Amir:2011}). 
However, they lack evidence to support the failure of independence through diversity. 
Moreover, some studies show that these techniques fail to provide real diversity~\cite{Snow:2013,Bittau:2014}. 
Additionally, the existent systems that implement time-triggered recoveries assume that it takes the same time to compromise each replica. 
This assumption is unrealistic, especially when the diversity of replicas is considered. 
Therefore, it is required tailored methods to evaluate the risk of a replicated system becoming compromised.


The few works have implemented and experimented with such mechanisms~\cite{Rodrigues:2001,Roeder:2010,Amir:2011}. 
However, despite the lack of evidence for supporting the diversity claim, they lack mechanisms that can make practical \gls{bft} systems using diversity in a continuous mode operation (i.e., with recoveries).  
Thus, reducing the costs typically associated with the management of such complex systems which deemed them as practical.


The problem is addressed from both a theoretical and practical perspective.


We address the problems of \emph{finding evidence for supporting diversity}, \emph{manage diversity in a dependable way}, and \emph{supporting diversity mechanism} in the thesis main contribution.
We use different \gls{osint} data sources to build a complete knowledge base about the possible vulnerabilities, exploits, and patches related to the systems of interest. 
Moreover, this data is used to create clusters of similar vulnerabilities, which potentially can be affected by (variations of) the same exploit. 
These clusters and the other collected data are used to assess the risk of the \gls{bft} system becoming compromised due to common vulnerabilities.
Once the risk increases, the system replaces the potentially vulnerable replica by another one, to maximize the failure independence of the replicated service.
The solution continuously collects data from the online sources and monitors the risk of the \gls{bft} in such a way that removes the human from the loop.
We developed these contributions in a solution named \system, and it is the first system that manages \gls{bft} replicated systems (e.g., \sieveq) in a dependable and automatic way.
\system experimental evaluation shows that its strategy reduces the number of executions where the system becomes compromised and that our prototype supports the execution of full-fledged \gls{bft} systems in diverse configurations with 17 \gls{os} versions, reaching a performance close to a homogeneous bare metal setup. 


The contributions of this work resulted in the following publications:

\begin{enumerate}

\item[3] \emph{Towards an Execution Environment for Intrusion-Tolerant Systems, Miguel Garcia, Alysson Bessani, and Nuno Neves, Poster session in the European Conference on Computer Systems (EuroSys), 2016}~\cite{Garcia:2016b}.


\item[4.] \emph{\system: Automatic Management of Diversity in BFT Systems, Miguel Garcia, Alysson Bessani, and Nuno Neves -- Submitted for publication}.

\end{enumerate}


\subsection*{Creating mechanisms to support intrusion-tolerance on long-lived systems}

Typical \gls{bft} protocols use one of the two following approaches to disseminate messages to the replicas: (1) traffic replicator before the replicas or (2) a leader is responsible to dessiminate the messages to the other replicas. 
The dissemination of a message to all replicas can be detrimental to the proper operation of the replicated service.
For example, a traffic replicator device (e.g., hub) can be placed at the entry of the system to transparently reproduce all messages~\cite{Sousa:2010,Roeder:2010}. 
The effect is an attack amplification caused by the replicator device.
Alternatively, a leader replica could receive the traffic and then disseminate the messages to the others~\cite{Amir:2011}.
The drawback is that the leader becomes a natural bottleneck, especially when under attack (instead of dispersing the attack load over all replicas).
%Then, new architectures can be designed to accommodate external attacks in a resourceful and resilient manner.

We design a new architecture to solve the \emph{dissemination problem}, and the envisioned solution applies filtering operations in two stages acting like a sieve.
We introduced a first pre-filtering stage that performs lightweight checks, making it efficient to detect and discard malicious messages from external adversaries.
This effort resulted in a \gls{bft} system named \sieveq, that mixes the firewall paradigm with a message queue service, with the goal of improving the state-of-the-art approaches under accidental failures and/or attacks. 
\sieveq was experimentally evaluated in different scenarios, and the results show that it is much more resilient to \gls{dos} attacks and various kinds of intrusions than existing replicated-firewall approaches.
The contributions of this work resulted in the following publications:

\begin{enumerate}
\item[1.] \emph{An Intrusion-Tolerant Firewall Design for Protecting SIEM Systems, Miguel Garcia, Nuno Neves, Alysson Bessani, in the Workshop on Systems Resilience in conjunction with the IEEE/IFIP International Conference on Dependable Systems and Networks, 2013}~\cite{Garcia:2013}.

\item[2.] \emph{\sieveq: A Layered BFT Protection System for Critical Services, Miguel Garcia, Nuno Neves, and Alysson Bessani, in IEEE Transactions on Dependable and Secure Computing, 2018}~\cite{Garcia:2016}.
\end{enumerate}


\subsection*{Thesis Statement}
We summarize our findings in the following thesis statement:


\vspace{2mm}
\fbox{ \begin{minipage}{35em}
%\begin{center}
\emph{
It is possible to build dependable BFT replicated systems by minimizing the number of replicas' common vulnerabilities through software's diversity.
Additionally, it is possible to continuously manage these systems while monitoring OSINT data and deciding when replicas should be diversified and deploying the most dependable configurations.
}
%\end{center}
\end{minipage}
}

\section{Thesis Overview}
Detailed description of each Chapter
