\chapter{Introduction}
\label{chap:introduction}

\section{Context and Motivation}
\paragraph{Intrusion Tolerance.}
\gls{bft}\footnote{In this thesis we interchange Byzantine fault tolerance (\emph{noun}) with Byzantine fault-tolerant (\emph{adjective}) using the BFT acronym for both.} is a well-established area of research that aims to guarantee the safety of replicated systems even in the presence of some faulty nodes that arbitrarily deviate from their specification.
In a nutshell, \gls{bft} protocols guarantee that replicas agree on the order of the message execution, ensuring that replicas work as a state machine.
The \gls{bft} safety holds as long as no more than $f$ out of a group of $n$ replicas are faulty.
Although not explicit, this assumption leverages on the strict condition that nodes (must) fail independently.
Otherwise, compromising $f$ replicas is virtually the same as compromising $f+1$ because the same vulnerabilities would exist in all nodes.


\gls{bft} was first proposed in 1980 by Pease~\etal{}~\cite{Pease:1980}, and in the following years a few other works continued to investigate on this subject~\cite{Reiter:1994,Kihlstrom:1998}.
However, only much later, with the work by Castro and Liskov~\cite{Castro:1999} did the distributed system community start to actively research on this area.
In the last twenty years, \gls{bft} systems have benefit from great advances on the performance (e.g.,~\cite{Kotla:2010,Aublin:2015,Behl:2015}), use of resources (e.g.,~\cite{Yin:2003,Wood:2011,Veronese:2013,Liu:2016,Behl:2017}), and robustness (e.g.,~\cite{Amir:2011,Bessani:2014,Clement:2009b}).
However, \gls{bft} in general and these works in particular, assume, either implicitly or explicitly, that replicas fail independently. 
This assumption implies that it should take more time to compromise a system if its nodes do not share weaknesses. 
A few works suggest some form of diversity mechanism (e.g.,~\cite{Roeder:2010,Avizienis:1995}) to avoid these common weaknesses or rule out the possibility of malicious failures from their system models.
However, only a few works have implemented and experimented with such diversity mechanisms (e.g.,~\cite{Castro:2003,Roeder:2010,Amir:2011}) but in a very limited way.


In other contexts, dependable applications take advantage of diversity mechanisms. 
In these cases, diversity is applied \emph{intuitively}.
For example, in avionics engineering, a few aircraft solutions embrace the diversity of components to avoid accidental failures~\cite{Yeh:2004}.
More specifically, in 2014, the U.S. Navy developed the Resilient Hull, Mechanical, and Electrical Security (\textsc{Rhimes}), which introduce diversity through a ``\emph{slightly different implementation}'' for each programmable logic controller~\cite{rhimes}.
Another example of a critical system that naturally adopted diversity are the \gls{dns} root servers. 
According to L. Liman (chair of the Root Server System Advisory Committee) each operator employs its own configurations as ``\emph{it allows for great operational diversity}''~\cite{dns_root}. 
The goal is to avoid that a single software bug would bring down all servers.
A more recent example is Blockchain technology, that in some cases, implements a \gls{bft} network in a large and decentralized way.
Therefore, it shares the same limitations of the enumerated \gls{bft} works, i.e., the network nodes may have common weaknesses.
Recent evidence already demonstrated that Bitcoin~\cite{bitcoin} can suffer from this problem with potential nefarious consequences.
Moreover, Ethereum stats show that a few \glspl{os} are deployed in the various nodes~\cite{Ethstats,Ethernodes} to deter attacks from exploiting common mode failures.


Despite the adoption of (naive) diversity solutions in practical systems, other variables must be considered to make systems dependable and secure.
For example, even considering an initial set of $n$ diverse replicas, long-running services need to be constantly monitored to remove possible failures and intrusions.
A few works on the proactive recovery of \gls{bft} systems~\cite{Castro:2002,Sousa:2010,Roeder:2010,Platania:2014,Distler:2011} propose periodically replicas restarts to clean up undetected faulty states introduced by a stealth attacker. 
However, these works (i) assume failure independence of the nodes, and (ii) do not provide a well-thought justification to support their diversity decisions or (iii) use limited solutions (e.g., memory randomization)~\cite{Snow:2013,Bittau:2014}.
Therefore, replicas may keep the same weaknesses unless careful diversity choices are introduced on recoveries.


Finally, contrary to the already established maturity of \gls{bft} solutions, there is a lack of research work debating how to apply diversity in a dependable way (e.g., choosing software configurations that are more dependable) and when these configurations should be changed (e.g., which is the most appropriate moment for recovering replicas with different software).

\paragraph{Vulnerabilities.}
The reason why \gls{bft} solutions have to make stronger assumptions on the replicas failure independence is due to the existence of vulnerabilities in their code.\footnote{In this document, a vulnerability is a special kind of software weakness (or bug or flaw) that can be exploited to break some security property.}
These weaknesses can be exploited by a malicious entity to compromise a security property (i.e., integrity, confidentiality, and availability).
If a vulnerability is found by a vendor or by a third-party software analyst, the vendor releases a patch that can be applied in the system to remove the weakness. 
On the contrary, if a vulnerability is found by malicious hackers, it is especially critical as their disclosure occurs only after the detection of its exploitation.
This particular type of weakness is called zero-day vulnerability.
Such vulnerabilities are a critical element present in many notable attacks (e.g., Stuxnet~\cite{stuxnet:2010}) and are traded on the black market at high prices~\cite{Symantec:2017,Allodi:2017}.
Moreover, big companies and organizations are promoting bug bounty programs as an attempt to find zero-day vulnerabilities before these are found by malicious hackers (e.g., Google Vulnerability Reward Program~\cite{google_reward}, Facebook Bug Bounty Program~\cite{facebook_whitehat}, Microsoft Bounty~\cite{microsoft_bounty}, and DARPA Cyber Grand Challenge (2016)~\cite{darpa}).



In a \gls{bft} context, an attacker can compromise a replica once a piece of code is developed to exploit a vulnerability in a software component of the node.
If the replicas have the same code and configuration, they share the same vulnerabilities.
Therefore, as soon as the attacker discovers a zero-day vulnerability, he or she will eventually be able to compromise $f+1$ replicas.
The rationale behind diversity on \gls{bft} is that different replicas do not share the same vulnerabilities. 
Therefore, it becomes more difficult to discover and exploit vulnerabilities in $f+1$ replicas than to compromise just $f$ replicas.




\section{Objectives and Contributions}

In the past, several researchers developed and improved techniques that made intrusion tolerance a cornerstone of security and dependability.
Nevertheless, a few problems were left open and still lack effective answers.
This thesis aims to push forward intrusion tolerance to a more effective and practical ground.
The main contributions of this thesis are summarized into the following sections, along with the related publications.


\subsection{Evidence for Implementing Diversity}% Diversity Study on Off-the-Shelf Operating Systems}


The results achieved during the author's MSc thesis encouraged us to extend the work on the diversity of \gls{ots} \glspl{os}~\cite{Garcia:2012}.
This is a first step towards the design and development of dependable \gls{bft} systems, and it aims to assess how accurate were the implicit or explicit assumptions on diversity (e.g.,~\cite{Abd-El-Malek:2005,Bessani:2008,Castro:2002,Castro:2003,Clement:2009,Correia:2004,Kapitza:2012,Kotla:2010,Moniz:2011,Yin:2003}).
We studied the information about vulnerabilities from the \gls{nvd}~\cite{nvd} to answer the question:
\emph{What are the dependability gains from using diverse \glspl{os} on a replicated  intrusion-tolerant system?} 
Some of the extensions to the original work, that are presented in this thesis are:
(i) we expanded the dataset from \gls{nvd} to further years, to extend previous observations;
(ii) we devised three manual strategies for selecting diverse software components, to minimize the incidence of common vulnerabilities in replicated systems; 
and (iii) we found out that employing different \gls{os} releases of the same \gls{os} is enough to warrant its adoption as a more straightforward and manageable configuration for replicated systems.
The described contributions (together with the ones published during the MSc~\cite{Garcia:2012}) are reported in the following publication:

\begin{enumerate}
\item[1.] \textbf{Analysis of Operating System Diversity for Intrusion Tolerance}, Miguel Garcia, Alysson Bessani, Ilir Gashi, Nuno Neves, and Rafael Obelheiro, in \emph{Software: Practice and Experience, 44(6), June 2014}.
\end{enumerate}




\subsection{BFT Multi-layer Resiliency} 
%%\system implements mechanisms that provide security and dependability for replicated systems.
%The previous contribution shows the benefits of using \glspl{os} diversity.
%This is particular relevant in \gls{bft} dependable systems. 
%However, not all systems need to run different \glspl{os} which brings additional management costs. % managed with such advanced mechanisms. 
%Some of the most critical applications that can benefit from such management are firewalls.
Firewalls are used as the primary protection against external threats, controlling the traffic that flows in and out of a network. 
Typically, they decide if a packet should go through (or be dropped) based on the analysis of its contents. 
Most generic firewall solutions suffer from two inherent problems: 
First, they have vulnerabilities as any other system, and as a consequence, they can also be the target of attacks. 
For example, \gls{nvd} shows that many security issues have been observed in commonly used firewalls. 
The following numbers of security issues are reported by \gls{nvd} for the interval 2010 and 2018:\footnote{To the date of Jul 12$^{th}$ 2018.} 205 for the Cisco Adaptive Security Appliance; 123 in Juniper Networks solutions; and 50 related to iptables/netfilter. 
Moreovoer, common protection solutions often have been the target of malicious actions as part of a wider scale attack (e.g., anti-virus software~\cite{Chauhan:2011}, \gls{ids}~\cite{Anderson:2001} or firewalls~\cite{Kamara:2003,Surisetty:2010,cisco1,cisco2}).
Second, firewalls are typically a single point of failure, which means that when they crash, the ability of the protected system to communicate may be compromised, at least momentarily.
Therefore, ensuring the correct operation of the firewall under a wide range of failure scenarios becomes imperative.
In the last decade, several significant advances occurred in the development of intrusion-tolerant systems.
However, to the best of our knowledge, very few works proposed intrusion-tolerant protection devices, such as firewalls.
Performance reasons might explain this gap, as \gls{bft} replication protocols are usually associated with significant overheads and limited scalability.
Additionally, achieving complete transparency to the rest of the system can be challenging to reconcile with the objective of having fast message filtering under attack.

We address these problems with a new protection system, called \sieveq, that mixes the firewall paradigm with a message queue service.
In \sieveq, we explore a different design for replicated protection devices, where we trade some transparency on senders and receivers for a more efficient and resilient firewall solution.
In particular, we propose an architecture in which critical services and devices can only be accessed through a message queue where application-level filtering is implemented.
It is assumed that these services have a limited number of senders, which can be appropriately configured to ensure that only they are authorized to communicate through \sieveq.
The solution is based on a fault- and intrusion-tolerant architecture that applies filtering operations in two stages, acting like a sieve.
The first stage, called \emph{pre-filtering}, performs lightweight checks, making it efficient to detect and discard malicious messages from external adversaries.
In particular, messages are only allowed to go through if they come from a pre-defined set of authenticated senders.
\gls{dos} traffic from external sources is immediately dropped, preventing those messages from overloading the next stage.
The second stage, named \emph{filtering}, enforces more refined application level policies, which can require the inspection of some message fields or need the enforcement of specific ordering rules.


\sieveq was evaluated in different scenarios and the results show that it is much more resilient to \gls{dos} attacks and various kinds of intrusions than existing replicated-firewall approaches.
We also evaluated \sieveq considering the protection of a \gls{siem} system.
The test environment emulated the setup of the 2012 Summer Olympic Games, where the same sort of security events was generated and transmitted across the network. 
The experiments demonstrate that \sieveq can handle a workload up to sixteen times higher than the observed load in the 2012 Summer Olympic Games, without noticeable degradation in performance.


The contributions of this work resulted in the following publications:

\begin{enumerate}

\item[2.] \textbf{An Intrusion-Tolerant Firewall Design for Protecting SIEM Systems}, Miguel Garcia, Nuno Neves, and Alysson Bessani, in the \emph{Workshop on Systems Resilience together with the {IEEE/IFIP} International Conference on Dependable Systems and Networks,  Budapest, June 2013}.

\item[3.] \textbf{\sieveq: A Layered BFT Protection System for Critical Services}, Miguel Garcia, Nuno Neves, and Alysson Bessani, in \emph{IEEE Transactions on Dependable and Secure Computing, 15(3), May 2018}.



\end{enumerate}



\subsection{Applying Diversity on BFT Systems}

Our initial attempt to validate diversity gave substantial evidence to the claim that employing different \glspl{os} ensures failure independence to some extent.
However, this preliminary analysis suffers from the limitation of being based on a single dataset, which might have imprecisions (e.g.,~\cite{Han:2009,Frei:2010,Shahzad:2012,Bozorgi:2010,Allodi:2014,Gorbenko:2017}).
For example, it is possible to find vulnerabilities in the \gls{nvd} feeds that are only reported for a subset of the \glspl{os} that are affected by them.
We have found this missing information on other sources (e.g., the vendors' security advisories).
Therefore, \gls{nvd}-based studies may provide optimistic conclusions about the benefits of selecting different software components based on common vulnerabilities.
Moreover, \gls{nvd} lacks data concerning exploits and patches that is relevant when considering the vulnerabilities life-cycle.

%An alternative (or complementary) way to \gls{ots} diversity is to generate diversity through automatic mechanisms (e.g.,~\cite{Roeder:2010,Amir:2011}).
%However, their produced vulnerability independence is difficult to substantiate~\cite{Snow:2013,Bittau:2014}. 

Systems that implement time-triggered recoveries consider that it takes the same time/effort to compromise each replica, by assuming that vulnerabilities are similarly easy to find and exploit. 
This assumption is unrealistic, especially when diversity is considered~\cite{Nayak:2014}. 
Therefore, tailored methods are required to evaluate the risk of a replicated system becoming vulnerable (i.e., $f+1$ replicas suffer from the same weaknesses).


In this thesis, we address these problems from both a theoretical and practical perspective, with the objective of \emph{finding evidence that demonstrates the gains of utilizing diversity}, \emph{managing diversity in a dependable way}, and \emph{implementing diversity with practical mechanisms}.
First, we suppress the aforementioned limitations of \gls{nvd} by using clustering techniques that group similar vulnerabilities, which potentially can be attacked by (variations of) the same exploit.
Moreover, we employ additional \gls{osint} data sources to build a more complete knowledge base about the possible vulnerabilities, exploits, and patches related to the systems of interest. 
Then, the clusters and the collected data are used to calculate a metric that is employed to assess the risk of a \gls{bft} system becoming compromised by the existence of common vulnerabilities.
Once the risk increases, the system replaces the potentially vulnerable replica by another one, to maximize the failure independence of the replicated service.
The solution continuously collects data from the online sources and monitors the risk of the \gls{bft} in such a way that removes the human from the loop.
These mechanisms were implemented, in a prototype named \system, to be fully automated and transparent to \gls{bft} systems.
Moreover, the current implementation of \system manages 17 \gls{os} versions, supporting the \gls{bft} replication of a set of representative applications.
The replicas run in \glspl{vm}, allowing provisioning mechanisms to configure them. 


We conducted two sets of experiments: The first one demonstrates that \system risk management can prevent a group of replicas from sharing vulnerabilities over time; 
The second one, reveals the potential negative impact that virtualization and diversity can have on performance. 
In this experiment, we evaluated \system with three \gls{bft} applications: (1) a Key-Value Store, (2) the \sieveq firewall (presented in the previous section), and (3) a \gls{bft} ordering service for the Hyperledger Fabric blockchain platform.
The overall results show that if naive configurations are avoided, \gls{bft} applications in virtualized diverse configurations can perform close to a homogeneous bare metal setup.

The contributions of this work resulted in the following publications:

\begin{enumerate}

\item[4.] \textbf{\textsc{Diversys}: DIVErse Rejuvenation SYStem}, Miguel Garcia, Nuno Neves, and Alysson Bessani, in the \emph{Simp\'{o}sio Nacional de Inform\'{a}tica (INFORUM),  Caparica, September 2012}.

\item[5.] \textbf{Towards an Execution Environment for Intrusion-Tolerant Systems}, Miguel Garcia, Alysson Bessani, and Nuno Neves, Poster session in the \emph{European Conference on Computer Systems (EuroSys), London, June 2016}.


\item[6.] \textbf{\system: Automatic Management of Diversity in BFT Systems}, Miguel Garcia, Alysson Bessani, and Nuno Neves -- \emph{Submitted for publication in 2019}.


\end{enumerate}


\paragraph{Additional contributions.}
A collaboration with a colleague resulted in a work on the design and implementation of a \gls{scada} system enhanced with \gls{bft} techniques. 
We documented the challenges of building such a system from a ``traditional'' non-\gls{bft} solution.
This effort resulted in a prototype, implemented by the colleague, that integrates the Eclipse NeoSCADA~\cite{eclipsescada} and the \textsc{BFT-SMaRt}~\cite{Bessani:2014} open-source projects.
Although this contribution is out of the scope of the thesis, it is easy to envision the integration of this prototype with \system.


\begin{enumerate}

\item[7.] \textbf{On the Challenges of Building a BFT SCADA}, Andr\'{e} Nogueira, Miguel Garcia, Alysson Bessani, and Nuno Neves, in \emph{Proceedings of the {IEEE/IFIP} International Conference on Dependable Systems and Networks, Luxembourg, June 2018}.
\end{enumerate}


\subsection{Thesis Statement}
We summarize our findings in the following thesis statement:

\vspace{2mm}
\fbox{ \begin{minipage}{38,5em}
%\begin{center}
\emph{
It is possible to build dependable BFT replicated systems by minimizing the number of shared vulnerabilities in the nodes through software diversity.
Additionally, it is possible to manage these systems by continuously monitoring OSINT data and deciding when replicas should be rejuvenated with diverse software, aiming at the deployment of a more dependable configuration.
}
%\end{center}
\end{minipage}
}

\section{Thesis Overview}
\paragraph{Chapter~\ref{chap:related_work}: \nameref{chap:related_work}.}
Provides a background overview of intrusion tolerance. 
It also presents the most relevant related works on the different areas of intrusion tolerance.
In particular, it is focused on \gls{bft} protocols, replica rejuvenations, and diversity.
It also identifies several works that implement diversity guided by \emph{intuition}, then it covers the works that analyze vulnerabilities.
Finally, it describes some solutions to manage intrusion-tolerant systems.


\paragraph{Chapter~\ref{chap:datasource}: \nameref{chap:datasource}.}
Gives evidence that diversity may improve systems' dependability.
It shows that with carefully made choices one can build dependable systems using different \glspl{os}.
Although some solutions are (now recognized by us as) naive, they were a significant contribution to push us towards building dependable systems with diversity.


\paragraph{Chapter~\ref{chap:sieveq}: \nameref{chap:sieveq}.}
Presents \sieveq that implements a firewall-like application.
The \sieveq architecture provides additional resilient mechanisms when compared with state-of-the-art solutions, such as resorting to several levels of filtering and alternative mechanisms for replication.
To conclude the chapter, we present an evaluation showing the performance and resilience of \sieveq under attack.


\paragraph{Chapter~\ref{chap:lazarus_design}: \nameref{chap:lazarus_design}.}
Describes our proposal for solving a few of the existing open problems with intrusion tolerance.
In particular, we show how to solve data source limitations that affected several works in the past.
Moreover, we present a new metric that is used to minimize the risk of \gls{bft} replicas becoming vulnerable by common mode failures.
Finally, we evaluate the efficacy of our approach by executing the algorithm against different configuration strategies that have been used in related works.


\paragraph{Chapter~\ref{chap:lazarus_implementation}: \nameref{chap:lazarus_implementation}.}
Presents the \system implementation and describes the decisions made on its several components.
For example, how \system collects the data from the different sources and uses it to build clusters; how we setup and deploy the different nodes using provision and virtualization tools.  
Moreover, we present an extensive performance evaluation of \system-managed systems.
In particular, we evaluate the performance of three real \gls{bft} applications, one of which is \sieveq.



\paragraph{Chapter~\ref{chap:conclusion}: \nameref{chap:conclusion}.}
Summarizes the results that were achieved with this thesis and identify possible directions for future research.   

