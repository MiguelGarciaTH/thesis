\chapter{Conclusion}
\label{chap:conclusion}

\section{Final Remarks}
\system addresses the long standing open problem of evaluating, selecting, and managing the diversity of a \gls{bft} system to make it resilient to malicious adversaries.
Our work focuses on two fundamental issues: how to select the best replicas to run together given the current threat landscape, and what is the performance overhead of running a diverse \gls{bft} system in practice.

\section{Future work}
Our promising results open many avenues for future work in this topic.

\textbf{Integration with other \gls{osint} and \glspl{ids}:}
\system monitors only five security data feeds on the internet looking for vulnerabilities, exploits, and patches in the \glspl{os} it manages, but it could be extended to monitor other indicators of compromise (e.g., IP black lists) extracted from much richer set of sources~\cite{Liao:2016,Sabottke:2015}.
Similarly, \system can be extended to additionally use the outputs of \glspl{ids} to assess the \gls{bft} system behaviour and trigger replica reconfigurations in case of need.

\textbf{Diversity-aware replication:}
The evaluation of BFT-SMaRt on top of \system shows that different replica group configurations can have a huge impact on the performance of applications, mostly due to the performance heterogeneity of the different \glspl{os}.
It would be interesting to consider protocols in which this heterogeneity is taken into account.
For example, the leader could be allocated in the fastest replica, or weighted-replication protocols such as WHEAT~\cite{Sousa:2015} could be used to assign higher weights to the replicas running in faster replicas.


\textbf{Trusted components:}
Our prototype implements the LTU as a trusted component isolated from the rest of the replica through virtualization, just like many previous works on hybrid \gls{bft}~\cite{Veronese:2013,Roeder:2010,Platania:2014,Sousa:2010,Distler:2011}.
The recent popularization of trusted computing technologies such as Intel SGX~\cite{sgx}, and its use for implementing efficient \gls{bft} replication~\cite{Behl:2017}, open interesting possibilities for using novel hardware to support services like \system on bare metal.
