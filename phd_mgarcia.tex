%
% Modelo para Capa final de tese de doutoramento
% do MEI.
%
% Incorpora elementos impostos pelo Regulamento de Estudos Pos-Graduados da
% Universidade de Lisboa (DR: 31/08/2017)
%
\documentclass[
 paper=A4,               % paper size --> A4 is default in Germany
    twoside=true,           % onesite or twoside printing
    openright,              % doublepage cleaning ends up right side
    parskip=full,           % spacing value / method for paragraphs
    chapterprefix=true,     % prefix for chapter marks
    11pt,                   % font size
    headings=normal,        % size of headings
    bibliography=totoc,     % include bib in toc
    listof=totoc,           % include listof entries in toc
    titlepage=on,           % own page for each title page
    captions=tableabove,    % display table captions above the float env
    draft=false,            % value for draft version
]{scrreprt}

%\DeclareOldFontCommand{\bf}{\normalfont\bfseries}
\usepackage{newstyle}
\usepackage[utf8]{inputenc}
\usepackage{amsmath,amsfonts,amssymb,amsthm,url,array}

% Quem tiver problemas com os acentos, trocar utf8 por latin1
\usepackage[portuguese,english]{babel}
\usepackage{times}
\usepackage{graphicx}
\usepackage{xspace}
\usepackage{setspace}
\usepackage[show]{chato-notes}
\usepackage[sort&compress,numbers]{natbib}
\usepackage[vlined,ruled,commentsnumbered,linesnumbered]{algorithm2e} 


% Indice remissivo
\usepackage{makeidx}
\makeindex

% Glossario
%\usepackage{glossaries}
%\makeglossary


% Links
\usepackage{hyperref}

% Package para cabecalhos
\usepackage{fancyhdr}
\usepackage{lastpage}
\usepackage{listings}
%\usepackage[subfigure]{tocloft}

\theoremstyle{definition}
\newtheorem{defn}{Definition}[]
\usepackage{tikz}
\newcommand*\circled[1]{\tikz[baseline=(char.base)]{
            \node[shape=circle,fill,inner sep=1pt] (char) {\textcolor{white}{#1}};}}

\definecolor{codegreen}{rgb}{0,0.6,0}
\definecolor{codegray}{rgb}{0.5,0.5,0.5}
\definecolor{codepurple}{rgb}{0.58,0,0.82}
\definecolor{backcolour}{rgb}{0.95,0.95,0.92}

\lstdefinestyle{mystyle}{
    backgroundcolor=\color{backcolour},
    commentstyle=\color{codegreen},
    keywordstyle=\color{blue},
    keywordstyle={[2]\color{magenta}},
    numberstyle=\tiny\color{codegray},
    stringstyle=\color{codepurple},
    basicstyle=\footnotesize,
    comment=[l]{\%},
    keywords={@relation,@attribute,@data,modifyvm, do, Vagrant, configure, id},
    morekeywords=[2]{real,integer,numeric,string,date,provider,customize,gui,v,provision, config,vm,box,ssh,insert_key,network},
    breakatwhitespace=false,
    breaklines=true,
    captionpos=b,
    keepspaces=true,
    numbers=left,
    numbersep=5pt,
    showspaces=false,
    showstringspaces=false,
    showtabs=false,
    tabsize=1
}

\lstset{style=mystyle}

\usepackage{xcolor}


\newtheorem*{validity}{Validity}{}
\newtheorem*{liveness}{Liveness}{}
\newtheorem*{security}{Compliance}{}
\newtheorem*{resilience}{Resilience}{}
\newtheorem{definition}{Definition}[]
\newtheorem{assumption}{Assumption}[]

\newcommand{\sender}{\emph{sender-SQ}\xspace}
\newcommand{\Sender}{\emph{Sender-SQ}\xspace}
\newcommand{\senders}{\emph{sender-SQs}\xspace}
\newcommand{\Senders}{\emph{Sender-SQs}\xspace}
\newcommand{\presieve}{\emph{pre-SQ}\xspace}
\newcommand{\Presieve}{\emph{Pre-SQ}\xspace}
\newcommand{\Presieves}{\emph{Pre-SQs}\xspace}
\newcommand{\presieves}{\emph{pre-SQs}\xspace}
\newcommand{\repsieve}{\emph{replica-SQ}\xspace}
\newcommand{\Repsieve}{\emph{Replica-SQ}\xspace}
\newcommand{\repsieves}{\emph{replica-SQs}\xspace}
\newcommand{\Repsieves}{\emph{Replica-SQs}\xspace}
\newcommand{\postsieve}{\emph{post-SQ}\xspace}
\newcommand{\Postsieve}{\emph{Post-SQ}\xspace}

\newcommand{\msg}{\texttt{Msg}\xspace}
\newcommand{\sn}{\emph{sn}\xspace}
\newcommand{\signature}{\emph{sgn}\xspace}
\newcommand{\mac}{\emph{mac}\xspace}
\newcommand{\senderi}{\emph{S}\xspace}
\newcommand{\sendersqi}{\emph{SQ}\xspace}
\newcommand{\presievei}{\emph{PS}\xspace}
\newcommand{\replicak}{\emph{RS}\xspace}
\newcommand{\postsievej}{\emph{PQ}\xspace}
\newcommand{\receiverj}{\emph{R}\xspace}
\newcommand{\privkey}{\emph{prk}\xspace}
\newcommand{\pubkey}{\emph{puk}\xspace}
\newcommand{\sharedkey}{\emph{shk}\xspace}


\newcommand{\system}{\textsc{Lazarus}\xspace}
\newcommand{\sieveq}{\textsc{SieveQ}\xspace}
\newcommand{\controller}{\textsc{Baton}\xspace}
\newcommand{\fetcher}{\emph{Data manager}\xspace}
\newcommand{\manager}{\emph{Deploy manager}\xspace}
\newcommand{\risk}{\emph{Risk manager}\xspace}

\newcommand{\block}{\emph{Diversity block}\xspace}
\newcommand{\blocks}{\emph{Diversity blocks}\xspace}
\newcommand{\replica}{\emph{replica}\xspace}
\newcommand{\replicas}{\emph{replicas}\xspace}
\newcommand{\configuration}{\emph{System configuration}\xspace}
\newcommand{\configurations}{\emph{System configurations}\xspace}
\newcommand{\configurationClean}{System configuration}


\newcommand{\systemformula}{risk\xspace}
\newcommand{\pairformula}{common\xspace}
\newcommand{\vulnerabilityformula}{score\xspace}


\SetKwData{RS}{\texttt{POOL}}
\SetKwData{ES}{\texttt{CONFIG}}
\SetKwData{PS}{\texttt{CANDIDATES}}
\SetKwData{RM}{\texttt{REMOVABLES}}
\SetKwData{QS}{\texttt{QUARANTINE}}
\SetKwData{MAX}{\texttt{MAXIMALS}}
\SetKwData{N}{\texttt{N}}
\SetKwData{r}{\texttt{r}}
\SetKwData{K}{\texttt{K}}
\SetKwData{toRemove}{\texttt{toRemove}}
\SetKwFunction{Rand}{rand}
\SetKwFunction{Size}{size}
\SetKwFunction{Remove}{rm}
\SetKwFunction{Risk}{risk}
\SetKwFunction{Common}{common}
\SetKwFunction{Older}{older}
\SetKwFunction{healing}{init\_healing}
\SetKwFunction{Inc}{increment}
\SetKwFunction{Dec}{decrement}
\SetKwProg{Fn}{Function}{}{}

\fancyhf{} %
\lhead{\nouppercase {\leftmark}} %
\rhead{\nouppercase {\bf \thepage}}
\renewcommand{\headrulewidth}{0.1pt}

% Comando para inserir pagina em branco (inserida na numeracao, mas sem
% numero impresso) para quando e' preciso obrigar um capitulo a comecar
% do lado direito (pagina impar)
\newcommand{\LIMPA}{
\newpage
\mbox{}
\thispagestyle{empty}
}

% Igual, mas insere pagina com numero impresso (normalmente nao se usa)
\newcommand{\LIMPAC}{
\newpage
\mbox{}
\thispagestyle{plain}
}

%
% ALTERAR AQUI AS INFORMACOES RELATIVAS AO PROJECTO
%
\newcommand{\TITULO}{DIVERSE INTRUSION-TOLERANT SYSTEMS}
\newcommand{\Autor}{Miguel Garcia Tavares Henriques}
\newcommand{\AutorNumAluno}{35054}

%Orientador e CoOrientador *sem* titulos (e.g. Prof. Doutor)
\newcommand{\Orientador}{Alysson Neves Bessani}
\newcommand{\CoOrientador}{Nuno Fuentecilla Maia Ferreira Neves} %se nao se aplicar, nao importa o que aqui esteja

%Se aplicavel, o supervisor pode ter um titulo (Dr., Eng.) colocado aqui
\newcommand{\SupervisorInstituicao}{Nome Completo do Supervisor}  %se nao se aplicar, nao importa o que aqui esteja

\newcommand{\AnoLectivo}{2017/2018}
\newcommand{\Ano}{\Large{2018}}

% Comentar/descomentar conforme conveniente
\newcommand{\TIPO}{DISSERTA\c{C}\~{A}O }
%\newcommand{\TIPO}{TRABALHO DE PROJETO }

% Comentar/descomentar conforme conveniente
%\newcommand{\MESTRADO}{MESTRADO EM -- PREENCHER --}
\newcommand{\DOUTORAMENTO}{Doutoramento em Inform\'{a}tica}

% Comentar/descomentar conforme conveniente
%\newcommand{\IdiomaTese}{\selectlanguage{portuguese}}
\newcommand{\IdiomaTese}{\selectlanguage{english}}

% Comentar/descomentar conforme conveniente
\newcommand{\Especialidade}{}

\newcommand{\Cabecalho}{
\vspace{1cm}\normalfont\normalfont
\vfill
\textsc{\normalsize\uppercase{Universidade de Lisboa}}\\
\normalsize\uppercase{Faculdade de Ci\^{e}ncias}\\
\vspace{1cm}
\includegraphics[scale=.45]{pic/logo_fcul_vertical.png}\\
}

\usepackage{ifpdf}
\ifpdf
\pdfinfo {
	/Author (\Autor)
	/Title (\TITULO)
	/Subject (\DOUTORAMENTO)
	/Keywords ()
	/CreationDate (D:20081112151803)
}
\fi

%\usepackage[dvips]{geometry}
%\geometry{a4paper=true,portrait=true,left=3cm,right=3cm,top=2.5cm,bottom=3.5cm}

\title{\TITULO}
\author{\Autor}
%\date{\today}

\usepackage{glossaries}
\makeglossaries

\begin{document}
\selectlanguage{english}
\pagestyle{empty}

% Primeira capa
% 
%
\begin{center}

\Cabecalho

\vspace{2.0cm}
\vfill
\IdiomaTese
\Large{\bf \TITULO}\\
\vspace{2cm}
\vfill

\large{\bf{\DOUTORAMENTO}}\\
%\normalsize{\Especialidade}\\
\vspace{2.0cm}
\vfill
\Large{\bf \Autor}\\
\vspace{1,8 cm}
\vfill
\large{Tese orientada por:}\\
%\normalsize{Trabalho de projeto orientado por:}\\
\large{Prof. Doutor \Orientador} \\
% DESCOMENTAR a linha relevante (se alguma), removendo o % no inicio
e co-orientada pelo Prof. Doutor \CoOrientador \\
\vspace{1 cm}
\vfill

\normalsize{Documento especialmente elaborado para a obten\c{c}\~{a}o do grau de doutor}
\vspace{0.5cm}
\vfill
\Ano
\end{center}
\newpage
\mbox{}
\newpage

%
% Segunda capa
%

\begin{center}

\Cabecalho

\vspace{0.5cm}
\vfill
\IdiomaTese
\Large{\bf \TITULO}\\
\selectlanguage{portuguese}
\vspace{1cm}
\vfill

\large{\bf{\DOUTORAMENTO}}\\
%\normalsize{\Especialidade }\\
\vspace{1cm}
\vfill
\Large{\bf \Autor}\\
\vspace{.5 cm}
\vfill
\large{Tese orientada por:}\\
%\normalsize{Trabalho de projeto orientado por:}\\
\large{Prof. Doutor \Orientador} \\
% DESCOMENTAR a linha relevante (se alguma), removendo o % no inicio
e pelo Prof. Doutor \CoOrientador \\
\vspace{.5 cm}
\vfill

\begin{flushleft}  
\large{J\'{u}ri}
\vfill
\setlength{\leftskip}{0.5cm}
\normalsize{Presidente:}
\begin{itemize}
\setlength\itemsep{-0.5 em}
\item{Nome do Presidente}
\end{itemize}
\normalsize{Vogais:}
\begin{itemize}
\setlength\itemsep{-0.5 em}
\item{Nome do Vogal}
\item{Nome do Vogal}
\item{Nome do Vogal}
\item{Nome do Vogal}
\end{itemize}
\setlength{\leftskip}{0cm}
\end{flushleft}
\vspace{0.0cm}
\vfill

\normalsize{Documento especialmente elaborado para a obten\c{c}\~{a}o do grau de doutor\par}
\vspace{0.0cm}
\normalsize{Este trabalho foi financiado pela Funda\c{c}\~{a}o para a Ci\^{e}ncia e Tecnologia (FCT) atrav\'{e}s da Bolsa Individual de Doutormento SFRH/BD/84375/2012 e atrav\'{e}s dos projectos da Comiss\~{a}o Europeia FP7-257475 (MASSIF) e H2020-700692 (DiSIEM).}
\vspace{0.cm}
\vfill
\Ano
\end{center}
\newpage
\mbox{}
\newpage
% Fim da capa
% ----------------------------------------------------------------------

% This work was supported by Fundação para a Ciência e a Tecnologia (FCT), through Multiannual
%Funding to the LaSIGE research unit and the Individual Doctoral Grant SFRH/BD/84375/2012.
\setcounter{page}{1}
\pagenumbering{roman}

\addcontentsline {toc} {chapter} {Acks}
\pagestyle{plain}

\vspace*{2cm}
\selectlanguage{english}
\chapter*{Acknowledgement}

%\begin{center}
%\selectlanguage{portuguese}
%\Large \bf Agradecimentos
%\selectlanguage{english}





%\Large \bf Acknowledgments
%\end{center}
%\vspace*{1cm} \setlength{\baselineskip}{0.6cm}



\LIMPA

\vfill

%\selectlanguage{portuguese}

\begin{flushright}\textit{To my grandmother.}\end{flushright}

\LIMPA


\addcontentsline {toc} {chapter} {Abstract (PT)}
% Pagina do resumo em portugues
%\pagestyle{empty}

% ----------------------------------------------------------------------
% P�gina do resumo em Portugu�s:
\selectlanguage{portuguese}
\chapter*{Abstract (PT)}
\label{sec:abstract}
\vspace*{-10mm}


\vfill

\begin{flushleft}
\textbf{Palavras-chave:}
Diversidade, Vulnerabilidades, Sistemas Operativos, Toler\^{a}ncia a Intrus\~{o}es, Recuper\c{c}\~{a}o Proactiva.
\end{flushleft}

\LIMPA
% Fim da p�gina do resumo em Portugu�s
% ----------------------------------------------------------------------


\addcontentsline {toc} {chapter} {Abstract (EN)}
% Pagina do resumo em ingles
% ----------------------------------------------------------------------
% P�gina do resumo em Ingl�s:
\selectlanguage{english}
%\vspace*{2cm}
%\begin{center}
%\Large \bf Abstract
%\end{center}
%\vspace*{1cm} \setlength{\baselineskip}{0.6cm}
\chapter*{Abstract (EN)}
\label{sec:abstract}
\vspace*{-10mm}


\vfill

\begin{flushleft}
\textbf{Keywords:}
Diversity, Vulnerabilities, Operating Systems, Intrusion Tolerance, Proactive Recovery.
\end{flushleft}

\LIMPA

% Fim da p�gina do resumo em Ingl�s.
% ----------------------------------------------------------------------




\tableofcontents

\LIMPA

%Lista de figuras
\listoffigures

\addcontentsline {toc} {chapter} {List of Figures}
\newpage
\thispagestyle{empty}
\mbox{}
\newpage

%Lista de tabelas
\listoftables

\addcontentsline {toc} {chapter} {List of Tables}
\newpage
\thispagestyle{empty} \mbox{}
\newpage

% ----------------------------------------------------------------------
% Inicio conteudo
\pagestyle{fancy}
\cleardoublepage

\setcounter{page}{1}
\pagenumbering{arabic}
\chapter{Introduction}
\label{chap:introduction}

\section{Context and Motivation}
\gls{bft}\footnote{In this thesis we interchange Byzantine fault tolerance (subject) with Byzantine fault-tolerant (adjetive) using the same acronym BFT for both.} is a well-established area of research that aims to guarantee safety on replicated systems even in the presence of some (Byzantine) faulty nodes.
In a nutshell, \gls{bft} protocols guarantee that replicas agree on the order of the message execution, and thus, working as a replicated state machine.
This holds even if a subpart $f$ out of $n$ is faulty, then there is always a sufficient number of correct nodes to execute correclty.
Although not explicit, this assumption leverages on the strict condition that the nodes (must) fail independly.
Otherwise, compromising $f+1$ is virtually the same than compromising $f$.


\gls{bft} was first proposed in 1982 by Lamport~\etal{}~\cite{Lamport:1982}, but it only as awaken the distributed systems research community to its relevancy in 1999 due to Castro and Liskov's Practical \gls{bft}~\cite{Castro:1999}. 
In the last twenty years of active research on \gls{bft} replication, there were made great advances on the performance (e.g.,~\cite{Kotla:2010,Aublin:2015,Behl:2015}), use of resources (e.g.,~\cite{Yin:2003,Wood:2011,Veronese:2013,Liu:2016,Behl:2017}), and robustness (e.g.,~\cite{Amir:2011,Bessani:2014,Clement:2009b}) of \gls{bft} systems.
However, \gls{bft} in general and these works in particular, assume, either implicitly or explicitly, that the replica nodes fail independtly. 
This assumption guarantees that the fault threshold is extended in time as it is more time consuming to compromise different replicas than compromise replicas that are equal, thus share the same weaknesses.
Nevertheless, a few works rely on some orthogonal mechanism (e.g.,~\cite{Roeder:2010,Chen:1995}) to avoid these common weaknesses, or rule out the possibility of malicious failures from their ystem models.
Moreover, a few works have implemented and experimented with such mechanisms~\cite{Rodrigues:2001,Roeder:2010,Amir:2011}, but in a very limited way.
Even if considering an initial set of $n$ diverse replicas (i.e., to the assumption), long-running services need to be cleaned from possible failures and intrusions.
A few works on the proactive recovery of \gls{bft} systems~\cite{Castro:2002,Sousa:2010,Roeder:2010,Platania:2014,Distler:2011} periodically restart their replicas to clean undetected faulty states introduced by a stealth attacker. 
However, a common limitation of these works is that they assume that these weaknesses will be cleaned after the recovery.
In practice, this will not happen unless the replica changes its software (e.g., via the previously described techniques) after its recovery.



In practice, diversity has shown that it is a fundamental building block of dependable services.
For example, in avionics~\cite{Yeh:2004}, military systems~\cite{rhimes}, and even in recent blockchain platforms such as Ethereum\footnote{\url{https://www.reddit.com/r/ethereum/comments/55s085/geth_nodes_under_attack_again_we_are_actively/}} (three essential applications of \gls{bft}), and the \gls{dns}~\cite{Shue:2013}.~\footnote{\url{https://secure64.com/dns-diversity/}}. 
AUMENTAR A EXPLICACAO DE CADA UM 


Despite the maturity of \gls{bft} solutions, no one addressed the challenge of building dependable \gls{bft} systems that consider evidence supported decisions on diversity and 



\section{Objectives and Contributions}

DEFINE THE GENERIC OBJECTIVES OF THE THESIS 

\subsection{Evidence for Supporting Diversity}% Diversity Study on Off-the-Shelf Operating Systems}
The current solutions that vouch for diversity as a way to guarantee failure independence, either lack of supporting evidence, or use evidence that is limited to some extent.
Therefore, some of the results may provide false conclusions.
We identify a need for finding \emph{accurate} sources for supporting sound evidence of the benefits diversity as a dependable mechanism.


The results achieved during the MSc thesis encouraged us to extend the work on \gls{ots} \glspl{os} common vulnerabilities~\cite{Garcia:2012}.
In this extension, we devised three manual strategies for selecting diverse software components to minimize the incidence of common vulnerabilities in replicated systems.
Moreover, we observed that using different \gls{os} releases of the same \gls{os} are enough to warrant its adoption as a more straightforward, less complicated, more manageable configuration for replicated systems.

\begin{enumerate}
\item[1.] \textbf{Analysis of operating system diversity for intrusion tolerance}, Miguel Garcia, Alysson Bessani, Ilir Gashi Nuno Neves, and Rafael Obelheiro, in \emph{Software: Practice and Experience, 2014}~\cite{Garcia:2014}.
\end{enumerate}



\subsection{Applying Diversity on BFT Systems}
%{\system, a Diversity and Recovery Manager for BFT systems}

A few works consider the diversity of replicas as a way to achieve such failure independence, however, it is mostly taken for granted.
For example, by using memory randomization techniques~\cite{Roeder:2010} or different \glspl{os}~\cite{Rodrigues:2001,Junqueira:2005} without providing evidence for such independence. 
Moreover, it has been shown that memory randomization does not suffice to impede common failures to occurring~\cite{Snow:2013,Bittau:2014}, and that although diversity promotes fault independence to some extent, it does not avoid utterly different \glspl{os} from sharing vulnerabilities~\cite{Garcia:2014}.


Therefore, some of the results may provide false conclusions.
We identify a need for finding \emph{accurate} sources for supporting sound evidence of the benefits diversity as a dependable mechanism.

A few works use automatic and artificial diversity (e.g.,~\cite{Roeder:2010,Amir:2011}). 
However, they lack evidence to support the failure of independence through diversity. 
Moreover, some studies show that these techniques fail to provide real diversity~\cite{Snow:2013,Bittau:2014}. 
Additionally, the existent systems that implement time-triggered recoveries assume that it takes the same time to compromise each replica, by assuming that vuleranbilies are all the same. 
This assumption is unrealistic, especially when the diversity of replicas is considered~\cite{Nayak:2014}. 
Therefore, it is required tailored methods to evaluate the risk of a replicated system becoming compromised.


The few works have implemented and experimented with such mechanisms~\cite{Rodrigues:2001,Roeder:2010,Amir:2011}. 
However, despite the lack of evidence for supporting the diversity claim, they lack mechanisms that can make practical \gls{bft} systems using diversity in a continuous mode operation (i.e., with recoveries).  
Thus, reducing the costs typically associated with the management of such complex systems which deemed them as practical.


The problem is addressed from both a theoretical and practical perspective.


We address the problems of \emph{finding evidence for supporting diversity}, \emph{manage diversity in a dependable way}, and \emph{supporting diversity mechanism} in the thesis main contribution.
We use different \gls{osint} data sources to build a complete knowledge base about the possible vulnerabilities, exploits, and patches related to the systems of interest. 
Moreover, this data is used to create clusters of similar vulnerabilities, which potentially can be affected by (variations of) the same exploit. 
These clusters and the other collected data are used to assess the risk of the \gls{bft} system becoming compromised due to common vulnerabilities.
Once the risk increases, the system replaces the potentially vulnerable replica by another one, to maximize the failure independence of the replicated service.
The solution continuously collects data from the online sources and monitors the risk of the \gls{bft} in such a way that removes the human from the loop.
We developed these contributions in a solution named \system, and it is the first system that manages \gls{bft} replicated systems (e.g., \sieveq) in a dependable and automatic way.
\system experimental evaluation shows that its strategy reduces the number of executions where the system becomes compromised and that our prototype supports the execution of full-fledged \gls{bft} systems in diverse configurations with 17 \gls{os} versions, reaching a performance close to a homogeneous bare metal setup. 


Patches take time to apply~\cite{Frei:2010}

MAking exploit from patches that were not yet released!!!\cite{Brumley:2008}
THE NEED FOR AUTOMATIC PATCHING~\cite{Nappa:2015} --shared code, one takes the patch and the other no


The contributions of this work resulted in the following publications:

\begin{enumerate}

\item[2.] \textbf{DIVERSYS: DIVErse Rejuvenation SYStem}, Miguel Garcia, Nuno Neves, and  Alysson Bessani in the \emph{Simp\'{o}sio Nacional de Inform\'{a}tica (INFORUM), 2012~\cite{Garcia:2012b}}.


\item[3.] \textbf{Towards an Execution Environment for Intrusion-Tolerant Systems}, Miguel Garcia, Alysson Bessani, and Nuno Neves, Poster session in the \emph{European Conference on Computer Systems (EuroSys), 2016}~\cite{Garcia:2016b}.


\item[4.] \textbf{\system: Automatic Management of Diversity in BFT Systems}, Miguel Garcia, Alysson Bessani, and Nuno Neves -- \emph{Submitted for publication}.

\end{enumerate}


\subsection{BFT Multi-layer Resiliency} 
Most generic firewall solutions suffer from two inherent problems: 
First, they have vulnerabilities as any other system, and as a consequence, they can also be the target of advanced attacks. 
For example, the \gls{nvd}~\cite{nvd} shows that there have been many security issues in commonly used firewalls. 
\gls{nvd}'s reports present the following numbers of security issues between 2010 and 2015: 157 for the Cisco Adaptive Security Appliance; 109 in Juniper Networks solutions; 29 for the Sonicwall firewall; and 24 related to iptables/netfilter. 
Common protection solutions often have been the target of malicious actions as part of a wider scale attack (e.g., anti-virus software~\cite{Chauhan:2011}, \gls{ids}~\cite{Anderson:2001} or firewalls~\cite{Kamara:2003,Surisetty:2010,cisco1,cisco2}).
Second, firewalls are typically a single point of failure, which means that when they crash, the ability of the protected system to communicate may be compromised, at least momentarily.
Therefore, ensuring the correct operation of the firewall under a wide range of failure scenarios becomes imperative.
To tolerate faults, one typically resorts to the replication of the components.

In the last decade, several important advances occurred in the development of intrusion-tolerant systems.
However, to the best of our knowledge, very few works proposed intrusion-tolerant protection devices, such as firewalls.
Performance reasons might explain this, as \gls{bft} replication protocols are usually associated with significant overheads and limited scalability.
Additionally, achieving complete transparency to the rest of the system can be challenging to reconcile with the objective of having fast message filtering under attack.


we propose a new protection system called \sieveq that mixes the firewall paradigm with a message queue service, with the goal of improving the state-of-the-art approaches under accidental failures and/or attacks.
The solution has a fault- and intrusion-tolerant architecture that applies filtering operations in two stages acting like a sieve.
The first stage, called \emph{pre-filtering}, performs lightweight checks, making it efficient to detect and discard malicious messages from external adversaries.
In particular, messages are only allowed to go through if they come from a pre-defined set of authenticated senders.
\gls{dos} traffic from external sources is immediately dropped, preventing those messages from overloading the next stage.
The second stage, named \emph{filtering}, enforces more refined application level policies, which can require the inspection of some message fields or need the enforcement of specific ordering rules.



Typical \gls{bft} protocols use one of the two following approaches to disseminate messages to the replicas: (1) traffic replicator before the replicas or (2) a leader is responsible to dessiminate the messages to the other replicas. 
The dissemination of a message to all replicas can be detrimental to the proper operation of the replicated service.
For example, a traffic replicator device (e.g., hub) can be placed at the entry of the system to transparently reproduce all messages~\cite{Sousa:2010,Roeder:2010}. 
The effect is an attack amplification caused by the replicator device.
Alternatively, a leader replica could receive the traffic and then disseminate the messages to the others~\cite{Amir:2011}.
The drawback is that the leader becomes a natural bottleneck, especially when under attack (instead of dispersing the attack load over all replicas).
%Then, new architectures can be designed to accommodate external attacks in a resourceful and resilient manner.

In \sieveq, we explore a different design for replicated protection devices, where we trade some transparency on senders and receivers for a more efficient and resilient firewall solution.
In particular, we propose an architecture in which critical services and devices can only be accessed through a message queue and implement the application-level filtering in this queue.
It is assumed that these services have a limited number of senders, which can be appropriately configured to ensure that only they are authorized to communicate through \sieveq.



The contributions of this work resulted in the following publications:

\begin{enumerate}
\item[5.] \textbf{An Intrusion-Tolerant Firewall Design for Protecting SIEM Systems}, Miguel Garcia, Nuno Neves, Alysson Bessani, in the \emph{Workshop on Systems Resilience in conjunction with the IEEE/IFIP International Conference on Dependable Systems and Networks, 2013}~\cite{Garcia:2013}.

\item[6.] \textbf{\sieveq: A Layered BFT Protection System for Critical Services}, Miguel Garcia, Nuno Neves, and Alysson Bessani, in \emph{IEEE Transactions on Dependable and Secure Computing, 2018}~\cite{Garcia:2016}.
\end{enumerate}



To conclude, the colaboration with Andr\'{e} Nogueira resulted in a (out of the scope of this thesis) work on \gls{scada} system enhanced with \gls{bft} techniques. 
We documented the challenges of building such system from a ``traditional'' non-\gls{bft} solution.
This effort resulted in a prototype that integrates the Eclipse NeoSCADA and the BFT-SMaRt open-source projects.
This solution could be managed by \system:


\begin{enumerate}

\item[7.] \textbf{On the Challenges of Building a BFT SCADA}, Andr\'{e} Nogueira, Miguel Garcia, Alysson Bessani, and Nuno Neves, in \emph{Proceedings of the International Conference on Dependable Systems and Networks, 2018 }~\cite{Nogueira:2018}.
\end{enumerate}


\subsection{Thesis Statement}
We summarize our findings in the following thesis statement:

\vspace{2mm}
\fbox{ \begin{minipage}{35em}
%\begin{center}
\emph{
It is possible to build dependable BFT replicated systems by minimizing the number of replicas' common vulnerabilities through software's diversity.
Additionally, it is possible to continuously manage these systems while monitoring OSINT data and deciding when replicas should be diversified and deploying the most dependable configurations.
}
%\end{center}
\end{minipage}
}





\section{Thesis Overview}
Detailed description of each Chapter

\chapter{Literature Review}
\label{chap:literaturereview}

\section{Byzantine Fault Tolerance}
\label{sec:bft}

Intrusion tolerance was first proposed by Fraga and Powell in 1985~\cite{Fraga:1985} as a solution to address faults without compromising the security of a system. 
Almost 15 years after, \gls{bft} replication became the most common solution to implement intrusion tolerance systems.
Castro and Liskov’s PBFT~\cite{Castro:1999} was the first practical \gls{bft} protocol. 
PBFT implements a \gls{smr} protocol that guarantees both liveness and safety for $\lfloor\frac{n-1}{3}\rfloor$ out of a total of n replicas. 
These properties hold even in asynchronous systems such as the internet. 
PBFT implements a SMR, therefore, it must guarantees that each replica executes the same commands, in the same order, and then it produces the same output. 
The protocol can be briefly described as follows: a client sends a message to all the replicas.
Then, the leader replica has to assign a sequence number to the request and multicast a pre-prepare message to the other replicas. 
If the replicas agree with the leader they send a prepare message to each other. 
At this phase of the protocol, every correct replica agrees on the ordering. 
A commit message is sent by every replica. When a replica receives the commit message from a quorum it executes the message. 
In the end, it replies to the client. 
Every replica shares a key with each other and with clients. These keys are used to authenticate messages with a \gls{mac}. 
The messages that are multicast by the clients are authenticated with a vector of \glspl{mac}. 
Then, each replica verifies its own \gls{mac}.
The authors implemented BFS, a Byzantine fault tolerant Filesystem, to validate this \gls{bft} library. 
The results show that when the workload increases the throughput and latency is nearly the same of a non-replicated system. 
This level of performance encouraged the use of \gls{bft} in common systems, and to develop optimizations to improve \gls{bft} protocols. 
Since the PBFT proposal, some work has been dedicated to improve \gls{bft} protocols (see Table~\ref{tab:bft}):


\begin{table}[!t]
\begin{center}
{\footnotesize
\begin{tabular}{ p{2.5cm}  p{11cm}  }\hline
Zyzzyva~\cite{Kotla:2010}  & Introduces speculation to avoid the expensive three phase commit before processing the requests. This might introduce some inconsistency in the state of the replicas when speculation fails, and the client needs to help replicas to fix the servers’ inconsistency. \\ \hline			
Upright~\cite{Clement:2009} & It provides a straightforward way to also add BFT to crash fault tolerant systems (CFT); \\ \hline	
Aardvark~\cite{Clement:2009b} & Shifts the paradigm to a new design. This design improves the performance under faulty scenarios trading some performance on the normal case; \\ \hline
BFT-SMaRt~\cite{Bessani:2014} & It is modular and multicore-aware. Supports replica reconfiguration and has a flexible programming interface; \\ \hline
COP~\cite{Behl:2015} & It is the most recent BFT implementation that reached 2.4 million operations per second. This was achieved mostly due to BFT architecture changes.\\  \hline  
\end{tabular}
}
\caption{Brief overview of the most relevant BFT works.}
\label{tab:bft}
\end{center}
\end{table}

\paragraph{Summary.} 
Intrusion tolerance has an important role in our work, namely BFT-replication.
We propose a system that enhances intrusion tolerance by adding mechanisms that work on
top of a \gls{bft} replicated system. We need \gls{bft} protocols to guarantee that replicas execute in
the same order and to be able to tolerate some malicious failures in a subset of the replicas.
The implementation of such \gls{bft} protocol is a complex task, particularly if we need to
include mechanisms for state transfer, reconfiguration, and guarantee a good performance.
We prefer to rely on existent libraries than to build a new one.

\section{Rejuvenations}
\label{sec:rejuvenations}

Software rejuvenation was proposed in the 90's~\cite{Huang:1993,Huang:1995} as a proactive approach to prevent performance degradation and failures due to software aging. 
The first proposals were based on periodical clean-up of the aging effects by restarting some parts of the software. 
This mechanism would postpone failures and restore the performance. 
This solution was implemented using three components: a watchdog process (\texttt{watchd}), a checkpointing library (\texttt{libft}) and a replication mechanism (\texttt{REPL}). 
A primary node executes the application while a backup node has the application inactive. 
The primary also runs the \texttt{watchd} to monitor the application crashes and hangs. 
In the backup, there is another watchdog watching the primary node. 
There is a routine (provided by \texttt{libft}) that periodically makes checkpoints and logging. 
These checkpoints are replicated with REPL in the backup node. 
When the primary node crashes or hangs, it is restarted, and if needed the backup takes his place on the execution.
PBFT-PR~\cite{Castro:2002} and COCA~\cite{Zhou:2002} introduced proactive recoveries in \gls{bft} replicated systems. 
Proactive Recovery is a mechanism that periodically rejuvenates the replicas to clean potential faults or stealth attacks. 
This mechanism allows that an adversary controls some replicas, usually $f$ , during a time window, where the system is vulnerable.
Castro and Liskov presented PBFT-PR~\cite{Castro:2002}, a BFT replication library that does proactive recovery (PR). 
PBFT-PR assumes that replicas will frequently recover. 
Additional assumptions are needed to guarantee the PBFT-PR's liveness and safety during the recoveries: 
(i) Each replica contains a trusted chip to store its private key, and it can sign and decrypt messages without revealing the key; 
(ii) The replicas’ public keys are stored in a BIOS read-only memory, which needs physical access to modify the BIOS; 
And last, (iii) a watchdog timer is used to avoid human interaction to restart replicas. 
The watchdog hands the execution to the recovery monitor, which cannot be interrupted.
Zhou~\etal{} presented COCA~\cite{Zhou:2002}, a fault-tolerant and secure online certification authority that has been built and deployed both in a local area network and on the internet.
COCA uses proactive signature sharing to ensure that when replicas fail and recover there
is no key-leakage. Contrary to PBFT-PR it does not rely on trusted components, but it may
need an administrator to refresh the COCA server keys.
Distler~\etal{} identified virtualization as a useful mechanism to implement proactive recovery in VM-FIT~\cite{Reiser:2007,Distler:2008}. 
The authors assume that an intrusion-tolerant replicated system executes in an untrusted domain, running in \gls{vm}. 
VM-FIT executes in a trusted domain, i.e., the \gls{vmm}. 
Virtualization allows the isolation between the untrusted and the trusted domains. 
Therefore, it can trigger recoveries from the trusted domain in a synchronous manner. 
Moreover, virtualization reduces downtime of the service during the recovery and makes the state transfer between replicas more efficient. 
The authors implemented this system using the Xen hypervisor to get the two domains: the trusted domain is the Xen Dom0, and the replicas run in the untrusted domains, DomUs.

Sousa~\etal{} improved the state-of-the-art recovery algorithms by introducing \gls{prrw}~\cite{Sousa:2010}. 
This technique removes the effects of faults ``immediately''. 
\gls{prrw} accelerates the rejuvenation process by detecting the faulty replicas behavior and forcing them to recover without sacrificing periodic rejuvenations. 
This type of technique can only be implemented with synchrony assumptions as the recoveries are time triggered~\cite{Sousa:2005}. 
To address this need the authors proposed an hybrid system model: the payload is an any-synchrony subsystem, and the wormhole is a synchronous subsystem. 
The authors implemented this system using the Xen hypervisor as the wormhole. 
Zhao~\etal{}~\cite{Zhao:2010} improved~\cite{Sousa:2010} with an algorithm to schedule the rejuvenations that is adaptable to a constant monitoring of the network and CPU/memory performance.

\paragraph{Summary.} \gls{bft} replication with proactive recovery represents one of the cornerstones of intrusion tolerance. 
PR allows the system to reduce the vulnerability window of the replicated system drastically. 
BFT replication with PR guarantees the system's correctness while the replicas recover faster than $f+1$ replicas become faulty. 
One limitation of PR is that there must be a trusted element somewhere to implement this mechanism. 
Some works use hardware timers while others resort to virtualization to separate the execution domains. 
In our proposal, we will adopt an architecture similar to~\cite{Distler:2008} and ~\cite{Sousa:2010}. 
However, our solution will implement a different (from~\cite{Sousa:2010}) algorithm to proactively trigger recoveries.


\section{Diversity}
\label{sec:diversity}
In some of the works presented before, the correctness is ensured under the assumption that replicas fail independently. 
Often the authors assume that fault independence is achieved using diversity~\cite{Castro:2002,Sousa:2010}. 
Diversity can be implemented in different ways~\cite{Deswarte:1999,Obelheiro:2006}, which may differ in the mean and the amount of diversity generated, but the goal is the same: to create different attack surfaces. 
The goal is to make the discovery of common vulnerabilities more difficult. 
We present at least three different ways to generate diversity: 
(i) N-version programming consists in design/implementation of different program binaries (from the same or different source code); 
(ii) Randomization implements different memory schemes; 
and (iii) Off-the-shelf diversity comprises the use of different products with the same functionality but taking advantage of implementations that are already available.


\paragraph{N-version programming}
N-version programming is a technique used to create diverse software components~\cite{Avizienis:1977,Knight:1986,Chen:1995}.
The main idea behind this mechanism is to develop N different implementations of the same specification. 
These implementations can be implemented by N different developers or using translators to N different programming languages.

\paragraph{Artificial diversity.} 
Forrest~\cite{Forrest:1997} suggested randomized program transformations to introduce application diversity. 
They have made modifications on the gcc, a C compiler, in such a way that during the compiling time gcc inserts random padding
into each stack frame. 
Linger presented stochastic diversification tool to swap code structures~\cite{Bairavasundaram:2009,Larsen:2014}. 
The diverse versions are generated during the source code compilation. 
The main idea of these works is to make the intruder’s work more difficult when exploiting buffer overflow vulnerabilities.
Bhatkar~\etal{}~\cite{Bhatkar:2003} proposed a solution based on memory address obfuscation (i.e., \gls{aslr}). 
Their solution transforms object files and executables at the link-time and load-time, without kernel or compiler modifications. 
The goal is to ensure that an attack that succeeds in one target will not succeed on the other targets. 
Each time the program is executed its virtual addresses and data are randomized, and therefore the attacker needs to find new ways to exploit memory errors, like buffer overflows. 
However, a recent work proved that solutions like ASLR are vulnerable to attacks~\cite{Bittau:2014}.
A few works propose the usage of compilers that generate different executables~\cite{Platania:2014,Roeder:2010,King:2016}.


\paragraph{Off-the-shelf diversity.}
The two previous solutions generate diversity before the software’s distribution. 
\gls{ots} diversity does not need pre-distribution of diverse mechanisms. 
It relies on the existence of different software components that are ready to be used.
There are plenty of different free products that provide the same functionality and were developed by different vendors. 
In other words, \gls{cots} diversity is like an opportunistic N-version programming.
Totel~\etal{} proposed an \gls{ids} based on design diversity~\cite{Totel:2005}.
The authors described an architecture that uses a set of replicated \gls{cots} servers, a proxy, and an \gls{ids}. 
The proxy is responsible for forwarding the requests from the client to every \gls{cots} server. 
When the servers reply, the proxy sends the responses to the \gls{ids} to be analyzed. 
The IDS compares the responses from the \gls{cots} servers and if it detects some differences an alarm is raised. 
Then the proxy votes the responses and replies to the clients. 
The authors developed an algorithm to detect intrusions and tested their solutions against Snort~\cite{snort} and WebStat~\cite{Vigna:2003}. 
The results show that using diverse-\gls{cots} service (e.g., http) allows the \gls{ids} to deliver fewer alarms without missing the intrusions.
Gashi~\etal{} made an experimental evaluation of the benefits of adopting diversity of SQL database servers~\cite{Gashi:2007}. 
The authors analyzed bug reports of four database servers (Post-greSQL, Interbase, Oracle, and Microsoft SQL Server) and verified which products were affected by each bug reported. 
They have found few cases of a single bug affecting more than one server. 
In fact, there were no coincident failures in more than two of the servers.
The conclusion is that \gls{ots} database servers’ diversity is an effective mean to improve system's reliability. 
However, the authors recognize the need for SQL translators, to increase the interoperability between servers in the replicated scenario.
Han~\etal{}. made a systematic analysis of the effectiveness of using \gls{ots} diversity to improve system's security~\cite{Han:2009}. 
First, the authors found if there were \gls{ots} software substitutes to provide the same functionality. 
Then, they determined if the \gls{ots} software shared the same vulnerabilities.
And if so, if the same vulnerability could be exploited with the same attack. 
In this study, more than 6k 1 vulnerabilities were analyzed, which were published in the \gls{nvd} on the year 2007. 
The results showed that 98.5$\%$ of the vulnerable software have substitutes. 
Moreover, the majority of them either did not had the same vulnerabilities or could not be compromised with the same exploit code. 
It is not expected that a single exploit works in different \glspl{os} because each one has a different memory scheme and different filesystems. 
Even between versions from the same \gls{os}, the low-level functions change across the versions. 
The study also concluded that 22.5$\%$ of the vulnerabilities were present in multiple software. 
However, only 7.1$\%$ from those vulnerabilities were present in software that offers the same service. 
The study findings are a good sign that diversity can improve a system’s dependability.
In a previous work, we studied \gls{ots} OSes diversity~\cite{Garcia:2014}. 
Similar to \cite{Han:2009} this study was carried out taking vulnerability feeds from \gls{nvd}. 
However, our work was only focused on OSes, and the data collected comprised 11-year of vulnerability reports. 
Our goal was to find to what extent different OSes shared common vulnerabilities. 
In this study, we analyzed 2270 vulnerabilities entries manually, where they were classified in different categories. 
Then, we defined three types of servers: (i) a server that contains most the packages/applications available (Fat server); (ii) a server that does not contain unnecessary applications to a certain
service (Thin server); and (iii) a thin server but with physically controlled access (Isolated hin server). 
We assumed that the third setting it is the most advisable for critical systems, since additional care is taken to install and setup its configuration. 
For each configuration, we compared all the common vulnerabilities in pairs of OSes. 
As expected, in the Isolated thin server, the number of common vulnerabilities was considerable less than in the other configurations. 
There was only 1 vulnerability that was shared among six OSes, 2 that were shared among five OSes, and 130 that are shared between two OSes. 
We went further in the study and looked for common vulnerabilities in different versions of the same OS. 
We have found evidence that suggests that using OS diversity in a replicated system can improve its dependability. 
Even if few OSes are employed, it is possible to achieve vulnerability independence just with different versions.



Larsen~\etal{} made a study with several examples on how to apply diversity~\cite{Larsen:2014,Larsen:2015}. 
The paper explains in detail what to diversify, from single instructions to an entire program. 
They also explain the three types of security impacts when diversity is applied: entropy analysis, attack-specific code analysis, and logical argument. 
However, they also point that there is no consensus on how to evaluate the efficacy of diversity. 



\paragraph{Summary.}
We presented three different techniques that support diversity as a fault-tolerant mechanism. 
N-version programming is the most costly, since it needs different developers or additional programs to verify if the automatic translation works. 
On the contrary, \gls{ots} and randomization diversity are almost for free. 
The first one can be obtained from free sources that are ready to be used, and the second one is automatic and already supported
in most OSes. 
Diversity still has some gaps that must be addressed to make it an effective building block of intrusion tolerance. 
For example, how to measure diversity in such a way that failure independence can be guaranteed?

%\section{Vulnerabilities}
%Several studies use vulnerabilities to assess the security of systems, using their life cycle and so on....

%We considered, as in previous works cite, vulnerabilities can be used for dependability

\section{Intrusion-tolerant Firewalls}
\label{sec:intrusiontolerantfirewalls}

\note{Improve-- make it more embeded}
Over the past years, there has been an important amount of research in the development of systems that are intrusion tolerant.
However, only very little work was devoted to design intrusion-tolerant firewall-like devices. Performance reasons might explain this, as \gls{bft} replication protocols are usually associated with reasonable overheads and limited scalability.

Sousa et al.~\cite{Sousa:2010} proposed the first replicated intrusion-tolerant system that implements proactive-reactive recovery. 
The paper presented a firewall for critical services, named \gls{cis}, which was integrated with a trusted component and works under the assumption of an hybrid synchrony model.


Roeder and Schneider~\cite{Roeder:2010} proposed a replicated intrusion-tolerant device that introduces diversity through software obfuscation techniques on each rejuvenation.
The authors main focus was in supporting diversity, a technique that could be integrated into our work.


There are several approaches to defend systems from \gls{dos}/\gls{ddos} (see a survey in~\cite{Zargar:2013}).
For example, Walfish et al.,~\cite{Walfish:2010} proposed a solution based on active response to \gls{ddos}. 
Basically, upon the detection of the attack, the server requests the client to send more data. 
The idea is that by increasing the load, the network congestion management mechanisms will make the channels be used in a more fair way among the correct and incorrect clients. 
Unfortunately, this solution is not resource efficient because it overloads the channels for the benefit of the client.
Jia~\etal{}, proposed a solution based on traffic redirection. It works on cloud-based services to solve \gls{ddos}.
Upon the detection of a \gls{ddos} the system redirects the correct client to non-attacked servers~\cite{Jia:2014}.
If the network support is available, this kind of technique could be used together with \sieveq to ensure \senders will always be able to reach a correct \presieve.

\paragraph{Summary}
\sieveq shares a few similarities with \gls{cis}, but there are two fundamental differences. 
The \gls{cis} needs traffic replicators, one for in-bound and other for out-bound messages, and therefore, an external DoS will be replicated to all replicas; another difference is that the \gls{cis} is a stateless firewall.


In the last decade, several important advances occurred in the development of intrusion-tolerant systems.
However, to the best of our knowledge, very few works proposed intrusion-tolerant protection devices, such as firewalls.
Performance reasons might explain this, as \gls{bft} replication protocols are usually associated with significant overheads and limited scalability.
Additionally, achieving complete transparency to the rest of the system can be challenging to reconcile with the objective of having fast message filtering under attack.



\section{Security Metrics}
\label{sec:securitymetrics}

Jumratjaroenvanit and Teng-amnuay~\cite{Jumratjaroenvanit:2008} described vulnerability life-cycles presenting their different stages. 
The authors goal was to understand how different dates can be useful to estimate the probability of an attack. 
The authors methodology was to collect and analyze data on the discovery, disclosure, and exploit dates, exploit and patch availability, all from public data sources. 
They used this information to define five life cycle types: (i) \gls{zda}, which typically is a done by a black-hat and is defined by the date of disclosure and exploit being the same; (ii) \gls{pzda}, is a \gls{zda} but with a patch already available, but not applied; 
(iii) \gls{ppzda}, is a \gls{pzda} but does not matter if the patch is available or not; 
(iv) \gls{poa}, it is a zero-day vulnerability. 
The most influential variables were the code that was scrutinized by a the tester with access to the source code, and the code that was analyzed with some static tool before. 
On the contrary, results showed that language safety was the less influent factor. 
The main result of the study was that two weeks of work were enough to have a 50\% chance of finding a zero-day vulnerability. 
Sometimes, 53 hours were enough to find one vulnerability with more than 95\% of probability.


Okhravi~\etal{} presented TALENT a framework for live migration of critical infrastructures across heterogeneous platforms~\cite{Okhravi:2014,Okhravi:2009}. 
The idea was to create a moving target that difficult the success of advanced targeted attacks. 
TALENT implements OS level virtualization with containers, which allows the system to migrate the OS between machines periodically or upon detection of malicious activity. 
The virtualization was implemented with OpenVZ and LXC for Linux, Virtuozzo for Windows, and Jail for FreeBSD. 
The network was also virtualized, more precisely, a second layer of virtualization was used to migrate the IP address from one container to another. 
Even an established ssh session was preserved during the migration. 
Additionally, the state of the application also had to be migrated by employing a checkpointing technique. 
When all the programs were checkpointed, the state was saved and then was migrated by mirroring the filesystem. 
The filesystem synchronization took 98.7\% of the migration time. 
The authors decided to focus on optimizing the filesystem synchronization. 
In the optimized version, the filesystem synchronizes in periodic intervals by sending the differences to the destination. 
Therefore, the migration was made seamless to the application. 
TALENT also had a risk assessment, called operation assessment, which monitored and adjusted the diverse components using vulnerability information. 
However, there was almost no details on how the risk was measured and on how to adapt to the threats in an efficient manner.

Guo and Bhattacharya addressed the problem of \gls{bft} replicas fault independence~\cite{Guo:2014}.
The authors proposed different replicas configurations using OTS OSes to increase the diversity among the replicas. 
There was a set of different configurations that could build a replicated system. 
Each configuration was composed of different \gls{os}. 
The authors proposed a formalization of the virtual replica diversification problem. 
They used a game-theory approach to find an optimal diversification strategy for the defender's side. 
The authors validated their model with the \gls{nvd} data used in~\cite{Garcia:2012}.
Wang~\etal{} proposed a different approach to measure the risk~\cite{Wang:2014}. 
The study explored how many zero-day vulnerabilities were required to compromise a network. 
They assumed that zero-day vulnerabilities typically do not occur at the same time, however the solution did not depend on any statistical model. 
The reasoning behind the proposal was novel, e.g., on the method to find the complexity of multiple zero-day vulnerabilities in a N-node network.

Newell~\etal{} ~\cite{Newell:2015} presented a solution to assign diversity variants among network nodes to increase the network's resiliency.

Sabote~\etal{}~\cite{Sabottke:2015} presented a study that used non-common vulnerability data sources, i.e., Twitter data (e.g., specific words, the number of retweets, and replies, information about
the users posting these messages), to find exploit data. 
The authors made a quantitative and qualitative exploration of the vulnerability-related information disseminated on Twitter.
They developed a Twitter-based exploit detector to provide an adequate response in a such short time frame. 
This allows the security community to foresee the exploit activity before the information reaches the de facto disclosure data sources like \gls{nvd} and ExploitDB.
However, to complement and strength their information results they also used sources like \gls{nvd}, \gls{osvdb}, Exploit-DB, Microsoft Security Advisories and Symantec WINE. 
The authors distinguished the real-world exploits from the proof-of-concept exploits, as the latter are by products of the disclosure process. 
The real-world exploits typically are not known until a critical zero-day attack occurs. 
The authors used \gls{svm} to classify exploits in the social media and tested how robust it was to attacks, i.e., if a powerful attack could manipulate the information on Twitter with a false account and false data. 
The results showed that with their proposal they could be confident with the predictions. 
For example, the \gls{svm} classifier set to a precision of 25\% could detect the Heartbleed exploits within 10 minutes of its first appearance on Twitter. 
They also showed that organizations like \gls{nvd} overestimate the severity of some vulnerabilities that never become in fact exploited.
However, their work did not focus on the discovery of the zero-day vulnerabilities, which is relevant for our research.

\paragraph{Summary.} We presented several works that carry out for risk assessment in a way to prevent exploits or to take action upon vulnerability detection. 
This is the last building block that intrusion tolerance is missing. 
There is a lot of relevant free information available to address the security and dependability of software. 
In a replicated context this information is even more relevant. 
First, to react upon vulnerability/exploit disclosures, and second to select what configurations are less vulnerable to common weakness. 
We want to explore the free available data to improve, the dependability and security of intrusion-tolerant systems, by ensuring the assumption of failure independence with evidence supported by  relevant data.


\section{Systems Management}
\label{sec:XX}
\note{Change title}

In the last two decades there were a number of \gls{bft} protocols and systems deemed ``practical'' (e.g.,~\cite{Castro:2002,Kotla:2010,Veronese:2013,Aublin:2015,Behl:2015,Behl:2017,Liu:2016,Yin:2003}).
Most of these systems either ignore the issue of fault independence or simply assume it is solved in some way (e.g., N-version programming~\cite{Chen:1995} or \gls{ots} diversity~\cite{Gashi:2007,Garcia:2014}).
In principle, \system can support the execution of all these systems/protocols, as long as they support, or are extended to support, replica group reconfigurations, just like BFT-SMaRt~\cite{Bessani:2014}.
In this section, we discuss the few previous works that address the issue of diversity in \gls{bft} systems. 

BASE~\cite{Rodrigues:2001} is an extension of PBFT~\cite{Castro:1999} that explores opportunistic \gls{ots} diversity in \gls{bft} replication. 
The system provides an abstraction layer for running diverse service codebases on top of the PBFT library.
The key issue addressed by BASE is how to deal with different representations of the replica's state, allowing a replica that recovers from a failure to rebuild its state from other replicas. 
BASE was evaluated considering four different OSes and their native \gls{nfs} implementations: Linux, OpenBSD, Solaris, and FreeBSD.
The results, from 16 years ago, show the same trends we observed: performance varies significantly when diversity is considered.
Differently from \system, BASE does not address the selection of replicas or the reconfiguration of a replica group.

In order to support long-running services, Castro and Liskov~\cite{Castro:2002} introduced the notion of proactive recovery for \gls{bft} services. 
The objective is to rejuvenate replicas periodically to remove stealth attackers and support the execution of long-running services. 
All works on \emph{safe} proactive recovery consider the use of trusted local component on each replica to trigger the periodic recoveries~\cite{Castro:2002,Sousa:2010,Roeder:2010,Platania:2014,Distler:2011}.
%, and some of them even support reactive recoveries to deal with service degradations caused by detectable attacks~\cite{sousa:pc}.
A common weakness of most of these works is that they do not support diversity, therefore a compromised replica can be attacked immediately after its recovery by exploiting the same vulnerability.
The two noticeable exceptions are discussed in the following.

Roeder and Schneider~\cite{Roeder:2010} propose an intrusion tolerance technique called \gls{po} for supporting a different type of diversity from \system.
The idea is, instead of changing \glspl{os} or other off-the-shelf component of the replica, \gls{po} changes the application and library code periodically preserving the original semantics using program transformations, i.e., system call obfuscation reordering, memory randomization, and functions return checks (through an IBM \textit{gcc} patch that inserts and checks a random value after the functions return).
Each replica generates its own obfuscated executable from a read-only device containing the ``master code'' based on time-triggered epochs that a controller triggers in a secure way through a trusted component similar to \system' \gls{ltu}.
All obfuscation mechanisms were implemented and evaluated on OpenBSD 4.0, and their results show that PO adds a little extra overhead to the non-\gls{po} execution.

Similarly to \gls{po}, Platania~\etal{}~\cite{Platania:2014} proposed a compiler-based diversity for the Prime \gls{bft} protocol~\cite{Amir:2011} using the MultiCompiler tool~\cite{Homescu:2013}. 
This compiler creates diverse binaries from the same source code through randomization and padding techniques.
The authors also proposed a theoretical rejuvenation model that receives as input: the probability of a replica being correct over a year ($c$); the number of rejuvenations per day across the whole system ($r$); the number of replicas ($n$); and the system's lifetime. 
Although operators can control only $r$ and $n$, the authors suggest (but does not show how) that $c$ can be estimated using \gls{osint} from the internet (CERT alerts, bug reports, and other historical data).
The authors implemented their solution in two settings, including a virtualized environment provided by the Xen Hypervisor.
In this setting, each replica executes in a Linux \gls{vm}, the recovery watchdog runs in a trusted domain of the same machine, and the proactive-recovery controller runs in another physical machine, in an architecture similar to ours.


\paragraph{Summary.} 
Diversity is becoming one of the building blocks of intrusion tolerance, alongside with \gls{bft} and rejuvenations. 
We presented few works that already used somehow diversity in intrusion-tolerant systems. Some of the works employed opportunistic \gls{ots} components to create diversity among the replicas. 
We do not discard the possibility of using other diversity techniques, such as randomization. 
However, in our work we are interested in \gls{ots} diversity because it is possible to estimate the vulnerability of each component. 
These works are still limited to that extent, i.e., there are none or few concerns on how to create diversity to avoid common failures. 
Most of the works implement diversity assuming complete fault independence, but that is unrealistic. 
For example, different \gls{ots} \glspl{os} can share the same weaknesses due to some shared libraries or kernel code. 
There is a need to understand how diversity can be efficiently employed in a replicated system to make the failure independence sound. 
Moreover, further work is still necessary to create automatic mechanisms that abstract diversity management for the administrators.

Both \gls{po} and the work from Platania~\etal{} improve the diversity of applications and replication libraries,\footnote{It should be noted, however, that recent studies have been shown that the benefits of these techniques are limited~\cite{Snow:2013,Bittau:2014}.} but not on \glspl{os} and other \gls{ots} components used by replicas.
Therefore, \system complements these services, as well as any proactive recovery system,\footnote{In particular, the ones that support proactive and reactive recoveries~\cite{Sousa:2010}.} by assessing and monitoring the potential risk of common-mode failures in a replicated service, and selecting appropriate replacements as the threat landscape evolves.

\system exploits the opportunistic \gls{ots} diversity derived from the existence of many version for \glspl{os}, \glspl{db}, Web Servers, \gls{jvm}, etc.
Several studies show that different versions of these components significantly increase chances of avoiding common-mode failures~\cite{Gashi:2007,Garcia:2014,Gorbenko:2011}.
Although we focus only on OSes in our \system prototype, the same techniques can be used to evaluate and deploy full diverse software stacks in the replicas.

\section{Moving Target Defenses}
A few works on \emph{Moving Target Defense} have goals similar to ours.
However, most of these works do not focus on replicated system nor use real data to generate different configurations.
Hong and Kim~\cite{Hong:2015} is an exception, as it proposes a formal security model for Shuffle, Diversity, and Redundancy techniques.
Although this work considers \gls{os} diversity, a few unrealistic assumptions were made, e.g., it assumes that any \gls{os} requires the same amount of time to compromise and that there are no common vulnerabilities among different \glspl{os}.
Moreover, their model and experiments were made with only two \glspl{os} and a handful of vulnerabilities for each of them.


\chapter{Overview and Preliminaries}
\label{chap:overview}

In this chapter, we present an overview of the whole thesis components, we present the system model, and some definitions.



\section{Overview}
In the end of this chapter, it should be clear how the different contributions are connected and how we materialize them in software systems. 


In a nutshell, as displayed in~Figure~\ref{fig:overview}, \system provides a distributed operating system for \gls{bft}-replicated services.
The system manages in its execution plane a set of nodes that can run \emph{unmodified replicas} (encapsulated in \glspl{vm} or containers). 
Each node must have a small \gls{ltu} that allows the activation and deactivation of replicas as demanded by the \system \emph{Controller}, in the control plane.
The controller decides which software should run at any given time by monitoring the existing vulnerabilities in the pool of replicas, aiming to minimize the risk of having several nodes being compromised by the same attack.

\begin{figure}[h]
\begin{center}
\includegraphics[width=0.7\columnwidth]{images/images/overview.pdf}
\vspace{-5mm}
\caption{\system overview.}
\label{fig:overview}
\end{center}
\end{figure}


\section{System Model}
\label{sec:systemmodel}

\system system model shares some similarity with previous works on the proactive recovery of \gls{bft} systems (e.g.,~\cite{Castro:2002,Platania:2014,Sousa:2010,Roeder:2010}).
More specifically, we consider a \emph{hybrid system model} composed of two planes with different properties and assumptions:

\begin{itemize}

\item \textbf{Execution Plane:} 
This plane is componsed of replica processes that can be subject to Byzantine failures.
Therefore, a Byzantine replica can try to mislead the other replicas or the clients.
These replicas communicate through an asynchronous network that can delay, drop or modify messages, just like most \gls{bft} system models~\cite{Castro:1999,Kotla:2010,Bessani:2014,Aublin:2015}.
This plane hosts $n$ replicas from which at most $f$ can be compromised at any given moment.
In this paper, we consider the typical scenario in which $n=3f+1$~\cite{Castro:2002,Kotla:2010,Aublin:2015}.  

\item \textbf{Control Plane:}  
For simplicity, we will address this component as a logical-centralized controller, which requires stronger assumptions. 
However, in Chapter~\ref{} we introduce \controller which requires weaker assumptions like the Execution Plane.

In this plane, we assume that each component can only fail by crashing. 
Each \emph{node} hosting processes contains a \gls{ltu}, and there is a logically-centralized controller to reconfigure the system, just like what has been used in several previous works on proactive recovery (e.g.,~\cite{Roeder:2010,Platania:2014,Sousa:2010}).
The failures of such components do not compromise the liveness and safety of the service as long as the control plane is recovered before $f$ replicas fail.


\end{itemize}

Besides the execution and control planes, we assume the existence of two types of external components: (1) clients of the replicated service, which can be subject to Byzantine failures; (2) \gls{osint} sources (e.g., \gls{nvd}, ExploitDB) that can not be subverted and controlled by the adversary.
In practice, this assumption lead us consider only well-established and authenticated data sources.
Dealing with untrusted sources is an active area of research in the threat intelligence community (e.g.,~\cite{Sabottke:2015,Liu:2015}), which we consider out of scope for this paper.


\section{Execution Plane}
\label{sec:executionplane}

The Execution Plane can accomodate any replicated system that already levareges on the existence of a controller node (e.g.,~\cite{Sousa:2010,Roeder:2010,Platania:2014,Garcia:2016}) or any \gls{bft} system that would benefit from the \system assistence (e.g.,~\cite{Sousa:2018}). 
However, a few requriements must be fulfilled, the Execution Plane is supported by virtulization therefore the \gls{bft} must run in a \gls{vm}.
Additionally, the \gls{bft} library must provide replicas configuration in order to allow replicas' replacement.
In this thesis we evaluate \system with three instances of differente Execution Planes: \gls{bft} ordering for Hyperledger Fabric, a \gls{kvs}, and \sieveq.
The latter, is one of the contributions of the thesis (see Chapter~\ref{chap:sieveq} and Chapter~\ref{chap:sieveqevaluation}, therefore detail its description.


\sieveq provides a message queue abstraction for critical services, applying various filtering rules to determine if messages are allowed to go through.
\sieveq is not a conventional firewall and we do not claim that it should replace existing firewalls in all deployment scenarios.
We are focusing on service- or information-critical systems that require a high-level of protection, and therefore, justify the implementation of advanced replication mechanisms.
The system we propose is able to deliver messages while guaranteeing authenticity, integrity, and availability.
As a consequence, and in contrast to conventional firewalls, we lose transparency on senders and receivers, since they are aware of the \sieveq's end-points.
The rest of the section explains how we address some of the mentioned issues and introduces the main design choices and the architecture of \sieveq.




Typical resilient firewall designs are based on primary-backup replication, and consequently, they are able to tolerate only crash failures.
Therefore, more elaborated failure modes may allow an adversary to penetrate into the protected network.


Some organizations deal with crashes (or \gls{dos} attacks) by resorting to several firewalls to support multiple entry points. 
This solution is helpful to address some (accidental) failures, but is incapable of dealing with an intrusion in a firewall.
In this case, the adversary gains access to the internal network, enabling an escalation of the attack, which at that stage can only be stopped if other protection mechanisms are in place.


Different fault tolerance mechanisms are employed at the two stages. 
Pre-filtering is implemented by a dynamic group of nodes named \presieves. 
\Presieves can be the target of various kinds of attacks and eventually may be intruded because they face the external network. Therefore, we take the conservative approach of assuming that \presieves can fail in an arbitrary (or Byzantine) way, meaning that they may crash or start to act maliciously.
When a failed \presieve is detected, it is simply replaced by a new one that is clean from errors.
Since \sieveq needs to support different message loads, e.g., due to additional senders, \presieves can be created dynamically to amplify the aggregated processing capabilities (within the constraints of the hardware).
The filtering stage is performed by a group of \repsieve components, which execute as a replicated state machine~\cite{Schneider:1990}.
\Repsieves may also fail in an arbitrary way, and therefore, we employ an intrusion-tolerant replication protocol that ensures correct operation in the presence of Byzantine faults.

Our solution was guided by the following design principles:

\begin{itemize}

\item \emph{Application-level filtering}: support sophisticated firewall filtering rules that take advantage of application knowledge. \sieveq implements this sort of rules by maintaining state about the existing flows, and this state has to be consistently replicated using a \gls{bft} protocol.

\item \emph{Performance}: address the most probable attack scenarios with highly efficient approaches, and as early as possible in the filtering stages; Reduce communication costs with external senders, as these messages may have to travel over high latency links (e.g., do not require message multicasts).	

\item \emph{Resilience}: tolerate a broad range of failure scenarios, including malicious external/internal attackers, compromised authenticated senders, and intrusions in a subset of the \sieveq components; Prevent malicious external traffic from reaching the internal network by requiring explicit message authentication.

\end{itemize}



\section{Control Plane}
\label{sec:controlplane}

\system is the first control plane that automatically changes the attack surface of a \gls{bft} system in a dependable way.
\system continuously collects security data from \gls{osint} feeds on the internet to build a knowledge base about the possible vulnerabilities, exploits, and patches related to the systems of interest.
This data is used to create clusters of similar vulnerabilities, which potentially can be affected by (variations of) the same exploit.
These clusters and other collected attributes are used to analyze the risk of the \gls{bft} system becoming compromised. % due to common vulnerabilities.
Once the risk increases, \system replaces the potentially vulnerable replica by another one, trying to maximize the failure independence. % of the replicated service.
Then, the replaced node is put on quarantine and updated with the available patches, to be re-used later.
These mechanisms were implemented to be fully automated, removing the human from the loop.

The current implementation of \system manages 17 \gls{os} versions, supporting the \gls{bft} replication of a set of representative applications.
The replicas run in \glspl{vm}, allowing provisioning mechanisms to configure them. 
We conducted two sets of experiments, one demonstrates that \system risk management can prevent a group of replicas from sharing vulnerabilities over time; the other, reveals the potential negative impact that virtualization and diversity can have on performance. 
However, we also show that if naive configurations are avoided, \gls{bft} applications in diverse configurations can actually perform close to our homogeneous bare metal setup.

\subsection{BFT-Control Plane}
\note{add details later}

\section{Diversity of Replicas}
\label{sec:diversityofreplicas}
\note{citar survey \cite{Baudry:2015}}
%BFT-replicated services running on \system are composed by $n$ replicas.
For our purposes, each \replica is composed of a stack of software, including an OS (kernel plus other software contained in an \gls{os} distribution), execution support (e.g., \gls{jvm}, \gls{dbms}), a \gls{bft} library, and the service that is provided by the system.
%(see Figure~\ref{fig:arch1}).
The set of $n$ replicas is called a \configuration.

It is possible to improve the \replicas fault independence by resorting to different \gls{ots} components in the software stack~\cite{Deswarte:1998}. 
For example, it has been shown that using distinct \glspl{os}~\cite{Garcia:2014}, filesystems~\cite{Rodrigues:2001,Bairavasundaram:2009}, and databases~\cite{Gashi:2007}, can yield important benefits in terms of fault independence. In addition, automatic techniques could enhance diversity, like randomization/obfuscation of \glspl{os}~\cite{Roeder:2010} and applications~\cite{King:2016}.

Although \system can exploit automatic techniques, in this paper we center our attention on diverse \gls{ots} components. 
In particular, \system monitors the disclosed vulnerabilities of all elements of the software stacks of the replicas to assess which of them may contain common vulnerabilities.  

However, in the experimental evaluation, we focus on the diversity of OSes (not only the kernel, but the whole product) for three fundamental reasons: (1) by far, most of the replica’s code is the \gls{os}; (2) such size and importance, make \glspl{os} a valuable target, with new vulnerabilities and exploits being discovered every day; and (3) there are many options of OSes that can be used.
The two last factors are particularly important to enrich the validity of our analysis.

Moreover, we do not explicitly consider the diversity of the \gls{bft} library (i.e., the protocol implementation) or the service code implemented on top of it.
Four facts justify this decision: (1) N-version programming is too costly for this~\cite{Avizienis:1977}; (2) there have been some works showing that such protocol implementations can be generated from formally verified specifications~\cite{Hawblitzel:2015,Rahli:2018}; (3) the relatively small size of such components (e.g., a key-value store on top of BFT-SMaRt has less than 15k lines of code~\cite{Bessani:2014}) make them relatively simple to test and assess with some confidence~\cite{Martins:2013,Lee:2014};  and (4) there are no reported vulnerabilities about these system to support our study.
Notice that, although we do not explicitly consider the diversity of \gls{bft} libraries, nothing prevents \system from monitoring them (when several alternatives become available). 
Additionally, as a pragmatic approach, we could employ automatic diversity techniques in this layer~\cite{Platania:2014,Roeder:2010}.



\chapter{\sieveq: A BFT Firewall}
\label{chap:sieveq}

%In the previous chapter, we presented a control plane for \gls{bft} systems, named \system. 
%Although we have already introduced \sieveq in the evaluation of \system, we will detail \sieveq contributions in this chapter.
In the previous chapter, we presented a study that demonstrates the potential gains of using diverse (\gls{os}) components in \gls{bft} replicated systems.
Here, we present a \gls{bft} solution that provides fault and intrusion tolerance by employing an architecture based on two filtering layers, enabling efficient removal of invalid messages at early stages in order to decrease the costs associated with \gls{bft} replication in the later stages.
This system is \sieveq, a message queue service that protects and regulates the access to critical systems, in a way similar to an application-level firewall.


\section{Firewalls Limitations}

Firewalls are one of the primary protection mechanisms against external threats, controlling the traffic that flows in and out of a network. 
Typically, they decide if a packet should go through (or be dropped) based on the analysis of its contents. 
Over the years, this analysis has been performed at different levels of the \gls{osi} model (see~\cite{Keromytis:2006} for a comprehensive survey), but the most sophisticated rules are based on the inspection of application data included in the packets.
State-of-the-art solutions for application-level firewalls include network appliances from several vendors, such as Juniper~\cite{juniper}, Palo Alto~\cite{paloalto}, and Dell~\cite{sonicwall}.
Such appliances are strategically placed on the network borders, and therefore, the security of the whole infrastructure relies on them.
A skilled attacker, however, may find weaknesses to compromise the firewall's detection/prevention capabilities.
When this happens, critical services under the protection of such devices may be affected, as in the \gls{scada} systems targeted in the Stuxnet~\cite{stuxnet:2010} or Dragonfly~\cite{dragonfly:2014} attacks.


Most generic firewall solutions suffer from two inherent problems: 
First, they have vulnerabilities as any other system, and as a consequence, they can also be the target of advanced attacks. 
For example, the \gls{nvd}~\cite{nvd} shows that there have been many security issues in commonly used firewalls. 
\gls{nvd}'s reports present the following numbers of security issues between 2010 and 2018: 205 for the Cisco Adaptive Security Appliance; 123 in Juniper Networks solutions; and 50 related to iptables/netfilter. 
Common protection solutions often have been the target of malicious actions as part of a wider scale attack (e.g., anti-virus software~\cite{Chauhan:2011}, \gls{ids}~\cite{Anderson:2001} or firewalls~\cite{Kamara:2003,Surisetty:2010,cisco1,cisco2}).
Second, firewalls are typically a single point of failure, which means that when they crash, the ability of the protected system to communicate may be compromised, at least momentarily.
Therefore, ensuring the correct operation of the firewall under a wide range of failure scenarios becomes imperative.
To tolerate faults, one typically resorts to the replication of the components.

%Primary-backup replication (also called $1+1$ replication) would suffice to tolerate a single crash fault on a firewall.
%If the primary replica crashes, the backup replica is able to replace it and deliver the requests. 
%If the system wants to tolerate arbitrary failures, e.g., replicas that suffer intentional or non-intentional Byzantine faults, then it needs a more complex type of replication. 
%For example, if one of the two replicas is misbehaving then the node implementing the critical service will be unable to distinguish which replica is delivering the correct message. 
%To address this difficulty, it is necessary to collect a majority of correct messages, which requires the addition of more replicas. 
%A system that needs to tolerate a single Byzantine fault must have at least $4$ replicas~\cite{Bracha:1985}.


In this chapter, we present a new protection system called \sieveq that mixes the firewall paradigm with a message queue service, with the goal of improving the state-of-the-art approaches under accidental failures and/or attacks.
The solution has a fault- and intrusion-tolerant architecture that applies filtering operations in two stages acting like a sieve.
The first stage, called \emph{pre-filtering}, performs lightweight checks, making it efficient to detect and discard malicious messages from external adversaries.
In particular, messages are only allowed to go through if they come from a pre-defined set of authenticated senders.
\gls{dos} traffic from external sources is immediately dropped, preventing those messages from overloading the next stage.
The second stage, named \emph{filtering}, enforces more refined application level policies, which can require the inspection of some message fields or need the enforcement of specific ordering rules.


%Different fault tolerance mechanisms are employed at the two stages. 
%Pre-filtering is implemented by a dynamic group of nodes named \presieves. 
%\Presieves can be the target of various kinds of attacks and eventually may be intruded because they face the external network. 
%Therefore, we take the conservative approach of assuming that \presieves can fail in an arbitrary (or Byzantine) way, meaning that they may crash or start to act maliciously.
%When a failed \presieve is detected, it is merely replaced by a new one that is clean from errors.
%Since \sieveq needs to support different message loads, e.g., due to additional senders, \presieves can be created dynamically to amplify the aggregated processing capabilities (within the %constraints of the hardware).
%The filtering stage is performed by a group of \repsieve components, which execute as a replicated state machine~\cite{Schneider:1990}.
%\Repsieves may also fail in an arbitrary way, and therefore, we employ an intrusion-tolerant replication protocol that ensures correct operation in the presence of Byzantine faults.

%\sieveq was experimentally evaluated in different scenarios.
%The results show that it is much more resilient to DoS attacks and various kinds of intrusions than existing replicated-firewall approaches.
%We also evaluated \sieveq considering the protection of a \gls{siem} system.
%The test environment emulated the setup of the 2012 Summer Olympic Games, where the same sort of security events was generated and transmitted across the network. 
%The experiments demonstrate that \sieveq can handle a workload up to sixteen times higher than the observed load in the 2012 Summer Olympic Games, without a noticeable degradation in performance.


\section{Intrusion-Tolerant Firewalls}
\label{building_blocks}

In the last decade, several significant advances occurred in the development of intrusion-tolerant systems.
However, to the best of our knowledge, very few works proposed intrusion-tolerant protection devices, such as firewalls.
Performance reasons might explain this, as \gls{bft} replication protocols are usually associated with significant overheads and limited scalability.
Additionally, achieving complete transparency to the rest of the system can be challenging to reconcile with the objective of having fast message filtering under attack.

\begin{figure}[t]
\begin{center}
\includegraphics[width=.7\columnwidth]{images/images/arch_traditional.pdf}
\caption{Architecture of a state-of-the-art replicated firewall.}
\label{fig:traditional}
\end{center}
\end{figure}

Figure~\ref{fig:traditional} shows an implementation of an intrusion-tolerant firewall, illustrating  existing works in this area~\cite{Sousa:2010,Roeder:2010}.
In this design, a sender transmits the messages through the network (e.g., the internet) towards the receiver.
As packets reach the firewall, they are disseminated to the replicas. 
Each replica applies the same filtering rules to decide whether the messages are acceptable. 
Invalid messages are discarded (and eventually logged). 
Messages deemed valid are conveyed to the receiver together with a proof of validity, which demonstrates that a sufficiently large quorum of replicas agrees on their validity.
The proof of validity is checked by a voter module at the receiver, before delivering the messages to the receiving application.
Thus, if a compromised replica produces a message with malicious content, it will be eliminated as it lacks the necessary proof of validity or it is in conflict with the messages transmitted by the other correct replicas.

Although this architecture has interesting characteristics, such as an increased failure resilience, it suffers from some fundamental limitations:

\begin{enumerate}

\item The dissemination of a message to all replicas can be detrimental to the proper operation of the firewall. 
For example, a traffic replicator device (e.g., hub) can be placed at the entry of the firewall to reproduce all messages~\cite{Sousa:2010,Roeder:2010} transparently. 
An obvious consequence of this approach is that malicious messages from an external attacker are also replicated, and therefore, all replicas have to spend the same effort to process them.
As a consequence, the attack is amplified by the replicator device.
Alternatively, a leader replica could receive the traffic and then disseminate the messages to the others~\cite{Roeder:2010}.
The drawback is that the leader becomes a natural bottleneck, especially when under attack (instead of dispersing the attack load over all replicas~\cite{Amir:2011}).

\item The support for stateful firewall filtering requires that all correct replicas process messages in the same order~\cite{Schneider:1990}.
As a consequence, to ensure an agreement in a common sequence of messages, replicas need to continuously run a \gls{bft} consensus protocol~\cite{Castro:2002} to establish message ordering.
A significant amount of work can be wasted with malicious messages since all messages have to be agreed. This is particularly relevant because a consensus protocol consumes both computational and network resources.

\item The creation and check of the proof of validity can be a complex task. For example, one approach requires a trusted component to be deployed in the replicas to generate a \gls{mac} as a proof that a message is valid~\cite{Sousa:2010}. 
The component only returns the \gls{mac} when a quorum of replicas accepted the message. 
Another solution uses threshold cryptography to ensure that every replica can individually produce a partial signature (that corresponds to a part of the proof)~\cite{Roeder:2010}. 
To recreate the full proof, the voter needs to wait for the arrival of a quorum of partial proofs. 
When building a firewall, it would be useful if a more straightforward approach could be employed, with no need for specialized trusted components or expensive threshold cryptography.

\end{enumerate}

\begin{figure}[t]
\begin{center}
\subfloat{\includegraphics[width=0.5\columnwidth]{images/gnuplot/sieveq/plots/latency_tradiotional_attack.pdf}}
\hspace{-5mm}
\subfloat{\includegraphics[width=0.5\columnwidth]{images/gnuplot/sieveq/plots/server_throughput_traditional_attack.pdf}}
\caption{Effect of a DoS attack (initiated at second 50) on the latency and throughput of the intrusion-tolerant firewall architecture displayed in Figure \ref{fig:traditional}.}
\label{fig:attack_traditional}
\end{center}
\end{figure}

These drawbacks can have a significant impact on the firewall performance depending on the considered setting. 
For example, a typical \gls{dos} attack can substantially decrease the throughput and lead to several orders of magnitude growth in the latency of message delivery. 
To illustrate this behavior we implemented the architecture of Figure~\ref{fig:traditional} and launched a \gls{dos} attack on this system (see Section~\ref{evaluation} for a description of the setup and environment). 
The results show that the system performance is significantly affected by the attack (see Figure~\ref{fig:attack_traditional}).




In \sieveq, we explore a different design for replicated protection devices, where we trade some transparency on senders and receivers for a more efficient and resilient firewall solution.
In particular, we propose an architecture in which critical services and devices can only be accessed through a message queue and implement the application-level filtering in this queue.
It is assumed that these services have a limited number of senders, which can be appropriately configured to ensure that only they are authorized to communicate through \sieveq.

\section{Overview of \sieveq}
\label{architecture}

Typical resilient firewall designs are based on primary-backup replication, and consequently, they are able to tolerate only crash failures.
Therefore, more elaborated failure modes may allow an adversary to penetrate the protected network.


Some organizations deal with crashes (or \gls{dos} attacks) by resorting to several firewalls to support multiple entry points. 
This solution is helpful to address some (accidental) failures but is incapable of dealing with an intrusion in a firewall.
In this case, the adversary gains access to the internal network, enabling an escalation of the attack, which at that stage can only be stopped if other protection mechanisms are in place.

\sieveq provides a message queue abstraction for critical services, applying various filtering rules to determine if messages are allowed to go through.
\sieveq is not a conventional firewall, and we do not claim that it should replace existing firewalls in all deployment scenarios.
We are focusing on service- or information-critical systems that require a high level of protection, and therefore, justify the implementation of advanced replication mechanisms.
The system we propose is able to deliver messages while guaranteeing authenticity, integrity, and availability.
As a consequence, and in contrast to conventional firewalls, we lose transparency on senders and receivers, since they are aware of the \sieveq's end-points.
The rest of the section explains how we address some of the mentioned issues and introduces the main design choices and the architecture of \sieveq.

\subsection{Design Principles}
Our solution was guided by the following  principles:

\begin{itemize}

\item \emph{Application-level filtering}: support sophisticated firewall filtering rules that take advantage of application knowledge. \sieveq implements this sort of rules by maintaining state about the existing flows, and this state has to be consistently replicated using a \gls{bft} protocol.

\item \emph{Performance}: address the most probable attack scenarios with highly efficient approaches, and as early as possible in the filtering stages; Reduce communication costs with external senders, as these messages may have to travel over high latency links (e.g., do not require message multicasts).	

\item \emph{Resilience}: tolerate a broad range of failure scenarios, including malicious external/internal attackers, compromised authenticated senders, and intrusions in a subset of the \sieveq components; Prevent malicious external traffic from reaching the internal network by requiring explicit message authentication.

\end{itemize}

\subsection{\sieveq Architecture}

A fundamental difference between the \sieveq architecture and the other replicated firewall designs (see Figure~\ref{fig:traditional}), is the separation of filtering in several stages.
The rationale for this change is to gain flexibility in the filtering operations while ensuring better performance under attack, retaining the ability to tolerate intrusions.
As observed previously, despite the significant improvements in state-of-the-art \gls{bft} implementations, there is an inherent trade-off between the benefits of \gls{bft} replication and its performance, namely due to the need to disseminate (and eventually authenticate) all messages at the replicas, which includes both valid and invalid messages.

\begin{figure}[t]
\begin{center}
\includegraphics[width=0.7\columnwidth]{images/images/arch.pdf}
\caption{\sieveq layered architecture.}
\label{fig:arch}
\end{center}
\end{figure}


Figure~\ref{fig:arch} presents the architecture of \sieveq. 
In this architecture, message processing starts with a first filtering layer that implements a message authentication mechanism and is responsible for discarding most of the malicious traffic efficiently. 
This layer is based on a set of \presieve modules, each of them in charge of the communications with a subgroup of senders. 
During a typical operation, a sender only interacts with its own \presieve.
The assignment of a \sender to a \presieve is done during the channel setup.
Initially, a sender connects with one of a few statically-configured \presieves.
If a \presieve becomes overloaded, it will request the creation of more \presieves (see details in Section~\ref{faultypresieve}) and/or hand of the new \sender channel to another node.



The messages are sent to the second filtering layer by the \presieve to perform a more detailed inspection, which can take advantage of state information kept from previous messages and application-related rules.
This layer is implemented by a group of \repsieves acting together as a \gls{bft} replicated state machine.
They receive all accepted messages from the \presieves and process them in the same order, which guarantees that every \repsieve reaches the same decision (discard or accept a message).
However, if $f$ replicas are faulty, their output could be different.
Consequently, as long as more than two-thirds of the \repsieves are correct, the right decision is taken by the \postsieve by performing a message voting.

In the following, we describe each module of \sieveq presented in Figure~\ref{fig:arch}:


\paragraph{\Sender.} The sender nodes cooperate with \sieveq to secure the messages by deploying a \sender module locally. 
Its primary role is to secure the messages and assist in the detection of some intruded \sieveq components. 
The module can be implemented inside the sender's \gls{os} (e.g., as a kernel module or a specialized device driver) or as a library to be linked with the applications.
The decision to have this module corresponds to a trade-off in our design, where we are willing to lose some transparency to improve the system's resilience.

\paragraph{\Presieve.}
These are the \sieveq front-end, and although it only performs stateless filtering to improve efficiency, it can deter the most common attacks. 
\Presieve modules discard invalid messages, while the approved ones are forwarded to the \repsieve using a Byzantine \gls{tom} protocol~\cite{Bessani:2014}. 
The \presieves can be deployed, for instance, as \glspl{vm}. 
The effect of this layer is a significant reduction in the communication and computational overhead caused by malicious packets at the \repsieve.

\paragraph{\Repsieve.}
These components implement a replicated filtering service that tolerates Byzantine faults. 
The actual filtering rules can be more or less complex depending on the needs of the critical service. 
The \gls{tom} ensures that \repsieves receive the messages in the same order. 
Consequently, identical rules are applied across the \repsieves, and therefore, the same decision should be reached on the validity of messages. 
Each \repsieve individually transmits approved messages to the final receiver (the others are dropped). 
Overall, this layer allows for sophisticated filtering as the replicas are stateful.

\paragraph{\Postsieve.} This module runs on the receiver side, and it is responsible for the delivery of messages to the application.
A \postsieve carries out a voting operation on the arriving data because a \repsieve might be intruded and corrupt messages.
It delivers a message to the application only after receiving the same approved message from a quorum of \repsieves.
From a deployment perspective, this module can be implemented in the \gls{os} or as a library, as in the sender.

\paragraph{Controller.} This module is a trusted component of \sieveq that runs with high privilege.
It takes input from the \repsieves to decide on the creation or destruction of \presieves if some misbehavior is observed.
Depending on the actual \sieveq implementation, it can be developed in different ways.
This sort of component was used in previous works, and it can be implemented both in a centralized~\cite{Roeder:2010,Platania:2014} or distributed~\cite{Sousa:2010} way.
%In the same way, we have presented two possible solutions for the \system controller in Chapter~\ref{chap:lazarus_implementation}.


\subsection{Resilience Mechanisms}

The \sieveq architecture is built to tolerate both faults with an accidental nature (e.g., crashes) and caused by malicious actions (e.g., a vulnerability is exploited, and a specific module is compromised).
To be conservative, we assume that all failed components are controlled by a single entity, which will make them act together in the worst possible manner to defeat the correctness of the system. 
Therefore, failed components can, for instance, stop sending messages, produce erroneous information, or try to delay the system. 
\sieveq performs several mitigation actions to guarantee a valid operation (as long as the number of faults is within the assumed bounds, see Section~\ref{fault_model}).


The most common attack scenario occurs when an external adversary attempts to attack a system that is being protected by \sieveq. 
He can deploy many nodes, whose aim is to delay the communications or bypass \sieveq protection and reach the internal network. 
\sieveq addresses these attacks by discarding unauthenticated or corrupted messages with minimal effort at the \presieve filtering stage.
As with any other firewall, if a \gls{dos} attack completely overloads the incoming channels, \sieveq cannot handle or react to the attack.
The network needs to include other defense mechanisms to deal with this sort of problem~\cite{Mishra:2011}.


As (authenticated) senders might be spread over many (outside) networks, it is advisable to consider a second scenario where an adversary is capable of taking control of some of these nodes. 
In this case, we assume that the adversary gains access to all data stored locally, including the \sender keys. 
Thus, he will be able to generate traffic that is correctly authenticated, allowing these messages to go through the first filtering step.
The messages are however still checked against the application-related rules (namely, the ones defined in the \repsieve), which can cause most malicious traffic to be dropped (e.g., a pre-defined \sender can only send messages to a particular \postsieve accordingly to a specific application protocol).
If the messages follow all the rules, the firewall has to forward them because they are indistinguishable from any other valid messages.


A third scenario occurs when the adversary can cause an intrusion in \sieveq and compromises a few of the \presieves and/or \repsieves. 
When this happens, these components can act in an erroneous (Byzantine) way. 
However, unlike with an intruded \sender, malicious \presieves cannot generate fully authenticated messages, since they lack all the required keys. 
They can still perform \gls{dos} attacks on the \repsieves, e.g., by transmitting many messages, but this strategy creates apparent misbehavior allowing immediate discovery. 
Malicious \presieves are detected with the assistance of correct \sender and \repsieves, eventually leading to their substitution.


\Repsieves modules are much harder to exploit because they do not face the external network. 
However, if they end up being intruded, \repsieves can produce arbitrary traffic to the internal network. 
\Postsieve addresses this issue by carrying out a voting step, which excludes these messages. Moreover, an alarm is generated and sent to the \emph{controller}.


The \sieveq architecture does not attempt to recover from intrusions in the controller and \postsieve.
The first is assumed to be trusted, as it is deployed in a separated administrative domain and is to be used only in a few particular operations.
Moreover, its simplicity allows the audit of its code and ensures correctness with a high level of confidence.
The \postsieve already runs in the internal network, and therefore, \sieveq can not preclude its misbehavior.


%%%%%%%%%%%%%%%%%%%%%%




\section{\sieveq Protocol}
\label{protocol}

This section details the \sieveq protocol and service properties. 
We conclude the section with an analysis of the behavior of the system under different kinds of attacks and component failures and highlight how the countermeasures integrated into our design mitigate such threats.

\subsection{System and Threat Model}
\label{fault_model}

The system is composed of a (potentially) large number of external nodes, called \emph{senders}, some internal nodes, called \emph{receivers}, and \sieveq nodes.
Senders run \sender modules to be able to transmit packets through the \sieveq, while receivers receive validated messages by using \postsieve modules. 

Communications can experience accidental faults or attacks.
Thus, packets might be lost, delayed, reordered or corrupted, but we assume that if messages are retransmitted, eventually they will be correctly received by \sieveq.
The fault model also assumes that \sender, \presieve, and \repsieve nodes can suffer from arbitrary (Byzantine) faults.
When this happens, failed nodes may perform actions that deviate from their specification, including colluding against the system.
However, at most $f_{ps}$ \presieves from a total of $N_{ps} = f_{ps} + k$ (with $k > 1$), and $f_{rs}$ \repsieve from a total of $N_{rs} = 3f_{rs}+1$ may fail.
Redundant components should fail independently by employing diversity techniques such as the ones described in Chapter~\ref{chap:lazarus_design}.
A component that is unable to communicate is also considered faulty because from a practical perspective it is indistinguishable from a crashed module.

The cryptographic operations used in the \sieveq protocol are assumed to be secure, and therefore, they cannot be subverted by an adversary. 
Consequently, traditional properties of digital signatures, \glspl{mac}, and hash functions will hold as long as the associated keys are kept safe.
The deployment of \sieveq requires a key distribution scheme to create shared keys between the \sender and the \presieve and to periodically re-issue private-public key pairs for the \sender.
We assume that the key distribution scheme is similar to solutions that already address this sort of problem (e.g.,~\cite{Harkins:1998}).
If required, the key distribution infrastructure could also be made intrusion-tolerant (e.g.,~\cite{Kreutz:2014,Zhou:2002}).


\subsection{Properties}
\label{properties}

\sieveq protocol guarantees the following three properties for messages transmitted from a sender to a receiver:

\begin{security}
If a message, transmitted by a correct sender, is delivered to a correct receiver then the message is in accordance with the security policy of \sieveq.
\end{security}

\begin{validity}
If a correct receiver delivers a message \msg.\texttt{DATA}, then the message was transmitted by \msg.\texttt{sender}.
\end{validity}

\begin{liveness}
If a correct sender sends a message, then the message eventually will be delivered to the correct receiver.
\end{liveness}

These properties require \sieveq to behave in a way similar to most firewalls while offering a few extra guarantees. 
Only external messages that are approved by the policies defined in \sieveq can reach the receivers, and the rest should be dropped (Compliance). 
\Postsieve can use the message field \msg.\texttt{sender} to find who transmitted the message contents (\msg.\texttt{DATA}), and accordingly decide if the message should be delivered to the receiver application (Validity). 
Progress is also ensured, as correct senders eventually can transmit their messages (Liveness).

Besides these functional properties (related to message filtering), \sieveq also ensures a \emph{resilience} property related to the detection and recovery of components of the system that exhibit faulty behavior:


\begin{resilience}
Every component exhibiting observable faulty behavior will be eventually removed or recovered.
\end{resilience}

In the following, we present the mechanisms for implementing the \sieveq functionalities, i.e., the mechanisms for satisfying the three functional properties stated above.
The mechanisms for ensuring resilience will be detailed after that.

\subsection{Message Transmission}
\label{transmittingmessage}


\paragraph{\Sender processing.}

This module gets a buffer with the \texttt{DATA} to be transmitted to a particular application placed behind \sieveq.
The buffer needs to be encapsulated in a message with some extra information required for protection (see Equation \ref{msg}): we add a \sender identifier $\senderi_i$ and a sequence number \sn that is incremented on each message.
This information is needed to prevent replay attacks, either from the network or from a compromised \presieve.
Some information is also added to protect the integrity and authenticate the message.
A signature $\signature_{Si}$ is performed over the message contents, and a \gls{mac} $\mac_{ski}$ is computed using a shared key established with the \presieve.
This \gls{mac} serves as an optimization to speed up checks~\cite{Clement:2009}.

\begin{equation}
\msg = \langle \senderi_i, \sn, \texttt{DATA}, \signature_{Si} \rangle_{\mac_{ski}}
\label{msg}
\end{equation}

\begin{figure}[t]
\centering
\includegraphics[scale=0.5]{images/images/filtering_steps_1.pdf}
\caption{Filtering stages at the \sieveq.}
\label{fig:filters}
\end{figure}


After constructing the message, it is sent to the \presieve assigned to the \sender, and a timer is started.
If this timer expires before the \sender receives an acknowledgment message, it re-sends the \msg to another \presieve.

\paragraph{\Presieve filtering.}

The \presieve determines if an arriving message \msg should be forwarded or discarded (stages (a)--(d) in Figure~\ref{fig:filters}). It applies the following checks to make this decision:

\begin{enumerate}

\item[(a)] \textbf{White list:} each \presieve maintains a list of the nodes that are allowed to transmit messages (i.e., which were authorized by the system administrator). 
Messages coming from other nodes are dropped.
This check is based on the address of the message sender, and therefore it serves as an efficient first test, but it is vulnerable to spoofing.

\item[(b)]  \textbf{Sequence number:} finds out if a message with sequence number \sn from \sender $\senderi_i$ was seen before.
In the affirmative case, the message is discarded to prevent replay attacks.
Messages are also dropped if their \sn is much higher than the largest sequence number ever observed from that \sender (a sliding window of acceptable sequence numbers is used).

\item[(c)]  \textbf{MAC test:} \gls{mac} $\mac_{ski}$ is verified to authenticate the message contents, including to find out if the expected $\senderi_i$  is the sender. 
If the check is invalid, the message is dropped, and the sequence number information is updated to forget that this message was ever received (the update carried out in the previous step needs to be undone).

\item[(d)] \textbf{Grant check:} each \presieve controls the amount of traffic that a \sender is transmitting.
Messages that fall outside the allocated amount are dropped to ensure that all senders get a fair share of the available bandwidth (and to avoid \gls{dos} attacks by compromised senders).
The amount of traffic that is allowed to each \sender is adjusted dynamically based on the available and consumed resources.


\end{enumerate}


Finally, the \presieve invokes the \gls{tom} primitive to forward the message to every correct \repsieve.

\paragraph{\Repsieve filtering.}
When a correct \repsieve delivers a message to be filtered, the following checks are applied:

\begin{enumerate}
\item[(e)] \textbf{Grant check, sequence number, and signature:} since a \presieve may have been intruded and may collude with malicious \sender, extra checks are required on the amount of forwarded traffic and the integrity of the message.
The grant check and the test on the sequence number are similar to the ones performed by the \presieve, and the signature ensures that all \repsieves reach the same decision regarding the validity of the message content.
\item[(f)] \textbf{Application-level rules:} apply the application-defined filtering rules to determine if the message is compliant with the security policy of the firewall.
\end{enumerate}

Although uncommon, \repsieves may receive messages in a different order from what is defined in their sequence numbers.
As a consequence, the \repsieve's application-level rules may drop some of the out-of-order messages, which later on will have to be re-transmitted by the \sender.
For example, if messages \texttt{A} and \texttt{B} should appear in this sequence but are re-ordered, then the rules may consider \texttt{B} invalid and then accept \texttt{A}.
At some point, the \sender would consider \texttt{B} as lost, and re-transmit it.\footnote{It is important to remark that, independently of any violation on the order of delivery of sender messages, all correct \repsieves receive the messages in the same order.}

To address this issue, each \repsieve enqueues messages with a sequence number greater than the expected for a while, as long as they do not exceed a threshold above the last processed sequence number. These messages are processed when either: 1) the missing messages with smaller sequence numbers arrive, and then they are all tested in order, or 2) the \repsieve gives up on waiting and checks the enqueued messages. This last decision is made after processing a pre-determined number of other messages.

Valid messages are encapsulated in a new format (see Equation \ref{msgl}) and are sent to their receivers.
Basically, \repsieve $\replicak_j$ substitutes the signature with a new \gls{mac}, $\mac_{skj}$.
This \gls{mac} is created with a shared key between the \repsieve and the \postsieve.


\begin{equation}
\msg' = \langle \senderi_i, \sn, \texttt{DATA}, \replicak_j \rangle_{\mac_{skj}}
\label{msgl}
\end{equation}

\paragraph{\Postsieve processing.}

\Postsieve accumulates the messages that arrive from \repsieves until enough evidence is collected to allow their delivery.

\begin{enumerate}
\item[(g)] \textit{vote:} A message can be delivered to the application when it is received from $f_{rs} + 1$ \repsieves.

\end{enumerate}

Algorithm \ref{algo:post_filter} presents the \postsieve voting protocol.
The \postsieve starts by checking if the message contains a valid \gls{mac} (lines 5 and 6).
Then, it finds out if the message carries an acceptable sequence number before storing it in a \texttt{WaitingQuorum} set (lines 7 and 8).
Notice that a different set is used for each sender $S_i$ and $sn$ pair.
Messages with sequence numbers already delivered or higher than a threshold (\texttt{snThreshold}) are discarded.

The expected sequence number is stored in the auxiliary variable $k$ (line 9).
Next, the \postsieve tries to find a message with this sequence number and with at least $f_{rs} + 1$ votes (by searching the corresponding set and using function \texttt{equalMsg} --- line 10).
For each different \texttt{DATA} value that may exist in the set, the \texttt{equalMsg} function counts the number of times its appears, and returns the largest count.\footnote{Notice that all correct \repsieves transmit messages with the same \texttt{DATA} value and that malicious replicas can send at most $f_{rs}$ arbitrary \texttt{DATA} values.
Therefore, eventually there will be a \texttt{DATA} value with at least $f_{rs}+1$ votes because there are at least $2f_{rs}+1$ correct \repsieves.}
Next, while there are \msg's with a $f_{rs} + 1$ quorum, \postsieve delivers \msg's in order (line 10-13).
The function \texttt{mostVotedMsg} returns the message with the \texttt{DATA} value with most votes (line 11).
Then, the \texttt{deliver} function delivers \texttt{DATA} to the application on the receiver and deletes the \texttt{WaitingQuorum} set.
Finally, the \texttt{snExpect} is incremented (line 12) and the $k$ index is updated (line 13).
This allows \postsieve to deliver messages in order while there are messages with the expected sequence number in its buffer.

\SetKwInOut{Input}{input}
\SetKwData{repsievealgo}{\repsieve}

\SetKwData{MAC}{$\mac_{si}$}
\SetKwData{Signature}{\signature}
\SetKwData{SeqNumber}{\texttt{\sn}}
\SetKwData{SeqNumberExp}{\texttt{snExpect}}
\SetKwData{SeqNumberThreshold}{\texttt{snThreshold}}
\SetKwData{MESSAGE}{\msg}
\SetKwData{MESSAGEX}{\texttt{aux}}
\SetKwData{DATA}{\texttt{DATA}}

\SetKwData{client}{\senderi$_{i}$}
\SetKwData{prefirewall}{\presieve\emph{$_{i}$}}
\SetKwData{prefirewalltwo}{\presieve\emph{$_{k}$}}
\SetKwData{RepFirewallID}{\repsieve\emph{$_{i}$}}
\SetKwData{WaitingQuorum}{\texttt{WaitingQuorum}}

\SetKwFunction{seqnumber}{expected\_seq\_numb}
\SetKwFunction{macverification}{verifyMAC}
\SetKwFunction{sendVoter}{send\_\postsieve}
\SetKwFunction{applicationRules}{application\_rules}
\SetKwFunction{discard}{discard}
\SetKwFunction{deliver}{deliver\_to\_Application}
\SetKwFunction{existsQuorom}{existsQuorom}
\SetKwFunction{deliver}{deliver}
\SetKwFunction{return}{return}
\SetKwFunction{equalMsg}{equalMsg}
\SetKwFunction{mostVotedMsg}{mostVotedMsg}

{\centering
\begin{minipage}{.8\linewidth}


\begin{algorithm}[H]
%\SetInd{0.3em}{0.2em}
\caption{\postsieve protocol}\label{algo:post_filter}
{
\small

  $Init:\ executed\ only\ once$\\
	$\SeqNumberExp_{Si}, k \leftarrow 0$\;
	$\WaitingQuorum_{Si,\sn} \leftarrow \perp$\;
  \BlankLine
\Fn{Filter (\MESSAGE')}{
    \If{( \macverification{\MESSAGE'} = FALSE )}{
        \return $errorMAC$\;
    }
	\If{ ($\SeqNumberExp_{Si}$ $\leq$ \MESSAGE'.\SeqNumber $<$ ($\SeqNumberExp_{Si}$ + \SeqNumberThreshold)) }{
		$\WaitingQuorum_{Si,\sn} \leftarrow \WaitingQuorum_{Si,\sn} \cup \{ \MESSAGE' \}$\;
	}
	$k \leftarrow \SeqNumberExp_{Si}$\;



	\While{ ( \equalMsg{$\WaitingQuorum_{Si,k}$} $\geq$ $f_{rs}+1$ ) }{

		  	\deliver{\mostVotedMsg{$\WaitingQuorum_{Si,k}$}} \;
		  	$\SeqNumberExp_{Si} \leftarrow \SeqNumberExp_{Si} + 1$ \;
			$k \leftarrow \SeqNumberExp_{Si}$\;

    }
}
}
\end{algorithm}
\end{minipage}
\par
}





\subsubsection{Correctness Argument}
In the following, we show that \sieveq design satisfies the three functional properties described in Section~\ref{properties}.
The proofs work under the assumptions of our system and threat model (defined in Section~\ref{fault_model}).

\begin{validity}

If a correct receiver delivers a message \msg.\texttt{DATA}, then the message was transmitted by \msg.\texttt{sender}.
\end{validity}

\begin{proof}
Assume that a receiver $\receiverj_j$ gets a message content \msg.\texttt{DATA} with the \msg.\texttt{sender} $=$ \msg.\texttt{$\senderi_i$}.
For this to happen, the \postsieve of $\receiverj_j$ waited for the arrival of at least $f_{rs}+1$ correctly authenticated messages\footnote{Note that \repsieves send $\msg'$ (defined in (\ref{msgl}))  instead of \msg (defined in (\ref{msg})). Both are similar, but \msg' has a \gls{mac} instead of a signature to authenticate its contents with \postsieve. For the sake of simplicity we will only use the $\msg$ notation in the rest of the proof.} with equal \msg.\texttt{$\senderi_i$}, \msg.$\sn$, and \msg.\texttt{DATA}.
Therefore, at least $f_{rs}+1$ \repsieves received $\msg$ signed by $\senderi_i$, and verified the signature $\signature_{Si}$ as valid.
This indicates that \msg.\texttt{DATA} was not modified by the network or by any \presieve.
Only a sender \msg.\texttt{$\senderi_i$} (correct or not) can create and authenticate a $\msg$ with its signature.
Therefore, $\senderi_i$ is the \msg.\texttt{sender}, i.e., the creator of \msg.\texttt{DATA}.
\end{proof}


\begin{security}
If a message, transmitted by a correct sender, is delivered to a correct receiver then the message is in accordance with the security policy of \sieveq.
\end{security}


\begin{proof}
The proof of the \emph{Compliance} property is very similar to the \emph{Validity} proof.
The difference is that, we also know that if $f_{rs}+1$ \repsieves sent $\msg$ to a \postsieve, then $\msg$ was verified against the security policy in at least one correct \repsieve, which approved it.
\end{proof}



\begin{liveness}
If a correct sender sends a message, then the message eventually will be delivered to the correct receiver.
\end{liveness}

\begin{proof}
For the sake of simplicity, our proof only considers nodes that drop or delay messages.
Attacks to the integrity will make the message be discarded.

Assume that a sender $\senderi_i$ transmits a message $\msg$ with the sequence number $\sn$ to a \presieve $\presievei_u$.
Then $\senderi_i$ sets a timer $\mathit{timer_{sn}}$ for \msg.\sn.
If $\presievei_u$ is correct, after it receives a message, it re-sends $\msg$ via \gls{tom} to the \repsieves.
At least $f_{rs}+1$ correct \repsieves will send $\msg$ to the \postsieve, which will deliver  \msg.\texttt{DATA} to the receiver $\receiverj_j$.
If the $\presievei_u$ is faulty, i.e., drops or delays messages, $timer_{sn}$ in $\senderi_i$ will eventually expire.
When this happens, $\senderi_i$ re-transmits $\msg$ to another \presieve.
Eventually, this process will make some correct \presieve forward $\msg$ to the \repsieves using the \gls{tom} primitive, which will cause \msg.\texttt{DATA} to be delivered to $\receiverj_j$.
\end{proof}


\subsection{Addressing Component Failures}
\label{sec:failures}

In this section, we discuss the implications of failures in the different components of the \sieveq, and how they are handled for ensuring the \emph{Resilience} property stated in Section~\ref{properties}.

In the Byzantine model, every failed component can behave arbitrarily, intentionally or accidentally. Therefore, the \sieveq design incorporates mechanisms that are resilient to different failure scenarios. 
Given the architecture of Figure~\ref{fig:arch}, one has to address faults in authenticated \senders, \presieves and \repsieves, as they are the main components subject to Byzantine failures in our model. 
The \postsieve is not considered in this section as it is co-located with the receiver.
Therefore, we can not make further assumptions about it. 
We assume the \emph{controller} as trusted in this chapter.
In Chapter~\ref{chap:lazarus_implementation} we present an intrusion-tolerant solution for a different controller component that can be adapted for this \emph{controller}.


In the following, we try to focus on complex scenarios whereas faulty components send syntactically-valid messages. 
Therefore avoiding cases that could be easily detected and recovered by existing network monitoring and protection tools (e.g., it is easy to discover that a \presieve is sending messages to another \presieve, something our protocol does not allow).

Since \presieves are directly exposed to the external network, there is a higher risk of them being compromised.
Then, to keep the \sieveq operational, it is required that failed \presieves be identified and recovered.
We leverage from the \repsieve setup to perform failure detection, and then use the \emph{controller} to restart erroneous \presieves.
Replacing these components is almost trivial because they are stateless.

\Repsieves execute as a \gls{bft} replicated state machine, processing messages in the same order and producing identical results.
Consequently, \repsieves faults can be tolerated by employing a voting technique on the \postsieve that selects results supported by a sufficiently large quorum (as explained above, an output with at least $f_{rs} + 1$ votes).
Below, we discuss in more detail a few failure scenarios.



\subsubsection{Faulty \Sender}
A faulty \sender is authorized to communicate while suffering from some arbitrary problem (e.g., intrusion). 
Therefore, it can produce correctly authenticated messages to attack the firewall. 
In some scenarios, it is possible to discard these messages. 
For example, if the \sieveq receives a correctly signed message with a sequence number is higher than what was expected, then it can be easily detected and eliminated (checks \emph{(b)} and \emph{(e)} in Figure~\ref{fig:filters}).

A more demanding scenario occurs when a \sender transmits faster than the allowed rate (verification (d) of Figure~\ref{fig:filters}).
In this case, some defense action has to be carried out, as these attacks can lead \sieveq to waste resources.
To be conservative, we decided to follow a simple procedure to protect the firewall: \sieveq maintains a counter per sender that is incremented whenever new evidence of failure can be attributed to it, e.g., the faulty \sender is overloading the system with invalid messages or if it is sending messages to \presieves which it was not assigned.
When the counter reaches a pre-defined value, the \sender is disallowed from communicating with \sieveq by temporarily removing it from the whitelist and adding it to a quarantine list (failing verification (a) of Figure~\ref{fig:filters}) and by giving a warning to the system administrator.
\emph{This ensures that faulty \emph{\senders} are eventually removed from the system.}

Excluded \senders may regain access to the service later on because the counter is periodically decreased (when the counter falls below a certain threshold, the \sender is moved back into the whitelist). 
Additionally, the administrator is free to update the white/quarantine lists. 
For instance, he may choose to manually add a faulty sender to the quarantine list or even deploy policies to do that under certain conditions (e.g., if the same sender is in the quarantine list for a certain number of times).


\subsubsection{Faulty \Presieve}
\label{faultypresieve}
Addressing failures in \presieves is difficult because these components may look as compromised even when they are correct.
Notably, when a \presieve is under a \gls{dos} attack, messages can start to be dropped due to buffer exhaustion, and this is indistinguishable from malicious behavior in which messages are selectively discarded.
In the same way, a failed signature check at a \repsieve indicates that either the \presieve is faulty (it is tampering/generating invalid messages) or that a \sender is misbehaving (recall that a \presieve verifies the message \gls{mac}, but not its signature).
Finally, a \repsieve may also detect problems if it observes a sudden increase in the arrival of messages (above the grant check), which could indicate a \gls{dos} attack by a malicious \presieve (maybe colluding with a compromised \sender).
This kind of ambiguity precludes accurate failure detection, and consequently, we aim to provide a mechanism that allows \sieveq to recover from end-to-end problems and continue to deliver a correct service.

An initial step to deal with these complex failure scenarios is to make \presieves evaluate their own state.
This is done by analyzing the amount of arriving traffic and by observing if it could overload the \presieve.
The analysis can be done by measuring the inter-arrival times of messages over a specified period.
If those intervals are small (on average), there is a chance that the \presieve is working at its full capacity or is even overloaded.
When this happens, the \presieve broadcasts (using the \gls{tom} primitive) a \texttt{WARNREQ} message to the \repsieves, so that they may take some action to solve the problem (see below).


When the \presieve is faulty, the \sender and \repsieves need to detect it together.
\sieveq provides a procedure to find how many messages are being discarded on the \presieve:


\begin{enumerate}

\item Periodically, the \sender sends a special \texttt{ACKREQ} request to \repsieves, in which it indicates the sequence number of the last message that was sent (plus a signature and a \gls{mac}).
This request is first sent to the preferred \presieve, but if no answer is received within some time window, and then it is forwarded to another \presieve.
The waiting period is adjusted in each retransmission by doubling its value.

\item When the \repsieves receive the request, the included sequence number together with local information is used to find how many messages are missing. 
The local information is the set of sequence numbers of the messages that were correctly delivered since the last \texttt{ACKREQ}.

\item Based on the number of missing messages, the \repsieves transmit through the same \presieve a response \texttt{ACKRES} to the sender, where they state the observed failure rate and other control information (plus a signature).
\Repsieves may also perform some recovery action if the failure rate is too high.

\end{enumerate}

Additionally, if a \repsieve detects that a signature is invalid or that it is receiving more messages than the expected (check (e) in Figure \ref{fig:filters}), it suspects the \presieve that sent the messages and takes some action.


Once the problem is detected, the \repsieves should attempt to fix the erroneous behavior by employing one of three possible remediation actions, depending on the extent of the perceived failures:

\begin{itemize}


\item \emph{Redistribute load:} if a \presieve has sent a warning about its load, or a high failure rate related with this component is observed, the first course of action is to move some of the messages flows from the problematic component to other \presieve.
This is achieved by specifying, in the \texttt{ACKRES} response to a \sender, the identifier of a new \presieve that should be contacted.
At that point, the \sender is expected to connect to the indicated \presieve and begin sending its traffic through it.

\item \emph{Increase the \emph{\presieves} capacity:} if the existing \presieves are unable to process the current load, then the \sieveq needs to create more \presieves (depending on the available resources).
To do that, \repsieves contact the \emph{controller} informing that an extra \presieve should be started.
When the controller receives $f_{rs} + 1$ messages, it performs the necessary steps to launch the new \presieve (which are dependent on the deployment environment).
The new \presieve begins with a few startup operations, which include the creation of a communication endpoint, and then it uses the \gls{tom} channel to inform the \repsieve that it is ready to accept messages from \senders.

\item \emph{Kill the \emph{\presieve}:} when there is a significant level of suspicion on a \presieve, the safest course of action is for \repsieves to ask the controller to destroy it.
Moreover, if the load on the firewall is perceived as having decreased substantially, the \repsieves select the oldest \presieve for elimination, allowing future aging problems to be addressed.
The controller carries out the needed actions when it gets $f_{rs} + 1$ of such requests (once again, which depend on how \sieveq is deployed).
The affected \senders will be informed about the \presieve replacement through the \texttt{ACKREQ} mechanism, i.e., they will eventually use another \presieve to send a request, and get the information about their newly assigned \presieve in the response.
Moreover, the new \presieve is informed about the expected sequence number for each \sender.
This information is stored by \repsieves, which contrary to \presieves are stateful.

\end{itemize}

Together, these mechanisms ensure that \emph{a faulty \emph{\presieve} affecting the \sieveq performance will be eventually removed from the system}.


\subsubsection{Faulty \Repsieve}
\label{faultyrepsieve}

A \postsieve only delivers a message if it receives $f_{rs}+1$ matching approvals for this message.
Given the number of \repsieves, this quorum is achievable for a correct message even if up to $f_{rs}$ \repsieves are faulty.
However, a faulty \repsieve can create a large number of messages addressed to other components of the system, effectively causing a \gls{dos} attack.
Therefore, we need countermeasures to disallow a faulty \repsieve to degrade the performance of the system in a similar way as illustrated in Figure~\ref{fig:attack_traditional}.
For example, a faulty \repsieve can send an unexpected amount of messages to a \presieve, making it slower and triggering suspicions that may lead it to be killed.
Similarly, it can attack other \repsieves to make the system slower.
Finally, a faulty \repsieve can also overload the \postsieve with invalid messages.

Each of these attacks requires a different detection and recovery strategy:

\begin{itemize}

\item When a \presieve is being attacked it complains to the controller, which first requests the \presieve replacement (assuming it might be compromised) and increases a suspect counter against the \repsieve.
If $f_{ps}+1$ \presieves also complain about the same \repsieve, the controller starts an \emph{individual recovery} in this \repsieve (see below).

\item If more than $f_{rs}$ other \repsieves are being attacked, they will probably become slower.
In this case, it is possible to detect such attacks if $f_{rs}+1$ \repsieves complain about a single \repsieve.
This would be possible because target \repsieves would observe unjustifiable high traffic coming from a single replica and because such spontaneously generated messages would be deemed invalid.
When this attack is detected, the controller starts an individual \repsieve recovery.

\item If only up to $f_{rs}$ other \repsieves are being attacked it is still possible to make the system slower without being detected by the previous mechanism.
This happens because $N_{rs}-f_{rs}$ replicas must participate in the \gls{tom} protocol~\cite{Bessani:2014}, and the target $f_{rs}$ plus the attacker intersect this quorum in a least one component. 
This intersecting component will define the pace of the messages coming, which typically will be slow.
We can mitigate this attack as each \repsieve periodically informs the \postsieve about its throughput (number of processed messages per second), piggybacking this value in some approved messages submitted for voting.
The \postsieve verifies that there are \repsieves presenting throughputs lower than expected and initiates a \emph{recovery round} on all the \repsieves (as described below).

\item When the \postsieve is being attacked it detects the abnormal behavior.
Then, it requests the individual recovery of the compromised replica.

\end{itemize}

The previous mitigation mechanisms suggest two kinds of recovery actions.
First, an individual recovery, where the machine is rebooted with clean code (we addressed this in Chapter~\ref{chap:lazarus_design}) and then is reintegrated in the system.
Second, a recovery round is used when a performance degradation attack is detected, but there is no certainty about which \repsieve was compromised.
%Therefore, in this case, we recover all replicas, one after another to avoid unavailability periods in the system, just like in proactive recovery systems~\cite{Castro:2002,Sousa:2010,Roeder:2010}.
%There are several works that present these techniques, and most of them are compatible with our architecture, therefore we refrain from describing them in more detail.
%Notice that the use of a recovery round is enough to ensure that \emph{a faulty \emph{\repsieve} will eventually be recovered in \sieveq}.


\section{Implementation}
\label{implementation}


We implemented a prototype of \sieveq following the specification of the previous section to validate our design.
The \sender and \postsieve were developed as libraries that are linked with the sender and receiver applications. 
The libraries offer an interface similar to the TCP sockets to simplify the integration and minimize changes both in the client and server applications. 
A sender can provide a buffer to be transmitted, and the receiver can indicate a buffer where the received data is to be stored.
Internally, the \sender library adds a \gls{mac} and a signature to each message. 
The \gls{mac} is created using \gls{hmac} with the \gls{sha} (256) hash function and a 256-bit key, and the signature employs \gls{rsa} with 512-bit keys. 
Messages are authenticated at the \postsieve also using \gls{hmac}. 
The \gls{rsa} keys were kept relatively small for performance reasons. 
However, since they should be updated periodically (e.g., every few hours), this precludes all practical brute force attacks~\cite{David:2015}.%%%%%%%%%%%%%%%%%%%%%%%%%%%%%%%%%

The \presieves operate as separate processes receiving the messages and forwarding them to the \repsieves.
We resorted to \textsc{BFT-SMaRt}~\cite{Bessani:2014} to implement the \gls{tom} and manage the \repsieves.
\textsc{BFT-SMaRt}, like most \gls{smr} systems (e.g., \textsc{Pbft}~\cite{Castro:2002}, \textsc{Zyzzyva}~\cite{Kotla:2010}, \textsc{Prime}~\cite{Amir:2011}) follows a client-server model, where a client transmits request messages to a group of server replicas and then receives the output messages with the results of some computation.
We had to modify \textsc{BFT-SMaRt} because this model does not fit well with the message flow of \sieveq.
Therefore, we have decoupled the client-server model into a client-server-client model.
The client sends messages (but does not wait for responses as in traditional \gls{smr}), and the server forwards them (after validation) to another client, which is the last receiver.
A reverse procedure is carried out for the traffic originating from the receiver.
Furthermore, in \textsc{BFT-SMaRt} the replicated servers are typically designed with a single-thread to process requests.
To improve performance, we also modified the system to allow CPU-costly operations (like a signature verification) to occur concurrently with the rest of the checks performed by \repsieves.

In the prototype, a \presieve can be replaced on two occasions: first, voluntarily by asking for a substitution to the \repsieves, when it is flooded with unauthorized messages (\gls{dos} attack); and second, when a \repsieve detects message corruptions by a \presieve.
In both situations, the \repsieves make a request to the \emph{controller}, which will replace a \presieve instance.
Notice that the \emph{controller} has to wait for $\mathit{f_{rs}+1}$ messages, requesting a \presieve replacement, to ensure that a faulty \repsieve cannot force the recovery of a correct \presieve.
The \repsieves are recovered when the collected information indicates that some component might be attacking other components.
The information is collected and sent to the trusted controller to evaluate and decide if the replicas are making progress as expected.
Based on this information, the controller can suspect on $\mathit{f}$ \repsieves and recover them avoiding the need to recover all of them at once.


Since \textsc{BFT-SMaRt} is programmed in Java, we decided to use the same language to develop the various \sieveq components. 
If the sender and receiver applications are coded in other languages, they can still be supported by implementing specific \sender and \postsieve libraries.


\section{Evaluation}
\label{evaluation}


In this section, we present the evaluation of \sieveq under different network and attack conditions.
We present the results of four types of experiments.
In the first one, we evaluate the latency for different \sender workloads, assessing the performance of \sieveq in the absence of failures.
The second experiment assesses the effect of filtering rules complexity on the performance of the system.
In the third experiment, we assess the throughput in three scenarios: \emph{i)} the normal case, \emph{ii)} during a DoS attack without countermeasures; and \emph{iii)} during  a DoS with all the resilience mechanisms enabled (in fact we considered two \gls{dos} attacks: external, from a malicious \sender, and internal, from a compromised \repsieve).
The last experiment considers the capability of \sieveq to safeguard a \gls{siem} system under a similar workload as the one observed in the 2012 Summer Olympic Games.

\subsection{Testbed Setup}

Figure~\ref{fig:testbed} illustrates the testbed, showing how the various \sieveq components were deployed in the machines.
We consider one \sender and one \postsieve deployed in different physical nodes, and an additional host acting as a malicious external adversary.
The \emph{controller} and \presieves were located in the same physical machine for convenience but in different \glspl{vm}.
Four \repsieves were placed in distinct physical nodes.
Every machine had two Quad-core Intel Xeon $2.27$ GHz CPUs, with $32$ GB of memory, and a Broadcom NetXtreme II Gigabit network card.
All the machines were connected by a 1Gbps switched network and run Ubuntu $10.04$ $64$-bit LTS (kernel $2.6.32$-server) and Java $7$ ($1.7.0\_67$).

\begin{figure}[t]
\centering
\includegraphics[width=0.7\columnwidth]{images/images/testbed1.pdf}
\caption{The \sieveq testbed architecture used in the experiments.}
\label{fig:testbed}
\end{figure}


\subsection{Methodology}

\sieveq acts as a highly resilient protection device, receiving messages on one side and forwarding them to the other side.
Therefore, the performance of \sieveq is assessed with latency/throughput measurements that can be attained under different network loads. 
In the experiments, the sender transmits data at a constant rate, i.e., 100 to 10000 messages per second, with three different message payload sizes, i.e., 100 bytes, 500 bytes, and 1k bytes.
We used the Guava library~\cite{guava} to control the message sending rate.


\emph{Latency} measures the time it takes to transmit a message from \sender until it is delivered to the application on the \postsieve. 
The following procedure was employed to compute the latency: the \sender obtains the local time before transmitting a message. 
When the message arrives and is ready to be delivered to the receiver application (after voting), the \postsieve returns an acknowledgment over a dedicated UDP channel.
The \Sender gets the current time again when the acknowledgment arrives. 
The latency of a message is the elapsed time calculated at \sender (receive time minus send time) subtracted by the average time it takes to transmit a UDP message from the \postsieve to the \sender.

\emph{Throughput} gives a measure of the number of messages per second that can be processed by \sieveq. 
It was calculated at the \postsieve using a counter. 
This counter is incremented every time a message is delivered to the receiver application, and the counter is reset to zero after one second. 
Consequently, the server can calculate the number of messages delivered every second. 
The throughput is computed as the average value of the individual measurements collected over a period of time (in our case, 5 minutes after the steady state was reached).

In some experiments, we wanted to assess the behavior of \sieveq under a \gls{dos} attack. 
The attack was made using \texttt{PyLoris}~\cite{pyloris}, a tool built to exploit vulnerabilities on TCP connection handling.
The tool implements the Slowloris attack method, which opens many TCP connections and keeps them open.
The tool allows the user to define parameters like group size of attack threads, the maximum number of connections, and the time interval between connections among others.
In our setup, \texttt{PyLoris} was configured to perform an unlimited number of connections, with $0.1$ milliseconds between each connection.

In all experiments, measurements were taken only after the \gls{jvm} was warmed-up, and the disks were not used (all data is kept in memory).


\subsection{Performance in Failure-free Executions}
\label{throughput_latency}

This experiment measures the latency of \sieveq with several message sizes and distinct message transmission rates. 
It demonstrates the overall performance of \sieveq in different scenarios, gradually stressing the \postsieve side as the workload is slowly increased. 
Measurements were collected after the system reached a steady state. 
The experiments were repeated 10 times for every workload, and the average result is reported.

\begin{figure}[t]
\centering
\includegraphics[width=\columnwidth]{images/gnuplot/sieveq/new_plot_latencyvsthroughput/latencyVSthroughput.pdf}
\caption{\sieveq latency for each workload (message size and transmission rate).}
\label{fig:lat_vs_throu}
\end{figure}

Figure~\ref{fig:lat_vs_throu} shows how the latency is affected by the transmission rate and message size.
As expected, when the system becomes increasingly loaded, the latency grows proportionally because resources have to be shared among the various messages.
The latency increase is approximately linear for the messages with 100 and 500 bytes until the throughput reaches $10k$ messages per second.
The messages with $1k$ bytes have a linear increase on latency until the throughput reaches approximately $7k$ messages per second, and then it has a higher increase as more load is put on the system.
This means that very high workloads can only be supported if applications have some tolerance to network delays.
Overall, the performance degrades gracefully when varying the message payload sizes, for rates under $7k$ messages per second.


\begin{table}[t]
\begin{center}
{\small
\begin{tabular}{ | c | c |  c | }\hline
\textbf{Payload size } & \textbf{Non-optimized}  & \textbf{Optimized}   \\
\textbf{$(bytes)$} & \textbf{$(msg/sec)$} & \textbf{$(msg/sec)$}  \\\hline \hline
$100  $ & 1360 & 10682 \\
$500  $ & 1320 & 10618 \\
$1000 $ & 1311 & 10332  \\
\hline
\end{tabular}
}
\vspace{2mm}
\caption{Maximum load induced by the \sender library with various message sizes\label{tab:client_evaluation}.}
\end{center}
\end{table}

We performed a more detailed analysis of the overheads introduced by the various components of \sieveq.
We observed that \sender performs the most expensive operations, which is interesting because it shows that our design offloads part of the effort to the edges, reducing bottlenecks.
The most significant overheads were caused by the tasks associated with securing the message payload (which are the most costly operations in the system). Several optimizations were made to mitigate the performance penalties during the message serialization (e.g., the creation of the signature), including the use of parallelization to take advantage of the multicore architecture (as done, for example, in~\cite{Kirsch:2014}).
Table~\ref{tab:client_evaluation} shows the gain of using the optimized version of the \sender library. 
Similar optimizations were employed in other components.



\subsection{Effect of Filtering Rules Complexity}


The results presented in the previous section do not consider any complex filtering rule set.
This section presents experiments with a fully loaded system and application-level filtering rules with different complexity at the \repsieves.

Given the diverse requirements imposed by application-level firewalls, we decided to approximate the complexity of the filtering rules by considering a variable number of string matchings on processed messages.  It is well recognized that the most expensive aspect of message filtering is exactly finding (or not) specific strings in the packet contents (besides crypto verifications, which were included in all our experiments). 
The high cost of running such algorithms leads for instance to several implementations in \glspl{fpga} and \glspl{gpu} to improve performance in firewalls and \gls{ids} (e.g.,~\cite{Moscola:2003,Lee:2015}).

We used a classical algorithm for string matching (Knuth–Morris–Pratt~\cite{Knuth:1977}) at the \repsieves to implement the filtering rules.
This algorithm is employed in \glspl{ids}~\cite{Prabha:2014} and firewalls like \texttt{iptables}~\cite{iptables}.
The algorithm employs a pre-computed table to execute string matching with $O(n+k)$ comparisons, where $n$ is the string length, and $k$ is the pattern length.


We measured the latency of the system by varying the message size from 100 to 1000 bytes and the string pattern size from 5 to 20 bytes. 
Both the message content and the strings were randomly generated. 
The experiments were performed with the maximum throughput of $10 000$ messages per second, as identified in the previous section. 
Figure~\ref{fig:KMP} shows the latency of \sieveq without message filtering (Baseline) and string matching of different sizes.
In the last group of experiments, where the message size is 1000 bytes, and the pattern is 20 bytes, the latency increases by $50\%$. 
In other cases, sometimes higher overheads were observed, e.g., with 500 bytes messages and 20 bytes pattern the overhead is approximately $200\%$. 
This result is expected as string matching is a slow operation and it needs to be performed in the critical path of message processing. 
Implementations with hardware support (as mentioned before) could be integrated with \sieveq to reduce these delays significantly.


\begin{figure}[t]
\centering
\includegraphics[width=\columnwidth]{images/gnuplot/sieveq/new_plot_fw/fw.pdf}
\caption{Comparison of the \sieveq's latency between the baseline and filtering rules with patterns of different sizes.}
\label{fig:KMP}
\end{figure}


\subsection{\sieveq Under Attack}

Our next set of experiments aims to evaluate the system under different attack scenarios.
Here, the \sender creates a steady load of 1000 500-byte messages per second.
We measured the latency and throughput of the system in three conditions: failure-free operation, a malicious external and internal \gls{dos} attack, and a malicious attack with remediation mechanisms.
Before presenting the results, we need to stress that these experiments must be compared with the results displayed in Figure \ref{fig:attack_traditional}, which were obtained in the same way but with a different architecture (see Figure \ref{fig:traditional}).

Figures~\ref{fig:latency_normal} and~\ref{fig:throughput_normal} show the latency and throughput observed in the failure-free scenario.
One can observe that latency stays (on average) around $3.4$ milliseconds.
The throughput is approximately constant during the whole period.
It is possible to observe some momentary spikes in the latency and throughput, which happens due to Java garbage collector and a queuing effect from the \gls{smr}.

\begin{figure}[t]

\subfloat[Baseline latency.]{\includegraphics[width=0.5\columnwidth]{images/gnuplot/sieveq/plots/latency_normal_t1000_s500.pdf}\label{fig:latency_normal}}
\hspace{-5mm}
\subfloat[Baseline throughput.]{\includegraphics[width=0.5\columnwidth]{images/gnuplot/sieveq/plots/server_throughput_normal.pdf}\label{fig:throughput_normal}}
\caption{Performance of \sieveq in fault-free executions.}
%\label{fig:performance_attacks}
\end{figure}




The behavior of \sieveq during a \gls{dos} attack is displayed in Figures~\ref{fig:latency_attack} and~\ref{fig:throughput_attack}.
In this scenario, we have disabled the \sieveq capability of replacing \presieves at runtime.
The attack consists in stressing the TCP socket interface of the \presieves by creating many TCP connections, which consumes network bandwidth and wastes resources at system and application levels. When the attack is started, it executes for 50 seconds. The latency graph displays a reasonable impact regarding an increase in the delays for message delivery. 
In some cases, the latency is not too affected, but in others, there is a drastic delay, with some messages taking more than 3 seconds to be received.
The attack also has consequences on the throughput as it is possible to observe an oscillation between 0 to 4000 messages per second (which correspond to the situation when the \postsieve processes a batch of messages that have been accumulated).


\begin{figure}[t]
\subfloat[DoS w/ no recovery.]{\includegraphics[width=0.5\columnwidth]{images/gnuplot/sieveq/plots/latency_normal_t1000_s500_under_attack.pdf}\label{fig:latency_attack}}
\hspace{-5mm}
\subfloat[DoS w/ no recovery.]{\includegraphics[width=0.5\columnwidth]{images/gnuplot/sieveq/plots/server_throughput_attack.pdf}\label{fig:throughput_attack}}
\hspace{-5mm}
\caption{Performance of \sieveq under DoS attack conditions.}
%\label{fig:performance_attacks}
\end{figure}


Figures~\ref{fig:latency_attack_replacement} and~\ref{fig:throughput_attack_replacement} show the latency and throughput when a similar \gls{dos} was carried out, but in this case the \sieveq replaced the \presieve under attack with a new \presieve.
When the \presieve finds out that it is being overloaded with messages coming from non-authorized senders, it asks for a replacement.
After that, the controller replaces the faulty \presieve, and the existing \senders are contacted to migrate their connections.
As the figures show, the impact of the attack is minimized, since only a few messages are delayed, and throughput is only affected momentarily while the \presieve is switched.
Once the new \presieve takes over, the messages lost during the switching period are retransmitted and delivered.
In practice, the attack becomes ineffective because, although it continues to consume network bandwidth, there is no longer a \presieve to process the malicious messages.
An adversary could increase the attack sophistication and try to find a new \presieve target.
However, even in this case, the attack has a limited effect because during an interval of time (while there is a search for a fresh target) the system can make progress.

\begin{figure}[t]
\subfloat[DoS w/ recovery.]{\includegraphics[width=0.5\columnwidth]{images/gnuplot/sieveq/plots/latency_t1000_s500_change_with_attack.pdf}\label{fig:latency_attack_replacement}}
\hspace{-5mm}
\subfloat[DoS w/ recovery.]{\includegraphics[width=0.5\columnwidth]{images/gnuplot/sieveq/plots/server_throughput_with_attack_change.pdf}\label{fig:throughput_attack_replacement}}
\hspace{-5mm}
\caption{Performance of \sieveq under DoS attack conditions with recovery.}
%\label{fig:performance_attacks}
\end{figure}


Figures~\ref{fig:replica_dos_no_recovery} and~\ref{fig:replica_dos_recovery} shows the \sieveq throughput when an internal attack is carried out by a compromised \repsieve. 
The experiment was made with a \repsieve ($r1$) launching a \gls{dos} to another \repsieve ($r2$).
The attack consists in overloading a \repsieve with \emph{state transfer} requests, which are the most demanding request a replica can receive in \textsc{BFT-SMaRt}~\cite{Bessani:2013}.
Figure~\ref{fig:replica_dos_no_recovery} shows the impact on the \sieveq throughput during an attack lasting 50 seconds, without any recovery capability on the system.
As can be seen, the performance of the system is severely disrupted during the attack.
Figure~\ref{fig:replica_dos_recovery} shows the same attack but with the detection and recovery mechanism described in Section~\ref{faultyrepsieve}.
The \postsieve detects the problem by noticing that $\mathit{f_{rs}+1}$ \repsieves are sending fewer messages than the others and then requests a recovery.
When the \repsieve is recovered, it requests the state from the other replicas, and then after applying the new state, the replica resumes the normal execution (end line in the figure).
In the experiment of Figure \ref{fig:replica_dos_recovery} we show a case in which the faulty \repsieve is the first to be recovered.
It could happen that \sieveq recovered $\mathit{f_{rs}}$ \repsieves before the faulty one.
This would take  $\mathit{(f_{rs}+1)} \times$ 3 seconds (in our setup) before the system resumes the normal execution.



\begin{figure}[t]
\subfloat[Internal DoS execution without recovery actions.]{\includegraphics[width=.5\columnwidth]{images/gnuplot/sieveq/new_internal_recovery/server_throughput_no_recovery_replicas_dos.pdf}\label{fig:replica_dos_no_recovery}}
\subfloat[Internal DoS execution with recovery actions.]{\includegraphics[width=0.5\columnwidth]{images/gnuplot/sieveq/new_internal_recovery/server_throughput_recovery_replicas_dos.pdf}\label{fig:replica_dos_recovery}}
\caption{Performance of \sieveq under internal DoS attack conditions without and with recovery.}
%\label{fig:performance_attacks}
\end{figure}

\subsection{\sieveq to Protect a SIEM System}
\label{use_case}

\gls{siem} systems offer various capabilities for the collection and analysis of security events and information in networked infrastructures~\cite{Miller:2010}.
Organizations are employing these systems as a way to help with the monitoring and analysis of their infrastructures.
They integrate an extensive range of security and network capabilities, which allow the correlation of thousands of events and the reporting of attacks and intrusions in near real-time.

A \gls{siem} operates by collecting data from the monitored network and applications through a group of sensors, which then forward the events towards a correlation engine at the core facility.
The engine performs an analysis of the stream of events and generates alarms and other information for post-processing by other \gls{siem} components.
Examples of such components are an archival subsystem for the storage of data needed to support forensic investigations, or a communication subsystem to send alarms to the system administrators.

\begin{figure}[t]
\centering
\includegraphics[width=0.70\columnwidth]{images/images/SIEM.pdf}
\caption{Overview of a SIEM architecture, showing some of the core facility subsystems protected by the \sieveq.}
\label{fig:siem}
\end{figure}


As part of the MASSIF European project~\cite{Vianello:2013}, we have implemented a resilient \gls{siem} system where \sieveq was used to protect the access to the core facility (see Figure~\ref{fig:siem}).
In the \sieveq architecture, the sensors integrate the \sender while the \postsieve was placed in the correlation engine.
Additionally, we had access to an anonymized trace with the security events collected during the 2012 Olympic Games.
Each event corresponds basically to a string describing some observed problem by a sensor. The strings of text had lengths varying between a minimum of 551 bytes and a maximum of 2132 bytes, with an average length of 1990 bytes (and a standard deviation of 420 bytes). Based on this log, we built a sensor emulator that generates traffic at a pre-defined rate. Basically, when it is time to produce a new event, the emulator selects an event from the trace and feeds it to the \sender.

During the 2012 Olympic Games, the workload was approximately 11 million events per day, i.e., around 127 events per second.
Figure~\ref{fig:massif} shows the latency imposed by \sieveq for this workload, and when it is scaled up from 2 to 16 times more. 
As can be observed, the latency is in the order of 4 milliseconds for the emulated scenario.
Even with the highest load (16 times), the observed values had a latency below $70$ milliseconds.
This means that \sieveq could potentially deliver 176 million events per day, which is more than enough to accommodate the expected growth in the number of events for the next Olympic Games.\footnote{In 2016 Rio's Olympic Games the number of events per second was approximated 174, source: \url{https://diginomica.com/2016/11/24/securing-the-olympics-lessons-for-enterprise-cyber-security/}}

\begin{figure}[t]
\centering
\includegraphics[width=\columnwidth]{images/gnuplot/sieveq/new_massif/massif.pdf}
\caption{\sieveq latency for the 2012 Summer Olympic Games scenario throughput requirement (127 events per second) and how the \sieveq scale.}
\label{fig:massif}
\end{figure}

\section{Discussion}
\label{discussion}

\sieveq differs from standard firewalls like \texttt{iptables}~\cite{iptables} that do not need client- or server-side code modifications.
However, our system requires these modifications to ensure end-to-end message integrity and tolerance of compromised components.
Complete transparency would be hard to achieve mainly due to the use of voting.
Nonetheless, it is worth stressing that \sieveq is to be used as an additional protection device in critical systems, not as a substitute to regular L3 firewalls.
\sieveq provides a protection similar to an application-level/L7 firewall, as one can implement arbitrary rules on the \repsieve module.


State machine replication is a well-known approach for replication~\cite{Schneider:1990}.
In this technique, every replica is required to process requests in a deterministic way.
This requirement traditionally implies in two limitations: (1) replicas cannot use their local clock during request processing, and (2) all requests are executed sequentially.
The first limitation can affect the capacity of \sieveq to process rules that use time.
We remove this limitation by making use of the timestamps generated by the leader replica and agreed upon on each consensus, as proposed in \textsc{Pbft}~\cite{Castro:2002} and implemented in \textsc{BFT-SMaRt}~\cite{Bessani:2014}.
The second limitation can constraint the performance of the system, especially when CPU-costly operations such as signature verifications are executed.
One of the optimizations we implemented in \sieveq was to add multi-threading support to the \textsc{BFT-SMaRt} replicas.
More precisely, the signature verification is done by a pool of threads that either accept or discard messages.
Once accepted, a message is added to a processing queue following the order established by the total order multicast protocol.
A single thread consumes messages from this queue, verifies them against the security policy, updates the firewall state (if needed) and forwards them to their destinations, without violating the determinism requirement.



\section{Final Remarks}
\label{sec:finalremarkssieveq}

We presented \sieveq, a new intrusion-tolerant protection system for critical services, such as \gls{ics} and \gls{siem} systems.
Our system exports a message queue interface which is used by senders and receivers to interact in a regulated way.
The main improvement of the \sieveq architecture, when compared with previous systems, is the separation of message filtering in several components that carry on verifications progressively more costly and complex.
This allows the proposed system to be more efficient than the state-of-the-art replicated firewalls under attack.
\sieveq also includes several resilience mechanisms that allow the creation, removal, and recovery of components in a dynamic way, to respond to evolving threats against the system effectively. 
Experimental results show that such resilience mechanisms can significantly reduce the effects of \gls{dos} attacks against the system.



\chapter{Finding evidence for supporting and manage diversity }
\label{chap:supportingdiversitymechanism}

 \note{We need to clarify that \system is proactive and \sieveq is reactive, they can be used together. E.g., when there is a problem the controller increases the risk}
\section{Introduction}

Practical \gls{bft} replication was initially proposed as a solution to handle Byzantine faults of both accidental and malicious nature~\cite{Castro:1999}.
The correctness of a BFT service comes from the existence of a quorum of correct nodes, capable of reaching consensus on the (total) order of messages to be delivered to the replicas.
For instance, to tolerate a single replica failure, the system typically must have four replicas~\cite{Castro:2002,Kotla:2010,Aublin:2015}. 
This model only works if nodes fail independently, otherwise, once an attacker discovers a vulnerability in one node, it is most likely that the remaining nodes suffer from the same weakness. 

In the last twenty years of \gls{bft} replication research, few efforts were made to justify or support this assumption. 
However, there were great advances on the performance (e.g.,~\cite{Kotla:2010,Aublin:2015,Behl:2015}), use of resources (e.g.,~\cite{Veronese:2013,Behl:2017,Liu:2016,Yin:2003}), and robustness (e.g.,~\cite{Amir:2011,Bessani:2014,Clement:2009b}) of BFT systems.
These works assume, either implicitly or explicitly, that replicas fail independently, relying on some orthogonal mechanism (e.g.,~\cite{Roeder:2010,Chen:1995}) to remove common weaknesses, or rule out the possibility of malicious failures from their system models.
A few works have implemented and experimented such mechanisms~\cite{Rodrigues:2001,Roeder:2010,Amir:2011}, but in a very limited way.
Nonetheless, in practice, diversity is a fundamental building block of dependable services in avionics~\cite{Yeh:2004}, military systems~\cite{rhimes}, and even in recent blockchain platforms such as Ethereum\footnote{\url{https://www.reddit.com/r/ethereum/comments/55s085/geth_nodes_under_attack_again_we_are_actively/}} -- three essential applications of \gls{bft}. 

For the few works that do consider the diversity of replicas, the absence of common-mode failures is mostly taken for granted.
For example, by using memory randomization techniques~\cite{Roeder:2010} or different OSes~\cite{Rodrigues:2001,Junqueira:2005}, it is assumed that such failures will not exist without providing evidence for it. 
In fact, researchers have argued that randomization techniques do not suffice to create fault independence~\cite{Snow:2013,Bittau:2014}.
In addition, although the use of distinct OSes promotes fault independence to some extent, \emph{per se} it is not enough to preclude vulnerability sharing among diverse \glspl{os}~\cite{Garcia:2014}.

Even if there was an initial diverse set of $n$ replicas that would have fault independence, long-running services eventually will need to be cleaned from possible failures and intrusions.
Proactive recovery of \gls{bft} systems~\cite{Castro:2002,Sousa:2010,Roeder:2010,Platania:2014,Distler:2011} periodically restarts the replicas to remove undetected faulty states introduced by a stealth attacker. 
However, a common limitation is that these works assume that the weaknesses will be eliminated after the recovery.
In practice, this does not happen unless the replica code changes after its recovery.

This paper presents \system, 
%a control plane integrated with BFT replication.
%It is the first to apply techniques from MTD to automatically change the attack surface of BFT systems in a dependable and automatic way.
%It is 
the first system that automatically changes the attack surface of a \gls{bft} system in a dependable way.
\system continuously collects security data from \gls{osint} feeds on the internet to build a knowledge base about the possible vulnerabilities, exploits, and patches related to the systems of interest.
This data is used to create clusters of similar vulnerabilities, which potentially can be affected by (variations of) the same exploit.
These clusters and other collected attributes are used to analyze the risk of the \gls{bft} system becoming compromised. % due to common vulnerabilities.
Once the risk increases, \system replaces the potentially vulnerable replica by another one, trying to maximize the failure independence. % of the replicated service.
Then, the replaced node is put on quarantine and updated with the available patches, to be re-used later.
These mechanisms were implemented to be fully automated, removing the human from the loop.

The current implementation of \system manages 17 \gls{os} versions, supporting the \gls{bft} replication of a set of representative applications.
The replicas run in \glspl{vm}, allowing provisioning mechanisms to configure them. 
We conducted two sets of experiments, one demonstrates that \system risk management can prevent a group of replicas from sharing vulnerabilities over time; the other, reveals the potential negative impact that virtualization and diversity can have on performance. However, we also show that if naive configurations are avoided, \gls{bft} applications in diverse configurations can actually perform close to our homogeneous bare metal setup.
%These results open avenues for many future works in the area. 

In summary, we make the following contributions: 

\begin{enumerate}

\item \system, a control plane that monitors \gls{osint} data and manages the \gls{bft} service replicas, selecting and reconfiguring the system to always run the ``most diverse'' set of replicas at any given time (Sections~\ref{sec:design} and~\ref{sec:implementation});

\item A method for assessing the risk of a group of replicas being compromised based on the security news feeds available on the internet. 
The method overcomes limitations from works that use NVD data for managing the replicas vulnerability independence (Section~\ref{sec:metric});

\item An evaluation of our risk management method based on real historical vulnerability data showing its effectiveness in keeping a group of replicas safe from common vulnerabilities (Section~\ref{sec:diversity});

\item An extensive evaluation of \system prototype using 17 \gls{os} versions, a \gls{bft} replication library, and some \gls{bft} applications (i.e., a \gls{kvs}, an application-level firewall/message queuing service, and a blockchain service) showing the costs of supporting diversity in \gls{bft} systems (Section~\ref{sec:overhead}).

\end{enumerate}


\section{Diversity-aware Reconfigurations}
\label{sec:metric}

The core of \system is the vulnerability evaluation method used to assess the risk of having replicas with shared vulnerabilities.
This section details this method.

\subsection{Finding Common Vulnerabilities}

\gls{nist}'s \gls{nvd}~\cite{nvd} is the authoritative data source for disclosure of vulnerabilities and associated information~\cite{Massacci:2010}. 
\gls{nvd} aggregates vulnerability reports from more than 70 security companies, advisory groups, and organizations, thus being the most extensive vulnerability database on the web. 
All data is made available as \gls{xml} data feeds, containing the reported vulnerabilities on a given period. 
Each \gls{nvd} vulnerability receives a unique identifier and a short description provided by the \gls{cve}~\cite{cveterm}. 
The \gls{cpe}~\cite{cpe} provides the list of products affected by the vulnerability and the date of the vulnerability publication.
The \gls{cvss}~\cite{cvss} calculates the vulnerability severity considering several attributes, such as the attack vector, privileges required, exploitability score, and the security properties compromised by the vulnerability (i.e., integrity, confidentiality, or availability).

Previous studies on diversity solely count the number of shared vulnerabilities among different \glspl{os}, assuming that less common vulnerabilities implies a smaller probability of compromising $f+1$ OSes~\cite{Garcia:2014}. 
Although this intuition may seem acceptable, in practice it underestimates the number of shared vulnerabilities due to imprecisions in the data sources. 
For example, Table~\ref{tab:missing_products} shows three vulnerabilities, affecting three different \glspl{os} at distinct dates.
At first glance, one may consider that these \glspl{os} do not share vulnerabilities.
However, a careful inspection of the descriptions shows that they are very similar.
Moreover, we checked this resemblance by searching for additional information on security web sites, and we found out that CVE-2016-4428, for example, also affects Solaris.\footnote{\url{https://www.oracle.com/technetwork/topics/security/bulletinjul2016-3090568.html}}

\begin{table}[!t]
\begin{center}
{\scriptsize
\begin{tabular}{| p{2.3cm} | p{10cm} | }\hline
\textbf{CVE (affected OS)} & \textbf{Description} \\\hline\hline
CVE-2014-0157 (Opensuse 13) & \scriptsize \gls{xss} vulnerability in the Horizon Orchestration dashboard in OpenStack Dashboard (aka Horizon) 2013.2 before 2013.2.4 and icehouse before icehouse-rc2 allows remote attackers to inject arbitrary web script or HTML via the description field of a Heat template. \\ \hline
CVE-2015-3988 (Solaris 11.2) & \scriptsize Multiple \gls{xss} vulnerabilities in OpenStack Dashboard (Horizon) 2015.1.0 allow remote authenticated users to inject arbitrary web script or HTML via the metadata to a (1) Glance image, (2) Nova flavor or (3) Host Aggregate. \\ \hline
CVE-2016-4428 (Debian 8.0) & \scriptsize \gls{xss} vulnerability in OpenStack Dashboard (Horizon) 8.0.1 and earlier and 9.0.0 through 9.0.1 allows remote authenticated users to inject arbitrary web script or HTML by injecting an AngularJS template in a dashboard form. \\ \hline
\end{tabular}
}
\caption{Similar vulnerabilities affecting different OSes.}
\label{tab:missing_products}
\end{center}
\end{table}

Even with these imperfections, \gls{nvd} is still the best data source for vulnerabilities.
Therefore, we exploit its curated data feeds for obtaining the unstructured information present in the vulnerability text descriptions and use this information to find similar weaknesses.
A usual way to find similarity in unstructured data is to use clustering algorithms~\cite{Jain:2010}.
Clustering is the process of aggregating related elements into groups, named clusters, and is one of the most popular unsupervised machine learning techniques. 
We apply this technique to build clusters of similar vulnerabilities (see Section~\ref{sec:details} for details), even if the data feed reports that they affect different products.
For example, the vulnerabilities in Table~\ref{tab:missing_products} will be placed in the same cluster as there is some resemblance among the descriptions, and they can potentially be activated by (variations of) the same exploit.

It is worth to remark that by using clusters to find similar vulnerabilities, we conservatively increase the chances of capturing shared weaknesses contributing to the score of a pair of replicas.

\subsection{Measuring risk}
\label{sec:measurerisk}

As discussed before, each vulnerability in \gls{nvd} has an associated \gls{cvss} severity score. 
Therefore, a straw man solution for measuring risk would be to sum the \gls{cvss} scores of all common vulnerabilities in the software stack of two replicas to get an estimate of how dangerous are their shared weaknesses.
However, \gls{cvss} has some limitations that make it unsuitable for managing the risk of replicated systems:
(1) In practice, it has been shown that there is no correlation between the \gls{cvss} exploitability score and the existence of real exploits for the vulnerability~\cite{Bozorgi:2010}; 
(2) \gls{cvss} does not provide information about vulnerabilities exploiting and patching times; 
(3) \gls{cvss} does not account for the vulnerability age, which means that severity remains the same over the years~\cite{Frei:2006}; 
and (4) some studies show that \gls{cvss} may overestimate  severity~\cite{Sabottke:2015}, as for example larger scores do not correspond to higher prices in the vulnerabilities' black markets~\cite{Allodi:2014}.

Given these limitations, we derive a novel, more refined, metric to measure the risk of a \gls{bft} system being affected by common vulnerabilities.
In our particular context, we are mostly interested in capturing information that relates to the window of exposure that vulnerabilities have, mainly when they are correlated among \replicas.
Therefore, we developed a risk metric that aims to overpass the identified limitations. 
We solved (1) and (2) by using additional \gls{osint} sources that provide information about the exploit and patch dates. 
Since NVD does not provide this information, we collect more data from other \gls{osint} sources like Exploit-DB~\cite{edb} for exploits, patching information from CVE-details~\cite{cvedetails}, and additional vendor websites, such as Ubuntu Security Notices~\cite{ubuntu}, Debian Security Tracker~\cite{debian}, and Microsoft Security Advisories and Bulletins~\cite{microsoft} (which also give additional product versions affected by the vulnerability).
We solve (3) using the vulnerability published date to calculate its age.
Finally, we only use the \gls{cvss} attributes that concern to integrity and availability, the properties traditionally related with \gls{bft} replication (4).



\begin{table}[h]
\begin{center}
{\small
\begin{tabular}{ c }\hline
\vbox{
\begin{equation}
\mathit{\systemformula(sc)}=\sum_{i=1}^{n-1} \sum_{j=i+1}^{n} \pairformula(rc_i,rc_j) \label{eq:3}
\end{equation}
}\\ \hline
\vbox{
\begin{equation}
\mathit{\pairformula(rc_i,rc_j)}=\sum_{v_k \in \mathcal{V}_{i,j}} \mathit{\vulnerabilityformula(v_k)} \label{eq:2}
\end{equation}
}\\ \hline
\vbox{
\begin{equation}
\mathit{\vulnerabilityformula(v_k)}= (A+I+\mathit{exp(v_k)}) \times \mathit{tdist(v_k)} \label{eq:1}
\end{equation}
\begin{equation}

\mathit{exp(v_k)}= \begin{cases}
		\mathit{max(\mathit{DP}-\mathit{DE},0)}+1 	& \text{$v_k$ exploited, patched}\\
  		\mathit{DE} 		& \text{$v_k$ exploited, not patched}\\
		0 		& \text{otherwise}
\end{cases} \label{eq:exposed}
\end{equation}
}\\ \hline

\end{tabular}
}
\label{tab:equations}
\end{center}
\end{table}

Our metric considers all this information to measure the risk of a set of $n$ replicas having active shared vulnerabilities.
More specifically, it works as an indicator of how fault-independent is a \configuration.
Equation~\ref{eq:3} shows the risk of a \configuration as the sum of the different \replica pairs' score.
This score is calculated based on the set of vulnerabilities $\mathcal{V}_{i,j}$ that affects both $r_i$ and $r_j$ or that are present in a cluster containing vulnerabilities affecting both \replicas (Equation~\ref{eq:2}).
Finally, we calculate the score of each vulnerability in $\mathcal{V}_{i,j}$ (Equation~\ref{eq:1}). 
We assign a \emph{dynamic score} to each vulnerability, considering the referred attributes:
\emph{(i)} we take two \gls{cvss} attributes to capture the extent to which a vulnerability $v_k$ affects availability ($A$) and integrity ($I$);
\emph{(ii)} we account for the number of days the vulnerability was exposed with $\mathit{exp}$, i.e., there was an exploit and no patch available.
This is calculated considering the number of days to patch ($DP$) and to exploit ($DE$) $v_k$ (Equation~\ref{eq:exposed}); 
\emph{(iii)} and we use an amortization function to reflect the fact that older vulnerabilities have are less likely to harm the system ($\mathit{tdist(v_k)} \in [0,1]$).


\subsection{Selecting Configurations}
\label{sec:configurations}


We use the risk metric to choose the \replicas that should be included in the \configuration. 
This is done by periodically evaluating the risk of the current \configuration. 
If the risk exceeds a pre-defined threshold, a mechanism is triggered to replace replicas and reduce the overall risk.
First, it decides which \replica (\r) should be removed and put in a quarantine set (\QS). 
Then, it selects (one of) the best candidate(s) replicas from all the available candidates (\RS) to make the substitution.
When the replacement takes place, the resulting \configuration (\ES) has lower risk than the previous one.
Additionally, we ensure that removed \replicas can sometime later re-enter the system, and the ones that are in the system, despite their overall score, are eventually replaced.
Therefore, each replica \r in \ES has an \emph{age} value that is incremented. 
On the contrary, each removed replica \r in \QS, has a \emph{healing} value that is decremented.


This procedure is detailed in Algorithm \ref{alg:algorithm2}.
The \emph{Monitor} function is called on each monitoring round (e.g., on every hour).
Consider a \ES that is already running with risk$=\alpha$.
First, the algorithm increments the \emph{age} of each \r in \ES (lines 6-7).
Then, it verifies if the risk of \ES (Equation~\ref{eq:3}) does not exceed the predefined $\mathit{threshold}$ (line 8).
In the affirmative case, a \replica replacement is started.
First, some local variables are initialized (lines 9-10).
Second, it randomly gets pairs $\langle i,j \rangle$ from \ES (line 11) and saves some of them that augment the risk in \MAX (Equation~\ref{eq:2}) (lines 12-14). 
Third, the algorithm picks the older replica (i.e., the one that is in \ES for more time) of all selected pairs (lines 15-18). 
This replica is removed from \ES (line 19) and added to \QS (line 21).
The \emph{healing} value is initialized with a value, different for each \replica, based on historical data about the time it takes for a patch to be published for this software (line 20). 
The algorithm calls a function that selects a new replica to join \ES (line 22). Finally, it decrements the \emph{healing} of each \r in \QS (line 24). When such value reaches zero, \r is removed from \QS and added to \RS (lines 25-28).

Function \emph{Find\_new\_config()} (line 22) solves the following optimization problem:

\vbox{
\begin{small}
\begin{equation*}
\begin{array}{ll@{}ll}
\text{\underline{min} } & \emph{risk}(\ES \cup \{r\}) 	 	\\
\text{\underline{subject to}}	&	 r \in \RS 			\\ 
\end{array}
\end{equation*}
\end{small}
}

\noindent
where \ES is the set of $n-1$ replicas that will stay in the system and $r$ is the new replica (which we have to find) among the ones in \RS.
The twist in our case is that we avoid deterministic solutions to increase the difficulty of an adversary guessing the next configurations.
Therefore, we developed a simple heuristic that finds the $k$ best replicas in \RS (e.g., $k=3$) and randomly picks one of them to be added to \ES.
The heuristic is quite simple: we just calculate the risk of a configuration with each candidate replica from \RS, and choose one of the $k$ replicas that induce lower risk. 

%The minimization problem presented is similar to a problem of solving the minimal cost of a $n$-clique for complete graphs. 
%However, as we mentioned before, we want to create unpredictability on the results.
%Therefore, we introduce randomness on the selection, meaning that the solution is minimal but not always the minimum.
%The minimization function (line 22) is a heuristic to select the valid candidates to reconfigure the \ES with a smaller $\alpha$ than before. 
%The selection of \r must have some degree of unpredictability to increase the difficulty of guessing future configurations.
%To meet this goal, we add randomness when picking \replicas while restricting the number of candidate elements to the ones that minimize the risk. 
%It starts iterating over the \RS and uses the Equation~\ref{eq:3} to calculate the risk of such configuration (line 30).
%If the risk is under a predefined $threshold$, the candidate \r is initialized (line 32), removed from the \RS (line 33) and added to a set with the admissible candidates (\PS) (line 34).
%Then, a random \r is picked from \PS and added to \ES (line 35) which is then returned (line 36).

Although better heuristics can be developed, this brute-force method works in \system because we do not expect \RS (our solution space) to be large.
In addition, this function is only called  if the risk exceeds the threshold. 

\begin{algorithm}[t]
\caption{Replica Set Reconfiguration}\label{alg:algorithm2}
{\footnotesize
%\N: number of replicas\;
%\K: number of replicas to remove\;
\ES: set replicas in the \configuration \;
\RS: set with the available replicas (not in use)\;
%\CS: set of candidate replicas\;
%\RM: set of removable replicas\;
\MAX: set of candidate replicas to remove\;
\QS: set of quarantine replicas\;
%\PS: set of valid replicas\;

\BlankLine
\Fn{Monitor ()}{
	\ForEach{r in \ES}{
		\Inc{r.age};
	}
	\If{\Risk{\ES} $>$ threshold}{
		$maxScore$, $maxAge$ $\leftarrow$ 0\;
		\toRemove $\leftarrow$ $\perp$\;
%		\While{\Size{\toRemove} $<$ \K}{
			\ForEach{$\langle i,j \rangle$ in \ES}{
				\If{\Common{i,j} $\geq$ maxScore}{	
					\MAX $\leftarrow$ \MAX $\cup$ $\{\langle i,j \rangle\}$\;
					$maxScore$ $\leftarrow$ \Common{i,j}\;
				} 
			}
				\ForEach{$\langle i,j \rangle$ in \MAX}{
					\If{\Older{i,j} $\geq$ maxAge}{	
						\toRemove $\leftarrow$ \Older{i,j}\;
						$maxAge$ $\leftarrow$ \toRemove.age\;	
					} 
			}		
			 \ES $\leftarrow$ \ES $\setminus$ $\{$ \toRemove$\}$\;	
             $\toRemove$.healing $\leftarrow$ \healing{\toRemove}\;
			 \QS $\leftarrow$ \QS $\cup$ $\{$ \toRemove$\}$\;
%		}
		 \ES $\leftarrow$ $\mathit{Find\_new\_config(\RS, \ES)}$\;
	}
	\ForEach{r in \QS}{
		\Dec{r.healing}\;
		\If{r.healing $=$ 0}{		
			\QS $\leftarrow$ \QS $\setminus$ $\{r\}$\;		
			$r.age$ $\leftarrow$ 0\;
            \RS $\leftarrow$ \RS $\cup$ $\{r\}$\;
		}
	}
}
%\Fn{Minimize (\RS, \ES)}{
%%	\If{\Size{\ES} is $\emptyset$}{
%%		\CS $\leftarrow$ $\emptyset$\;
%%		\CS $\leftarrow$ \Rand{\RS}\; 
%%	}
%%	\While{\Size{\CS} $\leq \N$}{
%		\PS $\leftarrow$ $\{\}$\;
%		\ForEach{r in \RS}{
%			\If{\Risk{\ES $\cup$ r} $<$ threshold}{
%				r.age $ \leftarrow$ 0\;
%				\RS $\leftarrow$ \RS $\setminus$ $\{r\}$\;		
%				\PS $ \leftarrow$ \PS $\cup$ $\{r\}$\;
%			}
%		}
%		\ES $\leftarrow$ \ES $\cup$ $\{\Rand{\PS}\}$\;
%%	}
%\Return	\ES\;
%}
%\end{multicols}
}
\end{algorithm}


\section{\system Implementation}
\label{sec:implementation}

This section details the implementation of each component of \system. 
It also briefly presents other aspects of our prototype.% like the management of replicas running in a virtualized environment.


\subsection{Control Plane}
\label{sec:lazarus}

Figure~\ref{fig:arch1} shows \system control plane with its four main modules, described below.

\begin{figure}[h]
\begin{center}
\includegraphics[width=.9\columnwidth]{images/images/architecture_new.pdf}
\vspace{-5mm}
\caption{\system architecture.}
\vspace{-5mm}
\label{fig:arch1}
\end{center}
\end{figure}


\circled{1} \textbf{\fetcher.} \system needs to know the software stack of each available replica to be able to look for vulnerabilities in this software.
The list of software products is provided following the \gls{cpe} Dictionary~\cite{cpe}, which is also used by \gls{nvd}. 
%The \fetcher determines if the CPE list is updated by checking the most recent CPE count in the NVD web page.
For each software, the administrator can indicate the time interval (in years) during which data should be obtained from \gls{nvd}' feeds.

The \fetcher parses the \gls{nvd} feeds considering only the vulnerabilities that affect the chosen products. 
The processing is carried out with several threads cooperatively assemblying as much data as possible about each vulnerability -- a queue is populated with requests pertaining a particular vulnerability, and other threads will look for related data in additional \gls{osint} sources. 
Typically, the other sources are not as well structured as \gls{nvd}, and therefore they have to be handled with specialized HTML parsers that we have developed. 
Currently, the prototype supports five other sources, namely Exploit DB, CVE-details, Ubuntu, Debian, and Microsoft. 
As previously mentioned, such sources provide complementary data, like additional affected products versions not mentioned in \gls{nvd}.

The collected data is stored in a relational database (MySQL).
For each vulnerability we keep its \gls{cve} identifier, the published date, the products it affects, its text description, some \gls{cvss} attributes (e.g., availability and integrity); and exploit and patching dates.


\circled{2} \textbf{\risk.} This component finds out when it is necessary to replace the currently running group of \replicas and discovers an alternative configuration that decreases the risk. 
As explained in Section~\ref{sec:measurerisk}, the risk is computed using score values that require two kinds of data: the information about the vulnerabilities, which is collected by the \fetcher; and the vulnerability clusters. 
A vulnerability cluster is a set of vulnerabilities that are related accordingly to their description (see Section~\ref{sec:details} for details).
The \risk also runs Algorithm~\ref{alg:algorithm2} to monitor the replicated system and trigger reconfigurations.


\circled{3} \textbf{\manager.} 
This component automates the setup and execution of the diverse replicas. 
It creates and deploys the \replicas in the execution environment implementing the decisions of the \risk, i.e., it dictates when and which \replicas leave and join the system. 
We developed a replica builder on top of Vagrant~\cite{vagrant}.
It is responsible for downloading, installing, and configuring the \replicas.
Moreover, it performs replica maintenance, where  \replicas in \QS are booted to carry out automatic software updates (i.e., patching). 


\paragraph{Setup.}
The box configuration is defined in a configuration file, named \emph{Vagrantfile}, consider the example in Listing~\ref{vagrantfile}.
In this file, it is possible to set several options of the box, to name a few:
the box name, the number of CPUs, the amount of memory RAM, the IP, the type of network, sync folders between host and the box, etc. 
Additionally, it is possible to pass some VM-specific parameters, e.g., some CPU/mother board flags such as enabling VT-x technology -- consider the \texttt{modifyvm} fields in Listing~\ref{vagrantfile} (lines 6-14).
It is possible to select different \gls{vm} providers (e.g., libvirt, VMware, VirtualBox, Parallels, Docker, etc), we rely on VirtualBox since it is the one with more diversity opportunities in the Vagrant Cloud~\cite{vagrantcloud}. 
Vagrant Cloud is a website that offers a plethora of different VMs.

\begin{lstlisting}[style=mystyle,caption=Windows Server 2016 Vagrantfile,label=vagrantfile]
Vagrant.configure(2) do |config|
	config.vm.box = "geerlingguy/ubuntu1604"
	config.ssh.insert_key = false
	config.vm.provider "virtualbox" do |v|
		v.customize ["modifyvm", :id, "--cpus", 4]
		v.customize ["modifyvm", :id, "--memory", 22000]
		v.customize ["modifyvm", :id, "--cpuexecutioncap", 100]
		v.customize ["modifyvm", :id, "--ioapic", "on"]
		v.customize ["modifyvm", :id, "--hwvirtex", "on"]
		v.customize ["modifyvm", :id, "--nestedpaging", "on"]
		v.customize ["modifyvm", :id, "--pae", "on"]
		v.customize ["modifyvm", :id, "--natdnshostresolver1", "on"]
		v.customize ["modifyvm", :id, "--natdnsproxy1", "on"]
	end
	config.vm.network "public_network", ip:"192.168.2.50", bridge:"em1"
	config.vm.provision :shell, path: "run_debian.sh", privileged: true
end
\end{lstlisting}

\todo{Add all fields that are mentioned, then add lines to the text}

\paragraph{Download.}
One of the fields of the \emph{Vagranfile} is the boxname, the \texttt{box} field is the key that is used to choose which box will be downloaded, each key is unique for each box.
There are plenty of \glspl{os} and versions ready-to-use, and the same OS/version can have different ``manufacturers" -- some of which are official.

\paragraph{Deploy and provision.}
Vagrant supports complex provisions mechanisms, such as Chef or Puppet, but shell script was sufficient for us. 
This is set in the \emph{Vagranfile} \texttt{provision} field, one can describe the provision steps inline or in an external file and link it to the \emph{Vagranfile}.
Then, when the OS is booting, after the basic setup, the \glspl{os} will execute the provision script.
In this script, we program which software will be downloaded and run in the \glspl{vm} and what configurations are needed. 
We have developed shell scripts for each \glspl{os}, some of which share the same script as Debian and Ubuntu. 
Each \glspl{os}, especially the ones from different \glspl{os} families (e.g., Solaris and BSD) have different commands to execute the same instructions.
Although different, all scripts were meant to do the same thing:
First the script installs the software that is missing in the box -- this is not true for all OSes -- like Java 8, \texttt{wget}, \texttt{unzip}, and any additional software that one wants to install/run after the \glspl{os} boot.



\paragraph{Command and Control.}
There are some simple commands to boot and halt a box, e.g., \texttt{vagrant up} and \texttt{vagrant halt}. 
Vagrant also provides an ssh command (\texttt{vagrant ssh}) that allows the host to connect to the box. 
Our manager component makes the bridge between the \risk and the execution environment.
We developed an API on top of Vagrant, another level of abstraction made to manage replicated systems.


\circled{4} \textbf{LTUs.} Each node that hosts a replica has a Vagrant daemon (see details in next section) running on its trusted domain.
This component is isolated from the internet and communicates only with the \system controller through \gls{tls} channels.

\subsection{Additional Details}
\label{sec:details}

\textbf{Vulnerability Clustering.}
A few steps are carried out to create the vulnerability clusters. 
First, the vulnerability description needs to be transformed into a vector, where a numerical value is associated with the most relevant words (up to 200 words). 
This operation entails, for example, converting all words to a canonical form and calculating their frequency (less frequent words are given higher weights).
Then, the K-means algorithm is applied to build the clusters~\cite{Jain:2010}, where the number of clusters to be formed is determined by the elbow method~\cite{Thorndike:1953}. 
%Currently, it is set $K=200$.
%The algorithm assigns each vulnerability to the cluster with the minimum distance to the cluster center. 
%Then, it computes the cluster centroid, i.e., the average of each data attribute using only the members of a cluster. 
%Next, it calculates the distance of every vulnerability to the centroid, potentially placing them in close clusters. 
%The algorithm stops when there are no further exchanges between clusters.
%We used the elbow method~\cite{Thorndike53whobelongs} to determine the number of clusters to be formed, in our case was $200$ clusters.
We used the open-source machine learning library Weka~\cite{weka} to build the clusters. 


\subsection{Clustering}\label{sec:clustering}
Clustering is the process of aggregating elements into similar groups, named clusters. 
For example, two elements from the same cluster have a higher probability of being similar than two elements from different clusters. 
We apply this technique to build clusters of similar vulnerabilities.
One of the benefits of applying clustering techniques to vulnerability-data is that the algorithm does not need prior knowledge about the data.
It is the process alone that discovers the hidden knowledge in the data.
Each cluster is used as a hint that similar vulnerabilities are likely to be activated through the same or similar exploit.
We considered the vulnerability description and published date to find these similarities. 
In the end, it is expected that the clusters have a minimal number of elements that just represent the same or similar vulnerabilities.


In order to apply a clustering technique to our data, we need to follow some steps:

\paragraph{1) Data representation}
We transform the data that is stored in a database into a format that is readable by the clustering algorithm. 
Since we are using Weka~\cite{weka}, the data must be represented as \gls{arff}. 
Basically, this is a CSV file with some meta information, see Listing~\ref{list:arff}.

\begin{lstlisting}[style=mystyle,caption=ARFF file describing a vulnerability.,label=list:arff]
@RELATION vulnerabilities
@ATTRIBUTE cve string
@ATTRIBUTE description string
@ATTRIBUTE published_date date "yyyy-MM-dd"
@DATA
CVE-2017-3301, 'vulnerability in the solaris component of oracle sun systems products suite subcomponent kernel the supported version that is [...] attacks of this vulnerability can result in unauthorized update insert or delete access to some of solaris accessible data cvss v base score integrity impacts', '2017-01-27'
...more
\end{lstlisting}


This file contains all the vulnerabilities entries in the database. 
As we are interested in the vulnerability similarities, we select only the \gls{cve} identifier, the text, that contains the most relevant and nonstructured information, and the published date.
These attributes add meaning and temporal reference to the clustering algorithm.


\paragraph{2) Data preparation}
In general, machine learning algorithms do not handle raw data, then the data needs to be prepared:
First, we transform the \gls{cve} string into a number, basically, for each \gls{cve}, there is a real number that identifies each \gls{cve} unequivocally. 
This attribute is transformed in numeric values using \emph{StringToNominal} filter. 
Each \gls{cve} identifier is mapped to a numeric value.
Second, we transform the published date to a number format that Weka can handle using the \emph{NumericToNominal} filter.
Third, the text description must be transformed into a vector, the vector will represent the frequencies of each word in the whole document of strings. 
We used the \emph{StringToWordVector} to transform a text string into a vector of word weights, these weights represent the relevance of each word in the document.
This filter contains the following parameters:
\begin{itemize}
\item \textbf{TF- and IDF-Transform}, both set to true, TF-IDF stands for term frequency-inverse document frequency. 
This is a statistical measure used to evaluate how relevant a word is to a document in a collection. 
The relevance increases proportionally to the number of times a word appears in the document but is offset by the frequency of the word in the corpus. 
For example, words that appear in all vulnerability descriptions, are less relevant than the ones that appear more in few vulnerabilities.
\item \textbf{lowerCaseTokens}, convert all the words to lower case.
\item \textbf{minTermFreq}, set to -1, this will preserve any word despite their occurrences in the document.
\item \textbf{normalizeDocLength}, set to normalize all data.
\item \textbf{stopwords}, we define a stop word list, then words from this list are discarded. We begin with a general English stop word list containing pronouns, articles, etc. 
\item \textbf{tokenizer}, set to \emph{WordTokenizer}, this filter will remove special characters from the text, there is a default set of characters but we added a few more.
\item \textbf{wordsTopKeep}, is the number of words that will be kept to make the clusters. We kept $200$ as it was the number of words that represent better the lexical of vulnerabilities after removing the \emph{stopwords}.
\end{itemize}


Our goal is to build clusters in such way that similar vulnerabilities, even if they affect different products, are put together in the same cluster. 
For example, recall the vulnerabilities in Table~\ref{tab:missing_products}, which can be put in the same cluster since they are very similar.
Some of the parameters listed above needed some tuning to achieve our goals.
For example, before the tuning, some of the clusters were representing types of vulnerabilities, e.g., buffer overflow, cross-site scripting, etc. 
Since we are not interested in that type of clusters we have refined our \emph{stopword list} to reduce the description vocabulary to contain only what matters for us.
We have done this by iterating the process and checking the most used words (top 200 words) for clustering (this can be seen in Weka's intermediary output).
Then, we added the most relevant from the 200 words that were deviating from our goal.
In the end, we have a stop word list~\footnote{Stop word list is available: here} without the security-vocabulary noise.


When the pre-filtering ends, Weka presents a file very similar to the previous \gls{arff} file with the difference that the features are the most relevant words in the corpus. And each instance contains a value of relevance for each word.


\paragraph{3) Making clusters}
K-means is an unsupervised machine learning algorithm that groups data in K clusters.
The \emph{K-means} has two important parameters: the number of clusters to be formed, we set to $200$ clusters. We used the elbow method~\cite{Thorndike:1953} to decide $k=200$; and the \emph{distance function}, to calculate the distance between objects, the \emph{Euclidean Distance} is the most adequate for this type of data;
The \emph{K-means} computes the distance from each data entry to the cluster center (randomly selected in the first round).
Then, it assigns each data entry to a cluster based on the minimum distance (i.e., Euclidean distance) to each cluster center.
Then, it computes the centroid, that is the average of each data attribute using only the members of each cluster.
Calculate the distance from each data entry to the recent centroids. 
If there is no modification (i.e., re-arrangement of the elements in the cluster), then the clusters are complete, or it recalculates the distance that best fits the elements. 
When the K-means finishes the execution, we take the cluster assignments of each vulnerability. 
The assignments will be added to a new \gls{arff} file, similar to the first one but with a new attribute that is the cluster name (e.g., cluster1, cluster2, etc). 



Sometimes the resulting clusters include vulnerabilities that are unrelated. Therefore, we use the Jaccard index (J-index) to measure the similarity between the vulnerabilities within the cluster. We calculate the J-index of each element, and then the average J-index of the cluster. 
The clusters with a smaller average J-index (below a certain threshold) are considered ill-formed. In this case, we select the vulnerabilities with lower J-index and move them to another cluster that would result in a better J-index. If no such cluster exists, then we create a new one with the ``orphan'' vulnerability.

\textbf{BFT replication.}
Although there has been relevant research on \gls{bft} protocols over the last twenty years, there are few open-source replication libraries that implement them. \system can use any of those libraries, as long as they support replica set reconfigurations.
More specifically, to manage the \replicas, we need the ability to add first a new \replica to the set and then remove the old \replica to be quarantined. 
Therefore, we employ BFT-SMaRt~\cite{Bessani:2014}, a stable \gls{bft} library that provides reconfigurations on the \replicas set.

\textbf{Replica Virtualization.}
\glspl{vm} can be used to implement replicated systems, leveraging on the isolation between the untrusted and the trusted domains~\cite{Sousa:2010,Platania:2014,Distler:2011}.
Recovery triggering can be initiated from the isolated domain in a synchronous manner, reducing the downtime of the service during the reconfigurations. 
In our implementation, we resort to the Vagrant~\cite{vagrant} provisioning tool to do fast deployment of ready-to-use \glspl{os} and applications on \glspl{vm}. 
Vagrant supports several virtualization providers, e.g., VMware and Docker. 
From the available alternatives, we chose VirtualBox~\cite{virtualbox} because it offers more diversity opportunities, i.e., it supports a more extensive set of different guests \glspl{os}.
%The VMs are available in the Vagrant Cloud~\cite{vagrantcloud}.


\section{Evaluation of Replica Set Risk}
\label{sec:diversity}

This section evaluates how \system performs on the selection of dependable \replica configurations.
As discussed in Section~\ref{sec:replica}, we focus our experimental evaluation solely on the OS diversity.
% These play a crucial role in any IT system, and most of the \replica's code is the OS. 
% Thus, they present a high potential to become the most vulnerable part of a \replica.
% Hence, in the following experiments, we explore OS diversity among the replicas. 

In these experiments, we emulate live executions of the system by dividing the collected data into two periods:
(i) a \emph{learning phase} covering all vulnerability data between \emph{2010-1-1} and \emph{2017-9-29}, which is used to setup the \risk's algorithm; and (ii) an \emph{execution phase} composed of the period between \emph{2017-10-1} and \emph{2018-3-30}.
This last period is divided into three intervals of two months (OUT-NOV, DEC-JAN, and FEB-MAR), allowing for three independent tests.
%JAN-FEB, MAR-APR, and MAY-JUN
The goal is to create a knowledge base in the \emph{learning phase} that is used to assess \system choices during each interval of the \emph{execution phase}. 
A run starts on the first day of an interval and then progresses through each day of the interval until the end. Every day, we check if the currently executing replica set could be compromised by an attack exploring the vulnerabilities released on that day. 
We take the most pessimist approach, which is to say that we consider the system to be broken if a vulnerability comes out that affects at least two OSes that would be executing at that time.

Three additional strategies, inspired by previous works, were defined to be compared with \system (Section~\ref{sec:configurations}):

\begin{itemize}
\item \textbf{Equal:} all the replicas use the same randomly-selected OS during the whole execution. 
This strategy corresponds to the scenario where most past \gls{bft} systems have been implemented and evaluated (e.g.,~\cite{Kotla:2010,Aublin:2015,Behl:2015,Veronese:2013,Behl:2017,Liu:2016,Yin:2003,Amir:2011,Bessani:2014,Clement:2009b}). 
Here, compromising a replica would mean an opportunity to intrude the remaining ones.

\item \textbf{Static:} a configuration of $n$ different \glspl{os} is randomly selected, and there are no changes during the whole execution. 
This corresponds to a diverse \gls{bft} system without reconfigurations (e.g.,~\cite{Rodrigues:2001}).

\item \textbf{Random:} a configuration of $n$ \glspl{os} is randomly selected, and at the beginning of each day, a new \gls{os} is randomly picked to replace an existing one. 
This solution represents a system with proactive recovery and diversity, but with no informed strategy for choosing the next \configuration.

%\item \textbf{\system}, $n$ OSes are chosen based the algorithm described in Section~\ref{sec:measurerisk}. This algorithm decides when it is time to replace OSes, which OS is out and which OS is in.
\end{itemize}

The experiments consider a pool of 38 \gls{os} versions to be deployed on four replicas. 
At the beginning of the execution phase, the OSes are assumed to be fully patched.

%\begin{figure}[t]
%\begin{center}
%\includegraphics[width=\columnwidth]{figs/gnuplot/executions/execution.pdf}
%\caption{Compromised system runs over 2 month slots.}
%\label{fig:all_vulns}
%\end{center}
%\end{figure}



\subsection{Diversity vs Vulnerabilities}
We evaluate how each strategy can prevent the replicated system from being compromised. 
Each strategy is analyzed over $5000$ runs throughout the execution phase in two-month slots. 
Different runs are initiated with distinct random number generator seeds, resulting in potentially different \gls{os} selections over the time slot. 
On each day, we check if there is a vulnerability affecting more than one replica in the current \configuration, and in the affirmative case the execution is stopped.

\begin{figure}[h]
\begin{center}
\includegraphics[width=\columnwidth]{images/gnuplot/executions_new/execution.pdf}
\caption{Compromised system runs over 2 month slots.}
\label{fig:all_vulns}
\end{center}
\end{figure}

\textbf{Results:} Figure~\ref{fig:all_vulns} compares the percentage of compromised runs of all strategies. 
Each bar represents the percentage of runs that did not terminate successfully (lower is better). 
In all three periods, \system presents the best results. 
The \emph{Random} strategy performs worse because eventually, it picks a group of \glspl{os} with common vulnerabilities. 
This result provides evidence for the claim that \system improves the dependability, reducing the probability that $f+1$ \glspl{os} eventually become compromised. 
Interestingly, and contrary to intuition, changing \glspl{os} every day with no criteria will always create unsafe configurations.
Therefore, it is paramount to have selection strategies like the ones we use in \system.
\note{Add all the months we already have}

\subsection{Risk evaluation}


In order to better understand how \system performed, we isolated one of the $5000$ runs to observe the risk evolution over time. 
We picked the \emph{Random} and \system strategies for this analysis, with results displayed in Figure~\ref{fig:run_all}. 
The graphs present the evolution of the common vulnerabilities, the common clusters, and our risk metric for both schemes. 
Notice that two \glspl{os} might appear in the same cluster but with no mutual flaw as clusters can include many distinct vulnerabilities.

\begin{figure*}[h]
\subfigure[Random]{\includegraphics[width=0.5\columnwidth]{images/gnuplot/score/score_random_all.pdf}\label{fig:random_all}}
\hspace{0.5cm}
\subfigure[\system]{\includegraphics[width=0.5\columnwidth]{images/gnuplot/score/score_final_all.pdf}\label{fig:intel_all}}
\caption{Execution phase for Random and \system OS configuration strategies (log scale).}
\label{fig:run_all}
\end{figure*}

\textbf{Results:} As shown in Figure~\ref{fig:random_all}, \emph{Random} survives only for $10$ days. 
The number of shared clusters and vulnerabilities remains small for the first days. 
Then, there is a replica replacement that adds to the configuration an OS that has common vulnerabilities with the others. 
%Thus, enabling an adversary to compromise enough replicas in the system.

\system survives until the end of the experiment, as the risk is continually managed to keep the system safe. Figure~\ref{fig:intel_all} shows that shared clusters sometimes increase, at the same pace as the risk.
But then, the next reconfigurations are carried out with the goal of decreasing the risk. 
Notice that the risk value is always under $1$ for \system, and in the \emph{Random} is mostly above $10$.


\subsection{Diversity vs Attacks}

\begin{table}[t]
\begin{center}
{%\small%
\footnotesize
\begin{tabular}{ | p{0.96\columnwidth} | }\hline

\textbf{Samba:} 
\emph{On February 2, 2017, security researchers published details about a zero-day vulnerability in Server Message Block (SMB) of Windows, affecting several versions such as 8.1, 10, Server 2012 R2, and Server 2016. 
Could cause a \gls{dos} condition when a client accesses a malicious SMB.}\\
\textbf{CVES:} 
CVE-2017-0016
\\ \hline

\textbf{Wanna Cry:} 
\emph{On Friday, May 12, 2017, the world was alarmed to discover a widespread ransomware attack that hit organizations in more than 100 countries. Based on a vulnerability in Windows' SMB protocol (nicknamed EternalBlue), discovered by the NSA and leaked by Shadow Brokers.} \\
\textbf{CVES:} 
CVE-2017-0143, CVE-2017-0144, CVE-2017-0145, CVE-2017-0146, CVE-2017-0147, CVE-2017-0148 \\ \hline

\textbf{PowerShell:} 
\emph{Security feature bypass vulnerabilities in Device Guard that could allow an attacker to inject malicious code into a Windows PowerShell session.} \\
\textbf{CVES:}
CVE-2017-0219, CVE-2017-0173, CVE-2017-0215, CVE-2017-0216, CVE-2017-0218\\ \hline

\textbf{Stackclash:} 
\emph{In its 2017 malware forecast, SophosLabs warned that attackers would increasingly target Linux. The flaw, discovered by researchers at Qualys, is in the memory management of several operating systems and affects Linux, OpenBSD, NetBSD, FreeBSD and Solaris.}\\
\textbf{CVES:}
CVE-2017-1000365, CVE-2017-1000366, CVE-2017-1000367, CVE-2017-1000369, CVE-2017-1000370, CVE-2017-1000370, CVE-2017-1000371, CVE-2017-1000372, CVE-2017-1000373, CVE-2017-1000374, CVE-2017-1000375, CVE-2017-1000376, CVE-2017-1000379, CVE-2017-1083, CVE-2017-1084, CVE-2017-3629, CVE-2017-3630, CVE-2017-3631\\ \hline

\end{tabular}
}
\caption{Notable attacks during 2017.}
\label{tab:special_vulns}
\end{center}
\end{table}

This experiment evaluates the strategies when facing notable attacks/vulnerabilities that appeared in $2017$. 
Each attack potentially exploits several flaws, some of which affecting different \glspl{os}. 
The attacks were selected by searching the security news sites for high impact problems, most of them related to more than one CVE. 
As some of the \glspl{cve} include applications, we added more vulnerabilities to the database for this purpose.
Table~\ref{tab:special_vulns} lists the attacks and related \glspl{cve}: Samba,\footnote{https://www.secureworks.com/blog/attacking-windows-smb-zero-day-vulnerability} WannaCry,\footnote{https://securityintelligence.com/wannacry-ransomware-spreads-across-the-globe-makes-organizations-wanna-cry-about-microsoft-vulnerability/} Powershell,\footnote{http://blog.talosintelligence.com/2017/06/ms-tuesday.html} and Stackclash.\footnote{https://nakedsecurity.sophos.com/2017/06/20/stack-clash-linux-vulnerability-you-need-to-patch-now/}


Since some of these attacks might have been prepared months before the vulnerabilities are publicly disclosed, we augmented the execution phase to the full six months. 
As before, the strategies are analyzed over $5000$ runs.


\begin{figure}[t]
\begin{center}
\includegraphics[width=\columnwidth]{images/gnuplot/special_vulns/execution-special.pdf}
\caption{Compromised runs with notable attacks.}
\label{fig:special_vulns}
\end{center}
\end{figure}

\textbf{Results:}
Figure~\ref{fig:special_vulns} shows the percentage of compromised runs for each attack and all attacks put together.
\system is clearly the best at handling the various scenarios, with no compromised executions.
\emph{Random} is the worse, as it does not use any criteria to select the OSes. 
Both \emph{Equal} and \emph{Static} may perform not so bad as they are static, i.e., the \glspl{os} selected by random chance might end up not being exploitable until the end of the run.

\section{Final Remarks}
\label{sec:finalremarkslazarus}

\system addresses the long-standing open problem of evaluating, selecting, and managing the diversity of a \gls{bft} system to make it resilient to malicious adversaries.
Our work focuses on two fundamental issues: how to select the best replicas to run together given the current threat landscape, and what is the performance overhead of running a diverse \gls{bft} system in practice.


\chapter{Supporting Diversity Mechanisms}
\label{chap:supportingdiversitymechanism}
This chapter presents \system iplementation, these mechanisms were implemented to be fully automated, removing the human from the loop.
The current implementation of \system manages 17 \gls{os} versions, supporting the \gls{bft} replication of a set of representative applications.
The replicas run in \glspl{vm}, allowing provisioning mechanisms to configure them. 
We conducted two sets of experiments, one demonstrates that \system risk management can prevent a group of replicas from sharing vulnerabilities over time; the other, reveals the potential negative impact that virtualization and diversity can have on performance. However, we also show that if naive configurations are avoided, \gls{bft} applications in diverse configurations can actually perform close to our homogeneous bare metal setup.
%These results open avenues for many future works in the area. 


\section{\system Implementation}
\label{sec:implementation}

This section details the implementation of each component of \system. 
It also briefly presents other aspects of our prototype.% like the management of replicas running in a virtualized environment.


\subsection{Control Plane}
\label{sec:lazarus}

Figure~\ref{fig:arch1} shows \system control plane with its four main modules, described below.

\begin{figure}[h]
\begin{center}
\includegraphics[width=.9\columnwidth]{images/images/architecture_new.pdf}
\vspace{-5mm}
\caption{\system architecture.}
\vspace{-5mm}
\label{fig:arch1}
\end{center}
\end{figure}


\circled{1} \textbf{\fetcher.} \system needs to know the software stack of each available replica to be able to look for vulnerabilities in this software.
The list of software products is provided following the \gls{cpe} Dictionary~\cite{cpe}, which is also used by \gls{nvd}. 
%The \fetcher determines if the CPE list is updated by checking the most recent CPE count in the NVD web page.
For each software, the administrator can indicate the time interval (in years) during which data should be obtained from \gls{nvd}' feeds.

The \fetcher parses the \gls{nvd} feeds considering only the vulnerabilities that affect the chosen products. 
The processing is carried out with several threads cooperatively assemblying as much data as possible about each vulnerability -- a queue is populated with requests pertaining a particular vulnerability, and other threads will look for related data in additional \gls{osint} sources. 
Typically, the other sources are not as well structured as \gls{nvd}, and therefore they have to be handled with specialized HTML parsers that we have developed. 
Currently, the prototype supports five other sources, namely Exploit DB, CVE-details, Ubuntu, Debian, and Microsoft. 
As previously mentioned, such sources provide complementary data, like additional affected products versions not mentioned in \gls{nvd}.

The collected data is stored in a relational database (MySQL).
For each vulnerability we keep its \gls{cve} identifier, the published date, the products it affects, its text description, some \gls{cvss} attributes (e.g., availability and integrity); and exploit and patching dates.


\circled{2} \textbf{\risk.} This component finds out when it is necessary to replace the currently running group of \replicas and discovers an alternative configuration that decreases the risk. 
As explained in Section~\ref{sec:measurerisk}, the risk is computed using score values that require two kinds of data: the information about the vulnerabilities, which is collected by the \fetcher; and the vulnerability clusters. 
A vulnerability cluster is a set of vulnerabilities that are related accordingly to their description (see Section~\ref{sec:details} for details).
The \risk also runs Algorithm~\ref{alg:algorithm2} to monitor the replicated system and trigger reconfigurations.


\circled{3} \textbf{\manager.} 
This component automates the setup and execution of the diverse replicas. 
It creates and deploys the \replicas in the execution environment implementing the decisions of the \risk, i.e., it dictates when and which \replicas leave and join the system. 
We developed a replica builder on top of Vagrant~\cite{vagrant}.
It is responsible for downloading, installing, and configuring the \replicas.
Moreover, it performs replica maintenance, where  \replicas in \QS are booted to carry out automatic software updates (i.e., patching). 


\paragraph{Setup.}
The box configuration is defined in a configuration file, named \emph{Vagrantfile}, consider the example in Listing~\ref{vagrantfile}.
In this file, it is possible to set several options of the box, to name a few:
the box name, the number of CPUs, the amount of memory RAM, the IP, the type of network, sync folders between host and the box, etc. 
Additionally, it is possible to pass some VM-specific parameters, e.g., some CPU/mother board flags such as enabling VT-x technology -- consider the \texttt{modifyvm} fields in Listing~\ref{vagrantfile} (lines 6-14).
It is possible to select different \gls{vm} providers (e.g., libvirt, VMware, VirtualBox, Parallels, Docker, etc), we rely on VirtualBox since it is the one with more diversity opportunities in the Vagrant Cloud~\cite{vagrantcloud}. 
Vagrant Cloud is a website that offers a plethora of different VMs.

\begin{lstlisting}[style=mystyle,caption=Windows Server 2016 Vagrantfile,label=vagrantfile]
Vagrant.configure(2) do |config|
	config.vm.box = "geerlingguy/ubuntu1604"
	config.ssh.insert_key = false
	config.vm.provider "virtualbox" do |v|
		v.customize ["modifyvm", :id, "--cpus", 4]
		v.customize ["modifyvm", :id, "--memory", 22000]
		v.customize ["modifyvm", :id, "--cpuexecutioncap", 100]
		v.customize ["modifyvm", :id, "--ioapic", "on"]
		v.customize ["modifyvm", :id, "--hwvirtex", "on"]
		v.customize ["modifyvm", :id, "--nestedpaging", "on"]
		v.customize ["modifyvm", :id, "--pae", "on"]
		v.customize ["modifyvm", :id, "--natdnshostresolver1", "on"]
		v.customize ["modifyvm", :id, "--natdnsproxy1", "on"]
	end
	config.vm.network "public_network", ip:"192.168.2.50", bridge:"em1"
	config.vm.provision :shell, path: "run_debian.sh", privileged: true
end
\end{lstlisting}

\todo{Add all fields that are mentioned, then add lines to the text}

\paragraph{Download.}
One of the fields of the \emph{Vagranfile} is the boxname, the \texttt{box} field is the key that is used to choose which box will be downloaded, each key is unique for each box.
There are plenty of \glspl{os} and versions ready-to-use, and the same OS/version can have different ``manufacturers" -- some of which are official.

\paragraph{Deploy and provision.}
Vagrant supports complex provisions mechanisms, such as Chef or Puppet, but shell script was sufficient for us. 
This is set in the \emph{Vagranfile} \texttt{provision} field, one can describe the provision steps inline or in an external file and link it to the \emph{Vagranfile}.
Then, when the OS is booting, after the basic setup, the \glspl{os} will execute the provision script.
In this script, we program which software will be downloaded and run in the \glspl{vm} and what configurations are needed. 
We have developed shell scripts for each \glspl{os}, some of which share the same script as Debian and Ubuntu. 
Each \glspl{os}, especially the ones from different \glspl{os} families (e.g., Solaris and BSD) have different commands to execute the same instructions.
Although different, all scripts were meant to do the same thing:
First the script installs the software that is missing in the box -- this is not true for all OSes -- like Java 8, \texttt{wget}, \texttt{unzip}, and any additional software that one wants to install/run after the \glspl{os} boot.



\paragraph{Command and Control.}
There are some simple commands to boot and halt a box, e.g., \texttt{vagrant up} and \texttt{vagrant halt}. 
Vagrant also provides an ssh command (\texttt{vagrant ssh}) that allows the host to connect to the box. 
Our manager component makes the bridge between the \risk and the execution environment.
We developed an API on top of Vagrant, another level of abstraction made to manage replicated systems.


\circled{4} \textbf{LTUs.} Each node that hosts a replica has a Vagrant daemon (see details in next section) running on its trusted domain.
This component is isolated from the internet and communicates only with the \system controller through \gls{tls} channels.

\subsection{Additional Details}
\label{sec:details}

\textbf{Vulnerability Clustering.}
A few steps are carried out to create the vulnerability clusters. 
First, the vulnerability description needs to be transformed into a vector, where a numerical value is associated with the most relevant words (up to 200 words). 
This operation entails, for example, converting all words to a canonical form and calculating their frequency (less frequent words are given higher weights).
Then, the K-means algorithm is applied to build the clusters~\cite{Jain:2010}, where the number of clusters to be formed is determined by the elbow method~\cite{Thorndike:1953}. 
%Currently, it is set $K=200$.
%The algorithm assigns each vulnerability to the cluster with the minimum distance to the cluster center. 
%Then, it computes the cluster centroid, i.e., the average of each data attribute using only the members of a cluster. 
%Next, it calculates the distance of every vulnerability to the centroid, potentially placing them in close clusters. 
%The algorithm stops when there are no further exchanges between clusters.
%We used the elbow method~\cite{Thorndike53whobelongs} to determine the number of clusters to be formed, in our case was $200$ clusters.
We used the open-source machine learning library Weka~\cite{weka} to build the clusters. 


\subsection{Clustering}\label{sec:clustering}
Clustering is the process of aggregating elements into similar groups, named clusters. 
For example, two elements from the same cluster have a higher probability of being similar than two elements from different clusters. 
We apply this technique to build clusters of similar vulnerabilities.
One of the benefits of applying clustering techniques to vulnerability-data is that the algorithm does not need prior knowledge about the data.
It is the process alone that discovers the hidden knowledge in the data.
Each cluster is used as a hint that similar vulnerabilities are likely to be activated through the same or similar exploit.
We considered the vulnerability description and published date to find these similarities. 
In the end, it is expected that the clusters have a minimal number of elements that just represent the same or similar vulnerabilities.


In order to apply a clustering technique to our data, we need to follow some steps:

\paragraph{1) Data representation}
We transform the data that is stored in a database into a format that is readable by the clustering algorithm. 
Since we are using Weka~\cite{weka}, the data must be represented as \gls{arff}. 
Basically, this is a CSV file with some meta information, see Listing~\ref{list:arff}.

\begin{lstlisting}[style=mystyle,caption=ARFF file describing a vulnerability.,label=list:arff]
@RELATION vulnerabilities
@ATTRIBUTE cve string
@ATTRIBUTE description string
@ATTRIBUTE published_date date "yyyy-MM-dd"
@DATA
CVE-2017-3301, 'vulnerability in the solaris component of oracle sun systems products suite subcomponent kernel the supported version that is [...] attacks of this vulnerability can result in unauthorized update insert or delete access to some of solaris accessible data cvss v base score integrity impacts', '2017-01-27'
...more
\end{lstlisting}


This file contains all the vulnerabilities entries in the database. 
As we are interested in the vulnerability similarities, we select only the \gls{cve} identifier, the text, that contains the most relevant and nonstructured information, and the published date.
These attributes add meaning and temporal reference to the clustering algorithm.


\paragraph{2) Data preparation}
In general, machine learning algorithms do not handle raw data, then the data needs to be prepared:
First, we transform the \gls{cve} string into a number, basically, for each \gls{cve}, there is a real number that identifies each \gls{cve} unequivocally. 
This attribute is transformed in numeric values using \emph{StringToNominal} filter. 
Each \gls{cve} identifier is mapped to a numeric value.
Second, we transform the published date to a number format that Weka can handle using the \emph{NumericToNominal} filter.
Third, the text description must be transformed into a vector, the vector will represent the frequencies of each word in the whole document of strings. 
We used the \emph{StringToWordVector} to transform a text string into a vector of word weights, these weights represent the relevance of each word in the document.
This filter contains the following parameters:
\begin{itemize}
\item \textbf{TF- and IDF-Transform}, both set to true, TF-IDF stands for term frequency-inverse document frequency. 
This is a statistical measure used to evaluate how relevant a word is to a document in a collection. 
The relevance increases proportionally to the number of times a word appears in the document but is offset by the frequency of the word in the corpus. 
For example, words that appear in all vulnerability descriptions, are less relevant than the ones that appear more in few vulnerabilities.
\item \textbf{lowerCaseTokens}, convert all the words to lower case.
\item \textbf{minTermFreq}, set to -1, this will preserve any word despite their occurrences in the document.
\item \textbf{normalizeDocLength}, set to normalize all data.
\item \textbf{stopwords}, we define a stop word list, then words from this list are discarded. We begin with a general English stop word list containing pronouns, articles, etc. 
\item \textbf{tokenizer}, set to \emph{WordTokenizer}, this filter will remove special characters from the text, there is a default set of characters but we added a few more.
\item \textbf{wordsTopKeep}, is the number of words that will be kept to make the clusters. We kept $200$ as it was the number of words that represent better the lexical of vulnerabilities after removing the \emph{stopwords}.
\end{itemize}


Our goal is to build clusters in such way that similar vulnerabilities, even if they affect different products, are put together in the same cluster. 
For example, recall the vulnerabilities in Table~\ref{tab:missing_products}, which can be put in the same cluster since they are very similar.
Some of the parameters listed above needed some tuning to achieve our goals.
For example, before the tuning, some of the clusters were representing types of vulnerabilities, e.g., buffer overflow, cross-site scripting, etc. 
Since we are not interested in that type of clusters we have refined our \emph{stopword list} to reduce the description vocabulary to contain only what matters for us.
We have done this by iterating the process and checking the most used words (top 200 words) for clustering (this can be seen in Weka's intermediary output).
Then, we added the most relevant from the 200 words that were deviating from our goal.
In the end, we have a stop word list~\footnote{Stop word list is available: here} without the security-vocabulary noise.


When the pre-filtering ends, Weka presents a file very similar to the previous \gls{arff} file with the difference that the features are the most relevant words in the corpus. And each instance contains a value of relevance for each word.


\paragraph{3) Making clusters}
K-means is an unsupervised machine learning algorithm that groups data in K clusters.
The \emph{K-means} has two important parameters: the number of clusters to be formed, we set to $200$ clusters. We used the elbow method~\cite{Thorndike:1953} to decide $k=200$; and the \emph{distance function}, to calculate the distance between objects, the \emph{Euclidean Distance} is the most adequate for this type of data;
The \emph{K-means} computes the distance from each data entry to the cluster center (randomly selected in the first round).
Then, it assigns each data entry to a cluster based on the minimum distance (i.e., Euclidean distance) to each cluster center.
Then, it computes the centroid, that is the average of each data attribute using only the members of each cluster.
Calculate the distance from each data entry to the recent centroids. 
If there is no modification (i.e., re-arrangement of the elements in the cluster), then the clusters are complete, or it recalculates the distance that best fits the elements. 
When the K-means finishes the execution, we take the cluster assignments of each vulnerability. 
The assignments will be added to a new \gls{arff} file, similar to the first one but with a new attribute that is the cluster name (e.g., cluster1, cluster2, etc). 



Sometimes the resulting clusters include vulnerabilities that are unrelated. Therefore, we use the Jaccard index (J-index) to measure the similarity between the vulnerabilities within the cluster. We calculate the J-index of each element, and then the average J-index of the cluster. 
The clusters with a smaller average J-index (below a certain threshold) are considered ill-formed. In this case, we select the vulnerabilities with lower J-index and move them to another cluster that would result in a better J-index. If no such cluster exists, then we create a new one with the ``orphan'' vulnerability.

\textbf{BFT replication.}
Although there has been relevant research on \gls{bft} protocols over the last twenty years, there are few open-source replication libraries that implement them. \system can use any of those libraries, as long as they support replica set reconfigurations.
More specifically, to manage the \replicas, we need the ability to add first a new \replica to the set and then remove the old \replica to be quarantined. 
Therefore, we employ BFT-SMaRt~\cite{Bessani:2014}, a stable \gls{bft} library that provides reconfigurations on the \replicas set.

\textbf{Replica Virtualization.}
\glspl{vm} can be used to implement replicated systems, leveraging on the isolation between the untrusted and the trusted domains~\cite{Sousa:2010,Platania:2014,Distler:2011}.
Recovery triggering can be initiated from the isolated domain in a synchronous manner, reducing the downtime of the service during the reconfigurations. 
In our implementation, we resort to the Vagrant~\cite{vagrant} provisioning tool to do fast deployment of ready-to-use \glspl{os} and applications on \glspl{vm}. 
Vagrant supports several virtualization providers, e.g., VMware and Docker. 
From the available alternatives, we chose VirtualBox~\cite{virtualbox} because it offers more diversity opportunities, i.e., it supports a more extensive set of different guests \glspl{os}.
%The VMs are available in the Vagrant Cloud~\cite{vagrantcloud}.
\section{Performance Evaluation}
\label{sec:overhead}

In this section, we evaluate how \system' affects the performance of diverse replicated systems.
First, we run the BFT-SMaRt microbenchmarks in our virtualized environment using 17 \gls{os} versions to understand how the performance of a \gls{bft} protocol varies with different \glspl{os}, and how they compare with the performance of a homogeneous bare metal setup.
Second, we use the same benchmarks to measure the performance of specific diverse setups.
Third, we analyze the performance of the system along \system-managed reconfigurations.
Fourth, we measure the time that each OS takes to boot.
Finally, we evaluate the performance of three \gls{bft} services running in the \system infrastructure.


These experiments were conducted in a cluster of Dell PowerEdge R410 machines, where each one has 32 GB of memory and two quad-core 2.27 GHz Intel Xeon E5520 processor with hyper-threading, i.e., supporting 16 hardware threads on each node.
The machines communicate through a gigabit Ethernet network.
Each server runs Ubuntu Linux 14.04 LTS (3.13.0-32-generic Kernel) and VirtualBox 5.1.28, for supporting the execution of VMs with different OSes. 
Additionally, Vagrant 2.0.0 was used as the provisioning tool to automate the deployment process.
In all experiments, we configure BFT-SMaRt v1.1 with four replicas ($f=1$), one replica per physical machine.

Table~\ref{tab:oses} lists the 17 \gls{os} versions used in the experiments and the number of cores used by their corresponding \glspl{vm}.
These values correspond to the maximum number of CPUs supported by VirtualBox with that particular \gls{os}.
The table also shows the \gls{jvm} used in each \gls{os}, and the amount of memory supported by each of these \glspl{vm}.
Given these limitations we setup our environment to establish a fair baseline by configuring an \emph{homogeneous} \gls{bm} environment that uses only four cores of the physical machine. 

%Given the limitations of the VMs we were able to setup in our environment, all experiments conducted in our \emph{homogeneous bare metal environment} (BM) were configured to make BFT-SMaRt replicas use only four cores of the machines, to establish a fair baseline.


\begin{table}[t]
\begin{center}
{\footnotesize
\begin{tabular}{| c | c | c | c | c |}\hline
\textbf{ID} & \textbf{Name}  & \textbf{Cores} & \textbf{JVM} & \textbf{Mem.} \\\hline\hline
UB14 & Ubuntu 14.04 & 4 & Java Oracle 1.8.0\_144 & 15GB \\ \hline
UB16 & Ubuntu 16.04 & 4 & Java Oracle 1.8.0\_144 & 15GB \\ \hline
UB17 & Ubuntu 17.04 & 4 & Java Oracle 1.8.0\_144 & 15GB \\ \hline
OS42 & OpenSuse 42.1 & 4 & Openjdk 1.8.0\_141 & 15GB \\ \hline
FE24 & Fedora 24 & 4 & Openjdk 1.8.0\_141 & 15GB \\ \hline
FE25 & Fedora 25 & 4 & Openjdk 1.8.0\_141 & 15GB \\ \hline
FE26 & Fedora 26 & 4 & Openjdk 1.8.0\_141 & 15GB \\ \hline
DE7 & Debian 7 & 4 & Java Oracle 1.8.0\_151 & 15GB \\ \hline
DE8 & Debian 8 & 4 & Openjdk 1.8.0\_131 & 15GB \\ \hline
W10 & Windows 10 & 4 & Java Oracle 1.8.0\_151 &1GB \\ \hline
WS12 & Win. Server 2012 & 4 & Java Oracle 1.8.0\_151 & 1GB \\ \hline
FB10 & FreeBSD 10 & 4 & Openjdk 1.8.0\_144 & 15GB \\ \hline
FB11 & FreeBSD 11 & 4 & Openjdk 1.8.0\_144 & 15GB \\ \hline
SO10 & Solaris 10 & 1 & Java Oracle 1.8.0\_141 & 15GB \\ \hline
SO11 & Solaris 11 & 1 & Java Oracle 1.8.0\_05 & 15GB \\ \hline
OB60 & OpenBSD 6.0 & 1 & Openjdk 1.8.0\_72 & 1GB \\ \hline
OB61 & OpenBSD 6.1 & 1 & Openjdk 1.8.0\_121 & 1GB \\ \hline
\end{tabular}
}
\caption{The different OSes used in the experiments and the configurations of their VMs and JVMs.}
\label{tab:oses}
\end{center}
\end{table}


\subsection{Homogeneous Replicas Throughput}


We start by running the BFT-SMaRt microbenchmark using the \emph{same \gls{os} version} in all replicas.
The microbenchmark considers an empty service that receives and replies variable size payloads, and is commonly used to evaluate \gls{bft} state-machine replication protocols (e.g., \cite{Castro:1999,Bessani:2014,Liu:2016,Behl:2015,Behl:2017}). 
Here, we consider the $0/0$ and $1024/1024$ workloads, i.e., $0$ and $1024$ bytes requests/response, respectively.
The experiments employ up to 1400 client processes spread on seven machines to create the workload.
% and the throughput was measured in the BFT-SMaRt leader replica.

\textbf{Results:}
Figure~\ref{fig:bftsmart} shows the throughput of each \gls{os} running the benchmark for both loads.
To establish a baseline, we executed the benchmark in our bare metal Ubuntu, without \system virtualization environment.

The results show that there are some significant differences between running the system on top of different \glspl{os}.
This difference is more significant for the $0/0$ workload as it is much more CPU intensive than the $1024/1024$ workload.
Ubuntu, OpenSuse, and Fedora OSes are well supported by our virtualization environment and achieved a throughput around $40k$ and $10k$ for the $0/0$ and $1024/1024$ workloads, which corresponds to approx. $66\%$ and $75\%$ of the bare metal results, respectively.
For Debian, Windows, and FreeBSD, the results are much worse for the CPU intensive $0/0$ workloads but close to the previous group for $1024/1024$.
Finally, single core VMs running Solaris and OpenBSD reached no more than $3000$ ops/sec with both workloads.

These results show that the virtualization platform limitations on supporting different \glspl{os}, strongly limits the performance of specific \glspl{os} in our testbed.

\begin{figure}[h]
\begin{center}
\includegraphics[width=\columnwidth]{images/gnuplot/vagrant/runs_new_new/throughput.pdf}
\caption{Microbenchmark for 0/0 and 1024/1024 (request/replying) for homogeneous OSes configurations.}
\label{fig:bftsmart}
\end{center}
\end{figure}

\subsection{Diverse Replicas Throughput}
\label{sec:performancediversity}

The previous results show the performance of BFT-SMaRt when running on top of different \glspl{os}, but with all replicas running in the same environment.
In this experiment, we evaluate three diverse sets of four replicas, one with the fastest \glspl{os} (UB17, UB16, FE24, and OS42), another with one replica of each \gls{os} family (UB16, W10, SO10, and OB61), and a last one with the slowest \glspl{os} (OB60, OB61, SO10, and SO11).
The idea is to set an upper and lower bound on all possible diverse sets throughput.

\begin{figure}[h]
\begin{center}
\includegraphics[width=\columnwidth]{images/gnuplot/vagrant/runs_diversity/throughput.pdf}
\caption{Microbenchmark for 0/0 and 1024/1024 (request/reply) for three diverse OS configurations.}
\label{fig:diversets}
\end{center}
\end{figure}


\textbf{Results:}
Figure~\ref{fig:diversets} shows that throughput drops from 39k to 6k for the $0/0$ workload ($65\%$ and $10\%$ of the bare metal performance), and from 11.5k to 2.5k for the $1024/1024$ workload ($82\%$ and $18\%$ of the bare metal performance).
When comparing these two sets with the non-diverse sets of Figure~\ref{fig:bftsmart}, the fastest set is in $7^{th}$, and the slowest set is in $16^{th}$.
It is worth to stress that the slowest set is composed of OSes that only support a single CPU -- due to the VirtualBox limitations -- therefore the low performance is somewhat expected.
The set with \glspl{os} from different families is very close to the slowest set, as two of the replicas use single-CPU \glspl{os}, and BFT-SMaRt always makes progress in the speed of the 3rd fastest replica (a Solaris \gls{vm}), since its Byzantine quorum needs three replicas for ordering requests.
These results show that running \system with current virtualization technology results in a significant performance variation, depending on the configurations selected by the system.
This opens interesting avenues for future work on protocols that consider such performance diversity, as will be discussed in Section~\ref{sec:discussion}.

\subsection{Performance During Reconfiguration}
\label{sec:reconfiguration}



In this experiment, we show how \system-triggered reconfigurations affect the replicas' performance.
Reconfigurations, in this case, subsumes to the addition of a new replica and the removal of the old one, using the BFT-SMaRt reconfiguration protocols~\cite{Bessani2014}. 
We execute this experiment with an in-memory \gls{kvs} service that comes with the BFT-SMaRT library.
%This service mimics the basic ``BFT Zookeeper'' application benchmark shown in recent BFT papers~\cite{Liu16,Behl:2017}.

The experiment was conducted with a \gls{ycsb}~\cite{Cooper:2010} workload of $50\%$ of reads and $50\%$ of writes, with values of 1024 bytes associated with small numeric keys, varying the state size of the \gls{kvs}.
We run this experiment for $400$ seconds, and reconfigure the set of replicas with a period of $100$ seconds. 
The initial \gls{os} configuration is the fastest OS set (i.e., UB17, UB16, FE24, OS42).
\begin{figure}[h]
\subfigure[KVS state size $\approx 30$MBs.]{\includegraphics[width=\textwidth]{images/gnuplot/vagrant/reconfiguration/reconfiguration_state10mbs.pdf}\label{fig:s10mbs}}
\subfigure[KVS state size $\approx 60$MBs.]{\includegraphics[width=\textwidth]{images/gnuplot/vagrant/reconfiguration/reconfiguration_state50mbs.pdf}\label{fig:s50mbs}}
\caption{KVS performance with \system-triggered reconfigurations on a 50/50 YCSB workload and 1kB-values.}
\label{fig:reconfiguration}
\end{figure}



\textbf{Results:}
Figure~\ref{fig:reconfiguration} shows two types of performance drops.
The first type happens at every 1000 writes and is due to the state checkpoints used to trim the operation logs.
The second type is more severe and less frequent, and happens during reconfigurations, mostly due to the state transfer to the new replica joining the system.
These types of performance perturbations, which become more severe with bigger states, were already identified, and mitigated in previous works~\cite{Bessani:2013}.\footnote{We employ the standard checkpoint and state transfer protocols of BFT-SMaRt and not the one introduced in~\cite{Bessani:2013} as the developers of the system pointed them as more stable.} 

The figure shows that the reconfigurations take around $10$ seconds for these setups, and did not change significantly with states of approx. $30$ and $60$ MBs.
Regarding diversity, the figure shows no noticeable disruption when changing Ubuntu 17 to Fedora 26 and Fedora 25 to Fedora 24, but the replacement of a Debian 8 by an OpenSuse 42 replica significantly decreases the performance, which is consistent with the results from Figure~\ref{fig:bftsmart}.

\subsection{Booting time}

Before triggering a reconfiguration, \system needs to start the new \gls{vm}, boot the \glspl{os}, and launch the BFT-SMaRt replica.
Therefore, the total time to replace a ``risky'' replica roughly comprises the time required to boot the new OS plus the reconfiguration time (discussed in the previous section).
This experiment measures the boot time of the \glspl{os} supported by \system prototype. 

\begin{figure}[h]
\begin{center}
\includegraphics[width=.8\columnwidth]{images/gnuplot/vagrant/updown/boot.pdf}
\vspace{-5mm}
\caption{OSes boot times (seconds).}
\label{fig:boot}
\end{center}
\end{figure}

\textbf{Results:}
Figure~\ref{fig:boot} shows the average boot time of $20$ executions boot for each \gls{os}.
As can be seen, most \glspl{os} take less than 40s to boot, with exceptions of Fedora 24 (50s), FreeBSD 10.3 (59s), and Solaris 11.2 (91s).
In any case, these results together with the reconfiguration times of the previous section show that \system can react to a new threat in less than two minutes (in the worst case).

\subsection{Application Benchmarks}
Our last set of experiments aims to measure the throughput of three existing \gls{bft} services built on top of BFT-SMaRt when running in \system.
The considered applications and workloads are:

\begin{itemize}

\item \emph{KVS} is the same BFT-SMaRt application employed in Section~\ref{sec:reconfiguration}.
It represents a consistent non-relational database that stores data in memory, similarly to a coordination service (an evaluation scenario used in many recent papers on \gls{bft}~\cite{Liu:2016,Behl:2017}).
In this evaluation, we employ the \gls{ycsb} $50\%/50\%$ read/write workload with values of 4k bytes.

\item \sieveq~\cite{Garcia:2016} is a \gls{bft} message queue service that can also be used as an application-level firewall, filtering messages that do not comply with a pre-configured security policy.
Its architecture, based on several filtering layers, reduces the costs of filtering invalid messages in a BFT-replicated state machine.
In our evaluation, we consider all the layers were running on the same four physical machines as the diverse BFT-SMaRt replicas (under different OSes).
The workload imposed to the system is composed of messages of 1k bytes.

\item \emph{BFT ordering for Hyperledger Fabric}~\cite{Sousa:2018} is the first \gls{bft} ordering service for Fabric~\cite{Androulaki:2018}. 
Fabric is an extensible blockchain platform designed for business applications beyond the basic digital coin.
The ordering service is the core of Fabric, being responsible for ordering and grouping issued transactions in signed blocks that form the blockchain.
In our evaluation, we consider transactions of 1k bytes, blocks of $10$ transactions and a single block receiver.

\end{itemize}

As in Section~\ref{sec:performancediversity}, we run the applications on the fastest and slowest diverse replica sets and compare them with the results obtained in bare metal.

\textbf{Results:}
Figure~\ref{fig:apps} shows the peak sustained throughput of the applications. 
The \gls{kvs} results show a throughput of 6.1k and 1.2k ops/sec, for the fastest and slowest configurations, respectively.
This corresponds to $86\%$ and $18\%$ of the 7.1k ops/sec achieved on bare metal.

The \sieveq results show a smaller performance loss when compared with bare metal results.
More specifically, \sieveq in the fastest replica set reaches $94\%$ of the throughput achieved on the bare metal.
Even with the slowest set, the system achieved $53\%$ of the throughput of bare metal.
This smaller loss happens due to the layered architecture of \sieveq, in which most of the message validations happen before the message reaches the \gls{bft} replicated state machine (which is the only layer managed by \system).

The Fabric ordering service results show that running the application on \system virtualization infrastructure lead to $91\%$ (fastest set) to $39\%$ (slowest set) of the throughput achieved on bare metal. 

Nonetheless, even the slowest configurations would not be a significant bottleneck if one takes into consideration the current performance of Fabric~\cite{Sousa:2018}
%Overall, the relatively poor results for the slowest set are due to the single-core and low-memory setups of their replicas (see Table~\ref{tab:oses}).

\begin{figure}[t]
\begin{center}
\includegraphics[width=\columnwidth]{images/gnuplot/vagrant/runs_apps/throughput.pdf}
\vspace{-5mm}
\caption{Different BFT applications running in the bare metal, fastest and slowest OS configurations.}
\vspace{-3mm}
\label{fig:apps}
\end{center}
\end{figure}



\section{Final Remarks}
\label{sec:finalremarkslazarus}

\system addresses the long-standing open problem of evaluating, selecting, and managing the diversity of a \gls{bft} system to make it resilient to malicious adversaries.
Our work focuses on two fundamental issues: how to select the best replicas to run together given the current threat landscape, and what is the performance overhead of running a diverse \gls{bft} system in practice.


\chapter{Untrusted (distributed) controller}
\label{chap:controler}

As in previous works~\cite{Roeder:2010,Platania:2014}, our current design for \sieveq and \system considers a centralized trusted control plane that analyze \gls{osint} and orchestrate replica group reconfigurations.
It would be desirable to have a distributed version of such control plane, not only for improving its dependability, but also to support the existence of multi-domain applications, such as blockchain platforms.


%\chapter{\sieveq Evaluation}
\label{chap:sieveqevaluation}




%\chapter{\system Evaluation}
\label{chap:lazarusevaluation}


\section{Evaluation of Replica Set Risk}
\label{sec:diversity}

This section evaluates how \system performs on the selection of dependable \replica configurations.
As discussed in Section~\ref{sec:replica}, we focus our experimental evaluation solely on the OS diversity.
% These play a crucial role in any IT system, and most of the \replica's code is the OS. 
% Thus, they present a high potential to become the most vulnerable part of a \replica.
% Hence, in the following experiments, we explore OS diversity among the replicas. 

In these experiments, we emulate live executions of the system by dividing the collected data into two periods:
(i) a \emph{learning phase} covering all vulnerability data between \emph{2010-1-1} and \emph{2017-9-29}, which is used to setup the \risk's algorithm; and (ii) an \emph{execution phase} composed of the period between \emph{2017-10-1} and \emph{2018-3-30}.
This last period is divided into three intervals of two months (OUT-NOV, DEC-JAN, and FEB-MAR), allowing for three independent tests.
%JAN-FEB, MAR-APR, and MAY-JUN
The goal is to create a knowledge base in the \emph{learning phase} that is used to assess \system choices during each interval of the \emph{execution phase}. 
A run starts on the first day of an interval and then progresses through each day of the interval until the end. Every day, we check if the currently executing replica set could be compromised by an attack exploring the vulnerabilities released on that day. 
We take the most pessimist approach, which is to say that we consider the system to be broken if a vulnerability comes out that affects at least two OSes that would be executing at that time.

Three additional strategies, inspired by previous works, were defined to be compared with \system (Section~\ref{sec:configurations}):

\begin{itemize}
\item \textbf{Equal:} all the replicas use the same randomly-selected OS during the whole execution. 
This strategy corresponds to the scenario where most past \gls{bft} systems have been implemented and evaluated (e.g.,~\cite{Kotla:2010,Aublin:2015,Behl:2015,Veronese:2013,Behl:2017,Liu:2016,Yin:2003,Amir:2011,Bessani:2014,Clement:2009b}). 
Here, compromising a replica would mean an opportunity to intrude the remaining ones.

\item \textbf{Static:} a configuration of $n$ different \glspl{os} is randomly selected, and there are no changes during the whole execution. 
This corresponds to a diverse \gls{bft} system without reconfigurations (e.g.,~\cite{Rodrigues:2001}).

\item \textbf{Random:} a configuration of $n$ \glspl{os} is randomly selected, and at the beginning of each day, a new \gls{os} is randomly picked to replace an existing one. 
This solution represents a system with proactive recovery and diversity, but with no informed strategy for choosing the next \configuration.

%\item \textbf{\system}, $n$ OSes are chosen based the algorithm described in Section~\ref{sec:measurerisk}. This algorithm decides when it is time to replace OSes, which OS is out and which OS is in.
\end{itemize}

The experiments consider a pool of 38 \gls{os} versions to be deployed on four replicas. 
At the beginning of the execution phase, the OSes are assumed to be fully patched.

%\begin{figure}[t]
%\begin{center}
%\includegraphics[width=\columnwidth]{figs/gnuplot/executions/execution.pdf}
%\caption{Compromised system runs over 2 month slots.}
%\label{fig:all_vulns}
%\end{center}
%\end{figure}



\subsection{Diversity vs Vulnerabilities}
We evaluate how each strategy can prevent the replicated system from being compromised. 
Each strategy is analyzed over $5000$ runs throughout the execution phase in two-month slots. 
Different runs are initiated with distinct random number generator seeds, resulting in potentially different \gls{os} selections over the time slot. 
On each day, we check if there is a vulnerability affecting more than one replica in the current \configuration, and in the affirmative case the execution is stopped.

\begin{figure}[h]
\begin{center}
\includegraphics[width=\columnwidth]{images/gnuplot/executions_new/execution.pdf}
\caption{Compromised system runs over 2 month slots.}
\label{fig:all_vulns}
\end{center}
\end{figure}

\textbf{Results:} Figure~\ref{fig:all_vulns} compares the percentage of compromised runs of all strategies. 
Each bar represents the percentage of runs that did not terminate successfully (lower is better). 
In all three periods, \system presents the best results. 
The \emph{Random} strategy performs worse because eventually, it picks a group of \glspl{os} with common vulnerabilities. 
This result provides evidence for the claim that \system improves the dependability, reducing the probability that $f+1$ \glspl{os} eventually become compromised. 
Interestingly, and contrary to intuition, changing \glspl{os} every day with no criteria will always create unsafe configurations.
Therefore, it is paramount to have selection strategies like the ones we use in \system.
\note{Add all the months we already have}

\subsection{Risk evaluation}


In order to better understand how \system performed, we isolated one of the $5000$ runs to observe the risk evolution over time. 
We picked the \emph{Random} and \system strategies for this analysis, with results displayed in Figure~\ref{fig:run_all}. 
The graphs present the evolution of the common vulnerabilities, the common clusters, and our risk metric for both schemes. 
Notice that two \glspl{os} might appear in the same cluster but with no mutual flaw as clusters can include many distinct vulnerabilities.

\begin{figure*}[h]
\subfigure[Random]{\includegraphics[width=0.5\columnwidth]{images/gnuplot/score/score_random_all.pdf}\label{fig:random_all}}
\hspace{0.5cm}
\subfigure[\system]{\includegraphics[width=0.5\columnwidth]{images/gnuplot/score/score_final_all.pdf}\label{fig:intel_all}}
\caption{Execution phase for Random and \system OS configuration strategies (log scale).}
\label{fig:run_all}
\end{figure*}

\textbf{Results:} As shown in Figure~\ref{fig:random_all}, \emph{Random} survives only for $10$ days. 
The number of shared clusters and vulnerabilities remains small for the first days. 
Then, there is a replica replacement that adds to the configuration an OS that has common vulnerabilities with the others. 
%Thus, enabling an adversary to compromise enough replicas in the system.

\system survives until the end of the experiment, as the risk is continually managed to keep the system safe. Figure~\ref{fig:intel_all} shows that shared clusters sometimes increase, at the same pace as the risk.
But then, the next reconfigurations are carried out with the goal of decreasing the risk. 
Notice that the risk value is always under $1$ for \system, and in the \emph{Random} is mostly above $10$.


\subsection{Diversity vs Attacks}

\begin{table}[t]
\begin{center}
{%\small%
\footnotesize
\begin{tabular}{ | p{0.96\columnwidth} | }\hline

\textbf{Samba:} 
\emph{On February 2, 2017, security researchers published details about a zero-day vulnerability in Server Message Block (SMB) of Windows, affecting several versions such as 8.1, 10, Server 2012 R2, and Server 2016. 
Could cause a \gls{dos} condition when a client accesses a malicious SMB.}\\
\textbf{CVES:} 
CVE-2017-0016
\\ \hline

\textbf{Wanna Cry:} 
\emph{On Friday, May 12, 2017, the world was alarmed to discover a widespread ransomware attack that hit organizations in more than 100 countries. Based on a vulnerability in Windows' SMB protocol (nicknamed EternalBlue), discovered by the NSA and leaked by Shadow Brokers.} \\
\textbf{CVES:} 
CVE-2017-0143, CVE-2017-0144, CVE-2017-0145, CVE-2017-0146, CVE-2017-0147, CVE-2017-0148 \\ \hline

\textbf{PowerShell:} 
\emph{Security feature bypass vulnerabilities in Device Guard that could allow an attacker to inject malicious code into a Windows PowerShell session.} \\
\textbf{CVES:}
CVE-2017-0219, CVE-2017-0173, CVE-2017-0215, CVE-2017-0216, CVE-2017-0218\\ \hline

\textbf{Stackclash:} 
\emph{In its 2017 malware forecast, SophosLabs warned that attackers would increasingly target Linux. The flaw, discovered by researchers at Qualys, is in the memory management of several operating systems and affects Linux, OpenBSD, NetBSD, FreeBSD and Solaris.}\\
\textbf{CVES:}
CVE-2017-1000365, CVE-2017-1000366, CVE-2017-1000367, CVE-2017-1000369, CVE-2017-1000370, CVE-2017-1000370, CVE-2017-1000371, CVE-2017-1000372, CVE-2017-1000373, CVE-2017-1000374, CVE-2017-1000375, CVE-2017-1000376, CVE-2017-1000379, CVE-2017-1083, CVE-2017-1084, CVE-2017-3629, CVE-2017-3630, CVE-2017-3631\\ \hline

\end{tabular}
}
\caption{Notable attacks during 2017.}
\label{tab:special_vulns}
\end{center}
\end{table}

This experiment evaluates the strategies when facing notable attacks/vulnerabilities that appeared in $2017$. 
Each attack potentially exploits several flaws, some of which affecting different \glspl{os}. 
The attacks were selected by searching the security news sites for high impact problems, most of them related to more than one CVE. 
As some of the \glspl{cve} include applications, we added more vulnerabilities to the database for this purpose.
Table~\ref{tab:special_vulns} lists the attacks and related \glspl{cve}: Samba,\footnote{https://www.secureworks.com/blog/attacking-windows-smb-zero-day-vulnerability} WannaCry,\footnote{https://securityintelligence.com/wannacry-ransomware-spreads-across-the-globe-makes-organizations-wanna-cry-about-microsoft-vulnerability/} Powershell,\footnote{http://blog.talosintelligence.com/2017/06/ms-tuesday.html} and Stackclash.\footnote{https://nakedsecurity.sophos.com/2017/06/20/stack-clash-linux-vulnerability-you-need-to-patch-now/}


Since some of these attacks might have been prepared months before the vulnerabilities are publicly disclosed, we augmented the execution phase to the full six months. 
As before, the strategies are analyzed over $5000$ runs.


\begin{figure}[t]
\begin{center}
\includegraphics[width=\columnwidth]{images/gnuplot/special_vulns/execution-special.pdf}
\caption{Compromised runs with notable attacks.}
\label{fig:special_vulns}
\end{center}
\end{figure}

\textbf{Results:}
Figure~\ref{fig:special_vulns} shows the percentage of compromised runs for each attack and all attacks put together.
\system is clearly the best at handling the various scenarios, with no compromised executions.
\emph{Random} is the worse, as it does not use any criteria to select the OSes. 
Both \emph{Equal} and \emph{Static} may perform not so bad as they are static, i.e., the \glspl{os} selected by random chance might end up not being exploitable until the end of the run.


\section{Discussion}
\label{sec:discussionlazarus}

\chapter{Conclusion and Future Research Directions}
\label{chap:conclusion}

\section{Conclusions}
The correctness of \gls{bft} systems is tied to the guarantee that replicas fail independtly. 
Otherwise, once a replica is compromised the next $f+1$ replicas would be compromised virtually within the same time or effort.
In this thesis, we addresses the long-standing open problem of many works on the past 20 years of \gls{bft} replication research: evaluate, select, and manage the failure independence (through diversity) of a \gls{bft} system to make it resilient to malicious adversaries.
That said, we developed a control plane that maximizes the failure inpdenpence of \gls{bft} replicated nodes in a practical and effective way.


The first step to achieve this objective was, following the line of the masters thesis, to validate the diversity hypothesis.  
To this end, we levarage on the study using \gls{nvd} data about the shared vulnerabilities among \glspl{os} and conducted and analysis more focused on the deciding which \glspl{os} provide a dependable set for replicated systems.
In particular, we presented several strategies to choose most diverse \glspl{os}.
These strategies comprise different approaches data depending on whether one (i) considers all common vulnerabilities as being of equal importance; (ii) place greater emphasis on more recent common vulnerabilities; and (iii) are primarily interested in the common vulnerabilities being reported less frequently in calendar time.
All strategies delivered the same best conbination of \glspl{os}, this emphasizes the importance of carefully select \glspl{os} that minimize the number of common weaknesses.



Despite the results that we have achieved in the previous contribution, we have identified some problems that leads us to pursue on further analysis.
Namely, we have identified that our (and several other) work(s) that used \gls{nvd} as a source of vulnerability studies were missing important data. 
In particular, not all products vulnerable to a certain vulnerability were reported as affected in \gls{nvd}.
We addressed this issue by using clustering techcniques to group similar vulnerabilities.
Moreover, we add other data sources with other relevant data (that \gls{nvd} does not provide) to enrich our data.
We devise a new metric that uses the vulnerabilities attributes provided by \gls{nvd}, \gls{cvss}, and the the ones that we added. 
This metric is used in an algorithm that builds and reconfigures replica sets on \gls{bft} systems.
The algorithm decides when and which replica should be replaced by estimating the risk of the \gls{bft} being vulnerable and potentially compromised.
The results show that our algorithm supports better decisions when selecting \glspl{os} to run in the replicated system. 
For example, for the evaluated time it \emph{compromised rate} varies between 2\%-5\% while a random selection varies between 98\%-99\%.
 

Besides the lack of proof for supporting diversity, most of works did not address diversity in a practical manner.
We have implemented \system to assess the limitations that diversity could introduce in \gls{bft} systems.
\system uses the algorithm previously described to manage real \gls{bft} systems. 
We conducted an extensive evaluation of the costs of using real \gls{os} diversity in \gls{bft} systems.
In the experiements, we have found a major limitation on the virtualization technology, which forced us to limit our evalution by reconfiguring the bare metal machines to use fewer resources.
Therefore, we manage to make a fair comparison (for some of the cases) between a bare metal configuration with a virtualized (yet limited) solution that accomates \system.
Although most of the results show a non-negligible overhead, we have shown that for \gls{os} that did not suffer virtualization limitations the overhead is minimal.
In these cases, the performance of the replicated system managed by \system is 86\%-94\% of the bare metal, which runs a replicated non-virtualized homogenous \gls{os} configuration.

One of the \gls{bft} systems used in the \system evaluation, is \sieveq. 
It is multi-layer firewall-like application that was designed to absorb most of the external and itnernal attacks with additional resilient mechanisms.
Firewalls represent one of the most critical roles in infrastrucres, as they controll which packets go in and out of the network.
The main improvement of the multi-layered architecture, when compared with previous systems, is the separation of message filtering in several components that carry on verifications progressively more costly and complex.
This allows the proposed system to be more efficient than the state-of-the-art replicated firewalls under attack.
\sieveq also includes several resilience mechanisms that allow the creation, removal and recovery of components in a dynamic way, to effectively respond to evolving threats against the system. Experimental results show that such resilience mechanisms can significantly reduce the effects of \gls{dos} attacks against the system.


\section{Future Work}
This thesis addresses the long-standing open problems of diversity in \gls{bft} system.
The results of this thesis open many avenues for future work on this research topic, which we address in the following topics:


\textbf{Integration of prevention techniques:}
In Chapter~\ref{chap:related_work}, we have briefly described a few prevention techniques.
Although we did not address them in this thesis, we initiate the discussion on how to integrate such techniques in \system.
The \system replicas have two phases were they are dormant, when they are in quarantine or before they are deployed in the execution plane.
Thus, this time could be used to apply a battery of automatic procedures (e.g., vulnerability detectors) that would (1) detect additional weaknesses, which could be reported, and (2) remediate those weakness with automatic mechanisms like automatic patching~\cite{Huang:2016}.
Such mechanisms would decrease the vulnerability surface, especially if common vulnerabilities are detected.

Additionally, use automatic attacks~\cite{Hu:2015} to exploit common vulnerabilities in differnet replicas. 
Although we are most concern with \gls{apts}, using automatic attacks could activate some common vulnerabilities that would trigger the algorithm to adjust the risk associated with such replicas.


\textbf{Extending the \system sources:}


\textbf{Use vulnerable clone decttors on Opens source software to aggravete pairs with more clones} with dependcy graphs and autidting tools~\cite{Kim:2017}


\textbf{Vulnreabilities in the black market as a inditicar of severity}~\cite{Allodi:2014}.


\textbf{Integration with other sensors:}
\system monitors only five security data feeds on the internet looking for vulnerabilities, exploits, and patches in the OSes it manages, but it could be extended to monitor other indicators of compromise (e.g., IP black lists) extracted from a much richer set of sources~\cite{Liao:2016,Sabottke:2015}.
Similarly, \system can be extended to additionally use the outputs of IDSes to assess the BFT system behavior and trigger replica reconfigurations in case of need.



\textbf{Virtualization technology:}
\system paid a performance penalty due to the limitations of the virtualization platform we used (VirtualBox).
VirtualBox was selected because it was the platform we could run more OSes.
Therefore its use enabled \system to support 17 different OSes for running BFT systems.
It would be great to have a VM technology capable of supporting all existing OSes without the resource limitations we experienced.

\textbf{Distributed control plane:}
As in previous works~\cite{Roeder:2010,Platania:2014}, our current design for \system considers a centralized trusted control plane that analyzes OSINT and orchestrates replica set reconfigurations.
It would be desirable to have a distributed version of such control plane, not only for improving its dependability but also to support the existence of multi-domain applications, such as blockchain platforms.

\textbf{Trusted components:}
Our prototype implements the LTU as a trusted component isolated from the rest of the replica in a VM, as many works on hybrid BFT~\cite{Veronese:2013,Roeder:2010,Platania:2014,Sousa:2010,Distler:2011}.
The recent popularization of trusted computing technologies such as Intel SGX~\cite{sgx}, and its use for implementing efficient BFT replication~\cite{Behl:2017}, open interesting possibilities for using novel hardware to support services like \system on bare metal.



\textbf{Diversity-aware replication:}
The evaluation of BFT-SMaRt on top of \system shows that different replica set configurations can impact on the performance of applications, mostly due to the performance heterogeneity of the different OSes.
It would be interesting to consider protocols in which this heterogeneity is taken into account.
For example, the leader could be allocated in the fastest replica, or weighted-replication protocols such as WHEAT~\cite{Sousa:2015} could be used to assign higher weights to the replicas running in faster replicas.




% Fim do conteudo
% ----------------------------------------------------------------------

% Glossario

%
% Para actualizar o glossario, e' preciso correr o script ./fazindice
% e voltar a gerar o PDF
%
\LIMPA
\renewcommand{\glossaryname}{Acronyms}

\newacronym{aslr}{ASLR}{Address Space Layout Randomization}
\newacronym{arff}{ARFF}{Attribute-Relation File Format}



\newacronym{bft}{BFT}{Byzantine Fault Tolerance}
\newacronym{bm}{BM}{Bare Metal}

\newacronym[plural=CVEs,firstplural=Common Vulnerabilities and Exposures (CVEs)]{cve}{CVE}{Common Vulnerabilities and Exposures}
\newacronym{cpe}{CPE}{Common Platform Enumeration}
\newacronym{cvi}{CVI}{Common Vulnerability Indicator}
\newacronym{cvcs}{CVCst}{Common Vulnerability Count Strategy}
\newacronym{cvc}{CVC}{Common Vulnerability Count}
\newacronym{cvis}{CVIst}{Common Vulnerability Indicator Strategy}
\newacronym{irt}{IRT}{Inter-Reporting Times}
\newacronym{irts}{IRTst}{Inter-Reporting Times Strategy}
\newacronym{ip}{IP}{Internet Protocol}

\newacronym[plural=GPUs,firstplural=Graphics Processing Units (GPUs)]{gpu}{GPU}{Graphics Processing Unit}
\newacronym[plural=FPGAs,firstplural=Field-Programmable Gate Arrays (FPGAs)]{fpga}{FPGA}{Field-Programmable Gate Array}


\newacronym{cvss}{CVSS}{Common Vulnerability Scoring System}
\newacronym{xss}{XSS}{Cross-site scripting}


\newacronym{cots}{COTS}{Components Off-the-Shelf}
\newacronym{cis}{CIS}{CRUTIAL Information Switch}
\newacronym{dos}{DoS}{Denial-of-Service}
\newacronym{ddos}{DDoS}{Distributed Denial-of-Service}
\newacronym{dbms}{DBMS}{Databases Management Systems}
\newacronym{dns}{DNS}{Domain Name System}

\newacronym[plural=DBs,firstplural=Databases (DBs)]{db}{DB}{Database}

\newacronym{hmac}{HMAC}{Hash-based Message Authentication Code}


\newacronym{jvm}{JVM}{Java Virtual Machine}
\newacronym{ics}{ICS}{Industrial Control Systems}

\newacronym{ots}{OTS}{Off-the-Shelf}
\newacronym{osi}{OSI}{Open Systems Interconnection}

\newacronym[plural=OSes,firstplural=Operating Systems (OSes)]{os}{OS}{Operating System}
\newacronym[plural=IDSs,firstplural=Intrusion Detection Systems (IDSs)]{ids}{IDS}{Intrusion Detection System}


\newacronym{sql}{SQL}{Structured Query Language}

\newacronym[plural=MACs,firstplural=Message Authentication Codes (MACs)]{mac}{MAC}{Message Authentication Code}

\newacronym{nist}{NIST}{National Institute of Standards and Technology}
\newacronym{nvd}{NVD}{National Vulnerability Database}
\newacronym{nfs}{NFS}{Network File System}


\newacronym{ltu}{LTU}{Logical Trusted Unit}


\newacronym{scada}{SCADA}{Supervisory Control and Data Acquisition}
\newacronym{smr}{SMR}{State Machine Replication}
\newacronym{siem}{SIEM}{Security Information and Event Management}

\newacronym{ycsb}{YCSB}{Yahoo! Cloud Serving Benchmark}



\newacronym{osint}{OSINT}{Open Source Intelligence}
\newacronym{osvdb}{OSVDB}{Open Sourced Vulnerability Database}

\newacronym{tom}{TOM}{Total Ordered Multicast}
\newacronym{tls}{TLS}{Transport Layer Security}


\newacronym{mtd}{MTD}{Moving Target Defense}


\newacronym{po}{PO}{Proactive Obfuscation}

\newacronym{pr}{PR}{Proactive Recovery}
\newacronym{prr}{PRR}{Proactive-Reactive Recovery}
\newacronym{prrw}{PRRW}{Proactive-Reactive Recovery Wormhole}

\newacronym{zda}{ZDA}{Zero-Day Attack}
\newacronym{pzda}{PZDA}{Pseudo Zero-Day Attack}
\newacronym{ppzda}{PPZDA}{Potential Pseudo Zero-Day Attack}
\newacronym{poa}{POA}{Potential for Attack}


\newacronym{rsa}{RSA}{Rivest–Shamir–Adleman}
\newacronym{kvs}{KVS}{Key-Value Store}
%\newacronym{kmp}{KMP}{Knuth–Morris–Pratt}


\newacronym{svm}{SVM}{Support Vector Machines}
\newacronym{scit}{SCIT}{Self-Cleaning Intrusion Tolerance}

\newacronym{sha}{SHA}{Secure Hash Algorithm }

\newacronym{tcp}{TCP}{Transmission Control Protocol}
\newacronym{wine}{WINE}{World Wide Intelligence Network Environment}


\newacronym[plural=VMs,firstplural=Virtual Machines (VMs)]{vm}{VM}{Virtual Machine}
\newacronym{vmm}{VMM}{Virtual Machine Manager}
\newacronym{vpn}{VPN}{Virtual Private Network}

\newacronym{xml}{XML}{Extensible Markup Language}




\printglossaries
\addcontentsline {toc} {chapter} {Acronyms}
% Bibliografia

\LIMPA
\bibliographystyle{abbrv}
\bibliography{chapters/references/references}
\addcontentsline {toc} {chapter} {Bibliography}

\end{document}
