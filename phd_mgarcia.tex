%
% Modelo para Capa final de tese de doutoramento
% do MEI.
%
% Incorpora elementos impostos pelo Regulamento de Estudos Pos-Graduados da
% Universidade de Lisboa (DR: 31/08/2017)
%
\documentclass[
 paper=A4,               % paper size --> A4 is default in Germany
    twoside=true,           % onesite or twoside printing
    openright,              % doublepage cleaning ends up right side
    parskip=full,           % spacing value / method for paragraphs
    chapterprefix=true,     % prefix for chapter marks
    11pt,                   % font size
    headings=normal,        % size of headings
    bibliography=totoc,     % include bib in toc
    listof=totoc,           % include listof entries in toc
    titlepage=on,           % own page for each title page
    captions=tableabove,    % display table captions above the float env
    draft=false,            % value for draft version
]{scrreprt}

%\DeclareOldFontCommand{\bf}{\normalfont\bfseries}
\usepackage{newstyle}
\usepackage{tabularx}
\usepackage[utf8]{inputenc}
\usepackage{amsmath,amsfonts,amssymb,amsthm,url}
\usepackage[portuguese,english]{babel}
\usepackage{times}
\usepackage{xspace}
\usepackage{setspace}
\usepackage[show]{chato-notes}
\usepackage[sort&compress,numbers]{natbib}
\usepackage[vlined,ruled,commentsnumbered,linesnumbered]{algorithm2e} 
\usepackage{graphicx,multirow}
\usepackage{array}
\usepackage{mathptmx} % use Times in math mode
\usepackage{moreverb}
\usepackage{lscape,rotating}
\usepackage{hyperref}
\usepackage{fancyhdr}
\usepackage{lastpage}
\usepackage{listings}
%\usepackage[tocloft]
\usepackage{subfig}
\usepackage{tikz}
\usepackage{xcolor}
\usepackage{makeidx}

\definecolor{codegreen}{rgb}{0,0.6,0}
\definecolor{codegray}{rgb}{0.5,0.5,0.5}
\definecolor{codepurple}{rgb}{0.58,0,0.82}
\definecolor{backcolour}{rgb}{0.95,0.95,0.92}

\lstdefinestyle{mystyle}{
    backgroundcolor=\color{backcolour},
    commentstyle=\color{codegreen},
    keywordstyle=\color{blue},
    keywordstyle={[2]\color{magenta}},
    numberstyle=\tiny\color{codegray},
    stringstyle=\color{codepurple},
    basicstyle=\footnotesize,
    comment=[l]{\%},
    keywords={@relation,@attribute,@data,modifyvm, do, Vagrant, configure, id},
    morekeywords=[2]{real,integer,numeric,string,date,provider,customize,gui,v,provision, config,vm,box,ssh,insert_key,network},
    breakatwhitespace=false,
    breaklines=true,
    captionpos=b,
    keepspaces=true,
    numbers=left,
    numbersep=5pt,
    showspaces=false,
    showstringspaces=false,
    showtabs=false,
    tabsize=1
}

\lstset{style=mystyle}


\newtheorem*{validity}{Validity}{}
\newtheorem*{liveness}{Liveness}{}
\newtheorem*{security}{Compliance}{}
\newtheorem*{resilience}{Resilience}{}
\newtheorem{definition}{Definition}[]
\newtheorem{assumption}{Assumption}[]

\newcommand{\sender}{\emph{sender-SQ}\xspace}
\newcommand{\Sender}{\emph{Sender-SQ}\xspace}
\newcommand{\senders}{\emph{sender-SQs}\xspace}
\newcommand{\Senders}{\emph{Sender-SQs}\xspace}
\newcommand{\presieve}{\emph{pre-SQ}\xspace}
\newcommand{\Presieve}{\emph{Pre-SQ}\xspace}
\newcommand{\Presieves}{\emph{Pre-SQs}\xspace}
\newcommand{\presieves}{\emph{pre-SQs}\xspace}
\newcommand{\repsieve}{\emph{replica-SQ}\xspace}
\newcommand{\Repsieve}{\emph{Replica-SQ}\xspace}
\newcommand{\repsieves}{\emph{replica-SQs}\xspace}
\newcommand{\Repsieves}{\emph{Replica-SQs}\xspace}
\newcommand{\postsieve}{\emph{post-SQ}\xspace}
\newcommand{\Postsieve}{\emph{Post-SQ}\xspace}

\newcommand{\msg}{\texttt{Msg}\xspace}
\newcommand{\sn}{\emph{sn}\xspace}
\newcommand{\signature}{\emph{sgn}\xspace}
\newcommand{\mac}{\emph{mac}\xspace}
\newcommand{\senderi}{\emph{S}\xspace}
\newcommand{\sendersqi}{\emph{SQ}\xspace}
\newcommand{\presievei}{\emph{PS}\xspace}
\newcommand{\replicak}{\emph{RS}\xspace}
\newcommand{\postsievej}{\emph{PQ}\xspace}
\newcommand{\receiverj}{\emph{R}\xspace}
\newcommand{\privkey}{\emph{prk}\xspace}
\newcommand{\pubkey}{\emph{puk}\xspace}
\newcommand{\sharedkey}{\emph{shk}\xspace}


\newcommand{\system}{\textsc{Lazarus}\xspace}
\newcommand{\sieveq}{\textsc{SieveQ}\xspace}
\newcommand{\controller}{\textsc{Baton}\xspace}
\newcommand{\fetcher}{\emph{Data manager}\xspace}
\newcommand{\manager}{\emph{Deploy manager}\xspace}
\newcommand{\risk}{\emph{Risk manager}\xspace}

\newcommand{\block}{\emph{Diversity block}\xspace}
\newcommand{\blocks}{\emph{Diversity blocks}\xspace}
\newcommand{\replica}{\emph{replica}\xspace}
\newcommand{\replicas}{\emph{replicas}\xspace}
\newcommand{\configuration}{\emph{System configuration}\xspace}
\newcommand{\configurations}{\emph{System configurations}\xspace}
\newcommand{\configurationClean}{System configuration}


\newcommand{\systemformula}{risk\xspace}
\newcommand{\pairformula}{common\xspace}
\newcommand{\vulnerabilityformula}{score\xspace}


\SetKwData{RS}{\texttt{POOL}}
\SetKwData{ES}{\texttt{CONFIG}}
\SetKwData{PS}{\texttt{CANDIDATES}}
\SetKwData{RM}{\texttt{REMOVABLES}}
\SetKwData{QS}{\texttt{QUARANTINE}}
\SetKwData{MAX}{\texttt{MAXIMALS}}
\SetKwData{N}{\texttt{N}}
\SetKwData{r}{\texttt{r}}
\SetKwData{K}{\texttt{K}}
\SetKwData{toRemove}{\texttt{toRemove}}
\SetKwFunction{Rand}{rand}
\SetKwFunction{Size}{size}
\SetKwFunction{Remove}{rm}
\SetKwFunction{Risk}{risk}
\SetKwFunction{Common}{common}
\SetKwFunction{Older}{older}
\SetKwFunction{healing}{init\_healing}
\SetKwFunction{Inc}{increment}
\SetKwFunction{Dec}{decrement}
\SetKwProg{Fn}{Function}{}{}

\makeindex



\theoremstyle{definition}
\newtheorem{defn}{Definition}[]

\newcommand*\circled[1]{\tikz[baseline=(char.base)]{
            \node[shape=circle,fill,inner sep=1pt] (char) {\textcolor{white}{#1}};}}

\fancyhf{} %
\lhead{\nouppercase {\leftmark}} %
\rhead{\nouppercase {\bf \thepage}}
\renewcommand{\headrulewidth}{0.1pt}

% Comando para inserir pagina em branco (inserida na numeracao, mas sem
% numero impresso) para quando e' preciso obrigar um capitulo a comecar
% do lado direito (pagina impar)
\newcommand{\LIMPA}{
\newpage
\mbox{}
\thispagestyle{empty}
}

% Igual, mas insere pagina com numero impresso (normalmente nao se usa)
\newcommand{\LIMPAC}{
\newpage
\mbox{}
\thispagestyle{plain}
}

%
% ALTERAR AQUI AS INFORMACOES RELATIVAS AO PROJECTO
%
\newcommand{\TITULO}{DIVERSE INTRUSION-TOLERANT SYSTEMS}
\newcommand{\Autor}{Miguel Garcia Tavares Henriques}
\newcommand{\AutorNumAluno}{35054}

%Orientador e CoOrientador *sem* titulos (e.g. Prof. Doutor)
\newcommand{\Orientador}{Alysson Neves Bessani}
\newcommand{\CoOrientador}{Nuno Fuentecilla Maia Ferreira Neves} %se nao se aplicar, nao importa o que aqui esteja

%Se aplicavel, o supervisor pode ter um titulo (Dr., Eng.) colocado aqui
\newcommand{\SupervisorInstituicao}{Nome Completo do Supervisor}  %se nao se aplicar, nao importa o que aqui esteja

\newcommand{\AnoLectivo}{2017/2018}
\newcommand{\Ano}{\Large{2018}}

% Comentar/descomentar conforme conveniente
\newcommand{\TIPO}{DISSERTA\c{C}\~{A}O }
%\newcommand{\TIPO}{TRABALHO DE PROJETO }

% Comentar/descomentar conforme conveniente
%\newcommand{\MESTRADO}{MESTRADO EM -- PREENCHER --}
\newcommand{\DOUTORAMENTO}{Doutoramento em Inform\'{a}tica}

% Comentar/descomentar conforme conveniente
%\newcommand{\IdiomaTese}{\selectlanguage{portuguese}}
\newcommand{\IdiomaTese}{\selectlanguage{english}}

% Comentar/descomentar conforme conveniente
\newcommand{\Especialidade}{}

\newcommand{\Cabecalho}{
\vspace{1cm}\normalfont\normalfont
\vfill
\textsc{\normalsize\uppercase{Universidade de Lisboa}}\\
\normalsize\uppercase{Faculdade de Ci\^{e}ncias}\\
\vspace{1cm}
\includegraphics[scale=.45]{pic/logo_fcul_vertical.png}\\
}

\usepackage{ifpdf}
\ifpdf
\pdfinfo {
	/Author (\Autor)
	/Title (\TITULO)
	/Subject (\DOUTORAMENTO)
	/Keywords ()
	/CreationDate (D:20081112151803)
}
\fi

%\usepackage[dvips]{geometry}
%\geometry{a4paper=true,portrait=true,left=3cm,right=3cm,top=2.5cm,bottom=3.5cm}

\title{\TITULO}
\author{\Autor}
%\date{\today}

\usepackage{glossaries}
\makeglossaries

\begin{document}
\selectlanguage{english}
\pagestyle{empty}

% Primeira capa
% 
%
\begin{center}

\Cabecalho

\vspace{2.0cm}
\vfill
\IdiomaTese
\Large{\textbf{\TITULO}}\\
\vspace{2cm}
\vfill

\large{\textbf{\DOUTORAMENTO}}\\
%\normalsize{\Especialidade}\\
\vspace{2.0cm}
\vfill
\Large{\textbf{\Autor}}\\
\vspace{1,8 cm}
\vfill
\large{Tese orientada por:}\\
%\normalsize{Trabalho de projeto orientado por:}\\
\large{Prof. Doutor \Orientador} \\
% DESCOMENTAR a linha relevante (se alguma), removendo o % no inicio
e co-orientada pelo Prof. Doutor \CoOrientador \\
\vspace{1 cm}
\vfill

\normalsize{Documento especialmente elaborado para a obten\c{c}\~{a}o do grau de doutor}
\vspace{0.5cm}
\vfill
\Ano
\end{center}
\newpage
\mbox{}
\newpage

%
% Segunda capa
%

\begin{center}

\Cabecalho

\vspace{0.5cm}
\vfill
\IdiomaTese
\Large{\textbf{\TITULO}}\\
\selectlanguage{portuguese}
\vspace{1cm}
\vfill

\large{\textbf{\DOUTORAMENTO}}\\
%\normalsize{\Especialidade }\\
\vspace{1cm}
\vfill
\Large{\textbf{\Autor}}\\
\vspace{.5 cm}
\vfill
\large{Tese orientada por:}\\
%\normalsize{Trabalho de projeto orientado por:}\\
\large{Prof. Doutor \Orientador} \\
% DESCOMENTAR a linha relevante (se alguma), removendo o % no inicio
e pelo Prof. Doutor \CoOrientador \\
\vspace{.3 cm}
\vfill

\begin{flushleft}  
\large{J\'{u}ri}
%\vfill
\setlength{\leftskip}{0.5cm}
\normalsize{Presidente:}
\begin{itemize}
%\setlength\itemsep{-0.3 em}
\item{Nome do Presidente}
\end{itemize}
\normalsize{Vogais:}
\begin{itemize}
\setlength\itemsep{-1 em}
\item{Nome do Vogal}
\item{Nome do Vogal}
\item{Nome do Vogal}
\item{Nome do Vogal}
\end{itemize}
\setlength{\leftskip}{0cm}
\end{flushleft}
\vspace{0.0cm}
%\vfill

\Ano
\end{center}
\newpage
\mbox{}
\newpage

\begin{center}
\normalsize{Documento especialmente elaborado para a obten\c{c}\~{a}o do grau de doutor\par}
\vspace{0.0cm}
\normalsize{Este trabalho foi financiado pela Funda\c{c}\~{a}o para a Ci\^{e}ncia e Tecnologia (FCT) atrav\'{e}s da Bolsa Individual de Doutormento SFRH/BD/84375/2012 e atrav\'{e}s dos projectos da Comiss\~{a}o Europeia FP7-257475 (MASSIF) e H2020-700692 (DiSIEM).}
\vspace{0.cm}
\vfill
\end{center}

% Fim da capa
% ----------------------------------------------------------------------


\pagenumbering{roman}			% roman page numbing (invisible for empty page style)
\pagestyle{empty}				% no header or footers

\cleardoublepage

\pagestyle{plain}				% display just page numbers
% ----------------------------------------------------------------------
% P�gina do resumo em Ingl�s:
\selectlanguage{english}
%\vspace*{2cm}
%\begin{center}
%\Large \bf Abstract
%\end{center}
%\vspace*{1cm} \setlength{\baselineskip}{0.6cm}
\chapter*{Abstract (EN)}
\label{sec:abstract}
\vspace*{-10mm}


\vfill

\begin{flushleft}
\textbf{Keywords:}
Diversity, Vulnerabilities, Operating Systems, Intrusion Tolerance, Proactive Recovery.
\end{flushleft}

\LIMPA

% Fim da p�gina do resumo em Ingl�s.
% ----------------------------------------------------------------------

\cleardoublepage

\pagestyle{plain}				% display just page numbers
%\pagestyle{empty}

% ----------------------------------------------------------------------
% P�gina do resumo em Portugu�s:
\selectlanguage{portuguese}
\chapter*{Abstract (PT)}
\label{sec:abstract}
\vspace*{-10mm}


\vfill

\begin{flushleft}
\textbf{Palavras-chave:}
Diversidade, Vulnerabilidades, Sistemas Operativos, Toler\^{a}ncia a Intrus\~{o}es, Recuper\c{c}\~{a}o Proactiva.
\end{flushleft}

\LIMPA
% Fim da p�gina do resumo em Portugu�s
% ----------------------------------------------------------------------

\cleardoublepage

\pagestyle{plain}				% display just page numbers
%\pagestyle{empty}
%1200 1500 palavras
% ----------------------------------------------------------------------
% P�gina do resumo em Portugu�s:
\selectlanguage{portuguese}
\inputencoding{latin1}
\chapter*{Resumo Alargado}
\label{chapter:abstract_pt}
\vspace*{-10mm}

A Toler�ncia a Faltas Bizantinas (BFT) � uma �rea de investiga��o bem estabelecida ao longo de v�rios anos. 
BFT tem como principal objectivo garantir a seguran�a de sistemas replicados mesmo na presen�a de faltas nos n�s que os comp�em. 
Resumidamente, os protocolos BFT garantem que as r�plicas do sistema processam os pedidos dos clientes de maneira semelhante, comportando-se como uma m�quina de estados replicada.
Tipicamente, nestes sistemas a seguran�a � assegurada mesmo que um subconjunto das r�plicas (normalmente, $f$ em $n$ r�plicas) seja incorrecta.
Ainda que n�o seja expl�cito, esta assump��o � verdadeira apenas se as r�plicas falharem de forma independente.
Caso contr�rio, comprometer $f$ r�plicas � virtualmente o mesmo que comprometer $f+1$ r�plicas.


A primeira publica��o de BFT � de 1980~\cite{Pease:1980} e foram sendo publicados mais trabalhos uns anos mais tarde~\cite{Reiter:1994,Kihlstrom:1998}.
No entanto, s� em 1999 � que BFT despertou a curiosidade da comunidade cientifica com o trabalho de Castro e Liskov~\cite{Castro:1999}.
Nos �ltimos vinte anos de investiga��o, foram feitos v�rios avan�os no desempenho (ex.,~\cite{Kotla:2010,Aublin:2015,Behl:2015}), uso dos recursos (ex.,~\cite{Yin:2003,Wood:2011,Veronese:2013,Liu:2016,Behl:2017}), e robustez (ex.,~\cite{Amir:2011,Bessani:2014,Clement:2009b}) destes sistemas.
No entanto, BFT no geral e estes trabalhos em particular assumem, explicitamente ou implicitamente, que as r�plicas falham de forma independente.
Est� impl�cito que � mais dif�cil comprometer todo o sistema (i.e., $f+1$ r�plicas) se as r�plicas n�o partilharem vulnerabilidades.
Alguns trabalhos apresentam mecanismos ortogonais (ex.,~\cite{Roeder:2010,Avizienis:1995}) para gerar diversidade e impedir vulnerabilidades comuns.
Al�m destes, apenas alguns trabalhos implementam mecanismos de diversidade ainda que de forma muito limitada (ex.,~\cite{Castro:2003,Roeder:2010,Amir:2011}).

Actualmente ainda n�o existe uma forma sistematizada para construir sistemas BFT confi�veis usando diversidade.
Ademais, considerar um conjunto inicial de $n$ r�plicas diferentes, n�o � suficiente para sistemas que precisem de executar servi�os durante muito tempo.
Algures no tempo o sistema ter� $f$ r�plicas comprometidas e precisar� de as limpar e recuperar dos efeitos de poss�veis falhas ou intrus�es.
J� existem v�rios trabalhos que implementam solu��es de recupera��o proactiva em sistemas BFT (ex.,~\cite{Castro:2002,Sousa:2010,Roeder:2010,Platania:2014,Distler:2011}).
Estes reiniciam periodicamente as r�plicas de forma a limpar o estado comprometido ou impedir ataques silenciosos em curso.
No entanto, estes trabalhos (i) assumem que as r�plicas falham de forma independente, (ii) n�o apresentam crit�rios para as escolhas de diversidade, ou (iii) apresentam solu��es mais limitadas, por exemplo ofusca��o de mem�ria~\cite{Snow:2013,Bittau:2014}.
Por estas raz�es e sem algum cuidado na escolha da diversidade das r�plicas, estas podem manter as mesmas vulnerabilidades ap�s as recupera��es.

Finalmente, ao contr�rio das solu��es BFT que j� apresentam bastante maturidade, poucos investigaram como aplicar diversidade de forma confi�vel e quando � que as r�plicas devem ser recuperadas.
Esta tese tem como principal objectivo dar uma resposta te�rica e pr�tica para este problema. 


Primeiramente, estendemos os resultados encorajadores do trabalho de mestrado do autor desta tese sobre diversidade de Sistemas Operativos (SO)~\cite{Garcia:2012}. 
Esse trabalho foi um dos primeiros passos para construir sistemas BFT confi�veis avaliando se existia realmente raz�o para assumir (expl�cita ou implicitamente) as vantagens da diversidade (ex.,~\cite{Abd-El-Malek:2005,Bessani:2008,Castro:2002,Castro:2003,Clement:2009,Correia:2004,Kapitza:2012,Kotla:2010,Moniz:2011,Yin:2003}).
Em particular, neste trabalho aument�mos o n�mero de anos da an�lise, desenvolvemos tr�s estrat�gias manuais para seleccionar SOs diversos de forma a minimizar a ocorr�ncia de vulnerabilidades comuns.
Al�m disso, observ�mos que mesmo utilizando vers�es diferentes do mesmo SO � poss�vel construir configura��es com poucas ou nenhumas vulnerabilidades partilhadas. 


Nem todos os sistemas necessitam de t�cnicas avan�adas como a toler�ncia a intrus�es.
T�cnicas que permitam a adop��o de diversidade nas r�plicas s�o especialmente relevantes em sistemas cr�ticos.
Estes, devido � sua natureza, justificam um custo adicional para garantir a sua confiabilidade e seguran�a.
Um bom exemplo deste tipo de aplica��es s�o as anteparas (do ingl�s \emph{firewalls}).
Estas s�o usadas tipicamente como a primeira barreira de protec��o contra ataques externos. 
As anteparas s�o respons�veis por gerir o tr�fego que entra e sai de uma rede de acordo com determinadas regras ou comportamentos.
As solu��es mais gen�ricas sofrem de algumas limita��es:
como qualquer outro programa tamb�m cont�m vulnerabilidades, normalmente s�o pontos �nicos de falhas, levando a protec��o seja comprometida caso uma vulnerabilidade seja explorada.
� por isso importante garantir o funcionamento correcto deste tipo de sistemas.


Propomos, para resolver as limita��es existentes, uma antepara aplicacional BFT chamada \sieveq.
Esta integra o conceito de antepara com o de fila de mensagens.
A solu��o apresenta uma arquitectura com v�rios n�veis de filtragem, como uma peneira (do ingl�s \emph{sieve}).
Estes n�veis permitem descartar mensagens que de outra forma seriam encaminhadas para todas as r�plicas da antepara.
Valid�mos este sistema em v�rios cen�rios de faltas, por exemplo com ataques de nega��o de servi�o.
Os resultados mostram uma melhoria no desempenho em compara��o com sistemas semelhantes.
Al�m disso, utiliz�mos uma amostra do tr�fego dos Jogos Ol�mpicos de Ver�o de 2012 e verific�mos que a nossa antepara tem capacidade para processar uma quantidade de dados realista.


Apesar dos primeiros resultados sobre diversidade serem encorajadores, encontr�mos algumas limita��es que podem comprometer a adop��o desta t�cnica para garantir independ�ncia das faltas. 
Estas limita��es afectam todos os trabalhos (ex.,~\cite{Han:2009,Frei:2010,Shahzad:2012,Bozorgi:2010,Allodi:2014,Gorbenko:2017}) que utilizam o \emph{National Vulnerability Database} (NVD).
� poss�vel encontrar vulnerabilidades no NVD que n�o foram reportadas para todos os SOs que na pr�tica s�o afectados.
Encontr�mos esta informa��o dispon�vel noutras fontes, por exemplo, nos sites dos pr�prios produtores de programas.
Al�m disso, o NVD carece de alguns dados relevantes sobre ataques e correc��es (do ingl�s \emph{patch}) que s�o relevantes quando consideramos o ciclo de vida das vulnerabilidades.
Identific�mos outro problema nas solu��es existentes, relacionado com uso de condi��es temporais para activar as recupera��es das r�plicas.
Est� impl�cito que as r�plicas necessitam do mesmo esfor�o at� serem comprometidas. 
Isto � particularmente relevante quando falamos de sistemas com diversidade~\cite{Nayak:2014}, pois n�o � realista assumir que r�plicas diferentes tenham as mesmas vulnerabilidades.
� necess�rio construir solu��es que adaptem o sistema de acordo com a avalia��o do risco deste ser comprometido, ou seja de $f+1$ r�plicas terem a mesma vulnerabilidade e terem a sua seguran�a quebrada com um s� ataque.

Desenvolvemos uma solu��o que mostra as vantagens de utilizar e gerir a diversidade de forma confi�vel.
Esta solu��o elimina as limita��es do NVD utilizando outras fontes de dados para enriquecer a nossa base de conhecimento sobre poss�veis vulnerabilidades, ataques e correc��es;
Al�m disso, utilizamos t�cnicas de agrupamento de dados (do ingl�s \emph{clustering}) que permitem agrupar informa��o n�o estruturada em grupos organizados por semelhan�a.
A nossa solu��o recolhe informa��o online para monitorizar o risco do sistema cont�nua e automaticamente.
Quando o risco aumenta, o sistema substitui as r�plicas potencialmente vulner�veis por uma r�plica que maximize a diversidade do conjunto.
Esta solu��o foi implementada num prot�tipo, chamado \system, que � o primeiro sistema a modificar a superf�cie de ataque de sistemas BFT de forma confi�vel.
O prot�tipo concretiza a gest�o das r�plicas com recurso � virtualiza��o de m�quinas e t�cnicas de aprovisionamento.
Actualmente o \system suporta a gest�o de 17 vers�es de SOs.


Valid�mos esta solu��o em dois conjuntos de experi�ncias.
A primeira demonstra que a gest�o de risco do \system consegue reduzir significativamente o n�mero de vulnerabilidades partilhadas num sistema replicado.
A segunda avalia o impacto que a virtualiza��o e a diversidade podem ter no desempenho.
Em particular, estud�mos o comportamento do \system com tr�s aplica��es diferentes: 
(1) uma solu��o de armazenamento chave-valor (KVS); (2) uma antepara de n�vel aplicacional (\sieveq); e (3) um sistema de ordena��o de mensagens para uma plataforma de \emph{blockchain}.
Os resultados mostram que alguns conjuntos de SOs conseguem ter um desempenho aproximado a uma solu��o sem diversidade e em m�quinas f�sicas.
No entanto, algumas decis�es de implementa��o ainda permitem algum espa�o para melhorar, uma vez que tiveram um custo consider�vel no desempenho do sistema.

\vfill
\begin{flushleft}
\textbf{Palavras-chave:}
Diversidade, Vulnerabilidades, Sistemas Operativos, Toler\^{a}ncia a Intrus\~{o}es, Rejuvenescimento.
\end{flushleft}
\inputencoding{utf8}
\selectlanguage{english}

%\LIMPA
% Fim da p�gina do resumo em Portugu�s
% ----------------------------------------------------------------------

\cleardoublepage

\pagestyle{plain}

\vspace*{2cm}
\selectlanguage{english}
\chapter*{Acknowledgement}

%\begin{center}
%\selectlanguage{portuguese}
%\Large \bf Agradecimentos
%\selectlanguage{english}





%\Large \bf Acknowledgments
%\end{center}
%\vspace*{1cm} \setlength{\baselineskip}{0.6cm}



\LIMPA

\vfill

%\selectlanguage{portuguese}

\begin{flushright}\textit{To my grandmother.}\end{flushright}

\LIMPA

\cleardoublepage
%
\setcounter{secnumdepth}{1}
\setcounter{tocdepth}{2}		% define depth of toc


%Lista de capitulos
\tableofcontents				% display table of contents
%\addcontentsline {toc} {chapter} {Content}
\newpage
\thispagestyle{empty}
\mbox{}
\newpage


%Lista de figuras
\listoffigures
\addcontentsline {toc} {chapter} {List of Figures}
\newpage
\thispagestyle{empty}
\mbox{}
\newpage

%Lista de tabelas
\listoftables
\addcontentsline {toc} {chapter} {List of Tables}
\newpage
\thispagestyle{empty} \mbox{}
\newpage

% --------------------------
% Body matter
% --------------------------
\pagenumbering{arabic}			% arabic page numbering
\setcounter{page}{1}			% set page counter
\pagestyle{maincontentstyle} 	% fancy header and footer


% ----------------------------------------------------------------------
% Inicio conteudo
%\pagestyle{fancy}
%\cleardoublepage

%\setcounter{page}{1}
%\pagenumbering{arabic}

\chapter{Introduction}
\label{chap:introduction}

\section{Context and Motivation}
\gls{bft}\footnote{In this thesis we interchange Byzantine fault tolerance (subject) with Byzantine fault-tolerant (adjetive) using the same acronym BFT for both.} is a well-established area of research that aims to guarantee safety on replicated systems even in the presence of some (Byzantine) faulty nodes.
In a nutshell, \gls{bft} protocols guarantee that replicas agree on the order of the message execution, and thus, working as a replicated state machine.
This holds even if a subpart $f$ out of $n$ is faulty, then there is always a sufficient number of correct nodes to execute correclty.
Although not explicit, this assumption leverages on the strict condition that the nodes (must) fail independly.
Otherwise, compromising $f+1$ is virtually the same than compromising $f$.


\gls{bft} was first proposed in 1982 by Lamport~\etal{}~\cite{Lamport:1982}, but it only as awaken the distributed systems research community to its relevancy in 1999 due to Castro and Liskov's Practical \gls{bft}~\cite{Castro:1999}. 
In the last twenty years of active research on \gls{bft} replication, there were made great advances on the performance (e.g.,~\cite{Kotla:2010,Aublin:2015,Behl:2015}), use of resources (e.g.,~\cite{Yin:2003,Wood:2011,Veronese:2013,Liu:2016,Behl:2017}), and robustness (e.g.,~\cite{Amir:2011,Bessani:2014,Clement:2009b}) of \gls{bft} systems.
However, \gls{bft} in general and these works in particular, assume, either implicitly or explicitly, that the replica nodes fail independtly. 
This assumption guarantees that the fault threshold is extended in time as it is more time consuming to compromise different replicas than compromise replicas that are equal, thus share the same weaknesses.
Nevertheless, a few works rely on some orthogonal mechanism (e.g.,~\cite{Roeder:2010,Chen:1995}) to avoid these common weaknesses, or rule out the possibility of malicious failures from their ystem models.
Moreover, a few works have implemented and experimented with such mechanisms~\cite{Rodrigues:2001,Roeder:2010,Amir:2011}, but in a very limited way.
Even if considering an initial set of $n$ diverse replicas (i.e., to the assumption), long-running services need to be cleaned from possible failures and intrusions.
A few works on the proactive recovery of \gls{bft} systems~\cite{Castro:2002,Sousa:2010,Roeder:2010,Platania:2014,Distler:2011} periodically restart their replicas to clean undetected faulty states introduced by a stealth attacker. 
However, a common limitation of these works is that they assume that these weaknesses will be cleaned after the recovery.
In practice, this will not happen unless the replica changes its software (e.g., via the previously described techniques) after its recovery.



In practice, diversity has shown that it is a fundamental building block of dependable services.
For example, in avionics~\cite{Yeh:2004}, military systems~\cite{rhimes}, and even in recent blockchain platforms such as Ethereum\footnote{\url{https://www.reddit.com/r/ethereum/comments/55s085/geth_nodes_under_attack_again_we_are_actively/}} (three essential applications of \gls{bft}), and the \gls{dns}~\cite{Shue:2013}.~\footnote{\url{https://secure64.com/dns-diversity/}}. 
AUMENTAR A EXPLICACAO DE CADA UM 


Despite the maturity of \gls{bft} solutions, no one addressed the challenge of building dependable \gls{bft} systems that consider evidence supported decisions on diversity and 



\section{Objectives and Contributions}

DEFINE THE GENERIC OBJECTIVES OF THE THESIS 

\subsection{Evidence for Supporting Diversity}% Diversity Study on Off-the-Shelf Operating Systems}
The current solutions that vouch for diversity as a way to guarantee failure independence, either lack of supporting evidence, or use evidence that is limited to some extent.
Therefore, some of the results may provide false conclusions.
We identify a need for finding \emph{accurate} sources for supporting sound evidence of the benefits diversity as a dependable mechanism.


The results achieved during the MSc thesis encouraged us to extend the work on \gls{ots} \glspl{os} common vulnerabilities~\cite{Garcia:2012}.
In this extension, we devised three manual strategies for selecting diverse software components to minimize the incidence of common vulnerabilities in replicated systems.
Moreover, we observed that using different \gls{os} releases of the same \gls{os} are enough to warrant its adoption as a more straightforward, less complicated, more manageable configuration for replicated systems.

\begin{enumerate}
\item[1.] \textbf{Analysis of operating system diversity for intrusion tolerance}, Miguel Garcia, Alysson Bessani, Ilir Gashi Nuno Neves, and Rafael Obelheiro, in \emph{Software: Practice and Experience, 2014}~\cite{Garcia:2014}.
\end{enumerate}



\subsection{Applying Diversity on BFT Systems}
%{\system, a Diversity and Recovery Manager for BFT systems}

A few works consider the diversity of replicas as a way to achieve such failure independence, however, it is mostly taken for granted.
For example, by using memory randomization techniques~\cite{Roeder:2010} or different \glspl{os}~\cite{Rodrigues:2001,Junqueira:2005} without providing evidence for such independence. 
Moreover, it has been shown that memory randomization does not suffice to impede common failures to occurring~\cite{Snow:2013,Bittau:2014}, and that although diversity promotes fault independence to some extent, it does not avoid utterly different \glspl{os} from sharing vulnerabilities~\cite{Garcia:2014}.


Therefore, some of the results may provide false conclusions.
We identify a need for finding \emph{accurate} sources for supporting sound evidence of the benefits diversity as a dependable mechanism.

A few works use automatic and artificial diversity (e.g.,~\cite{Roeder:2010,Amir:2011}). 
However, they lack evidence to support the failure of independence through diversity. 
Moreover, some studies show that these techniques fail to provide real diversity~\cite{Snow:2013,Bittau:2014}. 
Additionally, the existent systems that implement time-triggered recoveries assume that it takes the same time to compromise each replica, by assuming that vuleranbilies are all the same. 
This assumption is unrealistic, especially when the diversity of replicas is considered~\cite{Nayak:2014}. 
Therefore, it is required tailored methods to evaluate the risk of a replicated system becoming compromised.


The few works have implemented and experimented with such mechanisms~\cite{Rodrigues:2001,Roeder:2010,Amir:2011}. 
However, despite the lack of evidence for supporting the diversity claim, they lack mechanisms that can make practical \gls{bft} systems using diversity in a continuous mode operation (i.e., with recoveries).  
Thus, reducing the costs typically associated with the management of such complex systems which deemed them as practical.


The problem is addressed from both a theoretical and practical perspective.


We address the problems of \emph{finding evidence for supporting diversity}, \emph{manage diversity in a dependable way}, and \emph{supporting diversity mechanism} in the thesis main contribution.
We use different \gls{osint} data sources to build a complete knowledge base about the possible vulnerabilities, exploits, and patches related to the systems of interest. 
Moreover, this data is used to create clusters of similar vulnerabilities, which potentially can be affected by (variations of) the same exploit. 
These clusters and the other collected data are used to assess the risk of the \gls{bft} system becoming compromised due to common vulnerabilities.
Once the risk increases, the system replaces the potentially vulnerable replica by another one, to maximize the failure independence of the replicated service.
The solution continuously collects data from the online sources and monitors the risk of the \gls{bft} in such a way that removes the human from the loop.
We developed these contributions in a solution named \system, and it is the first system that manages \gls{bft} replicated systems (e.g., \sieveq) in a dependable and automatic way.
\system experimental evaluation shows that its strategy reduces the number of executions where the system becomes compromised and that our prototype supports the execution of full-fledged \gls{bft} systems in diverse configurations with 17 \gls{os} versions, reaching a performance close to a homogeneous bare metal setup. 


Patches take time to apply~\cite{Frei:2010}

MAking exploit from patches that were not yet released!!!\cite{Brumley:2008}
THE NEED FOR AUTOMATIC PATCHING~\cite{Nappa:2015} --shared code, one takes the patch and the other no


The contributions of this work resulted in the following publications:

\begin{enumerate}

\item[2.] \textbf{DIVERSYS: DIVErse Rejuvenation SYStem}, Miguel Garcia, Nuno Neves, and  Alysson Bessani in the \emph{Simp\'{o}sio Nacional de Inform\'{a}tica (INFORUM), 2012~\cite{Garcia:2012b}}.


\item[3.] \textbf{Towards an Execution Environment for Intrusion-Tolerant Systems}, Miguel Garcia, Alysson Bessani, and Nuno Neves, Poster session in the \emph{European Conference on Computer Systems (EuroSys), 2016}~\cite{Garcia:2016b}.


\item[4.] \textbf{\system: Automatic Management of Diversity in BFT Systems}, Miguel Garcia, Alysson Bessani, and Nuno Neves -- \emph{Submitted for publication}.

\end{enumerate}


\subsection{BFT Multi-layer Resiliency} 
Most generic firewall solutions suffer from two inherent problems: 
First, they have vulnerabilities as any other system, and as a consequence, they can also be the target of advanced attacks. 
For example, the \gls{nvd}~\cite{nvd} shows that there have been many security issues in commonly used firewalls. 
\gls{nvd}'s reports present the following numbers of security issues between 2010 and 2015: 157 for the Cisco Adaptive Security Appliance; 109 in Juniper Networks solutions; 29 for the Sonicwall firewall; and 24 related to iptables/netfilter. 
Common protection solutions often have been the target of malicious actions as part of a wider scale attack (e.g., anti-virus software~\cite{Chauhan:2011}, \gls{ids}~\cite{Anderson:2001} or firewalls~\cite{Kamara:2003,Surisetty:2010,cisco1,cisco2}).
Second, firewalls are typically a single point of failure, which means that when they crash, the ability of the protected system to communicate may be compromised, at least momentarily.
Therefore, ensuring the correct operation of the firewall under a wide range of failure scenarios becomes imperative.
To tolerate faults, one typically resorts to the replication of the components.

In the last decade, several important advances occurred in the development of intrusion-tolerant systems.
However, to the best of our knowledge, very few works proposed intrusion-tolerant protection devices, such as firewalls.
Performance reasons might explain this, as \gls{bft} replication protocols are usually associated with significant overheads and limited scalability.
Additionally, achieving complete transparency to the rest of the system can be challenging to reconcile with the objective of having fast message filtering under attack.


we propose a new protection system called \sieveq that mixes the firewall paradigm with a message queue service, with the goal of improving the state-of-the-art approaches under accidental failures and/or attacks.
The solution has a fault- and intrusion-tolerant architecture that applies filtering operations in two stages acting like a sieve.
The first stage, called \emph{pre-filtering}, performs lightweight checks, making it efficient to detect and discard malicious messages from external adversaries.
In particular, messages are only allowed to go through if they come from a pre-defined set of authenticated senders.
\gls{dos} traffic from external sources is immediately dropped, preventing those messages from overloading the next stage.
The second stage, named \emph{filtering}, enforces more refined application level policies, which can require the inspection of some message fields or need the enforcement of specific ordering rules.



Typical \gls{bft} protocols use one of the two following approaches to disseminate messages to the replicas: (1) traffic replicator before the replicas or (2) a leader is responsible to dessiminate the messages to the other replicas. 
The dissemination of a message to all replicas can be detrimental to the proper operation of the replicated service.
For example, a traffic replicator device (e.g., hub) can be placed at the entry of the system to transparently reproduce all messages~\cite{Sousa:2010,Roeder:2010}. 
The effect is an attack amplification caused by the replicator device.
Alternatively, a leader replica could receive the traffic and then disseminate the messages to the others~\cite{Amir:2011}.
The drawback is that the leader becomes a natural bottleneck, especially when under attack (instead of dispersing the attack load over all replicas).
%Then, new architectures can be designed to accommodate external attacks in a resourceful and resilient manner.

In \sieveq, we explore a different design for replicated protection devices, where we trade some transparency on senders and receivers for a more efficient and resilient firewall solution.
In particular, we propose an architecture in which critical services and devices can only be accessed through a message queue and implement the application-level filtering in this queue.
It is assumed that these services have a limited number of senders, which can be appropriately configured to ensure that only they are authorized to communicate through \sieveq.



The contributions of this work resulted in the following publications:

\begin{enumerate}
\item[5.] \textbf{An Intrusion-Tolerant Firewall Design for Protecting SIEM Systems}, Miguel Garcia, Nuno Neves, Alysson Bessani, in the \emph{Workshop on Systems Resilience in conjunction with the IEEE/IFIP International Conference on Dependable Systems and Networks, 2013}~\cite{Garcia:2013}.

\item[6.] \textbf{\sieveq: A Layered BFT Protection System for Critical Services}, Miguel Garcia, Nuno Neves, and Alysson Bessani, in \emph{IEEE Transactions on Dependable and Secure Computing, 2018}~\cite{Garcia:2016}.
\end{enumerate}



To conclude, the colaboration with Andr\'{e} Nogueira resulted in a (out of the scope of this thesis) work on \gls{scada} system enhanced with \gls{bft} techniques. 
We documented the challenges of building such system from a ``traditional'' non-\gls{bft} solution.
This effort resulted in a prototype that integrates the Eclipse NeoSCADA and the BFT-SMaRt open-source projects.
This solution could be managed by \system:


\begin{enumerate}

\item[7.] \textbf{On the Challenges of Building a BFT SCADA}, Andr\'{e} Nogueira, Miguel Garcia, Alysson Bessani, and Nuno Neves, in \emph{Proceedings of the International Conference on Dependable Systems and Networks, 2018 }~\cite{Nogueira:2018}.
\end{enumerate}


\subsection{Thesis Statement}
We summarize our findings in the following thesis statement:

\vspace{2mm}
\fbox{ \begin{minipage}{35em}
%\begin{center}
\emph{
It is possible to build dependable BFT replicated systems by minimizing the number of replicas' common vulnerabilities through software's diversity.
Additionally, it is possible to continuously manage these systems while monitoring OSINT data and deciding when replicas should be diversified and deploying the most dependable configurations.
}
%\end{center}
\end{minipage}
}





\section{Thesis Overview}
Detailed description of each Chapter

\chapter{Background}
\label{chap:related_work}

This thesis' contributions are built on the idea that intrusion tolerance is the last resort, to the best of our knowledge, to build secure and dependable systems.
In this chapter, we present the reasons why we need intrusion tolerance, and then we describe the most relevant related works in the different intrusion tolerance areas.


\section{The Need for Intrusion Tolerance}
The realistic way to provide security is to build mechanisms that protect highly vulnerable systems from powerful attackers.
This is particularly important when considering critical systems that attract highly motivated attackers. 
Then, there is a need to focus on designing mechanisms to make systems safe and resilient despite their number of vulnerabilities and the attacker's power.
In the following, we make an overview of the different approaches that can be combined to deal with the challenge of providing security and dependability for critical systems.

\subsection{Vulnerabilities Prevention}
One of the primary techniques to impede attackers' success is to try to avoid that vulnerabilities persist in the code throughout all the software development phases until it is in production. 
One way to do that is to detect and remove vulnerabilities during the development and testing stages.
We can identify a few standard techniques that are used to prevent the existence of vulnerabilities: static, dynamic, concolic analysis, and vulnerability workarounds.

A few examples of static analysis are source code verification and validation. 
However, the most common techniques typically over-simplify the protocols to make them formally verifiable or just use very particular version to verify~\cite{Klein:2009,Nelson:2017}. 
Moreover, this process consumes much time, and it is expensive for most of the companies, which makes this method difficult to scale to complex systems~\cite{Giuffrida:2013}.
For example, an \gls{os} kernel was formally verified~\cite{Klein:2009}, but contrary to the Linux kernel that has more than 20 million lines of code\footnote{Source: https://www.linuxcounter.net/statistics/kernel on 2nd July 2018} this only has 10k lines of code.

Fuzzing is a dynamic technique and its success in finding vulnerabilities or bugs results from a good set of input test cases.
These can be built manually and tuned for a particular application, or randomly generated, for the latter they are likely to fail on triggering more complex vulnerabilities.
Another dynamic technique, which is used in information security, is taint analysis.
In this technique, any program variable that can be modified by a user is seen as a vulnerability trigger. 
Then, when it is accessed, it becomes tainted for further inspection.
The main limitations of taint analysis are that vulnerabilities are detected only for the execution paths that have been explored by the tester, and it is limited to call/return functions being challenging to cover shared memory or global variables~\cite{Yamaguchi:2015}.

Concolic testing is a technique that performs symbolic execution along with a concrete execution path. 
Therefore, it is more accurate than fuzzing, but it tends to succumb to path explosion.
Some of these techniques can be combined to take the best of some of the approaches described before (e.g., using fuzzing with symbolic execution~\cite{Stephens:2016}).

Finally, some works propose vulnerability workarounds that intend to minimize the vulnerable window between vulnerability disclosure and the patch release.
Typically, these solutions have two phases: first, the detection phase where they look for vulnerabilities in the source code, and a second, where they instrument the code with the vulnerability workaround.
Although they may have good coverage of the vulnerabilities, there are a few caveats on applying such solution.
For example, some workarounds may crash the application upon the vulnerability activation, or they may disable the main functions that clients use in the software~\cite{Huang:2016}, i.e., compromising the availability. 
Nevertheless, these are initial steps towards more robust solutions.

\subsection{Fault Detection and Removal}
Most of the previously described techniques work offline and are not complete. 
Therefore, unknown vulnerabilities may arise when the system is already online, allowing the system to be compromised.
Then, we need a more complex approach to cope with the existence of vulnerabilities that can be exploited.
The detection and removal approach works by detecting intrusions, and then trigger a recovery mechanism to clear the resulting fault effects. 
Since some attacks use a combination of different vulnerabilities, which isolated would be harmless, it is hard to detect them when the system is in production. 
This approach presents three problems: 
First, there are no perfect intrusion detectors, and then, stealth attacks may not be detected; 
Second, the service can experience some periods of unavailability while the service is recovering; 
Finally, recovering a system might clean its faults, but the system remains vulnerable.
Therefore, it is easy for an attacker to repeat the same procedure to compromise the system every time it recovers.


\subsection{Fault Tolerance and Masking}
The previously described approaches assume that vulnerability prevention is complete to some extent or that is possible to detect all the attacks.
However, these assumptions are unrealistic, and it is not advisable to trust the security and dependability of a system in a single component, as it creates a single point-of-failure. 
Thus, once the system becomes compromised the whole infrastructure becomes exposed to more attacks. 
The established way to provide a system with tolerance and masking properties is to distribute it to a set of replicas, which execute the same commands in the same order. 
Primary-backup replication (i.e., $1 + 1$ replicas), would suffice if only crash faults are considered. 
If one of the replicas crashes, the other can replace it and deliver the requests correctly.
However, if arbitrary faults are considered, primary-backup replication is not enough because compromised replicas could deliver arbitrary outputs, impeding a client to decide which output is correct.

\gls{smr}~\cite{Lamport:1984} is an active approach that has been employed to ensure fault tolerance~\cite{Schneider:1990} of fundamental services in modern internet-scale infrastructures (e.g.,~\cite{Hunt:2010,Calder:2011,Corbett:2013}).
\gls{smr} is achieved in distributed systems that run an agreement protocol that guarantees that all the replica nodes (i) start from the same state, (ii) execute the same sequence of messages, and (iii) execute the same state transitions. 
These properties guarantee that a service runs deterministically in the distributed replicas.
However, \gls{smr} does not provide tolerance for malicious (Byzantine) faults.
Therefore, one must resort to intrusion tolerance techniques~\cite{Verissimo:2003} to address these more complex type of faults.
Intrusion tolerance was first proposed by Fraga and Powell in 1985~\cite{Fraga:1985} as a solution to address faults without compromising the security of a system. 
More formally, we adopt the following intrusion tolerance definition: 

\begin{defn}
\emph{``A replicated intrusion-tolerant system is a replicated system in which a malicious adversary needs to compromise more than $f$ out-of $n$ components in less than $T$ time units to make it fail.''}~\cite{Bessani:2011}
\label{def:def2}
\end{defn}

It is necessary to employ \gls{bft} \gls{smr}, a particular case of \gls{smr}, to build a system capable of operating correctly even in the presence of some compromised nodes.
Since no single replica can be trusted completely, the correctness of the system comes from the majority of correct nodes. 


Although \gls{bft} protocols provide safety to a bound of $f$ faulty nodes, with sufficient time, i.e., greater than $T$, an adversary eventually compromises $f+1$ nodes.
Then, additional mechanisms are needed to clean the faulty state (e.g., from time to time the nodes are recovered~\cite{Castro:2002}).
However, if a recovered node remains vulnerable to the same attack, the time to compromise $f+1$ replicas becomes smaller as the attacker already knows how to exploit those vulnerabilities.
For this reason, several authors have built their models assuming that nodes fail independently due to some mechanism that provides failure independence (e.g.,~\cite{Castro:2002,Bessani:2008,Veronese:2013,Sousa:2010}).
For instance, to increase the time or effort $T$ that takes to compromise $f+1$, one must employ recoveries to reset the replicas' faulty state. 
Moreover, to avoid reintroducing replicas with the same vulnerabilities in the system, the recovery should change the replica's code somehow.
Finally, such mechanism needs careful management otherwise the decisions are made with no criteria which could decrease $T$ after all.
In the following sections, we present and describe several works on specific areas of intrusion tolerance.


\section{Byzantine Fault Tolerance}

\side{OSDI 1999}
Castro and Liskov’s \textsc{Pbft}~\cite{Castro:1999} was the first Practical \gls{bft} replicated system, and it was initially proposed as a solution to handle Byzantine faults of both accidental and malicious nature.
The correctness of a \gls{bft} service arises from the existence of a quorum of correct nodes capable of reaching consensus on the (total) order of messages to be delivered to the replicas.
For instance, to tolerate a single replica failure, the system typically must have four replicas. 
\textsc{Pbft} implements a \gls{smr} protocol that guarantees both liveness and safety for $\lfloor\frac{n-1}{3}\rfloor$ out of a total of $n$ replicas are simultaneously faulty. 
These properties hold even in asynchronous systems such as the internet. 
\textsc{Pbft-pr} implements a \gls{smr}, therefore, it must guarantees that each replica executes the same commands, in the same order, and then it produces the same output. 
In Figure~\ref{fig:bft} we present an overview of \textsc{Pbft} protocol that can be summarized as follows:
A client \emph{(c)} sends a message to all the replicas \emph{(R1-R4)}.
Then, the leader replica has to assign a sequence number to the request and multicast a pre-prepare message to the other replicas. 
If the replicas agree with the leader, they send a prepare message to each other. 
At this phase of the protocol, every correct replica agrees on the ordering of the messsages.  
Every replica sends a commit message. 
When a replica receives the commit message from a quorum, it executes the message. 
In the end, it replies to the client. 
Every replica shares a key with each other and with clients. These keys are used to authenticate messages with a \gls{mac}. 
The messages that are multicast by the clients are authenticated with a vector of \glspl{mac}. 
Then, each replica verifies its own \gls{mac}.
The authors validated this \gls{bft} library implementing a Byzantine fault-tolerant filesystem. 
The results show that when the workload increases the throughput and latency is nearly the same as a non-replicated system. 

\begin{figure}[h]
\begin{center}
\includegraphics[width=.7\columnwidth]{images/images/bft.pdf}
\caption{Byzantine Fault Tolerance protocol overview.}
\label{fig:bft}
\end{center}
\end{figure}

The \textsc{Pbft} performance encouraged the use of \gls{bft} in common systems, and to develop optimizations to improve \gls{bft} protocols. 
Since the \textsc{Pbft-pr} proposal, some work has been dedicated to improve \gls{bft} protocols (see Table~\ref{tab:bft}):


\begin{table}[h]
\begin{center}
{\footnotesize
\begin{tabular}{ p{2.5cm}  p{10.0cm}  }\hline
\textsc{Zyzzyva}~\cite{Kotla:2010}  & Introduces speculation to avoid the expensive three phase commit before processing the requests. This might introduce some inconsistency in the state of the replicas when speculation fails, and the client needs to help replicas to fix the servers’ inconsistency. \\ \hline            
\textsc{Upright}~\cite{Clement:2009} & It provides a straightforward way to also add BFT to crash fault tolerant systems. \\ \hline    
\textsc{Aardvark}~\cite{Clement:2009b} & Shifts the paradigm to a new design. This design improves the performance under faulty scenarios trading some performance on the normal case. \\ \hline
\textsc{BFT-SMaRt}~\cite{Bessani:2014} & It is modular and multicore-aware. Supports replica reconfiguration and has a flexible programming interface. \\ \hline
\textsc{COP}~\cite{Behl:2015} & It is the most recent BFT implementation that reached 2.4 million operations per second. This was achieved mostly due to BFT architecture changes.\\  \hline  
\end{tabular}
}
\caption{Brief overview of the most relevant BFT works.}

\label{tab:bft}
\end{center}
\end{table}




\paragraph{Summary.} 

In this thesis, we propose a control plane for \gls{bft} systems, hence \gls{bft} systems have significant importance in this work.
They play an essential part as they guarantee that replicas execute correctly while tolerating malicious failures in a subset of the replicas.
The implementation of such \gls{bft} protocol is a complex task, mainly if we need to include mechanisms for state transfer, reconfiguration, and guarantee a good performance.
Therefore, we prefer to rely on existent libraries than to build a new one.
Nevertheless, for some contributions (see Chapter~\ref{chap:sieveq}) we needed to perform some modifications on the chosen library.

\side{Zyzzyva TOCS 2010, UpRight SOSP 2009, Aardvark NSDI 2009, BFT-SMART DSN 2014, COP Middleware 2015} 


\section{Replica Rejuvenation}
\side{International Symposium on Fault-Tolerant Computing 93, 95}
Software rejuvenation was proposed in the 90's~\cite{Huang:1993,Huang:1995} as a proactive approach to prevent performance degradation and failures due to software aging. 
This solution was implemented using three components: a watchdog process (\texttt{watchd}), a checkpoint library (\texttt{libft}) and a replication mechanism (\texttt{REPL}). 
A primary node executes the application and also runs the \texttt{watchd} to monitor the application crashes and hangs. 
The backup node, while its application is inactive, keeps on monitoring the primary node. 
Additionally, there is a routine (provided by \texttt{libft}) that periodically makes checkpoints and logging. 
These checkpoints are replicated with \texttt{REPL} in the backup node. 
When the primary node crashes or hangs, it is restarted, and if needed the backup takes his place on the execution.



Only some years later, proactive recovery was adopted on \gls{bft} replicated systems by Castro and Liskov~\cite{Castro:2002}.\side{TOCS 2002}
In order to support long-running services, they introduced the notion of proactive recovery for \gls{bft} services. 
The objective is to rejuvenate replicas periodically to remove stealth attackers and support the execution of long-running services. 
This mechanism allows that an adversary control up to $f$ replicas before a recovery.
Moreover, \textsc{Pbft-pr} introduces additional mechanisms to guarantee that \gls{smr} properties are maintained even with recoveries.
In particular, a \emph{state transfer} protocol, which allows that recovered replicas fetch a correct and up to date state from the other (correct) replicas.
Additional assumptions are needed to guarantee the \textsc{Pbft-pr}'s liveness and safety during the recoveries: 
(i) Each replica contains a trusted chip to store its private key, and it can sign and decrypt messages without revealing the key; 
(ii) The replicas' public keys are stored in a read-only memory, which needs physical access to be modified; 
And (iii) a watchdog timer is used to avoid human interaction to restart replicas. 
The watchdog hands the execution to the recovery monitor, which cannot be interrupted.


Zhou~\etal{}~\cite{Zhou:2002} presented \textsc{Coca}, a fault-tolerant online certification authority to be deployed both in a local area network and on the internet.\side{TOCS 2002}
It employs proactive recoveries to refresh replicas' state (e.g., faulty or incorrect states) as to refresh the private keys of the replicas.
The authors implemented \textsc{Coca} in Byzantine quorum systems as they do not require timing assumptions.
Moreover, and contrary to \textsc{Pbft-pr}, it does not rely on trusted components.
From these model decisions, \textsc{Coca} may need a trusted administrator to refresh the server keys.
However, and more importantly, the asynchronous model may compromise the safety (i.e., $f$ faulty nodes in-between recoveries) as it was shown by Sousa~\etal{}~\cite{Sousa:2007}. 
Sousa~\etal{} stated the need for hybrid-systems that require some level of synchrony to guarantee safety and liveness of such systems.


Reiser and Kapitza~\cite{Reiser:2007} and Distler~\etal{}~\cite{Distler:2008} identified virtualization as a useful mechanism to implement proactive recovery. \side{Workshops 2007 2008}
They proposed an architecture, named \textsc{VM-FIT}, that is divided into two domains: an untrusted domain and a trusted domain.
The intrusion-tolerant replicated system executes in the untrusted domain, running in \gls{vm}. 
While \textsc{VM-FIT} executes in a trusted domain, i.e., in the \gls{vmm}. 
Virtualization provides isolation between the untrusted and the trusted domains. 
Therefore, it can trigger recoveries from the trusted domain in a synchronous manner. 
Moreover, virtualization reduces downtime of the service during the recovery and makes the state transfer between replicas more efficient. 
The authors implemented this system using the Xen hypervisor to implement both domains: the trusted domain is the Xen Dom0, and the replicas run in the untrusted domains, DomUs.

Sousa~\etal{}~\cite{Sousa:2010} improved the state-of-the-art recovery algorithms by introducing \gls{prrw}. \side{TPDS 2010}
It removes the effects of faults ``immediately''. 
\textsc{\gls{prrw}} accelerates the rejuvenation process by detecting the faulty replicas behavior and forcing them to recover without sacrificing periodic rejuvenations. 
This type of technique can only be implemented with synchrony assumptions as the recoveries are time triggered~\cite{Sousa:2005}. 
To address this need, the authors proposed a hybrid system model: the payload is an any-synchrony subsystem, and the wormhole is a synchronous subsystem. 
The authors implemented this system using the Xen hypervisor as the wormhole. 
Zhao~\etal{}~\cite{Zhao:2012} improved \gls{prrw} with an algorithm to schedule the rejuvenations that are triggered by monitoring the network and CPU/memory performance.


\paragraph{Summary.} 
\gls{bft} replication with proactive recovery represents one of the cornerstones of intrusion tolerance. 
Proactive recovery allows the system to reduce the replicas' vulnerability window by cleaning the faulty states or impeding the attacker to be successful. 
Nevertheless, \gls{bft} replication with proactive recovery guarantees the system's correctness while replicas recover before $f+1$ replicas become faulty. 
All works on \emph{safe} proactive recovery consider the use of trusted local component on each replica to trigger the periodic recoveries~\cite{Castro:2002,Sousa:2010,Roeder:2010,Platania:2014,Distler:2011}.
Some works use hardware timers while others resort to virtualization to separate the execution domains. 
In our proposal, we adopt a similar architecture to~\cite{Distler:2008} and~\cite{Sousa:2010}. 
However, our solution implements a different (from~\cite{Sousa:2010}) algorithm to trigger recoveries proactively.
Most of these solutions assumed that replicas fail independently without addressing that issue. 
The following works address the failure independence through diversity. 




\section{Diversity}
In most of the works presented before, the correctness of the \gls{bft} is ensured under the assumption that replicas fail independently.
Therefore, they assume implicitly or explicitly that there is some mechanism that creates different vulnerable surfaces which would difficult the attacker's work.

Diversity can be implemented in different ways~\cite{Deswarte:1998,Larsen:2015}, which may differ in the mean and on the amount of diversity generated, but the goal is the same: to create different attack surfaces (details in the survey~\cite{Baudry:2015}).\side{Conference on Computer Security, Dependability, and Assurance: From Needs to Solutions 1998, Trans. Sec. Priv 2015, ACM Computer Surveys 2015}
In particular, it difficults the chances of finding a vulnerability that compromises $f+1$ replicas at the same time~\cite{Castro:2002}.
In other words, to avoid common vulnerabilities among replicas. 
We present at least three different ways to generate diversity: 
(i) N-version programming consists of designing and/or implementing  different versions from the same specification; 
(ii) Automatic diversity implements different memory schemes and binary executables (from the same source); 
and (iii) \gls{ots} diversity comprises the use of different products with the same functionality but taking advantage of the different implementations that are already available.


\subsection{N-version Programming}
The first work on diversity was published in 1975 by Randell~\etal{}~\cite{Randell:1975}. \side{IEEE Transactions on Software Engineering 1975}
They introduced the idea of using different software design (i.e., implementations) as a way to achieve reliable software. 
The idea is to deploy additional and different replicas along the primary replica, and once the replicas have a mismatch result, a spare replica follow up as primary.


Chen and Avizienis~\cite{Avizienis:1977,Chen:1978} defined the notion N-version programming with the goal of achieving software reliability in redundant systems.\side{ IEEE Computer Software and Applications 77,  IEEE Symposium on Fault-Tolerant Computing 78}
The principle was to define $N$ teams that would develop $N$ versions of the same software specification that would be semantically equivalent.
Then, the N-version programs would execute, compare the outputs, if the outputs verify the equivalent condition then the result is accepted.
These works opened avenues in the area of security and reliability research.
Knight and Leveson~\cite{Knight:1986} made an empirical study that concludes that N-version must be employed with particular care, as it for itself does not provide reliability guarantees. \side{IEEE Transactions on Software Engineering 1986}

\subsection{Automatic Diversity}  
The idea of having to $N$ different teams to develop $N$ software versions was not attractive, and sooner automatic approaches appeared.
Forrest~\cite{Forrest:1997} suggested randomized program transformations to introduce application diversity. \side{Workshop 97}
They have made modifications on the \texttt{gcc}, the GNU C Compiler, in such a way that during the compiling time \texttt{gcc} inserts random padding
into each stack frame. 
The diverse versions are generated during the source code compilation. 
The main idea of these works is to make the intruder’s work more difficult when exploiting buffer overflow vulnerabilities.
More recently, a few works propose the usage of diversity-based compilers to generate different executables~\cite{Platania:2014,Roeder:2010,King:2016}.\side{SRDS 2014, TOCS 2010, OSDI 2016}

Bhatkar~\etal{}~\cite{Bhatkar:2003} proposed a solution based on memory address obfuscation (i.e., \gls{aslr}). \side{Usenix Security 2003}
Their solution transforms object files and executables at the link- and load-time, without kernel or compiler modifications. 
The goal is to ensure that an attack that succeeds in one target will not succeed on the other targets. 
Each time the program is executed its virtual addresses and data are randomized, and therefore the attacker needs to find new ways to exploit memory errors like buffer overflows. 
However, recent works have shown that solutions like \gls{aslr} are still vulnerable to attacks~\cite{Bittau:2014,Jang:2016}.\side{Security Privacy 2014, CCS 2016}


Hosek and Cadar~\cite{Hosek:2015} proposed \textsc{Varan}, an N-version Execution framework.\side{Proceedings of the International Conference on Architectural Support for Programming Languages and Operating Systems 2015}
\textsc{Varan} is composed of a leader replica and its followers that share a memory ring where they execute in parallel.
These components are launched by a coordinator that orchestrates and setups the execution environment.
Each follower runs a binary that has system calls rewritten differently, and this optimization plays a significant performance improvement when compared with other solutions that rewrite the whole program.
The implementation of the memory ring is also a major improvement as it allows concurrent access by multiple producers and consumers.
Nevertheless, the results show that some system calls when executed in \textsc{Varan} can introduce an overhead of 36\% for \texttt{close}, 39\% for \texttt{write}, 135\% for \texttt{read}, and 240\% for \texttt{open} when compared with a native setup.
The authors also scale out the number of followers, which, as expected, increases the overhead as they are added. 
Although \textsc{Varan} was purposed for reliability, some principles could be adopted for security after solving some challenges, namely the use of the same address space that would benefit return-oriented programming attacks.

In the Larsen~\etal{}~\cite{Larsen:2015} survey on automatic diversity, one of the takeaways of the survey is how difficult it is to measure the efficacy of automatic techniques.
For example, entropy analysis may consider two versions with high entropy that are equally vulnerable to a specific attack.
Another way to evaluate these solutions is to test them against real attacks which raise coverage issues.
Moreover, to evaluate automatic diversity is time-consuming and it tends to over-generalize.


\subsection{Off-the-shelf Diversity}
The two previous solutions generate diversity before software’s distribution. 
On the contrary, \gls{ots} diversity does not need pre-distribution of diverse mechanisms. 
It relies on the existence of different software components that are ready to be used.
There are plenty of different free products that provide the same functionality and were developed by different vendors. 
In other words, \gls{cots} diversity is like opportunistic N-version programming.


\paragraph{Studies.}
Gashi~\etal{}~\cite{Gashi:2007} made an experimental evaluation of the benefits of adopting different \gls{sql} databases. \side{TDSC 2007}
The authors analyzed bug reports of four database servers (PostgreSQL, Interbase, Oracle, and Microsoft \gls{sql} Server) and verified which products were affected by each bug reported. 
They have found few cases of a single bug affecting more than one server. 
In fact, there were no coincident failures in more than two of the servers.
The conclusion is that \gls{ots} database servers’ diversity is an effective mean to improve system's reliability. 
However, the authors recognize the need for \gls{sql} translators, to increase the interoperability between servers in the replicated scenario.

Han~\etal{}~\cite{Han:2009} made a systematic analysis of the effectiveness of using \gls{ots} diversity to improve system's security. \side{Int, Conf. on Detection of Intrusions and Malware and Vulnerability Assessment 2009}
First, the authors tried to find if there were software substitutes to provide the same functionality. 
Then, they determined if the \gls{ots} software shared the same vulnerabilities and if so, if the same vulnerability could be exploited with the same attack. 
In this study, they analyzed more than 6k vulnerabilities from \gls{nvd}. 
The results show that 98.5$\%$ of the vulnerable software have substitutes. 
Moreover, the majority of them did not have the same vulnerabilities or could not be compromised with the same exploit code. 
It is not expected that a single exploit works in different \glspl{os} because each one has a different memory scheme and different filesystems. 
Even between versions from the same \gls{os}, the low-level functions change across the versions. 
The study also concluded that 22.5$\%$ of the vulnerabilities were present in multiple software. 
However, only 7.1$\%$ from those vulnerabilities were present in software that offers the same service. 
The study findings are a good sign that diversity can improve a system’s dependability.


Garcia~\etal{}~\cite{Garcia:2012} studied the vulnerabilities among \gls{ots} \glspl{os}. \side{DSN 2012}
Similar to \cite{Han:2009} this study was carried out taking vulnerability feeds from \gls{nvd}. 
However, this work was only focused on \glspl{os}, and the data collected comprised 11-year of vulnerability reports. 
Their goal was to find to what extent different \glspl{os} shared common vulnerabilities. 
To do that, they analyzed 2270 vulnerabilities entries manually and classified them into different categories. 
Then, the authors defined three types of servers: (i) a server that contains most the packages/applications available (Fat server); (ii) a server that does not contain unnecessary applications to a particular
service (Thin server); and (iii) a thin server but with physically controlled access (Isolated thin server). 
They assumed that the third setting it is the most advisable for critical systems since additional care is taken to install and setup its configuration. 
For each configuration, they compared all the common vulnerabilities in pairs of \glspl{os}. 
As expected, in the Isolated thin server, the number of common vulnerabilities was considerably less than in the other configurations. 
There was only one vulnerability that was shared among six \glspl{os}, two that were shared among five \glspl{os}, and 130 that are shared between two \glspl{os}. 
They went further in the study and looked for common vulnerabilities in different versions of the same \gls{os}. 
The authors have found evidence that suggests that using OS diversity in a replicated system can improve its dependability. 
Even if few \glspl{os} are employed, it is possible to achieve vulnerability independence just with different versions.


\paragraph{Network diversity.}
Newell~\etal{}~\cite{Newell:2015} presented a solution to assign diversity variants among network nodes to increase the network's resiliency (in particular its connectivity).\side{TDSC 2015}
The authors present this as a Diversity Assignment Problem, although it is an NP-hard problem the authors show that for medium-size random networks graphs it is feasible.
For medium-sized networks, they use mixed-integer programming and for larger networks propose a fast greedy approach.
For the first one, they achieve optimality, and for the latter an approximation of the optimal.
One of the results show is that random diversity assignment is far worse than a criteria assignment (as the one they propose).
Despite the different results that show improvements of diversity on the connectivity (i.e., resiliency), their models did not use realistic data, as the authors assumed that the compromise expectancy values are known and independent for each replica.


Zhang~\etal{}~\cite{Zhang:2016} propose a metric to evaluate the resilience of private networks.\side{IEEE Trans. on Information Forensics and Security 2016}
The idea is to define what is the least attacking effort to compromise some critical assets of the network (i.e., this work is focused on distributed yet not replicated systems).
Therefore, they measure the weakest path to target and compromise the main asset. 
The authors proposed three metrics which were evaluated in a simulated environment.
None of these metrics use vulnerability data to correlate the existence of common weaknesses, and on the contrary, the metrics use data about the topology, services installed, etc.
Nevertheless, the results show that increasing the diversity on the nodes reduces the number of (simulated) worms.
 

\paragraph{Summary.}
We presented three different techniques that support diversity as a fault tolerance mechanism. 
N-version programming is the most costly since it needs different developers or additional programs to verify if the automatic translation works. 
On the contrary, \gls{ots} and randomization diversity are almost for free. 
The first one can be obtained from free sources that are ready to be used, and the second one is automatic and already supported
in most OSes. 
Diversity still has some gaps that must be addressed to make it a useful building block of intrusion tolerance. 
For example, how to measure diversity in such a way that failure independence can be guaranteed?


\section{Practical Systems with Diversity}
Despite the existence of several studies on vulnerabilities, a few works implemented diversity in a ``random'' fashion.\side{Int. Conf. on Recent Advances in Intrusion Detection 2005}
Totel~\etal{}~\cite{Totel:2005} proposed an \gls{ids} based on design diversity.
The authors described an architecture that uses a set of replicated \gls{cots} servers, a proxy, and an \gls{ids}. 
The proxy is responsible for forwarding the requests from the client to every \gls{cots} server. 
When the servers reply, the proxy sends the responses to the \gls{ids} to be analyzed. 
The IDS compares the responses from the \gls{cots} servers, and if it detects some differences, an alarm is raised. 
Then the proxy votes the responses and replies to the clients. 
The authors developed an algorithm to detect intrusions and tested their solutions against Snort~\cite{snort} and WebStat~\cite{Vigna:2003}. 
The results show that using diverse-\gls{cots} service (e.g., HTTP) allows the \gls{ids} to deliver fewer alarms without missing the intrusions.

Bairavasundaram and Sundararaman~\cite{Bairavasundaram:2009} implemented \textsc{EnvyFS} a filesystem that uses N-version filesystems implementations to guarantee the reliability of stored data.
Their solution is able to tolerate filesystem mistakes (e.g., crash and integrity errors). \side{ATC 2009}
To keep the system simple, the authors used a virtual filesystem layer that abstract the specific filesystems underneath, therefore some challenges arise from the different implementations details.
Another challenge was to avoid the need for multiple storages as it uses N-variants of filesystems. 
They solved this issue using only one storage layer and the different filesystems in between. 
Then, after each filesystem executes the commands, a voter collects the majority of outputs to the storage layer.
The results show that \textsc{EnvyFS} is able to improve the reliability by leveraging on multiple filesystems.
However, the performance results show that \textsc{EnvyFS} pays the price of executing in multiple filesystems and waiting for a majority of them. 
In most of the times, it takes the sum of the time of the three filesystems used.

Diversity has also been adopted to work as a malware detector. 
Xu and Kim~\cite{Xu:2017} introduced \textsc{PlatPal} that analyzes the behavioral discrepancies of malicious document files on different platforms (e.g.,Windows or Macintosh).\side{Usenix Security 2017} 
\textsc{PlatPal} loads a file in both systems and monitors the execution traces/or crashes. 
In the end, both systems compare the outputs, if divergences are observed it is signaled an attack.
The authors focused on a particular application, Adobe Acrobat Reader, it reads pdf files which is a standard format and both \glspl{os} have the application available.
Therefore, besides the internal discrepancies, the correct executions should be very similar.
There are some limitations, namely, it fails to detect attacks that need some human interaction.
Although \textsc{PlatPal} is proposed to detect malicious payloads on files, it vouches (indirectly) for diversity as it shows that an attacker needs to deploy multiple payloads to compromise the system with the same attack (for the majority of the malware samples used).

\paragraph{Diversity on BFT.}
\textsc{Base}~\cite{Castro:2003} is an extension of PBFT~\cite{Castro:1999} that explores opportunistic \gls{ots} diversity in \gls{bft} replication. \side{SOSP 2001}
The system provides an abstraction layer for running diverse service codebases on top of the PBFT library.
The key issue addressed by \textsc{Base} is how to deal with different representations of the replica's state, allowing a replica that recovers from a failure to rebuild its state from other replicas. 
\textsc{Base} was evaluated considering four different \glspl{os} and their native \gls{nfs} implementations: Linux, OpenBSD, Solaris, and FreeBSD.
The results, from 17 years ago, show that performance varies significantly when diversity is considered, these are similar to the results we show in Chapter~\ref{chap:lazarus_implementation}.
Nevertheless, \textsc{Base} does not address the selection of replicas nor the reconfiguration of the replica group.

Roeder and Schneider~\cite{Roeder:2010} propose an automatic diversity technique called \gls{po}.\side{TOCS 2010}
The idea is to change the application and library code periodically preserving the original semantics using program transformations, i.e., system call obfuscation reordering, memory randomization, and functions return checks (through an IBM \texttt{gcc} patch that inserts and checks a random value after the functions return).
Each replica generates its own obfuscated executable from a read-only device containing the ``master code'' based on time-triggered epochs that a controller triggers in a secure way through a trusted component.
All obfuscation mechanisms were implemented and evaluated on OpenBSD 4.0, and their results show that \gls{po} adds little extra overhead to the non-\gls{po} execution.

Similarly to \gls{po}, Platania~\etal{}~\cite{Platania:2014} proposed a compiler-based diversity for the \textsc{Prime} \gls{bft} library~\cite{Amir:2011} using the MultiCompiler tool~\cite{Homescu:2013}. \side{TDSC 2010}
This compiler creates diverse binaries from the same source code through randomization and padding techniques.
The authors also proposed a theoretical rejuvenation model that receives as input: the probability of a replica being correct over a year ($c$); the number of rejuvenations per day across the whole system ($r$); the number of replicas ($n$); and the system's lifetime. 
Although operators can control only $r$ and $n$, the authors suggest (but do not show how) that $c$ can be estimated using \gls{osint} from the internet (CERT alerts, bug reports, and other historical data).
The authors implemented their solution in two settings, including a virtualized environment provided by the Xen Hypervisor.
In this setting, each replica executes in a Linux \gls{vm}, the recovery watchdog runs in a trusted domain of the same machine, and the proactive-recovery controller runs in another physical machine. In Chapter~\ref{chap:lazarus_implementation} we present an architecture similar to this one.


\paragraph{Summary.} 
Diversity is becoming one of the building blocks of intrusion tolerance, alongside with \gls{bft} and rejuvenations. 
We presented few works that already used somehow diversity in intrusion-tolerant systems. 
Although we do not discard the possibility of using other diversity techniques, such as randomization, we are more interested in \gls{cots} diversity.
\gls{cots} diversity allows us to use data that is freely available, to estimate how vulnerable is each component. 
In particular, it allows us to know \emph{a priori} which \gls{cots} shared vulnerabilities in the past.
Nevertheless, these works are still limited to that extent, i.e., there are none or few concerns on how to create diversity to avoid common failures. 
Most of the works implement diversity assuming complete fault independence, but that is unrealistic. 
For example, different \gls{ots} \glspl{os} can share the same weaknesses due to some shared libraries or kernel code. 
There is a need to understand how diversity can be efficiently employed in a replicated system to make the failure independence sound. 
Moreover, further work is still necessary to create automatic mechanisms that abstract diversity management for the administrators.



\section{Vulnerability Analysis}
Massacci and Nguyen~\cite{Massacci:2010} analyzed several data sources that provide data about vulnerabilities, exploits, and patches. \side{workshop 2010}
The authors raise some limitations on the data sources, namely the difficulty in mapping some of the listed sources and the need for integrating several sources.
In the following, we present different studies on vulnerability data that is used to measure and manage the security of computer systems.


Jumratjaroenvanit and Teng-amnuay~\cite{Jumratjaroenvanit:2008} described vulnerability life-cycles and presented their different stages. \side{International Symposium on Electronic Commerce and Security 2008}
The authors' goal was to understand how the different life-cycle dates can be useful to estimate the probability of an attack. 
The authors' methodology was to collect and analyze data from public data sources on the discovery-, disclosure-, exploit-date and exploit-, patch-availability. 
They used this information to define five life cycle types: (i) \gls{zda}, which typically is done by a black-hat and is defined by the date of disclosure and exploit date being the same; (ii) \gls{pzda}, is a \gls{zda} but with a patch already available, but not applied; 
(iii) \gls{ppzda}, is a \gls{pzda} but does not matter if the patch is available or not; 
(iv) \gls{poa}, it is a zero-day vulnerability. 
The most influential variables were the code that was scrutinized by the tester with access to the source code, and the code that was analyzed with some static analysis tool. 
The results showed that language safety was the less influent factor. 
The main result of the study was that two weeks of work were enough to have a 50\% chance of finding a zero-day vulnerability. 
Sometimes, 53 hours were enough to find one vulnerability with more than 95\% of probability.


Frei~\etal{}~\cite{Frei:2010} made a quantitative analysis of 27k vulnerabilities disclosed in the last decade.\side{Economics of Information Security and Privacy 2010}
The authors propose a model that describes the leading players in a vulnerability lifecycle.
All the dates associated with the vulnerability lifecycle are described in detail before the analysis.
They used a few public vulnerability databases (e.g., \gls{nvd} and \gls{osvdb}) to collect the different data attributes.
The results show that the number of patches increases after the vulnerability's disclosure.  
One interesting result is the \emph{the gap of insecurity} measurement, where they compare the patch availability against the exploit availability. 
The results show that exploits are always ahead of patches, in some cases after 30 days of the disclosure the exploits outnumber the number of patches by 90\%.


Shahzad~\etal{}~\cite{Shahzad:2012} obtained data between 1988 to 2011 from three data sources: \gls{nvd}, \gls{osvdb}, and the data from Frei and colleagues~\cite{Frei:2006}. \side{Proceedings of the International Conference on Software Engineering 2012}
They collected vulnerabilities that affected popular applications like Internet Explorer, Firefox, and Chrome, and popular \glspl{os} like Windows, Mac OS X, Solaris and several Linux distributions. 
The authors found that 2006 was the peak of vulnerability disclosures, and since then there was a decrease even if there are more software products available. 
However, the complexity of the disclosed vulnerabilities has increased, based on the \gls{cvss} score. 
From their dataset, 2.8\% vulnerabilities have an exploit released before their public disclosure. 
More dramatic is the percentage of exploits that are released on the day of the vulnerability disclosure, which is in the order of 88.2\%. 
The authors have found that 9.7\% of the vulnerabilities have exploits released after their disclosure
For example, as stated in~\cite{Gorbenko:2011}, the software’s popularity can be attractive to attackers. 
In this study, the authors found that Microsoft and Apple have more exploits before the vulnerability disclosure than the other vendors. 
This may primarily be because hackers find it more rewarding to exploit these products due to their broader market capitalization.
To what concerns patching, the authors argue that 10\% of the vulnerabilities had a patch before their disclosure, and 62\% had a patch on its disclosure day. 
There was a considerable number of vulnerabilities that were disclosed only after a patch was made, more precisely 28\%. 
However, the authors noticed that from 2007 there was an improvement from the vendors to respond to vulnerabilities with earlier patches. 
Nevertheless, the closed-source vendors are more efficient to patch vulnerabilities. 
Another conclusion is that it is easier to exploit opensource code vulnerabilities than in closed-source – this conclusion may seem obvious, but no one had shown it previously with statistical tests.

Bilge and Dumitras~\cite{Bilge:2012} made a systematic study on zero-day attacks between 2008-2011.\side{CCS 2012}
In their study, the main dataset is the \gls{wine} (developed by Symantec), it samples and aggregates data from several hosts running Symantec products (e.g., Norton Antivirus).
The \gls{wine} data is correlated with \gls{osvdb} and Symantec's Threat Explorer for attack/exploit information.
Their method allowed them to measure the duration of zero-day attacks.
They conclude that the average attack lasts ten months. 
Moreover, they have found that another threat has employed some of the vulnerabilities exploited during the Stuxnet in another attack two years before Stuxnet.
One of the main conclusions of this work is the evidence that the disclosure of zero-day vulnerabilities increases the risk, up to five orders of magnitude, of users being attacked.
The vendors are responsible for prioritizing the disclosure of such vulnerabilities and preparing a patch as soon as possible. 


Holm~\cite{Holm:2014} made an analysis of malware alarms to verify if there is a statistical distribution to model the number of intrusions or the time to compromise a computer system.\side{TDSC 2014}
The author collected information from different software configurations from 2009 to 2012 from an IT enterprise.
The enterprise had a total of 697 \glspl{os} and versions.
Each computer is equipped with anti-malware software, and when malware is detected, it sends the logging data to a central database.
The author's methodology applied a few filters to reduce the size of the data by removing redundant information (e.g., a single malware can generate several alarms in the systems).
The results show that the Pareto distribution is the one that best fits for the time of the first compromise.
Another impressive result is that the more the system is intruded, the more vulnerable it becomes. 
The author presents some hypothesis for this result, for example, once the system is broken it can no longer be trusted, but it seems odd to conclude something about how vulnerable the system is. 
Another and more plausible hypothesis is that the security awareness in the only temporary increased as a result of the intrusion, and then it is again neglected.
Most of the malware considered in this study was not result of targeted attacks but rather automatic ones, this is probably due to the type of enterprise. 
This may change the results if applied to targeted attacks as they require a different strategy to penetrate the system.


Wang~\etal{}~\cite{Wang:2014} proposed a different approach to measure the risk. \side{TDSC 2014}
Contrary to the most common approaches that attempt to measure the vulnerability state of the system, the authors explored how many zero-day vulnerabilities were required to compromise a network. 
As they do not consider replication, but instead a ``chained'' architecture, the use of diversity enhances the security of the overall network, but each service in the network is not dependable.
One of the limitations of this approach is that it considers all zero-day vulnerabilities equally likely to happen.
Moreover, the metric's calculation is complex, and no evaluation is provided to hint to the reader on how long can it take to calculate.
%Bopche and Mehtre~\cite{Bopche:2015} propose a very similar work.

Poolsappasit~\etal{}~\cite{Poolsappasit:2012} propose a security risk assessment method that incorporates the cause-effect relationships of the network states and the likelihood of exploiting such relationships.\side{TDSC 2012}
To do that, they measure the organizations' security risk using \gls{cvss} metric.
The authors implemented a tool that generates a map of the network as a Bayesian attack graph, then it collects data with a vulnerability scanner, and it fetches the vulnerability exposures for each vulnerability.
The result is helpful for system administrators that can have an overview of the network weaknesses and associate a cost to the assets in a way that countermeasures can be taken to avoid such weaknesses.
The countermeasures are indicated by the system administrator that will link them to the attributes model.
Then, the administrator assigns the damage costs to every attribute.
Finally, a genetic algorithm evolves the network for states where the cost of losing assets (i.e., being compromised) is minimal. 
Their empirical results show that they can minimize the costs of being compromised by assessing the security of the system and applying the most cost-efficient countermeasures.
Nevertheless, and besides the manual setup that the system administrator needs to perform, a time evaluation on how the system reacts when an attack is detected is missing.


Nappa~\etal{}~\cite{Nappa:2015} analyzed the patch deployment process over ten popular client applications over a five-year period, in some cases, which applications share libraries, the patching process takes different time to the same vulnerable shared code. \side{Security and Privacy 2015}
The authors have found that there is a strong correlation between the vulnerability disclosed and the start patching process among vendors, for 77\% of the studied vulnerabilities it occurs within seven days. 
They performed a \emph{survival analysis} (inspired in medicine and biology to measure the mortality rates associated with diseases) to measure the probability of a vulnerable host remains vulnerable beyond a specific time.
The vulnerability ``dies'' once a patch is installed or a non-vulnerable software is installed.
The experimental results show that using real-world WINE exploits still compromise 50\% of the vulnerable hosts (after vulnerabilities disclosure). 
The median percentage of hosts that have been patched before the exploit is released is at most 14\%.
Together with the number of shared vulnerable code with different patching delays, these values motivate attackers to re-used already known exploits or to create patched-based exploits~\cite{Brumley:2008}.
The authors also show evidence to support the natural intuition that silent or automatic updates reduce the survivability of vulnerabilities. 
They also performed an analysis on three different user profiles that show that security analysts are more careful on patching software than software developers or regular users.


%CVSS
Bozorgi~\etal{}~\cite{Bozorgi:2010} used machine learning techniques to classify existent vulnerabilities and predict future exploits.\side{KDD 2010}
Their results show that their trained classifiers outperform current exploitability measures like \gls{cvss} exploitability subscore.
They have built a dataset with two public databases Mitre' \gls{cve} (which is subpart of \gls{nvd}) and \gls{osvdb}.
They used the bag-of-words, a machine learning technique, to extract the textual attributes that are more relevant to analyze.
They used \gls{svm} to train the model and conduct offline and online experiments.
In the offline setting, which they train and evaluate data in a static fashion, they achieve an accuracy of 90\% of correct predictions.
In the online setting, the results show that after a small period the model stabilizes and reduces the overall error by 14\%.
In an online setting, they were able to make predictions up to two days before the exploit with an accuracy of 75-80\%.
The main result, for the interest of this thesis, is the comparison between their model and the \gls{cvss} as an indicator of how a vulnerability is likely to be exploited.
The \gls{cvss} assigns high scores to vulnerabilities that did not become exploited. 


Allodi and Massacci~\cite{Allodi:2014} made a in-depth analysis of four security databases (e.g., \gls{nvd}, Exploit-DB, SYM, and EKITS). \side{ACM Transactions on Information and System Security 2014}
The authors made a in deph \emph{break down} of \gls{cvss} on its several attributes.
For example, the exploitability subscore \emph{resembles more a constant than a variable}, thus not having a real impact on the \gls{cvss} overall score.
They used a randomized case-control study, which is applied to different study domains.
It is used to assess the effectiveness of treatment over a sample of subjects.
In this study, they measured how \gls{cvss} influences the \emph{risk factor} of the study cases.
A result that is worthy of note is that fixing vulnerabilities based on their \gls{cvss} is statistically equivalent to picking a random vulnerability to fix, from the security relevance standpoint.
On the contrary, the authors suggest that it is more critical to fix vulnerabilities that appear in the black market of vulnerabilities.
Their results indicate that the inclusion of the black market information as a \emph{risk factor} can increase the risk reduction up to 80\%.
However, the authors also point out that using the EKITS data source may be a threat to the validity of the study, as it is unstructured data and therefore it requires manual analysis.

Sabottke~\etal{}~\cite{Sabottke:2015} presented a study that used non-usual data sources on these type of studies.\side{Usenix Security 2015}
In particular, they used Twitter data (e.g., specific words, the number of retweets, replies, and information about the users posting these messages) to find information about exploits. 
The authors made a quantitative and qualitative exploration of the vulnerability-related information disseminated on Twitter.
They developed a Twitter-based exploit detector to provide an adequate response in such short time frame. 
This allows the security community to foresee the exploit activity before the information reaches the \emph{de facto} disclosure data sources like \gls{nvd} and ExploitDB.
However, to complement and strengthen their information results they also used sources like \gls{nvd}, \gls{osvdb}, Exploit-DB, Microsoft Security Advisories and Symantec WINE. 
The authors distinguished the real-world exploits from the proof-of-concept exploits, as the latter are by-products of the disclosure process. 
The real-world exploits typically are not known until a critical zero-day attack occurs. 
The authors used \gls{svm} to classify exploits in the social media and tested how robust it was to attacks, i.e., if a powerful attack could manipulate the information on Twitter with false accounts and false data. 
The results showed that with their proposal they could be confident about the predictions. 
For example, the \gls{svm} classifier set to a precision of 25\% could detect the Heartbleed exploits within 10 minutes of its first appearance on Twitter. 
They also showed that organizations like \gls{nvd} overestimate the severity using \gls{cvss} of some vulnerabilities that never become in fact exploited.


Gorbenko~\etal{}~\cite{Gorbenko:2017} used \gls{cve} and \gls{nvd} databases to study vulnerability life-cycles of different \glspl{os}.\side{ISSRE 2017}
The authors made a particular effort to analyze the relation between vulnerabilities' discovery and their fix, more precisely the time it takes to vendors patch the vulnerable software.
Moreover, they also studied the existent of common vulnerabilities among different \glspl{os}.
The results of their study show that between 2012 and 2016 most of \glspl{os} were patched for every vulnerability that was disclosed in that period, except Ubuntu Server 12.04.
Another interesting result is the results for \emph{forever-day vulnerability} (i.e., vulnerabilities that are publicly disclosed but not yet patched). 
They show that, for the analyzed period, Ubuntu never had a single day free from vulnerabilities, and Windows and Red Hat had only 12 and 10, respectively.
However, the most important results are the implications of \gls{cvss} on the \emph{days-of-gray-risk} (i.e., the number of days it takes to a vendor releases a patch). 
The authors show that there is no obvious implication on the level of \gls{cvss} severity and the time it takes to a vendor develop a patch.



\paragraph{Summary.} 
We presented several works that carry out for risk assessment in a way to prevent exploits or to take action upon vulnerability detection. 
This is the last building block that intrusion tolerance is missing. 
There is a lot of relevant free information available to address the security and dependability of software. 
In a replicated context this information is even more relevant. 
First, to react upon vulnerability/exploit disclosures, and second to select what configurations are less vulnerable to common weaknesses. 
We want to explore the free available data to improve the dependability and security of intrusion-tolerant systems, by ensuring the assumption of failure independence with evidence supported by relevant and sound data.
Additionally, some of the works proposed solutions for reliability and security of organization networks. 
In this thesis, we are concerned with the security and dependability of replicated systems. Hence we are focused on assuring that no more than $f$ replicas become compromised.



\section{Management of Vulnerable Systems}
In the previous sections, we presented a few works on recovery and diversity of replicas, some of which in \gls{bft} replicated systems.
In this section, we present some works that already combine in a very limited way, the idea of diversity and rejuvenations, and some more recent works already introduce the idea of risk management too (see~\cite{Yuan:2014} for a complete survey).


Reynolds~\etal{}~\cite{Reynolds:2002} presented \textsc{Hacqit}, fault-tolerant architecture. \side{DSN 2002}
Is built from out-of-bound networks and all connections go through \gls{vpn}, all the components were configured correctly and physically isolated.
It employs redundancy and diversity to detect and mask errors.
The same input is sent to identical nodes, and if the output is different, then some node is faulty. Thus it is possible to vote the majority of equal outputs as correct.
Moreover, it uses a forensic agent that analysis the logs of failed nodes to prevent bad requests in the future.
Finally, it implements a recovery mechanism that is triggered by the native security auditing of Windows NT/2000.
Unfortunately, no experimental evaluation is presented, nevertheless \textsc{Hacqit} is one of the first systems to promote all these mechanisms together.


Wang~\etal{}~\cite{Wang:2003} proposed an architecture, named \textsc{Sitar}, that aims to build intrusion-tolerant systems with diversity and dynamic reconfigurations. \side{DARPA Information Survivability Conference and Exposition 2003}
\textsc{Sitar} protects a set of potentially vulnerable \gls{cots} servers. 
The system is protected by an intrusion-tolerant multilayer architecture that uses redundancy and security protocols to guarantee that all nodes of the architecture correctly agree on the decisions.
It is composed of three main components, the proxy servers, the acceptor monitors and the ballot monitors.
These proxies receive the requests to audit them based on the current threat level, and then they propagate the messages to the \gls{cots} servers. 
On the way back, the acceptance monitors validate the response and run an agreement protocol to select a final response from the \gls{cots} servers.
There is an adaptive reconfiguration module that monitors the other modules, receiving heartbeats and intrusion triggers to take action on removing compromised modules and starting new ones.
No results on diversity were shown.


Arsenault~\etal{}~\cite{Arsenault:2007} presented a \gls{scit} a centralized architecture to support node cleansing inside a cluster.\side{International Conference on Availability, Reliability and Security 2007}
\gls{scit} goal is to clean the elements of a cluster without harming the availability of the overall service.
Recovering the cluster's nodes allows for resetting the state and avoiding huge windows of vulnerability on the nodes.
Nevertheless, the \gls{scit} service itself can be harmed. Therefore the authors leverage on trusted hardware to offer additional guarantees to the system.
The architecture supported by trusted hardware and full isolation allows \gls{scit} to operate in predictably and continuously manner regardless of the presence of attacks.

Junqueira~\etal{}~\cite{Junqueira:2005} proposed \emph{informed replication}, it is an approach to design distributed systems that can survive catastrophes. \side{ATC 2005}
Their proposal focuses on the diversity the system's components as a way to avoid common pathogens that compromise the entire system.
The authors propose a new system model that maps different software in attributes of a node. Thus these attributes represent characteristics that should be distinct on each node to avoid common failures.
While modeling the system, some biases were introduced.
For instance, the authors consider that two different services that use the same port are vulnerable to the same vulnerability, or that the same service running with two different ports are not vulnerable.
They propose a heuristic to build configurations in linear time.
Its goal is to distribute the \glspl{os} and applications in different configurations in such a way that minimizes the number of common attributes that would compromise the configuration.
They propose two metrics a uniform that selects \glspl{os} uniformly and a weighted that selects \glspl{os} on their popularity.
The results show that their solution mitigates the number of pathogens that can compromise an entire configuration as its attributes are selected in order to maximize their difference.
The heuristic guarantees 0.99 probability of data survival on an attack of single- and double-exploit pathogens while needing only three and five copies respectively.


Zhai~\etal{}~\cite{Zhai:2014} propose a system that looks for independence failures on the cloud from the reliability standpoint. \side{OSDI 2014}
The system collects and audits data from the system components to evaluate their independence by identifying potential correlated failures.
The authors collected information about the network's hardware and software data.
Then, the different configuration dependencies are analyzed and ranked in risk groups as a way to see how the different dependencies may make the system fail.
The main limitation of this work is the motivation that clouds may not have to join such system, i.e., provide private data to be analyzed by external agents.
The work's main focus is the reliability of cloud systems, thus no evaluation was done on vulnerabilities of such systems.
The authors evaluated the efficacy of their solution on finding the minimal risk groups (with fewer dependencies) on sampling data to make faster decisions while losing some accuracy.


Seondong~\etal{}~\cite{Seondong:2017} propose a system that uses data from \gls{nvd} to select the best configurations of diverse \glspl{os} and web servers.
\side{** NOT TOP ** IEICE Transactions on Information and Systems 2017}
The author's algorithm minimizes the \gls{cvss} of shared vulnerabilities among the different software.
Although their experiments explore different configurations of software stacks, the decisions are based on the \gls{cvss}.
All the experiments were done in a simulator (i.e., CSIM 20) which difficult a fair analysis of the performance evaluation. 
Moreover, the resilience of the system was evaluated against \gls{dos} attacks that affect particular vulnerabilities in the software used.
Depending on the type of the \gls{dos} it can cause performance decreases without exploiting any vulnerability.
Moreover, the evaluation should reflect the mechanisms that protect the system from having $f+1$ replicas compromised.
In other words, their \gls{dos} evaluation does not show if it compromises the security and dependability properties. 
This is the work that most resembles one of the main contributions of this thesis. 
However, it suffers from some limitations that we will address in Chapter~\ref{chap:lazarus_design}.



\paragraph{Moving Target Defenses.} 
An emerging area, called \gls{mtd}, re-defines some of the already existent solutions with the goal of continuous change attack surfaces. 
\gls{mtd} employs techniques such as  \gls{vm} migration, different types of diversity, etc.
A few works have been published in \gls{mtd} dedicated workshops.
Moreover, some authors have made an effort to formalize this sub-area of intrusion tolerance~\cite{Zhuang:2014}.  \side{workshop 2014}

Okhravi~\etal{}~\cite{Okhravi:2014} presented \textsc{Talent} a framework for live migration of critical infrastructures across heterogeneous platforms. \side{Trans. Sec. Priv 2014}
The idea was to create a moving target that difficult the success of advanced targeted attacks. 
TALENT implements \gls{os}-level virtualization with containers, which allows the system to migrate the \gls{os} between machines periodically or upon detection of malicious activity. 
The virtualization was implemented with OpenVZ and LXC for Linux, Virtuozzo for Windows, and Jail for FreeBSD. 
The network was also virtualized, more precisely, the second layer of virtualization was used to migrate the IP address from one container to another. 
Even an established ssh session was preserved during the migration. 
Additionally, the state of the application also had to be migrated by employing a checkpointing technique. 
When all the programs were checkpointed, the state was saved and then was migrated by mirroring the filesystem. 
The filesystem synchronization took 98.7\% of the migration time. 
The authors decided to focus on optimizing the file system synchronization. 
In the optimized version, the filesystem synchronizes in periodic intervals by sending the differences to the destination. 
Therefore, the migration was made seamless to the application. 
TALENT also had a risk assessment, called operation assessment, which monitored and adjusted the diverse components using vulnerability information. 
However, there were almost no details on how the risk was measured and on how to adapt to the threats efficiently.


Hong and Kim~\cite{Hong:2015} classified \gls{mtd} into three categories (e.g., shuffle, diversity and redundancy) and assess the effectiveness of each one. \side{TDSC 2015}
The authors integrated the three solutions in \textsc{Harm} and evaluate how each technique improve the security of distributed systems.
The solution is built on top of \glspl{vm}, as each \gls{vm} is able to host a few \glspl{os} in it.
Then, the evaluation scenarios consider three hosts are running from two to three \glspl{vm} inside.
The three techniques were evaluated separately, and they consider only two \glspl{os} on the diversity evaluation.
In the presented scenarios, deploying random diversity does not improve the risk of the system. However, according to their results employing diversity have some trade-offs while it may break the connectivity between clients and servers (i.e., in their architecture ideally each host should have at least one \gls{vm} with diverse \gls{os}. 
Otherwise, one host can be attacked and lose connectivity to the others).
However, these results are not surprising if only two \glspl{os} are considered. 

\paragraph{Summary.}
We presented a few works that propose control systems that manage nodes that are potentially vulnerable and exploitable.
Some of the works that we presented in other sections (e.g., Sousa~\etal{}~\cite{Sousa:2010} introduce reactive recovery upon detection of faulty behavior) already partially combined these mechanisms. 
However, only a few have implemented or discussed the real implications of using diversity (with the exceptions like BASE~\cite{Castro:2003} that implemented \gls{ots} diversity and assessed its performance overhead).


\section{Final Remarks}
In this chapter, we described the most relevant works that are related to this thesis.
We have made background revision of the intrusion tolerance area, and then we described the different topics that support intrusion-tolerant systems.
Our main contributions are on the management of \gls{bft} systems, in particular, how to create diverse replicas that effectively are independent and when the replicas need to be recovered. 
Although we could adopt solutions from artificial diversity, besides identified limitations~\cite{Snow:2013,Bittau:2014}, we are more interested in \gls{ots} diversity.
\gls{ots} diversity allow us to use data to make decisions on which diverse options work better together.
Such decisions would take action as rejuvenation triggers. 





\chapter{A Study on Common Vulnerabilities}
\label{chap:datasource}

In this chapter we answer the following question: \emph{What are the security gains from using diverse \glspl{os} on a replicated intrusion-tolerant system?}
To answer this question, we have analyzed to what extent different \glspl{os} share common vulnerabilities.
We made an analysis using a well-established database and devise three strategies to build diverse sets of \glspl{os}.
The promising results confirm, to some extent, the intuition that diversity implies vulnerability independence.

\section{Methodology}

This section presents the methodology adopted in this study, with a particular focus on how the dataset was selected, processed and analyzed.

%\subsection{Object of the study}

\subsection{Data Source}
We have analyzed \gls{os} vulnerability data from the \gls{nvd} database~\cite{nvd}. 
According to \gls{nist} a vulnerability is defined as follows:
%\gls{nvd} uses the \gls{cve} definition of vulnerability \cite{cveterm}, which is given below:

\begin{defn}
 \emph{``[...] A weakness in an information system, system security procedures, internal controls, or implementation that could be exploited by a threat source.''}~\cite{Nist:2012}
\end{defn}


%\gls{nvd} aggregates vulnerability reports from more than 70 security companies, forums, advisory groups and organizations,\footnote{See the complete list on \url{http://cve.mitre.org/compatible/alerts_announcements.html}.} being thus the most complete vulnerability database on the web.
%All data is made available as \gls{xml} files containing the reported vulnerabilities on a given period, called \emph{data feeds}.

\gls{nist}'s \gls{nvd}~\cite{nvd} is the authoritative data source for disclosure of vulnerabilities and associated information~\cite{Massacci:2010}. 
\gls{nvd} aggregates vulnerability reports from more than 70 security companies, advisory groups, and organizations, thus being the most extensive vulnerability database on the web. 
All data is made available as \gls{xml} data feeds, containing the reported vulnerabilities on a given period. 
Each \gls{nvd} vulnerability receives a unique identifier, in the format CVE-\textit{YEAR}-\textit{NUMBER}, and a short description provided by the \gls{cve}~\cite{cveterm}. 
The \gls{cpe}~\cite{cpe} provides the list of products affected by the vulnerability and the date of the vulnerability publication.
The \gls{cvss}~\cite{cvss} calculates the vulnerability severity considering several attributes, such as the attack vector, privileges required, exploitability score, and the security properties compromised by the vulnerability (i.e., integrity, confidentiality, or availability).


We developed a program that collects, parses and inserts the \gls{xml} data feeds into an SQL database, deployed with a custom schema to group vulnerabilities by affected products and versions.

\paragraph{Chosen Products.}
We are mostly interested in \gls{ots} components as they allow us to understand the correlation between vulnerabilities and products. 
Contrary to automatic diversity mechanisms that need testing to assess the vulnerability correltion.
In particular we focus on the diversity of \glspl{os} (not only the kernel, but the whole product) for three fundamental reasons: (1) by far, most of the replica’s code is the \gls{os}; (2) such size and importance, make \glspl{os} a valuable target, with new vulnerabilities and exploits being discovered every day; and (3) there are many options of OSes that can be used.



\subsection{Data Selection}
Despite the vast amount of information about each vulnerability available in \gls{nvd}, for this study, we are only interested in the name, publication date, summary (description), type of exploit (local or remote), and list of affected products.
We have collected vulnerabilities reported for 64 \gls{cpe}~\cite{cpe}.
Each one of these describes a system, i.e., a stack of software/hardware components in which the vulnerability may be exploited.
These \gls{cpe} were filtered, resulting in the following information that was stored in our database:

\begin{itemize}
\item \textbf{Part:} \gls{nvd} separates this in Hardware, \gls{os} and Application. For the purpose of this study we choose only enumerations marked as \gls{os};
\item \textbf{Product:} The product name of the platform;
\item \textbf{Vendor:} Name of the supplier or vendor of the product platform.
\end{itemize}

Those 64 CPEs were, by manual analysis, grouped in 11 \gls{os} distributions: \textit{OpenBSD, NetBSD, FreeBSD, OpenSolaris, Solaris, Debian, Ubuntu, Red Hat,\footnote{\textit{Red Hat} comprises the ``old'' Red Hat Linux (discontinued in 2003) and the newer Red Hat Enterprise Linux (RHEL).} Windows2000, Windows2003} and \textit{Windows2008}.
These distributions cover the most used \emph{\gls{os} products} of the BSD, Solaris, Linux and Windows families.



%We did not include Mac~OS~X because, unlike for the included OSes, we did not have an OS~X system available for cross-validation of vulnerability reports.

%\begin{figure}[!ht]
% \centering
% \includegraphics[width=0.9\columnwidth]{images/ea-crop.pdf}
% \caption{Simplified SQL schema of the database used to store and analyze the \gls{nvd} data. Fields not displayed are only used for auxiliary purposes.}
% \label{ea}
%\end{figure}

%The schema of the resulting database is displayed in Figure \ref{ea}.
%The tables with prefix \textit{cvss}, \textit{vulnerability\_type} and \textit{security\_protection} are employed to optimize the database.
%The most important tables are:

%\begin{itemize}
%\item \emph{cvss\_*} tables: refer directly to the Common Vulnerability Scoring System (CVSS) metrics \cite{cvss} of the stored vulnerabilities, which quantify the severity and impact of vulnerabilities;
%\item \emph{vulnerability}: stores basic data about a vulnerability (name, publication date, etc.) from the \gls{nvd} augmented with vulnerability life cycle information from the Open Source Vulnerability Database (OSVDB) \cite{osvdb};
%\item \emph{vulnerability\_type}: stores the vulnerability type assigned by us (see Section \ref{vulntypes});
%\item \emph{os}: stores the \glspl{os} platforms of interest in this study;
%\item \emph{os\_vuln}: stores the relationship between vulnerabilities and \glspl{os}, and their affected versions.
%\end{itemize}

The use of an SQL database brings at least three benefits when compared with analyzing the data directly from the \gls{xml} feeds.
First, it allows us to \emph{enrich the data set} by hand, for example, by associating release times and family names to each affected \gls{os} distribution.
Second, it allows us to modify the \gls{cve} fields to correct problems.
For instance, one of the problems with \gls{nvd} is that the same product is occasionally registered with distinct names in different entries;
For instance, (\textit{debian\_linux}, \textit{debian}) and (\textit{linux}, \textit{debian}) are two (product, vendor) pairs we have found for the Debian Linux distribution.
Other users of \gls{nvd} data feeds previously observed this same problem~\cite{cvedetails}.
Finally, an SQL database is much more convenient to work with than parsing the feeds on demand.

\subsection{Filtering the Data}\label{filtering_data}


From the more than 44,000 vulnerabilities published by \gls{nvd} at the time of this study, we selected 2563 vulnerabilities.
These vulnerabilities are the ones classified as \gls{os}-level vulnerabilities (``/o'' in their \gls{cpe}) for the \glspl{os} under consideration.

When manually inspecting the data set, we discovered and removed vulnerabilities that contained tags in their descriptions such as \emph{Unknown} and \emph{Unspecified}. 
These correspond to vulnerabilities for which \gls{nvd} does not know precisely where they occur or why they exist (however, they are usually included in the \gls{nvd} database because they were mentioned in some patch released by a vendor). 
We also found few vulnerabilities flagged as \emph{**DISPUTED**}, meaning that product vendors disagree that the vulnerability exists, and \emph{Duplicate}, used for vulnerabilities in which the \emph{summary} points to a duplicate entry or duplicate entry suspicion.
Due to the uncertainty that surrounds these vulnerabilities, we decided to exclude them from the study as well.

Table \ref{tab:unknowns} shows the distribution of these vulnerabilities across the analyzed \glspl{os}, together with the total number of valid vulnerabilities.

%TABLE 1
\begin{table}[!ht]
\begin{center}
{\scriptsize
\begin{tabular}{|l||c | c | c | c | c|}\hline
\textbf{OS} & \textbf{Valid} & \textbf{Unknown} & \textbf{Unspecified} & \textbf{Disputed} & \textbf{Duplicate}  \\\hline\hline % total
OpenBSD & 153 & 1 & 1 & 1 & 0 \\
NetBSD & 143 & 0 & 1 & 2 & 0  \\
FreeBSD & 279 & 0 & 0 & 2 & 0 \\
OpenSolaris & 31 & 0 & 52 & 0 & 0  \\
Solaris & 426 & 40 & 145 & 0 & 3  \\
Debian & 213 & 3 & 1 & 0 & 0  \\
Ubuntu & 90 & 2 & 1 & 0 & 0  \\
Red Hat & 444 & 13 & 8 & 1 & 1  \\
Windows2000 & 495 & 7 & 28 & 5 & 5  \\
Windows2003 & 56 & 5 & 34 & 3 & 5  \\
Windows2008 & 334 & 0 & 8 & 0 & 3   \\\hline\hline
\textbf{\#distinct vulns.} & 2270 & 63 & 210 & 8 & 12 \\ \hline
\end{tabular}
\caption{Distribution of OS vulnerabilities in NVD.}
\label{tab:unknowns}
}
\end{center}
\end{table}

An important observation about Table \ref{tab:unknowns} is that the columns do not add up to the number of distinct vulnerabilities (last row of the table) because some vulnerabilities are shared among \glspl{os} and are counted only once.
Notice that about 60\% of the removed vulnerabilities affected Solaris and OpenSolaris.
Moreover, these two systems are the only ones that have more than 10\% of its vulnerabilities removed.
We should remark that this manual filtering was necessary to increase the confidence that only valid vulnerabilities were used in the study.


\section{OS Diversity Study}\label{study}
This section analyzes the vulnerabilities that were shared in \gls{os} pairs over the period of 1994 to 2011. 
In the investigation, we consider two possible server machine setups offering an increasingly more secure platform. 
This accommodates the case where system administrators create differentiated \gls{os} installations, which contain more or less vulnerabilities depending on the services and applications available in the server. 
Of course, other setups could be used, but we decided to concentrate on these configurations because they are quite generic and they lead to results that can be directly obtained from the \gls{nvd} data. 
The setups are the following:

\begin{itemize}

\item \textbf{Fat Server:} The server contains most of the software packages for a given \gls{os}, and consequently, it can be used to run various kinds of applications by locally or remotely connected users. This server has potentially all vulnerabilities that were reported for the corresponding \gls{os};

%\item \textbf{Thin Server:} This setting corresponds to a platform that does not contain any applications (except for the replicated service). The server has a decreased security risk because the attack vectors related to applications have been mostly eliminated. 
%In this setting, vulnerabilities classified as \textit{Applications} are excluded from the statistics;

\item \textbf{Isolated Thin Server:} This setting corresponds to a platform that does not contain any applications (except for the replicated service). 
The server has a decreased security risk because the attack vectors related to applications have been mostly eliminated. 
Moroever, is placed in a room physically protected from illegal accesses, with remote logins disabled, and therefore it can only be compromised by receiving malicious packets from the network. In this setting, we only consider remotely exploitable vulnerabilities (those with ``Network'' or ``Adjacent Network'' values in their CVSS\_ACCESS\_VECTOR field) that are not classified as \textit{Applications}.

\end{itemize}

The Fat Server setup corresponds to a case where an attack may target an application available in the \gls{os} distribution. 
Therefore, it provides an upper bound estimate on the number of vulnerabilities that can be exploited. 
The Isolated Thin Server setups is the counterpart case, where the system is deployed without applications that come bundled with the \gls{os} and thus provide a lower bound estimate on the number of common vulnerabilities and it has the vulnerabilities that can be remotely exploited. 
Of course, in practice, at least one application (such as a name service or a distributed file service) would be deployed in an intrusion-tolerant system, and the vulnerabilities of this application would add up to the flaws that we will report. 
In any case, since we expect that most of the complexity is in the \gls{os}, we anticipate that a single application will have a small contribution to the overall number of vulnerabilities.



Intrusion-tolerant systems are usually built using four or more replicated servers~\cite{Castro:2002}. 
Therefore, it is useful to understand if there are many vulnerabilities that involve groups of \glspl{os} larger than two -- if this is the case, then it will be hard to find \gls{os} configurations for the various servers that do not share vulnerabilities, and a goal of using \gls{ots} diversity will be difficult to be achieved in practice. 
Figure \ref{top} portrays the number of common vulnerabilities that exist simultaneously in increasingly more extensive sets of \glspl{os} (with Isolated Thin Servers). 
The graph shows a rapid decrease in the shared vulnerabilities as the number of \glspl{os} grows. 
The largest group that was affected by the same vulnerability had six \glspl{os}, and this occurred only for a single flaw, and there were also two vulnerabilities in sets with five \glspl{os}. Vulnerabilities in two and three \glspl{os} usually occur in systems from the same family, whose common ancestry implies the reuse of more significant portions of the code base.
%As we have seen in previous tables, this is particularly true for the Windows and BSD families.

\begin{figure}[!h]
 \centering
 \includegraphics[]{images/gnuplot/spe/top/top.pdf}
 \caption{Common vulnerabilities for $n$ different OSes (with Isolated Thin Servers).}
 \label{top}
\end{figure}


Table \ref{tab:spreaded_vulns} lists in more detail the vulnerabilities that can be exploited in larger groups (four to six) of \glspl{os}. 
The first three bugs have a considerable impact because they allow a remote adversary to run arbitrary commands on the local system using a high-privilege account. 
They occurred either in widespread login services (telnet and rlogin) or in a primary system function, and consequently, several products from the BSD family were affected, as well as Solaris. 
The \gls{nvd} entry for the CVE-2001-0554 vulnerability also had external references to the Debian and Red Hat websites, which could indicate that these systems might suffer from a similar (or the same) problem. 
Vulnerability CVE-2008-1447 occurs in a large number of systems because it results from a bug in the BIND implementation of the \gls{dns}. Since BIND is a highly popular service, more \glspl{os} could potentially be affected. 
A closer look at the corresponding \gls{nvd} entry reveals external references to the sites of OpenBSD, NetBSD, and FreeBSD, indicating that they would be vulnerable in case this software was being used. 
This sort of vulnerability confirms that from an intrusion-tolerance perspective it is unwise to run the same server software everywhere, and that diverse implementations must be selected (in this case for recursive \gls{dns} servers).


\begin{table}[!ht]
\begin{center}
{\scriptsize
\begin{tabular}{|c||c| p{7,8cm} | }\hline
\textbf{CVE} & \#affected OS  & Description  \\\hline\hline
CVE-1999-0046  &  4  & Buffer Overflow of \emph{rlogin} allows admin access. Affects: NetBSD, FreeBSD, Solaris and Debian. \\ \hline
CVE-2001-0554  &  4  & Buffer overflow in telnetd (telnet daemon) allows remote attackers to execute arbitrary commands. Affects: OpenBSD, NetBSD, FreeBSD and Solaris.  \\ \hline
CVE-2003-0466  &  4  & Off-by-one error in the Kernel function \emph{fb\_realpath()} allows admin access. Affects: OpenBSD, NetBSD, FreeBSD and Solaris.  \\ \hline
CVE-2005-0356  &  4  & TCP implementations allow a denial of service (DoS) via spoofed packets with large timer values, when used with Protection Against Wrapped Sequence Numbers. Affects: OpenBSD, FreeBSD, Windows2000 and Windows2003. \\ \hline
CVE-2008-1447  &  5  & The BIND 8 and 9 implementation of the DNS protocol allow attackers to spoof DNS traffic via a birthday attack to conduct cache poisoning against recursive resolvers. Affects: Debian, Ubuntu, Red Hat, Windows2000 and Windows2003.  \\ \hline
CVE-2001-1244  &  5  & TCP implementations allow a DoS by setting a maximum segment size very small to force the generation of more packets, amplifying network traffic and CPU consumption. Affects: OpenBSD, NetBSD, FreeBSD, Solaris and Windows2000. \\ \hline
CVE-2008-4609  &  6  & TCP implementation allows a DoS via multiple vectors that manipulate TCP state table. Affects: OpenBSD, NetBSD, FreeBSD, Windows2000, Windows2003 and Windows2008.   \\ \hline
\end{tabular}
}
\caption{Vulnerabilities that affect more than four OSes.}
\label{tab:spreaded_vulns}
\end{center}
\end{table}


The remaining three vulnerabilities are related to the TCP/IP protocol stack. 
All of them affect system availability, allowing different forms of \gls{dos} attacks. 
CVE-2008-4609 is the vulnerability that affects more \glspl{os}, according to the \gls{nvd} database. 
Given that TCP/IP stack code is often reused across \glspl{os}, we checked the websites of the other \glspl{os} for reports related to this vulnerability. 
We only found a disclaimer by Red Hat~\footnote{\url{https://access.redhat.com/kb/docs/DOC-18730}} stating that this flaw affected some releases but that they would not provide an update (they only offered a mitigation solution based on IPtables, the Linux firewall software).

Overall, the above results look encouraging because over a significant period (around 18 years) there are very few vulnerabilities that appear in many \glspl{os}. 
A good portion of them are in the TCP/IP stack implementation, which is probably the most shared software component across \glspl{os}, but they only impact on the availability of the system, leaving confidentiality and integrity of data unharmed.


\section{Strategies for OS Diversity Selection}\label{evaluation}
The prelimary analysis indicate that it is possible to find that some \glspl{os} share less vulnerabilities than others. 
Therefore, in this section we present three alternative strategies to select \glspl{os}, based on the built dataset, to decrease the chance of common vulnerabilities on replicated systems. 
We start the section by first describing an approach we called the \gls{cvi}, which is used by one of the strategies. 
For each strategy, we then give example sets of \glspl{os} that exhibit a reasonable level of diversity, and perform an evaluation based on the collected data. 
Finally, we conclude the section with an analysis of potential diversity between releases of the same \glspl{os} (concentrating on the \glspl{os} that the three strategies picked as the best configuration).


\subsection{Common Vulnerability Indicator}\label{cvi}
In a previous work~\cite{Garcia:2012} we have found that the number of common vulnerabilities that are observed between different \glspl{os} varies over time. 
Thus, in order to be able to evaluate which \gls{os} pairs are more (or less) likely to experience common flaws while taking into account the timing of vulnerability disclosures, we developed a new metric, called the \emph{\gls{cvi}}. 
This indicator is calculated for a given year $y$, based on the vulnerabilities that were shared by \glspl{os} A and B over a period of $\mathit{tspan}$ previous years. \gls{cvi} is built to ensure the following desirable properties:
\begin{enumerate}
\newcommand{\OLDtheenumi}{\theenumi}
\renewcommand{\theenumi}{\roman{enumi}}
\item $\mathit{CVI}_y(A,B) = 0$ if A and B have no common vulnerabilities in the $\mathit{tspan}$ interval;
\item $\mathit{CVI}_y(A,B) < \mathit{CVI}_y(C,D)$ if A and B shared less vulnerabilities than C and D in each year of the considered period;
\item $\mathit{CVI}_y(A,B) < \mathit{CVI}_y(C,D)$ if A and B had $N$ common vulnerabilities in the distant past while C and D had $N$ shared vulnerabilities more recently;
\item $\mathit{CVI}_{y_2}(A,B) < \mathit{CVI}_{y_1}(A,B)$, with $y_1 < y_2$, if the number of common vulnerabilities for A and B has decreased over the years.
\renewcommand{\theenumi}{\OLDtheenumi}
\end{enumerate}

Therefore, \gls{cvi} is useful for comparison purposes, allowing the identification of \gls{os} pairs that have a smaller number of common flaws, while considering the instant when vulnerabilities were found. 
This last point is particularly crucial because \glspl{os} are continually evolving, potentially getting more (or less) diverse, and consequently, one should take into consideration the time dimension when selecting \gls{os} configurations (e.g., \gls{os} pairs that have had less common flaws recently are likely better candidates).
\gls{cvi} is computed as follows:

First, a weighting factor $\alpha_{i}$ is defined for each year   $i \in \{y-\mathit{tspan}+1, ..., y-2, y-1, y\}$.

\begin{equation}
\alpha_{i} = 1 - \frac{y-i}{\mathit{tspan}}
\end{equation}

Then, CVI is obtained using the number of vulnerabilities $v_{i}(A,B)$ that appeared in both \glspl{os} A and B for every year $i$ from the start of the time span up to reference year $y$.

\begin{equation}
\mathit{CVI}_{y}(A,B)= \sum_{i = y-\mathit{tspan}+1}^{y} \alpha_{i}\cdot v_{i}(A,B)
\end{equation}


\begin{table}[!ht]
\begin{center}
{\scriptsize
\begin{tabular}{|l|c c c|| c c c |}
\cline{2-7}
\multicolumn{1}{c}{} &  \multicolumn{3}{|c||}{\textbf{Fat Server}}  &  \multicolumn{3}{|c|}{\textbf{Isolated Thin Server}} \\ \hline
OS             & 2009 &	2010 & 	2011        & 	2009 & 	2010 & 2011 \\ \hline
OpenBSD-NetBSD  & 17.1 & 13.8 & 14.8        & 8.4 & 7.1 & 6.9   \\
OpenBSD-FreeBSD & 22.2 & 18.0 & 18.1        & 14.1  & 11.6 & 10.3  \\
OpenBSD-Solaris & 4.0 & 3.1 & 3.3           &  1.8 & 1.4 & 0.9  \\
OpenBSD-Debian  & 1.7 & 1.5 & 1.4           &  0.0 & 0.0 & 0.0  \\
OpenBSD-Red Hat & 5.0 & 4.1 & 3.2           &  1.6 & 1.4 & 1.1  \\
OpenBSD-Win2000 & 1.8 & 1.5 & -             &  1.8 & 1.5 & -  \\  \hline
NetBSD-FreeBSD  & 19.7 & 18.3 & 18.8        & 11.0 & 9.3 & 8.6 \\
NetBSD-Solaris  & 4.9 & 4.0 & 4.2           &  1.5 & 1.1 & 0.7  \\
NetBSD-Debian   & 1.4 & 0.9 & 0.5           &  0.6 & 0.4 & 0.2  \\
NetBSD-Red Hat  & 1.8 & 1.1 & 0.5           &  0.6 & 0.4 & 0.2  \\
NetBSD-Win2000  & 1.6 & 1.4 & -             & 1.6 & 1.4 & -   \\ \hline
FreeBSD-Solaris & 6.5 & 5.4 & 6.3           & 2.3 & 1.7 & 1.2  \\
FreeBSD-Debian  & 1.3 & 0.8 & 0.5           & 0.0 & 0.0 & 0.0  \\
FreeBSD-Red Hat & 5.6 & 4.4 & 3.3           & 1.7 & 1.4 & 1.1  \\
FreeBSD-Win2000 & 2.3 & 1.9 & -             & 2.3 & 1.9 & -  \\ \hline
Solaris-Debian  & 1.0 & 0.8 & 0.6           & 0.0 & 0.0 & 0.0  \\
Solaris-Red Hat & 5.3 & 6.5 & 5.5           & 1.0 & 0.7 & 0.5  \\
Solaris-Win2000 & 5.5 & 5.7 & -             & 1.0 & 0.7 & -  \\ \hline
Debian-Red Hat  & 26.1 & 20.5 & 15.7        & 3.2 & 2.1 & 1.4  \\
Debian-Win2000  & 0.9 & 0.8 & -             & 0.9 & 0.8 & -   \\ \hline
Red Hat-Win2000 & 1.8 & 1.6 & -             & 0.9 & 0.8 & -   \\ \hline
\end{tabular}
\caption{CVI for the years 2009 to 2011, with $\mathit{tspan}$ of 10 years.}
\label{tab:cvi-2011-2009}
}
\end{center}
\end{table}

Table~\ref{tab:cvi-2011-2009} presents \gls{cvi} values for the years 2009 to 2011. 
We have excluded from this analysis \glspl{os} with vulnerability information missing for more than one year over the period, to avoid using incomplete data in the calculation of the indicators. Therefore, the following \glspl{os} were not considered in Table~\ref{tab:cvi-2011-2009}: Windows2003, Windows2008, Ubuntu, and OpenSolaris. 
The \gls{cvi} values are computed for a $\mathit{tspan}$ of 10 years to reflect a reasonable history. 
It is possible to see that, except for one system (Solaris with FreeBSD/Red Hat/Windows2000 in the Fat Server), in all remaining cases \gls{cvi} shows a decreasing trend. 
Several of the \glspl{os} that have evolved over a considerable period are having less reported vulnerabilities, and this causes a decline in shared vulnerabilities in the recent years. 
For some of the \gls{os} pairs the drop in the \gls{cvi} value is quite significant, becoming almost one-third of the 2009 value (NetBSD with Debian or Red Hat). 
In the Isolated Thin Server case, there are three pairs with $\mathit{CVI}(A,B)=0$, which shows that they have shared no vulnerabilities during the past 12 years, and are thus particularly good candidates to include in intrusion-tolerant configurations. 
These systems are Debian with either OpenBSD or FreeBSD or Solaris.

%According to Table~\ref{tab:pairs_vulns_by_year}, both \gls{os} pairs had 10 common vulnerabilities in the 2000--2011 period, suggesting that they both provide the same degree of diversity. 
From the \gls{cvi} values in Table~\ref{tab:cvi-2011-2009}, it is apparent that OpenBSD and Red Hat have become more diverse, in recent years, than Red Hat and Solaris, and to make it advisable, from a diversity standpoint, to choose the former \gls{os} pair over the latter.



\subsection{Building Replicated Systems with Diversity}
\label{build_diversity}


This section describes three strategies for choosing diverse sets of \glspl{os}. 
These strategies utilize the data analyzed earlier as a basis to make decisions, by assuming that the information reported by \gls{nvd} on vulnerabilities can be correlated to the amount of diversity among \glspl{os}. 
Of course, there are some caveats associated with this approach, but the alternatives can be even harder to put in practice, especially if one wants to consider a large number of \glspl{os}. 
For example, for closed systems (e.g., Windows) it is challenging to determine the level of sharing of \gls{os} components, and therefore, diversity estimations based on the code cannot be performed. Moreover, even if this estimate could be obtained, there is the risk that it does not reflect the number of vulnerabilities that occur simultaneously in several \glspl{os} (e.g., two distinct implementations of the same flawed algorithm are vulnerable).

The first strategy, called \gls{cvcs}, is based on raw data collected over a considerable interval, and is the most straightforward approach for selecting \gls{os} pairs. 
It should be used when one wants to treat all vulnerabilities, regardless of the time which they were reported, as equally important to make choices. 
The second strategy, \gls{cvis}, uses the \gls{cvi} described in the previous section to select \gls{os} pairs taking into account the incidence of common vulnerabilities over the years. 
It is indicated when one wants to give greater importance to more recent vulnerabilities because it is a weighted sum. 
The third strategy, \gls{irts}, follows a different approach from the previous ones, focusing not so much on common vulnerabilities directly, but on the frequency in which vulnerabilities appear in the two OSes. If one wants to give more importance to the time interval between successive reports of common vulnerabilities, this is the best strategy. 
Since this last criterion complements the previous two, one could explore strategies \gls{cvcs} and \gls{cvis} in conjunction with \gls{irts}.

For every strategy, we present example \gls{os} sets for the Fat Server and Isolated Thin Server configurations. 
Fat Server configurations can be pessimistic in the sense that they may account for common vulnerabilities in applications that are not present in the servers, while Isolated Thin Server configurations reflect more accurately the expected setup of dedicated servers in a replicated system. 
An intrusion-tolerant system usually requires $3f+1$ replicas to tolerate $f$ intrusions (e.g., \cite{Castro:2002}). 
Therefore, we will focus on sets with four \glspl{os} to deploy a hypothetical replicated system with four replicas, which allows one fault to be tolerated.

As a cautious note, one should take into account that Ubuntu and Windows2008 were first released in 2004 and 2008 respectively, so the data for these two \glspl{os} was collected for a smaller number of years. 
Windows2000 is presented in the tables because there are published vulnerabilities until 2010, although it has been gradually replaced by Windows2003 and Windows2008 in the organizations. Consequently, we do not use Windows2000 when choosing the \gls{os} sets. 
We have excluded OpenSolaris from the study because there is data available for only a limited period. 
In each strategy and configuration we present two sets: \emph{setCon} is more conservative, since it does not contain Ubuntu and Windows2008; and \emph{setUpdt} is more up-to-date because it can include Ubuntu and Windows2008. 
When looking at these two sets, one should keep in mind that setCon is selected from a group of \glspl{os} for which there is a significant amount of \gls{nvd} data, which contributes for higher confidence on the result. 
On the other hand, setUpdt uses \glspl{os} with different amounts of \gls{nvd} data, which can cause small levels of inaccuracy when making comparisons (e.g., in the \gls{cvcs}, any \gls{os} pair featuring Windows2008 has zero common vulnerabilities until 2007). 
One way to address this would be only to consider vulnerabilities that appear later than 2007 when choosing setUpdt. 
We opted not to follow this approach because it has the drawback of discarding too much data.


\subsection{Common Vulnerability Count (CVCst)} 

The results from the previous section give a strong indication that it should be possible to choose groups of \glspl{os} with few common vulnerabilities over reasonable intervals of time. 
However, we would like to understand if the data from the \gls{nvd} database is effective at suggesting these groups of \glspl{os}. 
To address this point, we divided the data into two subsets: the \emph{history period}, comprising the data for the interval between 2000 to 2009, and the \emph{observed period}, from 2010 to 2011. 
The objective is to employ the historical period to pick the sets of \glspl{os} to use on the replicated system (as if the choice was made at the beginning of 2010). Then we use the data for the observed period to verify if these choices would have been adequate, i.e., if they have a small (preferably the smallest) number of common vulnerabilities in this period.

\gls{cvcs} makes decisions based directly on the empirical data for the number of common vulnerabilities across all \gls{os} pairs. 
This data is displayed in Table \ref{tab:strat_i} for \glspl{os} with a Fat Server configuration. 
Numbers to the right and above the diagonal line represent the history period, while numbers to the left and below the line stand for the observed period. 
For example, the entry corresponding to OpenBSD-Red Hat to the right of the diagonal line has the number 10, which means that these \glspl{os} shared 10 vulnerabilities between 2000 and 2009. 
The equivalent entry, but to the left of the diagonal line, is 0 because they had no common flaws reported in 2010 and 2011. 
As expected, \gls{os} pairs from the same family had the highest counts of common vulnerabilities. 
The only case where there were more vulnerabilities in the observed period than the historical period is for the Windows2008--Windows2003 pair, which is explained by the recent release date of Windows2008. 
It is interesting to notice, however, that most pairs had zero common vulnerabilities in the observed period.


\begin{table}[!ht]
\begin{center}
{\footnotesize
\begin{tabular}{|l|c|c|c|c|c|c|c|c|c|c|c|c|}\cline{2-11}
 \multicolumn{1}{c|}{} &
\begin{sideways}\vspace{2mm}\parbox{2mm}OpenBSD\end{sideways}&
\begin{sideways}\vspace{2mm}\parbox{2mm}NetBSD\end{sideways}&
\begin{sideways}\vspace{2mm}\parbox{2mm}FreeBSD\end{sideways}&
\begin{sideways}\vspace{2mm}\parbox{2mm}Solaris  \end{sideways}&
\begin{sideways}\vspace{2mm}\parbox{2mm}Debian\end{sideways}&
\begin{sideways}\vspace{2mm}\parbox{2mm}Ubuntu \end{sideways}&
\begin{sideways}\vspace{2mm}\parbox{2mm}Red Hat\end{sideways}&
\begin{sideways}\vspace{2mm}\parbox{3mm}Win2000 \end{sideways}&
\begin{sideways}\vspace{2mm}\parbox{3mm}Win2003 \end{sideways}&
\begin{sideways}\vspace{2mm}\parbox{3mm}Win2008  \end{sideways}&
\multicolumn{2}{|c}{} \\ \cline{1-11}  \cline{1-12}
OpenBSD & - & 33 & 43 & 9 & 2 & 3 & 10 & 3 & 2 & 1&   \multirow{13}{1mm}{\begin{sideways}\hspace{8mm}\parbox{15mm}{2000-2009}\end{sideways}} \\ \cline{1-11}
NetBSD & 4 & - & 36 & 9 & 4 & 0 & 6 & 3 & 2 & 1  &\\ \cline{1-11}
FreeBSD & 4 & 6 & - & 12 & 4 & 2 & 12 & 4 & 3 & 1& \\ \cline{1-11}
Solaris & 1 & 1 & 2 & - & 2 & 2 & 8 & 8 & 7 & 0 &\\ \cline{1-11}
Debian & 0 & 0 & 0 & 0 & - & 14 & 52 & 1 & 1 & 0 &\\ \cline{1-11}
Ubuntu & 0 & 0 & 0 & 0 & 0 & - & 27 & 1 & 1 & 0 &\\ \cline{1-11}
Red Hat & 0 & 0 & 0 & 2 & 0 & 0 & - & 2 & 2 & 0 &\\ \cline{1-11}
Win2000 & 0 & 0 & 0 & 1 & 0 & 0 & 0 & - & 216 & 42 &\\ \cline{1-11}
Win2003 & 0 & 0 & 0 & 1 & 0 & 0 & 0 & 49 & - & 53 & \\ \cline{1-12}
Win2008 & 0 & 0 & 0 & 1 & 0 & 0 & 0 & 38 & 229 & -	& \multicolumn{1}{|c}{}  \\ \cline{1-11}
 \multicolumn{1}{c|}{}& \multicolumn{9}{|c|}{2010-2011} & \multicolumn{2}{|c}{}\\ \cline{2-10}
\end{tabular}
\caption{History/observed period for CVCst with Fat Servers.}
\label{tab:strat_i}
}
\end{center}
\end{table}


The strategy for building sets of \glspl{os} is based on a simple cost function. 
Given any potential \gls{os} pair A and B that could be added to the set, one can perform a lookup in Table \ref{tab:strat_i} to determine the pair's number of common vulnerabilities in the historical period. 
This number corresponds to the cost of adding this \gls{os} pair to the group. 
Similarly, when including a third \gls{os} C, it is possible to find in the table the entries for A--C and B--C and take their sum as the cost of integrating C in the group. 
When building a set with $n$ \glspl{os}, the total cost is the addition of each individual cost for all combinations of \gls{os} pairs.
The sets that lead to smaller values of total cost are considered the best choices for deployment in the replicated system.
Accordingly, based on the table, the best groups of four \glspl{os} are:

\begin{itemize}
  \setlength\itemsep{0em}
\item setCon = \{\emph{OpenBSD, Solaris, Debian, Windows2003}\}, with a total cost of 23;
\item setUpdt = \{\emph{NetBSD, Solaris, Ubuntu, Windows2008}\}, with a total cost of 12.%, but one vulnerability affects more than two \glspl{os} simultaneously, hence two pairs.
\end{itemize}


One, however, should keep in mind that sometimes the total cost may be only an approximation of the actual number of shared vulnerabilities among the \glspl{os} in the set, as specific vulnerabilities might be counted more than once. 
This will likely not be a problem since overcounting vulnerabilities provides a conservative estimate, or an estimate worse than reality. 
For example, setUpdt only has 11 shared vulnerabilities for a total cost of 12, since one of the vulnerabilities appears in three of the \glspl{os} (and is therefore included in two table entries).

Next we can check to what extent our choice of the best group of four \glspl{os} that we would pick from the historical period (2000--2009), as prescribed by \gls{cvc} cost calculation, remains consistent with the choice of the best group of four \glspl{os} from the observed period (2010--2011). 
We see that both setCon and setUpdt have only two shared vulnerabilities. 
They are not the best sets in the observed period (2010--2011), since there are groups of \glspl{os} with zero common vulnerabilities in the observed period (e.g., by replacing Solaris with Red Hat), though they do exhibit a high level of diversity. 
A graphical representation of the sets as Venn diagrams is available in Figures~\ref{fig-venn}(a) and~\ref{fig-venn}(b). 
Below the \gls{os} name is the total number of vulnerabilities during the observed period, and the number inside each intersection shows the count of common flaws for the corresponding \glspl{os}. 
For example, in setCon with Fat Servers (Figure~\ref{fig-venn}(a)) there is one vulnerability that appears both on Solaris and OpenBSD and another on Solaris and Windows2003.


\begin{table}[!ht]
\begin{center}
{\footnotesize
\begin{tabular}{|l|c|c|c|c|c|c|c|c|c|c|c|c|}\cline{2-11}
 \multicolumn{1}{c|}{} &
\begin{sideways}\vspace{2mm}\parbox{2mm}OpenBSD\end{sideways}&
\begin{sideways}\vspace{2mm}\parbox{2mm}NetBSD\end{sideways}&
\begin{sideways}\vspace{2mm}\parbox{2mm}FreeBSD\end{sideways}&
\begin{sideways}\vspace{2mm}\parbox{2mm}Solaris  \end{sideways}&
\begin{sideways}\vspace{2mm}\parbox{2mm}Debian\end{sideways}&
\begin{sideways}\vspace{2mm}\parbox{2mm}Ubuntu \end{sideways}&
\begin{sideways}\vspace{2mm}\parbox{2mm}Red Hat\end{sideways}&
\begin{sideways}\vspace{2mm}\parbox{3mm}Win2000 \end{sideways}&
\begin{sideways}\vspace{2mm}\parbox{3mm}Win2003 \end{sideways}&
\begin{sideways}\vspace{2mm}\parbox{3mm}Win2008  \end{sideways}&
\multicolumn{2}{|c}{} \\ \cline{1-11}  \cline{1-12}
OpenBSD & - & 13 & 26 & 5 & 0 & 0 & 3 & 3 & 3 & 1&\multirow{13}{1mm}{\begin{sideways}\hspace{8mm}\parbox{15mm}{2000-2009}\end{sideways}} \\ \cline{1-11}
NetBSD & 1 & - & 18 & 4 & 2 & 0 & 2 & 3 & 1 & 1& \\ \cline{1-11}
FreeBSD & 1 & 1 & - & 6 & 0 & 0 & 3 & 4 & 2 & 1& \\ \cline{1-11}
Solaris & 0 & 0 & 0 & - & 0 & 0 & 2 & 2 & 1 & 0& \\ \cline{1-11}
Debian & 0 & 0 & 0 & 0 & - & 2 & 8 & 1 & 1 & 0 &\\ \cline{1-11}
Ubuntu & 0 & 0 & 0 & 0 & 0 & - & 1 & 1 & 1 & 0 &\\ \cline{1-11}
Red Hat & 0 & 0 & 0 & 0 & 0 & 0 & - & 1 & 1 & 0 &\\ \cline{1-11}
Win2000 & 0 & 0 & 0 & 0 & 0 & 0 & 0 & - & 81 & 13 &\\ \cline{1-11}
Win2003 & 0 & 0 & 0 & 0 & 0 & 0 & 0 & 4 & - & 14 &\\ \cline{1-12}
Win2008 & 0 & 0 & 0 & 0 & 0 & 0 & 0 & 3 & 26 & - & \multicolumn{1}{|c}{}  \\ \cline{1-11}
 \multicolumn{1}{c|}{}& \multicolumn{9}{|c|}{2010-2011} & \multicolumn{2}{|c}{}\\ \cline{2-10}
\end{tabular}
\caption{History/observed period for strategy CVCst with Isolated Thin Servers.}
\label{tab:strat_i_iso}
}
\end{center}
\end{table}


Table \ref{tab:strat_i_iso} presents the common vulnerabilities with the Isolated Thin Server configuration data. This configuration represents a class of servers that have a dedicated function and are protected against physical intruders. 
The same approach can be applied as above: first, we choose the best pairs based on the historical period to build a set of four \glspl{os}; next, we evaluate the sets by comparing the results with the values for the observed period. 
The history period provides two candidate sets: 

\begin{itemize}
\item setCon = \{\emph{NetBSD, Solaris, Debian, Windows2003}\}, with a total cost of 9;
\item setUpdt = \{\emph{Solaris, Debian, Ubuntu, Windows2008}\}, with a total cost of 2.
\end{itemize}

In the observed period, setCon and setUpdt have no common vulnerabilities, showing that the strategy would have chosen sufficiently diverse groups of \glspl{os}. A graphical representation of the sets is displayed in Figures~\ref{fig-venn}(c) and~\ref{fig-venn}(d).

\begin{figure}[!ht]
 \centering
 \includegraphics[scale=0.7]{images/images/grayscale_ven_dia_one_fig_update.pdf}
 \caption{Venn diagrams for vulnerabilities in setCon and setUpdt for strategies CVCst and CVIst in Fat Server and Isolated Thin Server configurations. The first four diagrams (a, b, c and d) represent the results for CVC, and the last four \glspl{os} (e, f, g and h) represent the results for CVI.}
 \label{fig-venn}
\end{figure}


\subsubsection*{Common Vulnerability Indicator Strategy (CVIst)} 
This strategy employs the \gls{cvi} value, defined at the beginning of this section, to make decisions about including/excluding particular \glspl{os}.
Therefore, besides taking advantage of the available data on total counts of shared vulnerabilities, it also uses the information on how these numbers have evolved through the years.

\gls{cvis} is applied by executing the following method, which is based on minimizing a cost function. 
For a given year and time span, the \gls{cvi} value is calculated for each of the \gls{os} pairs. 
Typically, one should use the most recent year for which there is available data. 
The time span should cover a reasonable interval so that the indicator reflects the trend of discovered vulnerabilities. 
In some cases, however, one may have to resort to smaller time spans due to lack of data, namely with \glspl{os} released recently. 
In this case, the indicator will give a higher weight to the vulnerabilities reported in the last year. 
In \gls{cvis} the cost of creating a group with two \glspl{os} A and B is $\mathit{CVI}(A,B)$. 
Extending this idea to a group of $n$ \glspl{os}, the total cost becomes the sum of the individual \gls{cvi} for all combinations of \gls{os} pairs.
In order to choose the best groups, the strategy searches for sets of \glspl{os} that together have the smallest total cost.


\begin{table}[!ht]
\begin{center}
{\footnotesize
\begin{tabular}{|l|c|c|c|c|c|c|c|c|c|c|c|c|}\cline{2-11}
 \multicolumn{1}{c|}{} &
\begin{sideways}\vspace{2mm}\parbox{2mm}OpenBSD\end{sideways}&
\begin{sideways}\vspace{2mm}\parbox{2mm}NetBSD\end{sideways}&
\begin{sideways}\vspace{2mm}\parbox{2mm}FreeBSD\end{sideways}&
\begin{sideways}\vspace{2mm}\parbox{2mm}Solaris  \end{sideways}&
\begin{sideways}\vspace{2mm}\parbox{2mm}Debian\end{sideways}&
\begin{sideways}\vspace{2mm}\parbox{2mm}Ubuntu \end{sideways}&
\begin{sideways}\vspace{2mm}\parbox{2mm}Red Hat\end{sideways}&
\begin{sideways}\vspace{2mm}\parbox{3mm}Win2000 \end{sideways}&
\begin{sideways}\vspace{2mm}\parbox{3mm}Win2003 \end{sideways}&
\begin{sideways}\vspace{2mm}\parbox{3mm}Win2008  \end{sideways}&
\multicolumn{2}{|c}{} \\ \cline{1-11}  \cline{1-12}
OpenBSD & - & 17.1 & 22.2 & 4.0 & 1.7 & 2.5 & 5.0 & 1.8 & 1.5 & 0.9 & \multirow{13}{1mm}{\begin{sideways}\hspace{8mm}\parbox{16mm}{$\mathit{CVI}_{2009}$(A,B)}\end{sideways}} \\ \cline{1-11}
NetBSD & 4 &  - & 19.7 & 4.9 & 1.4 & 0.0 & 1.8 & 1.6 & 1.5 & 0.9&\\ \cline{1-11}
FreeBSD & 4 & 6 & - & 6.5 & 1.3 & 1.3 & 5.6 & 2.3 & 2.0 & 0.9&\\ \cline{1-11}
Solaris & 1 & 1 & 2 & - & 1.0 & 1.5 & 5.3 & 5.5 & 5.6 & 0.0&\\ \cline{1-11}
Debian & 0 & 0 & 0 & 0 & - & 10.5 & 26.1 & 0.9 & 0.9 & 0.0&\\ \cline{1-11}
Ubuntu & 0 & 0 & 0 & 0 & 0 & - & 18.5 & 0.9 & 0.9 & 0.0&\\ \cline{1-11}
Red Hat & 0 & 0 & 0 & 2 & 0 & 0 &  - & 1.4 & 1.8 & 0.0&\\ \cline{1-11}
Win2000 & 0 & 0 & 0 & 1 & 0 & 0 & 0 & - & 163.6 & 39.5&\\ \cline{1-11}
Win2003 & 0 & 0 & 0 & 1 & 0 & 0 & 0 & 49 & - & 0.0 &\\ \cline{1-12}
Win2008 & 0 & 0 & 0 & 1 & 0 & 0 & 0 & 38 & 229 & - &\multicolumn{1}{|c}{}  \\ \cline{1-11}
 \multicolumn{1}{c|}{}& \multicolumn{9}{|c|}{2010-2011} & \multicolumn{2}{|c}{}\\ \cline{2-10}
\end{tabular}
\caption{History/observed period for CVIst with Fat Servers.}
\label{tab:strat_ii}
}
\end{center}
\end{table}

To evaluate this strategy, we split the time into two intervals as we did for \gls{cvcs}. 
Table~\ref{tab:strat_ii} presents in the cells for the history period the $\mathit{CVI}_{2009}(A,B)$ for a time span of 10 years in a Fat Server configuration\footnote{In some cases, we had to use smaller time spans due to the more recent release date of the \glspl{os} (e.g., Ubuntu and Windows 2008). When this happened, the CVI value was calculated using the maximum time span that is allowed by the available data.}. 
The cells at left and bottom of the diagonal line correspond to the observed period, and they count as before the number of shared vulnerabilities in 2010 and 2011. After applying \gls{cvis}, the following sets are the best with four \glspl{os}:

\begin{itemize}
\item setCon = \{\emph{OpenBSD, Solaris, Debian, Windows2003}\}, with a total cost of $14.7$;
\item setUpdt = \{\emph{NetBSD, Solaris, Ubuntu, Windows2008}\}, with a total cost of $7.3$.
\end{itemize}

To verify if \gls{cvi} is a good indicator for selecting diverse sets, we can look at the number of common vulnerabilities in the observed period (2010--2011). 
By analyzing Table~\ref{tab:strat_ii}, it is possible to see that setCon and setUpdt remain good sets, each one with two shared flaws. 
As before with CVCst, one can find better sets in observed period, where no vulnerabilities appear in common, for example by replacing Solaris with Red Hat.
The Venn diagrams for these two sets are shown in Figures~\ref{fig-venn}(e) and~\ref{fig-venn}(f).

\begin{table}[!ht]
\begin{center}
{\footnotesize
\begin{tabular}{|l|c|c|c|c|c|c|c|c|c|c|c|c|}\cline{2-11}
 \multicolumn{1}{c|}{} &
\begin{sideways}\vspace{2mm}\parbox{2mm}OpenBSD\end{sideways}&
\begin{sideways}\vspace{2mm}\parbox{2mm}NetBSD\end{sideways}&
\begin{sideways}\vspace{2mm}\parbox{2mm}FreeBSD\end{sideways}&
\begin{sideways}\vspace{2mm}\parbox{2mm}Solaris  \end{sideways}&
\begin{sideways}\vspace{2mm}\parbox{2mm}Debian\end{sideways}&
\begin{sideways}\vspace{2mm}\parbox{2mm}Ubuntu \end{sideways}&
\begin{sideways}\vspace{2mm}\parbox{2mm}Red Hat\end{sideways}&
\begin{sideways}\vspace{2mm}\parbox{3mm}Win2000 \end{sideways}&
\begin{sideways}\vspace{2mm}\parbox{3mm}Win2003 \end{sideways}&
\begin{sideways}\vspace{2mm}\parbox{3mm}Win2008  \end{sideways}&
\multicolumn{2}{|c}{} \\ \cline{1-11}  \cline{1-12}
OpenBSD & - & 8.4 & 14.1 & 1.8 & 0.0 & 0.0 & 1.6 & 1.8 & 1.8 & 0.9  & \multirow{13}{1mm}{\begin{sideways}\hspace{8mm}\parbox{16mm}{$\mathit{CVI}_{2009}$(A,B)}\end{sideways}} \\ \cline{1-11}
NetBSD & 1 & - & 11.0 & 1.5 & 0.6 & 0.0 & 0.6 & 1.6 & 0.9 & 0.9&\\ \cline{1-11}
FreeBSD & 1 & 1 & - & 2.3 & 0.0 & 0.0 & 1.7 & 2.3 & 1.5 & 0.9&\\ \cline{1-11}
Solaris & 0 & 0 & 0 & - & 0.0 & 0.0 & 1.0 & 1.0 & 0.6 & 0.0&\\ \cline{1-11}
Debian & 0 & 0 & 0 & 0 & - & 1.9 & 3.2 & 0.9 & 0.9 & 0.0&\\ \cline{1-11}
Ubuntu & 0 & 0 & 0 & 0 & 0 &  - & 0.9 & 0.9 & 0.9 & 0.0&\\ \cline{1-11}
Red Hat & 0 & 0 & 0 & 0 & 0 & 0 &- & 0.9 & 0.9 & 0.0&\\ \cline{1-11}
Win2000 & 0 & 0 & 0 & 0 & 0 & 0 & 0 & - & 58.7 & 12.5&\\ \cline{1-11}
Win2003 & 0 & 0 & 0 & 0 & 0 & 0 & 0 & 4 & - & 13.5&\\ \cline{1-12}
Win2008 & 0 & 0 & 0 & 0 & 0 & 0 & 0 & 3 & 26 & -&\multicolumn{1}{|c}{}  \\ \cline{1-11}
 \multicolumn{1}{c|}{}& \multicolumn{9}{|c|}{2010-2011} & \multicolumn{2}{|c}{}\\ \cline{2-10}
\end{tabular}
\caption{History/observed period for CVIst with Isolated Thin Servers.}
\label{tab:strat_ii_iso}
}
\end{center}
\end{table}




Table \ref{tab:strat_ii_iso} provides the data for applying CVIst in Isolated Thin Server configurations. 
It is possible to observe that \gls{cvi} values have significantly decreased when compared to the previous table. 
The best groups of four \glspl{os} in the historical period are:
 
\begin{itemize}
\item setCon = \{\emph{NetBSD, Solaris, Debian, Windows2003}\}, with a total cost of $4.5$;
\item setUpdt = \{\emph{Solaris, Debian, Ubuntu, Windows2008}\}, with a total cost of $1.9$.
\end{itemize}

By checking the data for the observed period, one can see that both sets do not share a single vulnerability, which indicates that the strategy would have made a good selection of \glspl{os} (see also the Venn diagrams in Figures~\ref{fig-venn}(g) and~\ref{fig-venn}(h)).

%TABLE VII

\begin{table}[!ht]
\begin{center}
{\scriptsize
\begin{tabular}{|l||c c c c c|}\hline
&	0 $\leq$ IRT $\leq 1$	&	$1$ \textless IRT $\leq 10$	&	$10$ \textless IRT $\leq100$	&	$100$ \textless IRT $\leq 1000$	& $1000$ \textless IRT $\leq10000$\\\hline
OpenBSD-Win2008 & 0 & 0 & 0 & 0 & 0 \\
NetBSD-Ubuntu & 0 & 0 & 0 & 0 & 0 \\
NetBSD-Win2003 & 0 & 0 & 0 & 0 & 0 \\
NetBSD-Win2008 & 0 & 0 & 0 & 0 & 0 \\
FreeBSD-Win2008 & 0 & 0 & 0 & 0 & 0 \\
Solaris-Win2008 & 0 & 0 & 0 & 0 & 0 \\
Debian-Win2000 & 0 & 0 & 0 & 0 & 0 \\
Debian-Win2003 & 0 & 0 & 0 & 0 & 0 \\
Debian-Win2008 & 0 & 0 & 0 & 0 & 0 \\
Ubuntu-Win2000 & 0 & 0 & 0 & 0 & 0 \\
Ubuntu-Win2003 & 0 & 0 & 0 & 0 & 0 \\
Ubuntu-Win2008 & 0 & 0 & 0 & 0 & 0 \\
Red Hat-Win2008 & 0 & 0 & 0 & 0 & 0 \\ \hline
OpenBSD-Win2003 & 0 & 0 & 0 & 0 & 1 \\
FreeBSD-Win2003 & 0 & 0 & 0 & 0 & 1 \\
Solaris-Debian & 0 & 0 & 0 & 0 & 1 \\
Red Hat-Win2000 & 0 & 0 & 0 & 0 & 1 \\
OpenBSD-Win2000 & 0 & 0 & 0 & 0 & 2 \\ \hline
OpenBSD-Debian & 0 & 0 & 0 & 1 & 0 \\
Solaris-Ubuntu & 0 & 0 & 0 & 1 & 0 \\
NetBSD-Win2000 & 0 & 0 & 0 & 1 & 1 \\
FreeBSD-Win2000 & 0 & 0 & 0 & 2 & 1 \\ \hline
FreeBSD-Ubuntu & 0 & 0 & 1 & 0 & 0 \\
Red Hat-Win2003 & 0 & 0 & 1 & 0 & 0 \\
OpenBSD-Solaris & 0 & 0 & 3 & 4 & 2 \\
FreeBSD-Solaris & 0 & 0 & 5 & 8 & 0 \\ \hline
FreeBSD-Debian & 0 & 1 & 0 & 2 & 0 \\ \hline
OpenBSD-Ubuntu & 1 & 0 & 0 & 1 & 0  \\
NetBSD-Debian & 1 & 0 & 0 & 1 & 0 \\
NetBSD-Solaris & 1 & 0 & 3 & 4 & 1 \\
Solaris-Win2003 & 1 & 1 & 1 & 4 & 0 \\
Solaris-Win2000 & 1 & 1 & 1 & 4 & 1 \\
NetBSD-Red Hat & 2 & 0 & 0 & 3 & 0 \\
Solaris-Red Hat & 2 & 0 & 0 & 7 & 0\\
FreeBSD-Red Hat & 2 & 1 & 2 & 5 & 1\\
Debian-Ubuntu & 3 & 2 & 3 & 5 & 0\\
OpenBSD-Red Hat & 4 & 0 & 1 & 4 & 0\\
NetBSD-FreeBSD & 4 & 3 & 20 & 14 & 0 \\
OpenBSD-NetBSD & 7 & 3 & 14 & 12 & 0 \\
Ubuntu-Red Hat & 11 & 2 & 7 & 6 & 0 \\
OpenBSD-FreeBSD & 11 & 5 & 16 & 14 & 0 \\
Debian-Red Hat & 22 & 3 & 18 & 8 & 0 \\
Win2000-Win2008 & 54 & 7 & 15 & 3 & 0 \\
Win2000-Win2003 & 167 & 29 & 66 & 2 & 0 \\
Win2003-Win2008 & 222 & 16 & 41 & 2 & 0 \\ \hline
\end{tabular}
\caption{Number of two consecutive vulnerabilities occurring in each \gls{irt} period, between 2000 and 2011 (with Fat Servers).}
\label{tab:pairs_irt}
}
\end{center}
\end{table}



\subsubsection*{Inter-Reporting Times Strategy (IRTst)} \label{IRT} 
This strategy is mainly concerned with the \emph{\gls{irt}}, i.e., the number of days between successive reports of common vulnerabilities in different \gls{os} pairs, rather than vulnerability counts. 
The assumption underlying this strategy is that lower inter-reporting times suggest a greater similarity between \glspl{os}, and thus it would be advisable, from a diversity standpoint, to select \glspl{os} with higher \gls{irt}.

Table \ref{tab:pairs_irt} presents the number of vulnerabilities for each pair of \glspl{os} in five \gls{irt} intervals, from 0 to 10000 days.
The values in the table were obtained in the following manner: first, for a given \gls{os} pair A and B, we collected the dates for common vulnerabilities; next, we calculated the \gls{irt} in days of every two consecutive vulnerabilities; and then, we counted the number of vulnerabilities that were within each interval.
The table is organized such that on the top are the \glspl{os} without common vulnerabilities, which are then followed by the ones that had larger \gls{irt} values. 
Therefore, each horizontal line separates groups of \gls{os} pairs that have positive \gls{irt} values in the same leftmost column, starting from the rightmost column (i.e., with the longer \gls{irt}). 
From a diversity perspective, it is interesting to notice that in the table there are $29\%$ of the pairs that do not have two consecutive vulnerabilities, and that $11\%$ only have consecutive vulnerabilities from $1000$ days on (last column).

The \gls{irts} strategy allows the selection of \glspl{os} with longer \gls{irt}. 
This criteria is essential if one wants to deploy a system that has a short lifetime, e.g., a batching process, which ideally would only be in operation between the discovery of common vulnerabilities. 
\gls{irts} tries first to select \gls{os} pair with zero common vulnerabilities; when this is not possible, it chooses next pair whose common vulnerabilities appear in the rightmost columns. 
By inspecting the table, we can see that the best two sets of four \glspl{os} are:

\begin{itemize}
\item setCon = \{\emph{OpenBSD, Solaris, Debian, Windows2003}\};
\item setUpdt = \{\emph{NetBSD, Solaris, Ubuntu, Windows2008}\}.
\end{itemize}


Table \ref{tab:pairs_irt_iso} presents the \gls{irt} for an Isolated Thin Server configuration. 
Since each \gls{os} pair has less common vulnerabilities, this often translates to larger \gls{irt}.  
The percentage of lines with zero \gls{irt} in all intervals is higher, $51\%$, but remained the same for the \gls{os} pairs that share vulnerabilities with longer \gls{irt} ($11\%$). 
When applying the strategy to this table, the best sets of \glspl{os} are: 

\begin{itemize}
\item setCon = \{\emph{OpenBSD, Solaris, Debian, Windows2003}\};
\item setUpdt = \{\emph{OpenBSD, Debian, Ubuntu, Windows2008}\}.
\end{itemize}


\begin{table}[!ht]
\begin{center}
{\scriptsize
\begin{tabular}{|l||c c c c c|}\hline
&	$0$ $\leq$ IRT $\leq 1$	& $1$ \textless IRT $\leq 10$ & $10$ \textless IRT $\leq 100$ & $100$ \textless IRT $\leq 1000$ & $1000$ \textless IRT $\leq 10000$\\\hline
OpenBSD-Debian & 0 & 0 & 0 & 0 & 0 \\
OpenBSD-Ubuntu & 0 & 0 & 0 & 0 & 0 \\
OpenBSD-Win2008 & 0 & 0 & 0 & 0 & 0 \\
NetBSD-Ubuntu & 0 & 0 & 0 & 0 & 0 \\
NetBSD-Win2003 & 0 & 0 & 0 & 0 & 0 \\
NetBSD-Win2008 & 0 & 0 & 0 & 0 & 0 \\
FreeBSD-Debian & 0 & 0 & 0 & 0 & 0 \\
FreeBSD-Ubuntu & 0 & 0 & 0 & 0 & 0 \\
FreeBSD-Win2008 & 0 & 0 & 0 & 0 & 0 \\
Solaris-Debian & 0 & 0 & 0 & 0 & 0 \\
Solaris-Ubuntu & 0 & 0 & 0 & 0 & 0 \\
Solaris-Win2003 & 0 & 0 & 0 & 0 & 0 \\
Solaris-Win2008 & 0 & 0 & 0 & 0 & 0 \\
Debian-Win2000 & 0 & 0 & 0 & 0 & 0 \\
Debian-Win2003 & 0 & 0 & 0 & 0 & 0 \\
Debian-Win2008 & 0 & 0 & 0 & 0 & 0 \\
Ubuntu-Red Hat & 0 & 0 & 0 & 0 & 0 \\
Ubuntu-Win2000 & 0 & 0 & 0 & 0 & 0 \\
Ubuntu-Win2003 & 0 & 0 & 0 & 0 & 0 \\
Ubuntu-Win2008 & 0 & 0 & 0 & 0 & 0 \\
Red Hat-Win2000 & 0 & 0 & 0 & 0 & 0 \\
Red Hat-Win2003 & 0 & 0 & 0 & 0 & 0  \\
Red Hat-Win2008 & 0 & 0 & 0 & 0 & 0 \\ \hline
OpenBSD-Win2003 & 0 & 0 & 0 & 0 & 1 \\
FreeBSD-Win2003 & 0 & 0 & 0 & 0 & 1 \\
Solaris-Red Hat & 0 & 0 & 0 & 0 & 1 \\
Solaris-Win2000 & 0 & 0 & 0 & 0 & 1  \\
OpenBSD-Win2000 & 0 & 0 & 0 & 0 & 2 \\ \hline
Debian-Ubuntu & 0 & 0 & 0 & 1 & 0 \\
NetBSD-Win2000 & 0 & 0 & 0 & 1 & 1 \\
FreeBSD-Win2000 & 0 & 0 & 0 & 2 & 1 \\ \hline
NetBSD-Solaris & 0 & 0 & 1 & 2 & 0 \\
OpenBSD-Solaris & 0 & 0 & 1 & 3 & 0 \\
FreeBSD-Solaris & 0 & 0 & 1 & 4 & 0 \\ \hline
NetBSD-Debian & 1 & 0 & 0 & 0 & 0  \\
NetBSD-Red Hat & 1 & 0 & 0 & 0 & 0 \\
Debian-Red Hat & 1 & 0 & 1 & 3 & 2 \\
OpenBSD-Red Hat & 2 & 0 & 0 & 0 & 0 \\
FreeBSD-Red Hat & 2 & 0 & 0 & 0 & 0 \\
OpenBSD-NetBSD & 2 & 0 & 3 & 7 & 1 \\
NetBSD-FreeBSD & 2 & 0 & 6 & 9 & 1 \\
OpenBSD-FreeBSD & 6 & 0 & 9 & 10 &  1\\
Win2000-Win2008 & 8 & 1 & 3 & 3 & 0 \\
Win2003-Win2008 & 16 & 2 & 19 & 2 & 0 \\
Win2000-Win2003 & 35 & 8 & 36 & 5 & 0\\ \hline
\end{tabular}
\caption{Number of two consecutive vulnerabilities occurring in each IRT period, between 2000 and 2011 (Isolated Thin Servers).}
\label{tab:pairs_irt_iso}
}
\end{center}
\end{table}

\subsubsection*{Comparing the three strategies}
The three strategies explore distinct characteristics of the data to pick the \glspl{os} to be deployed. 
Qualitatively they differ in the method of selection, and potentially the result of applying them to our data could lead to distinct sets being chosen.
On the other hand, they all try to find \glspl{os} that when placed together in the same system have a low probability of experiencing common vulnerabilities in the future. 
Therefore, if there is a small collection of \gls{os} sets with this property, then all strategies should elect one of these sets as the best choice. 
This is precisely what we observed with the \gls{nvd} data, and consequently, the selected best sets are not too different from each other.
Only when one needs to find many \gls{os} sets with reasonable levels of diversity, such as with the implementation of proactive recovery mechanisms in intrusion-tolerant systems \cite{Castro:2002,Sousa:2010}, then the distinctions among the strategies start to become apparent.
We leave it as future work a more refined study on the comparison of the strategies with larger groups of \gls{os} sets.

The \gls{os} sets that resulted from the execution of the strategies achieved reasonable levels of diversity when evaluated in the observed period (years 2010 and 2011). 
As expected from our analysis, several of the best sets contained an \gls{os} from each of the families. 
The exception to this rule occurred with the Linux family, wherein a few cases it had two representatives (Debian and Ubuntu) because these \glspl{os} had very few vulnerabilities reported in the last three years. 
Even though the BSD family also had a small number of recent vulnerabilities, since these occasionally affected more \glspl{os}, the strategies opted for including just one of the BSD \glspl{os}.

In the next section, we will look into \gls{os} versions as a way to increase diversity. 
For this study, we will use the set that was most selected by the different strategies: \{\emph{OpenBSD, Debian, Solaris, Windows2003}\} (4 out of 12 choices, considering both Fat Server and Isolated Thin Server configurations).


%%%%%%%%%%%%%%%%%%%%%%%%%%%%%%%%%%%%%%%%%%%%%%%%%%%%%%%%%%%%%%%%%%%%%%%%%%%%%%%%%%%%%%%%%%%
\subsection{Exploring Diversity Across OS Releases}
\label{releases}

If one wants to build systems capable of tolerating a few intrusions, our results show that it is possible to select \glspl{os} for the replicas with a small collection of common vulnerabilities. 
It is hard, however, to support critical services that need to remain correct with higher numbers of compromised replicas or to use some \gls{bft} algorithms that trade off performance for extra replicas (e.g., \cite{Abd-El-Malek:2005,Kotla:2010,Serafini:2010}). 
The number of available \glspl{os} is limited, and consequently, one rapidly runs out of different \glspl{os} (e.g., 13 distinct \glspl{os} are needed to tolerate $f=4$ faults in a $3f+1$ system). 
On the other hand, our experiments are relatively pessimistic in the sense that they are based on long periods of time, and no distinctions are made between \gls{os} releases.

Newer releases of an \gls{os} can contain essential code changes, and therefore, old vulnerabilities may disappear and/or new vulnerabilities may be introduced. 
As a result, if we consider (OS, release) pairs, one may augment the number of different systems that do not share vulnerabilities. 
Nevertheless, one should keep in mind that the use of older \gls{os} releases does not come without a cost. Namely, there might be incompatibilities with the current hardware, and some older software packages might be challenging to find.

In the next two sections we explore \emph{n-diverse} sets, where we extend the OS, as an element in the set, to the \gls{os} release. 
First we study a \emph{4-diverse} set, and then a \emph{2-diverse} set built with only two \glspl{os}. 
We will concentrate on Isolated Thin Server configurations because they correspond to the most common way to deploy intrusion-tolerant systems.


\subsubsection*{4-diverse sets}
Here we analyze in more detail the vulnerabilities for the set \{\emph{OpenBSD, Debian, Solaris, Windows2003}\} across their releases between 2000 and 2011. 
Despite the year of the release, since some vulnerabilities can be inherited from older versions of the code, we will include all vulnerabilities no matter the published date (i.e., even the flaws before 2000).

From all releases available for our 4-version replicated system,\footnote[1]{OpenBSD~2.7, OpenBSD~2.8, OpenBSD~2.9, OpenBSD~3.0, OpenBSD~3.1, OpenBSD~3.2, OpenBSD~3.3, OpenBSD~3.4, OpenBSD~3.5, OpenBSD~3.6, OpenBSD~3.7, OpenBSD~3.8 OpenBSD~3.9, OpenBSD~4.0, OpenBSD~4.1, OpenBSD~4.2, OpenBSD~4.3, OpenBSD~4.4,   Solaris~10.0, Solaris~11.0, Solaris~8.0, Solaris~8.1, Solaris~8.2, Solaris~9.0, Solaris~9.1, Debian~2.2, Debian~2.3, Debian~3.0,   Debian~3.1, Debian~4.0, Debian~6.0, Debian~6.2, and Windows~2003.} we looked at the major releases that had non-zero vulnerabilities. 
Since OpenBSD follows a fixed six-month release cycle, in this case, we selected one version every three years, which is reasonable given our 12-year time span (considering all 18 releases would require 154 entries in the table). 
Therefore, the \gls{os} releases that are taken into the study are: OpenBSD 5.0, OpenBSD 4.4, OpenBSD 3.8, OpenBSD 3.2, Solaris 8.0, Solaris 9.0, Solaris 10.0, Solaris 11.0, Debian 2.2, Debian 3.0, Debian 4.0, Debian 5.0, Debian 6.0 and Windows2003.

\begin{table}[!ht]
\begin{center}
{\scriptsize
\begin{tabular}{|l|c|}\hline
\textbf{OS Versions} & Total  \\\hline\hline
Solaris 8.0-Solaris 9.0 	&21\\
Solaris 9.0-Solaris 10.0 	&8\\
Solaris 8.0-Solaris 10.0 	&7\\
OpenBSD 3.2-OpenBSD 3.8 & 4 \\
OpenBSD 3.2-Solaris 9.0 & 2 \\
OpenBSD 3.2-Windows2003 & 2 \\
OpenBSD 3.2-Solaris 8.0 & 1 \\
OpenBSD 3.8-Windows2003 & 1 \\
Solaris 8.0-Windows2003 & 1 \\
Solaris 9.0-Windows2003 & 1 \\
Solaris 10.0-Windows2003 & 1 \\
Solaris 10.0-Solaris 11.0	& 1 \\
Solaris 8.0-Debian 2.2 &	1 \\
Debian 4.0-Windows2003 & 1 \\\hline
\end{tabular}
\caption{Common vulnerabilities between OS releases.}
\label{tab:vulns_releases}
}
\end{center}
\end{table}

Table \ref{tab:vulns_releases} shows the number of common vulnerabilities for each pair of OS-release (pairs with zero values are not displayed). 
The first observation is that there are many releases within this set of 4 \glspl{os} that appear to be free of common flaws. 
Second, these shared bugs occur more often between releases of the same \gls{os}. 
This is anticipated because more code is re-used within the same \gls{os}.
Additionally, one can notice that releases of the same \gls{os} typically have less shared vulnerabilities when comparing older and newer versions. 
This is particularly evident for the OpenBSD and Debian releases. 
This result is quite promising because it supports our thesis that one should be able to explore diversity across releases, as a way to increase the number of available candidates for the construction of the diverse \gls{os} sets.


\subsubsection*{2-diverse sets} 
To reduce the development and maintenance costs of an intrusion-tolerant system, it is reasonable to investigate solutions that attempt to decrease the number of distinct \glspl{os} but still ensure a high level of security. 
Here, it makes sense to consider an approach that offers diversity with only two \glspl{os}, while still exploiting the diversity available within the \gls{os} releases. 
In the extreme case, one can select \glspl{os} from the same family, for instance, to simplify system management.

To study this sort of solution, we looked at the vulnerabilities that appeared in several versions of Debian and Red Hat (all released after 2000). 
In total, seven Debian and ten Red Hat releases were considered, which gives a total of $136$ combinations. 
Table \ref{tab:debian_RedHat} provides a summary of the vulnerabilities that were found in \gls{nvd} for pairs of \gls{os} releases (to save space, we omit pairs with zero common vulnerabilities). 
The reader should notice that Red Hat went through two distinct distributions from 2000, and they can be distinguished by the version number: Red Hat Linux existed between 2000 and 2003, and it has a dot in the version number (e.g., Red Hat 7.1); Red Hat Enterprise Linux is available from 2003, and its releases only have a single digit (e.g., Red Hat 3).

\begin{table}[!ht]
\begin{center}
{\scriptsize
\begin{tabular}{|l|c||l|c|}\hline
\textbf{OS Versions} & Total &  \textbf{OS Versions} & Total \\\hline\hline
Debian 2.2 - Debian 2.3 & 2 & Red Hat 7.1 - Red Hat 9.0 & 2  \\\hline
Debian 3.1 - Debian 4.0 & 1 & Red Hat 7.2 - Red Hat 7.3 & 5  \\\hline
Debian 4.0 - Red Hat 3 & 1 &  Red Hat 7.2 - Red Hat 8.0 & 5  \\\hline
Debian 4.0 - Red Hat 4 & 1 & Red Hat 7.2 - Red Hat 9.0 & 2  \\\hline
Debian 4.0 - Red Hat 5 & 1 & Red Hat 7.3 - Red Hat 8.0 & 5  \\\hline
Red Hat 7.0 - Red Hat 7.1 & 2 & Red Hat 7.3 - Red Hat 9.0 & 2  \\\hline
Red Hat 7.0 - Red Hat 7.2 & 1 & Red Hat 8.0 - Red Hat 9.0 & 2  \\\hline
Red Hat 7.1 - Red Hat 7.2 & 3 &  Red Hat 3 - Red Hat 4 & 2  \\\hline
Red Hat 7.1 - Red Hat 7.3 & 2 & Red Hat 3 - Red Hat 5 & 1  \\\hline
Red Hat 7.1 - Red Hat 8.0 & 2 &  Red Hat 4 - Red Hat 5 & 1  \\\hline
\end{tabular}
\caption{Common vulnerabilities between Debian and Red Hat releases.}
\label{tab:debian_RedHat}
}
\end{center}
\end{table}

The table only shows $14.7\%$ of the \gls{os} releases combinations, which means $85.3\%$ of the \gls{os} pairs are free from common flaws. 
Debian has very few vulnerabilities that are shared between versions. 
Moreover, accordingly to \gls{nvd}, these few vulnerabilities only affect two releases, i.e., no vulnerability had an impact over an extensive period. 
There are two vulnerabilities, CVE-2003-0248 and CVE-2003-0364, that appear in five Red Hat releases (from Red Hat 7.1 to Red Hat 9.0). 
The first is a vulnerability in the \emph{mxcsr} kernel code which lets attackers modify CPU state registers via a malformed address. 
This is a critical vulnerability because it has an impact on availability, confidentiality, and integrity. 
The second one is in the Red Hat TCP/IP implementation, and allows a \gls{dos} by CPU consumption. 
Three vulnerabilities occur in Red Hat 7.2, 7.3 and 8.0, all related to particular versions and configurations of the OpenSSL cryptographic toolkit. 
When exploited, they cause a \gls{dos} (CVE-2004-0079, CVE-2004-0081 and CVE-2004-0112).
%Again, we can observe that few high impact vulnerabilities cross many versions of the same \gls{os}. 

These results demonstrate that with a careful selection of the Debian and Red Hat releases, it is possible to avoid most of the common vulnerabilities, and apparently, it becomes viable to build an intrusion-tolerant replicated system based on only two \glspl{os}.


\section{Decisions About Deploying Diversity}\label{decisions}
We have underscored that these results are only \textit{prima facie} evidence for the usefulness of diversity. 
On average, we would expect our estimates to be conservative as we analyzed aggregated vulnerabilities across releases: common vulnerabilities could be much smaller in a ``specific set'' of diverse \gls{os} releases. 
However, there are limitations on what can be claimed from the analysis of the \gls{nvd} data alone without further manual analysis (other than what we have done, e.g., developing/finding and running exploit scripts on every \gls{os} for each vulnerability).
A better analysis would be obtained if the \gls{nvd} vulnerability reports were combined with the exploit reports (including exploiting counts), and even better if they also had indications about the users' usage profile.
Moreover, we have seen that \gls{nvd} has some limitations, as we have found that some of the vulnerabilites were not reported in the \gls{nvd}, but several security advisory websites have reported those vulnerabilities.
%However, vendors are often wary of sharing such detailed dependability and security information with their customers.
Additionaly, there are partial exploit reports available from other sites (e.g., \cite{cvedetails}).


Given these limitations, \emph{how can individual user organizations decide whether diversity is a suitable option for them, with their specific requirements and usage profiles?}
The cost is reasonably easy to assess: costs of the software products, the required middleware (if any), added the complexity of management, difficulties with client applications that require vendor-specific features, hardware costs, run-time cost of the synchronization and consistency enforcing mechanisms, and possibly more complex recovery after some failures.
The gains in improved security (from some tolerance to zero-day vulnerabilities and easier recovery from some exploits, set against possible extra vulnerabilities due to the increased complexity of the system) are difficult to predict except empirically. 


The first step onto adopting diversity as a dependable mechanism is done. 
The other limitations will be addressed in the rest of this thesis. 


\section{Final Remmarks}

In this chapter we presented results from an analysis of 18 years (1994--2011) of vulnerability reports from the \gls{nvd}  database. 
We analyzed 2270 vulnerabilities of eleven \gls{os} distributions. 
We enriched the dataset with manual fixes and further classification of vulnerabilities, depending on which part of the \gls{os} they affected, into the kernel, driver, system software, and applications. The enriched dataset allowed us to perform various analysis to answer the main questions that drove our research for this thesis: 
\emph{what are the potential security gains that could be attained from using diverse \glspl{os} in a replicated intrusion-tolerant system?}
 

\chapter{BFT Diversity Management}
\label{chap:lazarus_design}

This chapter presents the design of a control plane, named \system, to manage the diversity on \gls{bft} services.
The design improves the work presented in the previous chapter with new data and clustering techniques, introduces a new metric to evaluate vulnerabilities severity, and propose a strategy to minimize the risk of having common failure modes in replicated systems.
In the end, we validate this proposal against other strategies.

\note{We need to clarify that \system is proactive and \sieveq is reactive, they can be used together. E.g., when there is a problem the controller increases the risk}
\note{patches:Patches take time to apply~\cite{Frei:2010}
MAking exploit from patches that were not yet released!!!\cite{Brumley:2008}
THE NEED FOR AUTOMATIC PATCHING~\cite{Nappa:2015} --shared code, one takes the patch and the other no}

\section{Overview}
\system is a control plane solution that automatically changes the attack surface of a \gls{bft} system in a dependable way.
\system continuously collects security data from \gls{osint} feeds on the internet to build a knowledge base about the possible vulnerabilities, exploits, and patches related to the systems of interest.
This data is used to create clusters of similar vulnerabilities, which potentially can be affected by (variations of) the same exploit.
These clusters and other collected attributes are used to analyze the risk of the \gls{bft} system becoming compromised due to common vulnerabilities.
Once the risk increases, the control plane replaces a potentially vulnerable replica by another one, trying to maximize the failure independence of the replicated service.
Then, the replaced node is put on quarantine and updated with the available patches, to be re-used later.


%In summary, we make the following contributions: 

%\begin{enumerate}

%\item A method for assessing the risk of a group of replicas being compromised based on the security news feeds available on the internet. 
%The method overcomes limitations from works that use the \gls{nvd} data for managing the replicas vulnerability independence (Section~\ref{sec:metric});

%\item An evaluation of our risk management method based on real historical vulnerability data showing its effectiveness in keeping a group of replicas safe from common vulnerabilities (Section~\ref{sec:diversity});

%\end{enumerate}


In a nutshell, as displayed in~Figure~\ref{fig:overview}, \system provides a distributed operating system for \gls{bft}-replicated services.
The system manages in its execution plane a set of nodes that run \emph{unmodified replicas} (encapsulated in \glspl{vm} or containers). 
Each node must have a small \gls{ltu} that allows the activation and deactivation of replicas as demanded by the \system \emph{Controller}, in the control plane.
The controller decides which software should run at any given time by monitoring the vulnerabilities that potentially exist in the pool of replicas, aiming to minimize the risk of having several nodes being compromised by the same attack.

\begin{figure}[h]
\begin{center}
\includegraphics[width=0.7\columnwidth]{images/images/overview.pdf}
\vspace{-5mm}
\caption{\system overview.}
\label{fig:overview}
\end{center}
\end{figure}


\section{System Model}
\label{sec:systemmodel}

\system system model shares some similarity with previous works on the proactive recovery of \gls{bft} systems~\cite{Castro:2002,Platania:2014,Sousa:2010,Roeder:2010}.
More specifically, we consider a \emph{hybrid system model} composed of two planes (see Figure~\ref{fig:overview}) with different properties and assumptions:

\begin{itemize}

\item \textbf{Execution Plane:} 
This plane is composed of replica processes that can be subject to Byzantine failures.
Therefore, a Byzantine replica can try to mislead the other replicas or the clients.
These replicas communicate through an asynchronous network that can delay, drop or modify messages, just like most \gls{bft} system models~\cite{Castro:2002,Kotla:2010,Bessani:2014,Aublin:2015}.
This plane hosts $n$ replicas from which at most $f$ can be compromised at any given moment.
We consider the typical scenario in which $n=3f+1$~\cite{Castro:2002,Bessani:2008,Moniz:2011,Kotla:2010,Aublin:2015}.  

\item \textbf{Control Plane:}  
In this plane, we assume that each component can only fail by crashing. 
Each \emph{node} hosting processes contains a \gls{ltu}, and there is a logically-centralized controller to reconfigure the system, just like what has been used in several previous works on proactive recovery (e.g.,~\cite{Roeder:2010,Platania:2014,Sousa:2010}).
The failures of such components do not compromise the liveness and safety of the service as long as the control plane is recovered before $f$ replicas fail.
For simplicity, in this chapter, we address this component as a logical-centralized controller, which requires stronger assumptions. 
However, in Chapter~\ref{chap:lazarus_implementation} we introduce and discuss a \gls{bft}-control plane version, which requires weaker assumptions.


\end{itemize}

Besides the execution and control planes, we assume the existence of two types of external components: (1) clients of the replicated service, which can be subject to Byzantine failures; (2) \gls{osint} sources (e.g., \gls{nvd}, ExploitDB) that cannot be subverted and controlled by the adversary.
In practice, this assumption led us to consider only well-established and authenticated data sources.
Dealing with untrusted sources is an active area of research in the threat intelligence community (e.g.,~\cite{Sabottke:2015,Liu:2015}), which we consider out of the scope of this thesis.


%\section{Execution Plane}
%\label{sec:executionplane}

%The Execution Plane can accomodate any replicated system that already levareges on the existence of a controller node (e.g.,~\cite{Sousa:2010,Roeder:2010,Platania:2014,Garcia:2016}) or any \gls{bft} system that would benefit from the \system assistence (e.g.,~\cite{Sousa:2018}). 


%\section{Control Plane}
%\label{sec:controlplane}

%\system is the first control plane that automatically changes the attack surface of a \gls{bft} system in a dependable way.
%\system continuously collects security data from \gls{osint} feeds on the internet to build a knowledge base about the possible vulnerabilities, exploits, and patches related to the systems of interest.
%This data is used to create clusters of similar vulnerabilities, which potentially can be affected by (variations of) the same exploit.
%These clusters and other collected attributes are used to analyze the risk of the \gls{bft} system becoming compromised due to common vulnerabilities.
%Once the risk increases, \system replaces the potentially vulnerable replica by another one, trying to maximize the failure independence of the replicated service.
%Then, the replaced node is put on quarantine and updated with the available patches, to be re-used later.
%These mechanisms were implemented to be fully automated, removing the human from the loop.

%The current implementation of \system manages 17 \gls{os} versions, supporting the \gls{bft} replication of a set of representative applications.
%The replicas run in \glspl{vm}, allowing provisioning mechanisms to configure them. 
%We conducted two sets of experiments, one demonstrates that \system risk management can prevent a group of replicas from sharing vulnerabilities over time; the other, reveals the potential negative impact that virtualization and diversity can have on performance. 
%However, we also show that if naive configurations are avoided, \gls{bft} applications in diverse configurations can actually perform close to our homogeneous bare metal setup.

\section{Diversity of Replicas}
\label{sec:diversityofreplicas}

%BFT-replicated services running on \system are composed by $n$ replicas.
For our purposes, each \replica is composed of a stack of software, including an \gls{os} (kernel plus other software contained in an \gls{os} distribution), execution support (e.g., \gls{jvm}, \gls{dbms}), a \gls{bft} library, and the service that is provided by the system.
%(see Figure~\ref{fig:arch1}).
The set of $n$ replicas is called a \configuration.



The fault independence of the replicas is improved when different \gls{ots} components are employed in the software stack~\cite{Deswarte:1998}. 
For example, it has been shown that using distinct \glspl{db}~\cite{Gashi:2007}, \glspl{os}~\cite{Garcia:2013}, and filesystems~\cite{Castro:2003,Bairavasundaram:2009}, can yield important benefits in terms of fault independence. 
In addition, automatic techniques could enhance diversity, like randomization/obfuscation of \glspl{os}~\cite{Roeder:2010} and applications~\cite{King:2016}.
Although \system can also exploit these automatic techniques, we center our attention on diverse \gls{ots} components. 
In particular, \system monitors the disclosed vulnerabilities of all elements of the replicas' software stacks to assess which of them may contain common vulnerabilities.  
In Chapter~\ref{chap:datasource} we explained why we focoused on \gls{os} diversity. 
Moreover, we do not explicitly consider the diversity of the \gls{bft} library (i.e., the protocol implementation) or the service code implemented on top of it.
Four facts justify this decision: (1) N-version programming is too costly for this~\cite{Avizienis:1977}; (2) there have been some works showing that such protocol implementations can be generated from formally verified specifications~\cite{Hawblitzel:2015,Rahli:2018}; (3) the relatively small size of such components (e.g., a key-value store on top of BFT-SMaRt has less than 15k lines of code~\cite{Bessani:2014}) make them relatively simple to test and assess with some confidence~\cite{Martins:2013,Lee:2014}; and (4) there are no reported vulnerabilities about these systems to support our study.
Notice that, although we do not explicitly consider the diversity of \gls{bft} libraries, nothing prevents \system from monitoring them (when several alternatives become available). 
Additionally, as a pragmatic approach, we could employ automatic diversity techniques in this layer~\cite{Platania:2014,Roeder:2010}.


\section{Diversity-aware Reconfigurations}
\label{sec:metric}

One of the main contributions of this thesis is the vulnerability evaluation method used to assess the risk of having replicas with shared vulnerabilities.
This section details this method.

\subsection{Finding Common Vulnerabilities}
\label{sec:common_vulnerabilities}

%\gls{nist}'s \gls{nvd}~\cite{nvd} is the authoritative data source for disclosure of vulnerabilities and associated information~\cite{Massacci:2010}. 
%\gls{nvd} aggregates vulnerability reports from more than 70 security companies, advisory groups, and organizations, thus being the most extensive vulnerability database on the web. 
%All data is made available as \gls{xml} data feeds, containing the reported vulnerabilities on a given period. 
%Each \gls{nvd} vulnerability receives a unique identifier and a short description provided by the \gls{cve}~\cite{cveterm}. 
%The \gls{cpe}~\cite{cpe} provides the list of products affected by the vulnerability and the date of the vulnerability publication.
%The \gls{cvss}~\cite{cvss} calculates the vulnerability severity considering several attributes, such as the attack vector, privileges required, exploitability score, and the security properties compromised by the vulnerability (i.e., integrity, confidentiality, or availability).
%WE USE THE NVD (as described in Chapter~\ref{chap:datasource}


Previous studies on diversity solely count the number of shared vulnerabilities among different \glspl{os}, assuming that less common vulnerabilities imply a smaller probability of compromising $f+1$ OSes~\cite{Garcia:2014}. 
Although this intuition may seem acceptable, in practice it underestimates the number of shared vulnerabilities due to imprecisions in the data sources. 
For example, Table~\ref{tab:missing_products} shows three vulnerabilities, affecting three different \glspl{os} at distinct dates.
At first glance, one may consider that these \glspl{os} do not share vulnerabilities.
However, a careful inspection of the descriptions shows that they are very similar.
Moreover, we checked this resemblance by searching for additional information on security websites, and we found out that CVE-2016-4428, for example, also affects Solaris~\cite{solaris_report}.

\begin{table}[!t]
\begin{center}
{\scriptsize
\begin{tabular}{| p{2.3cm} | p{10cm} | }\hline
\textbf{CVE (affected OS)} & \textbf{Description} \\\hline\hline
CVE-2014-0157 (Opensuse 13) & \scriptsize \gls{xss} vulnerability in the Horizon Orchestration dashboard in OpenStack Dashboard (aka Horizon) 2013.2 before 2013.2.4 and icehouse before icehouse-rc2 allows remote attackers to inject arbitrary web script or HTML via the description field of a Heat template. \\ \hline
CVE-2015-3988 (Solaris 11.2) & \scriptsize Multiple \gls{xss} vulnerabilities in OpenStack Dashboard (Horizon) 2015.1.0 allow remote authenticated users to inject arbitrary web script or HTML via the metadata to a (1) Glance image, (2) Nova flavor or (3) Host Aggregate. \\ \hline
CVE-2016-4428 (Debian 8.0) & \scriptsize \gls{xss} vulnerability in OpenStack Dashboard (Horizon) 8.0.1 and earlier and 9.0.0 through 9.0.1 allows remote authenticated users to inject arbitrary web script or HTML by injecting an AngularJS template in a dashboard form. \\ \hline
\end{tabular}
}
\caption{Similar vulnerabilities affecting different OSes.}
\label{tab:missing_products}
\end{center}
\end{table}

Even with these imperfections, \gls{nvd} is still the best data source for vulnerabilities.
Therefore, we exploit its curated data feeds to obtain the unstructured information present in the vulnerability text descriptions and use this information to find similar weaknesses.
A usual way to find similarity in unstructured data is to use clustering algorithms~\cite{Jain:2010}.
Clustering is the process of aggregating related elements into groups, named clusters, and is one of the most popular unsupervised machine learning techniques. 
We apply this technique to build clusters of similar vulnerabilities (see Section~\ref{sec:clustering} for details), even if the data feed reports that they affect different products.
For example, the vulnerabilities in Table~\ref{tab:missing_products} will be placed in the same cluster as there is some resemblance among the descriptions, and they can potentially be activated by (variations of) the same exploit.

It is worth to remark that by using clusters to find similar vulnerabilities, we conservatively increase the chances of capturing shared weaknesses contributing to the score of a pair of replicas.


\subsection{Measuring Risk}
\label{sec:measurerisk}


Once the set of common vulnerabilities is found, \system needs to assign a score to each one of these vulnerabilities based on their severity.

As presented in Chapter~\ref{chap:datasource}, each vulnerability in \gls{nvd} has associated a few \gls{cvss} severity scores and metrics (two versions are currently offered, v2~\cite{cvssv2} and v3~\cite{cvssv3}). 
The scores provide a way to marshal several vulnerability attributes in a value reflecting various aspects that impact security. 
The value can also be translated into a qualitative representation to assist on the vulnerability management process. 
For example, the \gls{cvss} base score categorizes a vulnerability in a severity scale as presented in Table~\ref{tab:cvss_scale}.


\begin{table}[h]
\begin{center}
%{\scriptsize
\begin{tabular}{| c | c | }\hline
\textbf{Rating} & \textbf{CVSS Score} \\\hline\hline
\texttt{NONE} & 0.0 \\
\texttt{LOW} & 0.1-3.9 \\
\texttt{MEDIUM} & 4.0-6.9 \\
\texttt{HIGH} & 7.0-8.9 \\
\texttt{CRITICAL} & 9.0-10.0 \\ \hline
\end{tabular}
%}
\caption{Qualititative CVSS severity rating scale.}
\label{tab:cvss_scale}
\end{center}
\end{table}

There are some differneces between \gls{cvss} v2 and \gls{cvss} v3.
In particular, the latter introduces extra metrics that allow more granularity when assessing the vulnerability characteristics (e.g., it includes \emph{User Interaction}, specifying whether the exploitation of a given vulnerability requires human intervention).


However, \gls{cvss} has some limitations that can make it inappropriate for managing the risk associated with a replicated system:
(1)  it was shown that there is no correlation between the \gls{cvss} exploitability score and the availability of exploits in the wild for the vulnerability~\cite{Bozorgi:2010}; 
(2) \gls{cvss} does not provide information about the date when a vulnerability starts to be exploited and when the patch becomes ready; 
and (3) \gls{cvss} does not account for the vulnerability age, which means that severity remains the same over the years~\cite{Frei:2006}; therefore, a very old vulnerability can end up being considered as critical as a recent one, even though for the former there has been plenty of time to update the component and/or the defenses.  
% Nuno: tirei este ponto porque aparentemente não conseguimos soluciona-lo
%; and (4) some studies have shown that CVSS may overestimate severity~\cite{Sabottke:15}; for example, vulnerabilities with larger scores do not necessarily command higher prices at black markets~\cite{Allodi:2014}.
Additionally, both \gls{cvss} specifications support temporal metrics, i.e., Temporal Metric Group. 
However, this metric group is made to be \emph{updated} by security teams of each asset. 
Therefore, it is not provided in the \gls{nvd} feeds. 
Moreover, the specification shows that \gls{cvss} does not account for increasing the base score once there is information about an existent exploit. 
For example, the score of a vulnerability with a known exploit (i.e., Exploit Code Maturity as High) has the same score as a vulnerability with no information about an exploit (i.e., Exploit Code Maturity as Not defined). 

%One of the reasons for such shortcomings is related with the fact that (both versions of) the CVSS base score is mostly static, not evolving increasing its value when exploits become widely distributed. 
%In addition, metrics belong to Temporal Metric Group (described in both CVSS specifications) are not available on NVD data feeds.

%One of the reasons why such limitations occur may be related with the fact that (both versions of) CVSS do not account for increasing the base score once there is information about an existent exploit. For example, the score of a vulnerability with a known exploit (i.e., Exploit Code Maturity as High) has the same score as a vulnerability with no information about an exploit (i.e., Exploit Code Maturity as Not defined). Additionally, these metrics belong to Temporal Metric Group (described in both CVSS specifications) which is not available on NVD data feeds.


Given these shortcomings, we propose a more refined metric that is adapted to measure the risk of a \gls{bft} system configuration having replicas with shared vulnerabilities.
The metric is mostly focused on differentiating vulnerabilities by their current \emph{potential exploitability}, aiming to surpass the limitations identified above.
In this process, (1) and (2) were addressed by using additional \gls{osint} sources that provide information about exploits and dates. 
We collect data from sources like Exploit-DB~\cite{edb} for exploits, CVE-details~\cite{cvedetails} and vendor websites to get details about the patches, such as Ubuntu Security Notices~\cite{ubuntu}, Debian Security Tracker~\cite{debian}, Solaris~\cite{solaris}, Redhat~\cite{redhat}, FreeBSD~\cite{freebsd}, and Microsoft Security Advisories and Bulletins~\cite{microsoft}. In fact, often vendor sites also give additional product versions compromised by the vulnerability, thus improving the accuracy of the analysis. 
Limitation (3) is settled by decreasing the criticality of a vulnerability gradually through time.


\begin{figure}[h]
\begin{center}
\includegraphics[width=0.7\columnwidth]{images/images/scale.pdf}
\caption{Scoring system of vulnerabilities based on age, patch, and exploit.}
\label{fig:scale}
\end{center}
\end{figure}


Our novel metric uses four factors that together contribute to the overall score. 
The starting factor is the \gls{cvss} v3 core score, as it is a reasonable basis that takes into consideration several attributes of the vulnerability.
The other three factors adjust the score taking into account the age and the availability of a patch and an exploit. 
The rationale is to allow the ranking of vulnerabilities according to their possible exploitation at a given moment in time.
The resulting scale is represented in Figure~\ref{fig:scale}.
The worst scenario (higher severity score) corresponds to a vulnerability that is new (N) (i.e., recently published), for which there is an exploit already being distributed (E), and that is not yet patched -- called NE.
The best scenario (lowest score) is when a vulnerability is old (O), and there is a patch (P), and apparently, no viable exploit has been crafted -- named OP. 
Between these two extremes, several cases of vulnerabilities are considered and their score calculated accordingly.


\begin{table}[h]
\begin{center}
%{\footnotesize
\begin{tabular}{ c }
%\hline

\vbox{
\begin{equation}
\begin{split}
\text{score(v$_i$)}=\text{CVSSv3(v$_i$)}\times\text{oldness(v$_i$)}\times\text{patched(v$_i$)}\times\text{exploited(v$_i$)}
\label{eq:score}
\end{split}
\end{equation}
}
%\vspace{-1mm}
\\
\vbox{
\begin{equation} 
\begin{split}
\text{oldness(v$_i$)}=\text{max}\left((1-0.25\times\frac{(\text{now()}-\text{v$_i$.published\_date)}}{\text{Oldness Threshold}}), 0.75\right)
\label{eq:oldness}
\end{split}
\end{equation}
}
%\vspace{-1mm}
\\
\vbox{
\begin{equation} 
\text{patched(v$_i$)}=0.5^{\text{v$_i$.patched}}
\label{eq:patched}
\end{equation} 
}
%\vspace{-1mm}
\\
\vbox{
\begin{equation}
\text{exploited(v$_i$)} = 1.25^{\text{v$_i$.exploited}}
\label{eq:exploited}
\end{equation}  
}
\\ 
%\hline
\end{tabular}
%}
\end{center}
\end{table}

The metric is defined in Equation~\ref{eq:score}.
It is a multiplication of the \gls{cvss} v3 core score (as specified in~\cite{cvssv3}) with the three adjusting factors. 
The first is \emph{Oldness}, which causes criticality to decrease over time. 
It is harmonized by the \emph{Oldness Threshold} and the interval\footnote{\textit{Oldness Threshold} is set to 365 in our calculations; \textit{now()} and $v_i.published\_date$ return the current day and the day when the vulnerability was published, respectively.} that has passed since the vulnerability publication (Equation~\ref{eq:oldness}). 
In addition, this factor is bounded by a minimum value that impedes it from reaching zero (which would cause the vulnerability to be left unnoticed).
The second is \emph{Patched}, which reduces the severity by half when a patch is available (Equation~\ref{eq:patched}; $v_{i}.patched$ is a on/off flag).
Finally, the \emph{Exploited} factor grows severity by a quarter when an exploit is made available (Equation~\ref{eq:exploited}; $v_{i}.exploited$ is again a on/off flag).
The constants in these equations were defined to ensure the aggregated modifiers corresponds to Figure~\ref{fig:scale}.
Overall, the range of our metric is from $0.0$ to $12.5$ (similar to \gls{cvss} score $[0.0-10.0]$).


\begin{figure*}[h]
\subfloat[CVE-2018-8303.]{\includegraphics[width=0.35\columnwidth]{images/images/NE.pdf}\label{fig:ne}} %for Windows 10
\subfloat[CVE-2016-7180.]{\includegraphics[width=0.35\columnwidth]{images/images/OP.pdf}\label{fig:op}}%Debian 9.0
\subfloat[CVE-2018-8012.]{\includegraphics[width=0.35\columnwidth]{images/images/NPE.pdf}\label{fig:npe}} %Debian 9.0
\caption{Examples of \emph{score(v$_i$)} for three vulnerabilities.}
\label{fig:scoreplot}
\end{figure*}


Figure~\ref{fig:scoreplot} displays our score and \gls{cvss} v3 for three example vulnerabilities. 
Case (a) corresponds to an NE, i.e., a vulnerability that is new and has no patch yet, but an exploit was made available a few days after publication. 
Our score starts by decaying slowly but then there is a jump on severity when the exploit is published.
%On the contrary, the CVSS score remains constant (approx. $6.5$).
Case (b) presents the best scenario (i.e., OP), where the vulnerability is \emph{old} and a patch was eventually distributed (and no exploit is available). Here, the severity decreases once there is a patch and over time, losing its relevance from a security perspective. 
Finally, case (c) illustrates a vulnerability that has an exploit a few days after publishing and then a patch is also created (i.e., NPE).
First, the score is raised once the exploit starts to be distributed, next it decreases three days later after the patch is released, and then continues to decay with the passage of time.


\subsection{Measuring Configurations Risk}
\label{sec:risk}

After discovering common vulnerabilities and assigning a dynamic score to them, the third step of our method is to calculate the \emph{risk of a set of replicas} to be compromised in the current threat landscape.


\begin{table}[h]
\begin{center}
%{\footnotesize
\begin{tabular}{ c }
%\hline

\vbox{
\begin{equation}
\mathit{risk(\ES)}=\sum_{r_i,r_j \in \ES}^{} \hspace{1.2mm} \sum_{v \in \mathcal{V}(r_i,r_j)}^{}\mathit{score(v)}
\label{eq:risk}
\end{equation}
}
\\
% \hline
\end{tabular}
%}
\end{center}
\end{table}


The risk associated with a \configuration (\ES) with $n$ replicas is given by Equation~\ref{eq:risk}. 
It sums up the score (Equation~\ref{eq:score}) of the vulnerabilities that would allow an attack to compromise simultaneously a pair of replicas $r_i,r_j \in \ES$.
%Function \emph{common()} employs our score, as it aggregates security features of a vulnerability, and other information collected from OSINT sources. To begin, the function gathers all the vulnerabilities that would allow an attack to compromise simultaneously a pair of replicas ($rc_i,rc_j$). 
More precisely, the vulnerabilities in $\mathcal{V}(r_i,r_j)$ aggregate: (i) the vulnerabilities that affect the software running in both replicas as listed in \gls{nvd} (and other \gls{osint} sites); and (ii) groups of vulnerabilities that are placed in the same cluster and affect each replica in the pair. 
%The final score is normalized with function \emph{norm()} to generate a value between 0 and 1.
%Next, the function calculates the score of each vulnerability, and returns the sum of all scores. 
%The risk metric computes \emph{common()} for all combinations of replica pairs in the configuration and accumulates the values. 
%\vspace{-3mm}
%{\footnotesize



This metric penalizes configurations that include replica pairs containing more common weaknesses, as this is an indication that they are less fault-independent.
In addition, the level of penalty is kept proportional to the severity of these vulnerabilities as observed at the time of the calculation. 
Thus, for example, replicas that share weaknesses only in the distant past are considered less risky to run in a configuration than replicas that together had recently highly exploitable vulnerabilities.

\subsection{Selecting Configurations}
\label{sec:configurations}

The last step of our method is to periodically assess the risk of a deployed configuration and its future replacement of replicas according to the evaluation of the metric of the previous section.

The bootstrap of the monitoring algorithm is simple. 
The first replica is selected randomly from the available candidates (\RS).
Then, it selects the next $n$ elements by minimizing the risk of the \ES in construction (using Equation~\ref{eq:risk}).
After the initialization, the \ES is periodically monitored and reconfigured if needed, as detailed in Algorithm \ref{alg:algorithm2}.
This is done by periodically evaluating the risk of the current \ES. 
If the risk exceeds a predefined \emph{threshold}, a mechanism is triggered to replace replicas and reduce the overall risk.
When this happens, the algorithm will pick a new replica, from \RS, to join \ES. 
In addition, it removes the replica that will reduce the risk.
The removed replica is set aside in a special group (\QS) to impede re-selection.
While in \QS, the replicas wait for patches before they re-join \RS and be ready to be chosen again.
Additionally, the algorithm ensures that all replicas that are running will eventually be replaced, despite their overall score.
Otherwise, the algorithm favors configurations that have lower scores by keeping them online for longer periods, allowing an attacker to gain time to discover new vulnerabilities.
To avoid this problem, the algorithm looks for the unpatched vulnerabilities with higher score affecting singles replicas.
This approach is conservative in the sense that it penalizes replicas expecting that such vulnerabilities can be unknown as common to other replicas.

\begin{algorithm}[h]
\begin{multicols}{2}
\SetInd{0.3em}{0.2em}
\caption{Replica Set Reconfiguration}\label{alg:algorithm2}
{\scriptsize
\ES: set replicas in the \configuration \;
$n$: number of replicas in \ES\;
\RS: set with the available replicas (not in use)\;
\QS: set of quarantine replicas\;
\BlankLine
\Fn{Monitor()}{
	\If{\Risk{\ES} $\geq$ threshold}{	
		\texttt{cadidates\_list} $\leftarrow$ $\perp$\;
		\COMBS = ${n\choose n-1}$\ES\;
		\ForEach{$ r $ in \RS}{
			\ForEach{$ \COMB $ in \COMBS}{
				\ES' $\leftarrow$ \COMB $\cup$ \{r\}\;
				\texttt{score} $\leftarrow$ \Risk{\ES'}\;
				\texttt{cadidates\_list}.add($\langle$\ES', \texttt{score}$\rangle$)\;
			}
		}
	\SC $\leftarrow$ \Rand{\texttt{cadidates\_list}} \;
	\Replace{\SC}\;		
	}
    \Else{
		\toRemove $\leftarrow$ $\perp$\;
		\texttt{maxScore} $\leftarrow$ \texttt{HIGH}\;
		\ForEach{$ r $ in \ES}{
			\texttt{avgScore} $\leftarrow$ \oldestVulnerability{r}\;
			\If{ avgScore $\geq$ maxScore}{	
				\toRemove $\leftarrow$ $r$\;
				\texttt{maxScore} $\leftarrow$ avgScore\;
			}
	}
		\If{ \toRemove $\neq$ $\perp$}{
			\texttt{cadidates\_list} $\leftarrow$ $\perp$\;				
			\ForEach{$ r' $ in \RS}{
				\ES' $\leftarrow$ (\ES $\setminus$ \{toRemove\}) $\cup$ \{r'\}\;
				%\ES' $\leftarrow$ \ES' $\cup$ \{r'\}\;
				\texttt{score} $\leftarrow$ \Risk{\ES'}\;	
				\texttt{cadidates\_list}.add($\langle$\ES', \texttt{score}$\rangle$) \;
			}
			\SC $\leftarrow$ \Rand{\texttt{cadidates\_list}} \;
			\Replace{\SC}\;
		}
	}
	\ForEach{$ r $ in \QS}{	
		\If{\patched{r} = TRUE}{	
			\QS $\leftarrow$ \QS $\setminus$ $\{r\}$\;
			\RS $\leftarrow$ \RS $\cup$ $\{r\}$\;
		}
	}
}
\Fn{updateSets(\SC)}{
		\toRemove $\leftarrow$ $x \in (\ES \setminus \SC)$\;
		\toJoin $\leftarrow$ $y \in (\SC \setminus \ES)$\;
		\QS $\leftarrow$ \QS $\cup$ $\{toRemove\}$\;
		\ES $\leftarrow$ (\ES $\setminus$ $\{toRemove\}$) $\cup$ $\{toJoin\}$\;

}
}
\end{multicols}
\end{algorithm}


Function \emph{Monitor()} (line 5) is called on each monitoring round (e.g., at midnight every day) to evaluate the configuration that is executing.
%Consider a \ES that is already running with a risk$=\alpha$.
If the risk of \ES is greater or equal than a certain \emph{threshold} (line 6), the algorithm will assess which replica should be replaced.
The first step is to initialize two variables, the candidates list (line 7) and all possible combinations of $n-1$ out of $n$ elements of \ES (line 8).
Then, each element \texttt{r} in \RS (line 9) is tested as a potential substitute, i.e., as the $n^{th}$ element that would complete each of the combinations \COMB with $n-1$ replicas (line 10).
Next, we define a \ES' as the union of \COMB and $r$ (line 11).
The risk of \ES' is calculated (line 12) and added to a list that stores the tuple $\langle$\ES', \texttt{score}$\rangle$ (line 13). 
At the end of the these nested loops, we have a list with all the possible combinations of \ES and \RS together with their risk.
Then, the function \Rand selects a configuration randomly from the list of candidates (line 14). 
Although it is not present in the algorithm, it only selects configurations which are below the \emph{threshold} score. 
Then the algorithm updates all the sets (line 15) using the function \Replace (lines 36-40). 
To decide which element needs to be removed from \ES it makes the difference between the current \ES and \SC (line 37) and the contrary to select the element to join the \ES (line 38).
Then, \toRemove is added to \QS (line 39), removed from \ES and the new element is added to the \ES (line 40).
 

If the risk of \ES is less than the \emph{threshold} (line 16) we need to assess if the \ES needs a reconfiguration.
In this scenario, we start by initializing the variable \toRemove as empty (line 17) and \texttt{maxScore} with the value of the \gls{cvss} score rating \texttt{HIGH} (line 18).
For each element of \ES (line 19) the algorithm calculates the average score of the vulnerabilities affecting \r using Equation~\ref{eq:score} (line 20).
Then, if the average vulnerability score is equals or greater than \texttt{HIGH} (or \texttt{CRITICAL}) (line 21), the variable \toRemove is set to the element \r (line 22) and the value of \texttt{maxScore} with the \texttt{avgScore} (line 23).
At the end of the loop, the algorithm knows which is the element \r from \ES that has vulnerabilities with a highest average score.
If \toRemove is empty then algorithm proceeds to line 32. 
Otherwise, the algorithm will select a new replica (line 24).
First, the \texttt{candidate\_list} is set as empty (line 25).
Next, for each element \r' in \RS (line 26) the algorithm tests a \ES' with \r' instead of \toRemove (line 27).
Then, the score for this configuration is calculated (line 28) and together with the \ES' is stored in a list of configuration candidates (line 29).
As already described, the function \Replace updates the elements of the sets (line 31).

Finally, the \RS and \QS are updated. 
For each \QS element (line 32) is checked if it is fully patched (line 33).
In the affirmative case, the element is removed from \QS (line 34) and added to the \RS again (line 35).


For the sake of simplicity we do not present two corner cases where a \emph{system admin} may need to intervene: if the \RS runs out of elements or \emph{rand()} cannot return a configuration that minimizes the risk of the system.
One can take at least two decisions to guarantee the availability under some risk of becoming compromised: (1) increase the \emph{threshold} or (2) reset the \RS by removing the elements from \QS with fewer unpatched vulnerabilities.
Both cases are not likely to occur, but if they happen there is not much we can make to solve as we depend on the availability of replicas in \RS and patch availability to clean \QS from time to time. 

\section{Evaluation}
\label{sec:diversity}


This section evaluates how \system performs on the selection of dependable replica configurations.
As discussed in Section~\ref{sec:diversityofreplicas}, we focus our experimental evaluation solely on the OS diversity.
In particular, we considered 21 \gls{os} versions to be deployed on four replicas with the following distributions: OpenBSD, FreeBSD, Solaris, Windows, Ubuntu, Debian, Fedora, and Redhat. 

In these experiments, we emulate live executions of the system by dividing the collected data into two periods:
(i) a \emph{learning phase} covering all vulnerability data between \emph{2014-1-1} and the beginning of the \emph{execution phase}, which is used to setup the \risk's algorithm; 
and (ii) an \emph{execution phase} period that goes from January to August of 2018. 
This last period is divided into monthly intervals (JAN-AUG) allowing for eight independent tests.
For example, to run the experiment in May the \emph{learning phase} starts on \emph{2014-1-1} and ends on \emph{2018-4-30}, then the \emph{execution phase} starts on the May $1^{st}$ until the end of the month. 

The goal is to create a knowledge base in the \emph{learning phase} that is used to assess \system' choices during each interval of the \emph{execution phase}. 
A run starts on the first day of an interval and then progresses through each day of the interval until the end. 
Every day, we check if the currently executing replica set could be compromised by an attack exploring the vulnerabilities that were published in that month. 
We take the most pessimistic approach, which is to say that we consider the system to be broken if a single vulnerability comes out affecting at least two \glspl{os} that would be executing at that time.
Therefore, if one of the \glspl{os} already has a patch for the tested vulnerability, it is not counted as compromised.


Three additional strategies, inspired by previous works, were defined to be compared with \system (Section~\ref{sec:configurations}):

\begin{itemize}

\item \textbf{Equal:} all the replicas use the same randomly-selected \gls{os} during the whole execution. 
This corresponds to the scenario where most past \gls{bft} systems have been implemented and evaluated (e.g.,~\cite{Kotla:2010,Aublin:2015,Behl:2015,Veronese:2013,Behl:2017,Liu:2016,Yin:2003,Amir:2011,Bessani:2014,Clement:2009b}). 
Here, compromising a replica would mean an opportunity to intrude the remaining ones.

\item \textbf{Random:} a configuration of $n$ \glspl{os} is randomly selected, and at the beginning of each day, a new \gls{os} is randomly picked to replace an existing one. 
This solution represents a system with proactive recovery and diversity, but with no informed strategy for choosing the next \configuration.

\item \textbf{Common:} This strategy is the straw man solution to prevent the existence of shared vulnerabilities among \glspl{os}. 
This strategy minimizes the number of common vulnerabilities for each set and was introduced in previous vulnerability studies~\cite{Garcia:2013}.

\item \textbf{CVSS v3:} This strategy is very similar to ours as it tries different combinations to find the best one that minimizes the sum of \gls{cvss} v3 score.


\end{itemize}


\subsection{Diversity vs Vulnerabilities}
We evaluate how each strategy can prevent the replicated system from being compromised. 
Each strategy is analyzed over $1000$ runs throughout the execution phase in monthly slots. 
Different runs are initiated with distinct random number generator seeds, resulting in potentially different \gls{os} selections over the time slot. 
On each day, we check if there is a vulnerability affecting more than one replica in the current \configuration, and in the affirmative case the execution is stopped.

\begin{figure}[h]
\begin{center}
\includegraphics[width=\columnwidth]{images/gnuplot/executions/execution.pdf}
\caption{Compromised system runs over eight months.}
\label{fig:all_vulns}
\end{center}
\end{figure}


\textbf{Results:} Figure~\ref{fig:all_vulns} compares the percentage of compromised runs of all strategies. 
Each bar represents the percentage of runs that did not terminate successfully (lower is better). 
In all three periods, \system presents the best results. 
The \emph{Random} and \emph{Equal} strategies perform worse because eventually, they pick a group of OSes with common vulnerabilities as they do not make decisions with specific criteria. 
Although some criteria guide the other strategies, most of them present a majority of executions compromised during the experiments.
This result provides evidence for the claim that \system improves the dependability, reducing the probability that $f+1$ \glspl{os} eventually become compromised. 
Nevertheless, the results from May deserved a more careful inspection. 
We have identified some \glspl{cve} that make it very difficult to survive to common vulnerabilities even using the \system strategy.
For example, CVE-2018-1125, CVE-2018-8897, and CVE-2018-8897 that not only affect a few Ubuntu releases but also affect Debian releases simultaneously.
There are also a set of vulnerabilities affecting several Windows releases (e.g., CVE-2018-8134 and CVE-2018-0959). 
We also found a vulnerability (i.e., CVE-2018-1111) that affects few Fedora releases and one Redhat release that was repeatedly compromising the executions.
We have made a more careful inspection of the dispersion of vulnerabilities across these months. 
Although May is not the month with most vulnerabilities, it has on average $3.05$ \glspl{os} affected by a vulnerability, and it is the highest.
This result was already observed in our previous work (see Figure~\ref{top}) where the number of shared vulnerabilities decreases drastically between three affected \glspl{os} to four, almost $50$ vulnerabilities affecting three \glspl{os} to less than ten affecting four \glspl{os}. 

One of the outcomes of this experiment is that contrary to the intuition, not guided strategies will eventually create unsafe configurations.
%The most relevant result shows that contrary to intuition, changing \glspl{os} every day with no criteria will always create unsafe configurations.
Therefore, it is paramount to have selection strategies like the ones we use in \system.

\subsection{Diversity vs Attacks}

This experiment evaluates the strategies when facing notable attacks/vulnerabilities that appeared in the last year. 
Each attack potentially exploits several flaws, some of which affecting different OSes. 
The attacks were selected by searching the security news sites for high impact problems, most of them related to more than one CVE. 
As some of the CVEs include applications, we added more vulnerabilities to the database for this purpose. 
We have considered the attacks described in Table~\ref{tab:special_vulns}: 
WannaCry~\cite{wannacry}, Stackclash~\cite{stacklash}, and Petya~\cite{petya}.


 \begin{table}[h]
 \begin{center}
 {
\small%
% \scriptsize
 \begin{tabular}{ | p{0.96\columnwidth} | }\hline
 \textbf{WannaCry:} 
 \emph{On Friday, May 12, 2017, the world was alarmed to discover a widespread ransomware attack that hit organizations in more than 100 countries. Based on a vulnerability in Windows' SMB protocol (nicknamed EternalBlue), discovered by the NSA and leaked by Shadow Brokers.} \\
 %CVE-2017-0143, CVE-2017-0144, CVE-2017-0145, CVE-2017-0146, CVE-2017-0147, CVE-2017-0148 \\ 
 \hline

 \textbf{StackClash:} 
 \emph{In its 2017 malware forecast, SophosLabs warned that attackers would increasingly target Linux. The flaw, discovered by researchers at Qualys, is in the memory management of several operating systems and affects Linux, OpenBSD, NetBSD, FreeBSD and Solaris.}\\
 %CVE-2017-1000365, CVE-2017-1000366, CVE-2017-1000367, CVE-2017-1000369, CVE-2017-1000370, CVE-2017-1000370, CVE-2017-1000371, CVE-2017-1000372, CVE-2017-1000373, CVE-2017-1000374, CVE-2017-1000375, CVE-2017-1000376, CVE-2017-1000379, CVE-2017-1083, CVE-2017-1084, CVE-2017-3629, CVE-2017-3630, CVE-2017-3631\\ 
 \hline

 \textbf{Petya:} 
 \emph{Another month, another global ransomware attack. [...] Just as it seemed that the threat of WannaCry has dissipated, organisations around the world are finding themselves under siege from a new threat. This cyberattack first hit targets in Ukraine, including its central bank, main international airport and even the Chernobyl nuclear facility before quickly spreading around the globe, infecting organisations across Europe, North America and even Australia. A day after the incident began, at least 2,000 attacks have been recored across at least 64 countries.}\\
 %CVE-2017-0199, CVE-2017-0204, CVE-2017-0207, CVE-2017-2605, CVE-2017-0197, CVE-2017-0195, CVE-2017-0194, CVE-2017-0106\\ 
 \hline
 \end{tabular}
 }
 \caption{Notable attacks during 2017.}
 \label{tab:special_vulns}
 \end{center}
 \end{table}


Since some of these attacks might have been prepared months before the vulnerabilities are publicly disclosed, we augmented the execution phase to the full eight months. 
Therefore, we set \emph{learning phase} to begin \emph{2014-1-1} and to end on \emph{2017-12-31}. 
The \emph{execution phase} period that starts on 2018 January and finishes on August 2018. 
As before, the strategies are executed over $1000$ runs. 


\begin{figure}[h]
\begin{center}
\includegraphics[width=\columnwidth]{images/gnuplot/attacks/attacks.pdf}
\caption{Compromised runs with notable attacks.}
\label{fig:special_vulns}
\end{center}
\end{figure}

\textbf{Results:}
Figure~\ref{fig:special_vulns} shows the percentage of compromised runs for each attack and all attacks put together.
\system is the best at handling the various scenarios, with almost no compromised executions.
The StackClash is the most destructive attack as it is the one affecting more \glspl{os}.
Therefore, decisions guided by criteria that aim to avoid common vulnerabilities may also fail. 
Nevertheless, the results show that such strategies do improve the resilience to attacks.



\section{Final Remarks}
\label{sec:finalremarkslazarusdesign}

\system addresses the long-standing open problem of evaluating, selecting, and managing the diversity of a \gls{bft} system to make it resilient to malicious adversaries.
This thesis focuses on two fundamental issues: (1) how to select the best replicas to run together given the current threat landscape, which was presented in this chapter, and (2) what is the performance overhead of running a diverse \gls{bft} system in practice, which is presented in the next chapter.


\chapter{\system Implementation}
\label{chap:lazarus_implementation}

\chapter{\sieveq: A BFT Firewall}
\label{chap:sieveq}

%In the previous chapter, we presented a control plane for \gls{bft} systems, named \system. 
%Although we have already introduced \sieveq in the evaluation of \system, we will detail \sieveq contributions in this chapter.
In the previous chapter, we presented a study that demonstrates the potential gains of using diverse (\gls{os}) components in \gls{bft} replicated systems.
Here, we present a \gls{bft} solution that provides fault and intrusion tolerance by employing an architecture based on two filtering layers, enabling efficient removal of invalid messages at early stages in order to decrease the costs associated with \gls{bft} replication in the later stages.
This system is \sieveq, a message queue service that protects and regulates the access to critical systems, in a way similar to an application-level firewall.


\section{Firewalls Limitations}

Firewalls are one of the primary protection mechanisms against external threats, controlling the traffic that flows in and out of a network. 
Typically, they decide if a packet should go through (or be dropped) based on the analysis of its contents. 
Over the years, this analysis has been performed at different levels of the \gls{osi} model (see~\cite{Keromytis:2006} for a comprehensive survey), but the most sophisticated rules are based on the inspection of application data included in the packets.
State-of-the-art solutions for application-level firewalls include network appliances from several vendors, such as Juniper~\cite{juniper}, Palo Alto~\cite{paloalto}, and Dell~\cite{sonicwall}.
Such appliances are strategically placed on the network borders, and therefore, the security of the whole infrastructure relies on them.
A skilled attacker, however, may find weaknesses to compromise the firewall's detection/prevention capabilities.
When this happens, critical services under the protection of such devices may be affected, as in the \gls{scada} systems targeted in the Stuxnet~\cite{stuxnet:2010} or Dragonfly~\cite{dragonfly:2014} attacks.


Most generic firewall solutions suffer from two inherent problems: 
First, they have vulnerabilities as any other system, and as a consequence, they can also be the target of advanced attacks. 
For example, the \gls{nvd}~\cite{nvd} shows that there have been many security issues in commonly used firewalls. 
\gls{nvd}'s reports present the following numbers of security issues between 2010 and 2018: 205 for the Cisco Adaptive Security Appliance; 123 in Juniper Networks solutions; and 50 related to iptables/netfilter. 
Common protection solutions often have been the target of malicious actions as part of a wider scale attack (e.g., anti-virus software~\cite{Chauhan:2011}, \gls{ids}~\cite{Anderson:2001} or firewalls~\cite{Kamara:2003,Surisetty:2010,cisco1,cisco2}).
Second, firewalls are typically a single point of failure, which means that when they crash, the ability of the protected system to communicate may be compromised, at least momentarily.
Therefore, ensuring the correct operation of the firewall under a wide range of failure scenarios becomes imperative.
To tolerate faults, one typically resorts to the replication of the components.

%Primary-backup replication (also called $1+1$ replication) would suffice to tolerate a single crash fault on a firewall.
%If the primary replica crashes, the backup replica is able to replace it and deliver the requests. 
%If the system wants to tolerate arbitrary failures, e.g., replicas that suffer intentional or non-intentional Byzantine faults, then it needs a more complex type of replication. 
%For example, if one of the two replicas is misbehaving then the node implementing the critical service will be unable to distinguish which replica is delivering the correct message. 
%To address this difficulty, it is necessary to collect a majority of correct messages, which requires the addition of more replicas. 
%A system that needs to tolerate a single Byzantine fault must have at least $4$ replicas~\cite{Bracha:1985}.


In this chapter, we present a new protection system called \sieveq that mixes the firewall paradigm with a message queue service, with the goal of improving the state-of-the-art approaches under accidental failures and/or attacks.
The solution has a fault- and intrusion-tolerant architecture that applies filtering operations in two stages acting like a sieve.
The first stage, called \emph{pre-filtering}, performs lightweight checks, making it efficient to detect and discard malicious messages from external adversaries.
In particular, messages are only allowed to go through if they come from a pre-defined set of authenticated senders.
\gls{dos} traffic from external sources is immediately dropped, preventing those messages from overloading the next stage.
The second stage, named \emph{filtering}, enforces more refined application level policies, which can require the inspection of some message fields or need the enforcement of specific ordering rules.


%Different fault tolerance mechanisms are employed at the two stages. 
%Pre-filtering is implemented by a dynamic group of nodes named \presieves. 
%\Presieves can be the target of various kinds of attacks and eventually may be intruded because they face the external network. 
%Therefore, we take the conservative approach of assuming that \presieves can fail in an arbitrary (or Byzantine) way, meaning that they may crash or start to act maliciously.
%When a failed \presieve is detected, it is merely replaced by a new one that is clean from errors.
%Since \sieveq needs to support different message loads, e.g., due to additional senders, \presieves can be created dynamically to amplify the aggregated processing capabilities (within the %constraints of the hardware).
%The filtering stage is performed by a group of \repsieve components, which execute as a replicated state machine~\cite{Schneider:1990}.
%\Repsieves may also fail in an arbitrary way, and therefore, we employ an intrusion-tolerant replication protocol that ensures correct operation in the presence of Byzantine faults.

%\sieveq was experimentally evaluated in different scenarios.
%The results show that it is much more resilient to DoS attacks and various kinds of intrusions than existing replicated-firewall approaches.
%We also evaluated \sieveq considering the protection of a \gls{siem} system.
%The test environment emulated the setup of the 2012 Summer Olympic Games, where the same sort of security events was generated and transmitted across the network. 
%The experiments demonstrate that \sieveq can handle a workload up to sixteen times higher than the observed load in the 2012 Summer Olympic Games, without a noticeable degradation in performance.


\section{Intrusion-Tolerant Firewalls}
\label{building_blocks}

In the last decade, several significant advances occurred in the development of intrusion-tolerant systems.
However, to the best of our knowledge, very few works proposed intrusion-tolerant protection devices, such as firewalls.
Performance reasons might explain this, as \gls{bft} replication protocols are usually associated with significant overheads and limited scalability.
Additionally, achieving complete transparency to the rest of the system can be challenging to reconcile with the objective of having fast message filtering under attack.

\begin{figure}[t]
\begin{center}
\includegraphics[width=.7\columnwidth]{images/images/arch_traditional.pdf}
\caption{Architecture of a state-of-the-art replicated firewall.}
\label{fig:traditional}
\end{center}
\end{figure}

Figure~\ref{fig:traditional} shows an implementation of an intrusion-tolerant firewall, illustrating  existing works in this area~\cite{Sousa:2010,Roeder:2010}.
In this design, a sender transmits the messages through the network (e.g., the internet) towards the receiver.
As packets reach the firewall, they are disseminated to the replicas. 
Each replica applies the same filtering rules to decide whether the messages are acceptable. 
Invalid messages are discarded (and eventually logged). 
Messages deemed valid are conveyed to the receiver together with a proof of validity, which demonstrates that a sufficiently large quorum of replicas agrees on their validity.
The proof of validity is checked by a voter module at the receiver, before delivering the messages to the receiving application.
Thus, if a compromised replica produces a message with malicious content, it will be eliminated as it lacks the necessary proof of validity or it is in conflict with the messages transmitted by the other correct replicas.

Although this architecture has interesting characteristics, such as an increased failure resilience, it suffers from some fundamental limitations:

\begin{enumerate}

\item The dissemination of a message to all replicas can be detrimental to the proper operation of the firewall. 
For example, a traffic replicator device (e.g., hub) can be placed at the entry of the firewall to reproduce all messages~\cite{Sousa:2010,Roeder:2010} transparently. 
An obvious consequence of this approach is that malicious messages from an external attacker are also replicated, and therefore, all replicas have to spend the same effort to process them.
As a consequence, the attack is amplified by the replicator device.
Alternatively, a leader replica could receive the traffic and then disseminate the messages to the others~\cite{Roeder:2010}.
The drawback is that the leader becomes a natural bottleneck, especially when under attack (instead of dispersing the attack load over all replicas~\cite{Amir:2011}).

\item The support for stateful firewall filtering requires that all correct replicas process messages in the same order~\cite{Schneider:1990}.
As a consequence, to ensure an agreement in a common sequence of messages, replicas need to continuously run a \gls{bft} consensus protocol~\cite{Castro:2002} to establish message ordering.
A significant amount of work can be wasted with malicious messages since all messages have to be agreed. This is particularly relevant because a consensus protocol consumes both computational and network resources.

\item The creation and check of the proof of validity can be a complex task. For example, one approach requires a trusted component to be deployed in the replicas to generate a \gls{mac} as a proof that a message is valid~\cite{Sousa:2010}. 
The component only returns the \gls{mac} when a quorum of replicas accepted the message. 
Another solution uses threshold cryptography to ensure that every replica can individually produce a partial signature (that corresponds to a part of the proof)~\cite{Roeder:2010}. 
To recreate the full proof, the voter needs to wait for the arrival of a quorum of partial proofs. 
When building a firewall, it would be useful if a more straightforward approach could be employed, with no need for specialized trusted components or expensive threshold cryptography.

\end{enumerate}

\begin{figure}[t]
\begin{center}
\subfloat{\includegraphics[width=0.5\columnwidth]{images/gnuplot/sieveq/plots/latency_tradiotional_attack.pdf}}
\hspace{-5mm}
\subfloat{\includegraphics[width=0.5\columnwidth]{images/gnuplot/sieveq/plots/server_throughput_traditional_attack.pdf}}
\caption{Effect of a DoS attack (initiated at second 50) on the latency and throughput of the intrusion-tolerant firewall architecture displayed in Figure \ref{fig:traditional}.}
\label{fig:attack_traditional}
\end{center}
\end{figure}

These drawbacks can have a significant impact on the firewall performance depending on the considered setting. 
For example, a typical \gls{dos} attack can substantially decrease the throughput and lead to several orders of magnitude growth in the latency of message delivery. 
To illustrate this behavior we implemented the architecture of Figure~\ref{fig:traditional} and launched a \gls{dos} attack on this system (see Section~\ref{evaluation} for a description of the setup and environment). 
The results show that the system performance is significantly affected by the attack (see Figure~\ref{fig:attack_traditional}).




In \sieveq, we explore a different design for replicated protection devices, where we trade some transparency on senders and receivers for a more efficient and resilient firewall solution.
In particular, we propose an architecture in which critical services and devices can only be accessed through a message queue and implement the application-level filtering in this queue.
It is assumed that these services have a limited number of senders, which can be appropriately configured to ensure that only they are authorized to communicate through \sieveq.

\section{Overview of \sieveq}
\label{architecture}

Typical resilient firewall designs are based on primary-backup replication, and consequently, they are able to tolerate only crash failures.
Therefore, more elaborated failure modes may allow an adversary to penetrate the protected network.


Some organizations deal with crashes (or \gls{dos} attacks) by resorting to several firewalls to support multiple entry points. 
This solution is helpful to address some (accidental) failures but is incapable of dealing with an intrusion in a firewall.
In this case, the adversary gains access to the internal network, enabling an escalation of the attack, which at that stage can only be stopped if other protection mechanisms are in place.

\sieveq provides a message queue abstraction for critical services, applying various filtering rules to determine if messages are allowed to go through.
\sieveq is not a conventional firewall, and we do not claim that it should replace existing firewalls in all deployment scenarios.
We are focusing on service- or information-critical systems that require a high level of protection, and therefore, justify the implementation of advanced replication mechanisms.
The system we propose is able to deliver messages while guaranteeing authenticity, integrity, and availability.
As a consequence, and in contrast to conventional firewalls, we lose transparency on senders and receivers, since they are aware of the \sieveq's end-points.
The rest of the section explains how we address some of the mentioned issues and introduces the main design choices and the architecture of \sieveq.

\subsection{Design Principles}
Our solution was guided by the following  principles:

\begin{itemize}

\item \emph{Application-level filtering}: support sophisticated firewall filtering rules that take advantage of application knowledge. \sieveq implements this sort of rules by maintaining state about the existing flows, and this state has to be consistently replicated using a \gls{bft} protocol.

\item \emph{Performance}: address the most probable attack scenarios with highly efficient approaches, and as early as possible in the filtering stages; Reduce communication costs with external senders, as these messages may have to travel over high latency links (e.g., do not require message multicasts).	

\item \emph{Resilience}: tolerate a broad range of failure scenarios, including malicious external/internal attackers, compromised authenticated senders, and intrusions in a subset of the \sieveq components; Prevent malicious external traffic from reaching the internal network by requiring explicit message authentication.

\end{itemize}

\subsection{\sieveq Architecture}

A fundamental difference between the \sieveq architecture and the other replicated firewall designs (see Figure~\ref{fig:traditional}), is the separation of filtering in several stages.
The rationale for this change is to gain flexibility in the filtering operations while ensuring better performance under attack, retaining the ability to tolerate intrusions.
As observed previously, despite the significant improvements in state-of-the-art \gls{bft} implementations, there is an inherent trade-off between the benefits of \gls{bft} replication and its performance, namely due to the need to disseminate (and eventually authenticate) all messages at the replicas, which includes both valid and invalid messages.

\begin{figure}[t]
\begin{center}
\includegraphics[width=0.7\columnwidth]{images/images/arch.pdf}
\caption{\sieveq layered architecture.}
\label{fig:arch}
\end{center}
\end{figure}


Figure~\ref{fig:arch} presents the architecture of \sieveq. 
In this architecture, message processing starts with a first filtering layer that implements a message authentication mechanism and is responsible for discarding most of the malicious traffic efficiently. 
This layer is based on a set of \presieve modules, each of them in charge of the communications with a subgroup of senders. 
During a typical operation, a sender only interacts with its own \presieve.
The assignment of a \sender to a \presieve is done during the channel setup.
Initially, a sender connects with one of a few statically-configured \presieves.
If a \presieve becomes overloaded, it will request the creation of more \presieves (see details in Section~\ref{faultypresieve}) and/or hand of the new \sender channel to another node.



The messages are sent to the second filtering layer by the \presieve to perform a more detailed inspection, which can take advantage of state information kept from previous messages and application-related rules.
This layer is implemented by a group of \repsieves acting together as a \gls{bft} replicated state machine.
They receive all accepted messages from the \presieves and process them in the same order, which guarantees that every \repsieve reaches the same decision (discard or accept a message).
However, if $f$ replicas are faulty, their output could be different.
Consequently, as long as more than two-thirds of the \repsieves are correct, the right decision is taken by the \postsieve by performing a message voting.

In the following, we describe each module of \sieveq presented in Figure~\ref{fig:arch}:


\paragraph{\Sender.} The sender nodes cooperate with \sieveq to secure the messages by deploying a \sender module locally. 
Its primary role is to secure the messages and assist in the detection of some intruded \sieveq components. 
The module can be implemented inside the sender's \gls{os} (e.g., as a kernel module or a specialized device driver) or as a library to be linked with the applications.
The decision to have this module corresponds to a trade-off in our design, where we are willing to lose some transparency to improve the system's resilience.

\paragraph{\Presieve.}
These are the \sieveq front-end, and although it only performs stateless filtering to improve efficiency, it can deter the most common attacks. 
\Presieve modules discard invalid messages, while the approved ones are forwarded to the \repsieve using a Byzantine \gls{tom} protocol~\cite{Bessani:2014}. 
The \presieves can be deployed, for instance, as \glspl{vm}. 
The effect of this layer is a significant reduction in the communication and computational overhead caused by malicious packets at the \repsieve.

\paragraph{\Repsieve.}
These components implement a replicated filtering service that tolerates Byzantine faults. 
The actual filtering rules can be more or less complex depending on the needs of the critical service. 
The \gls{tom} ensures that \repsieves receive the messages in the same order. 
Consequently, identical rules are applied across the \repsieves, and therefore, the same decision should be reached on the validity of messages. 
Each \repsieve individually transmits approved messages to the final receiver (the others are dropped). 
Overall, this layer allows for sophisticated filtering as the replicas are stateful.

\paragraph{\Postsieve.} This module runs on the receiver side, and it is responsible for the delivery of messages to the application.
A \postsieve carries out a voting operation on the arriving data because a \repsieve might be intruded and corrupt messages.
It delivers a message to the application only after receiving the same approved message from a quorum of \repsieves.
From a deployment perspective, this module can be implemented in the \gls{os} or as a library, as in the sender.

\paragraph{Controller.} This module is a trusted component of \sieveq that runs with high privilege.
It takes input from the \repsieves to decide on the creation or destruction of \presieves if some misbehavior is observed.
Depending on the actual \sieveq implementation, it can be developed in different ways.
This sort of component was used in previous works, and it can be implemented both in a centralized~\cite{Roeder:2010,Platania:2014} or distributed~\cite{Sousa:2010} way.
%In the same way, we have presented two possible solutions for the \system controller in Chapter~\ref{chap:lazarus_implementation}.


\subsection{Resilience Mechanisms}

The \sieveq architecture is built to tolerate both faults with an accidental nature (e.g., crashes) and caused by malicious actions (e.g., a vulnerability is exploited, and a specific module is compromised).
To be conservative, we assume that all failed components are controlled by a single entity, which will make them act together in the worst possible manner to defeat the correctness of the system. 
Therefore, failed components can, for instance, stop sending messages, produce erroneous information, or try to delay the system. 
\sieveq performs several mitigation actions to guarantee a valid operation (as long as the number of faults is within the assumed bounds, see Section~\ref{fault_model}).


The most common attack scenario occurs when an external adversary attempts to attack a system that is being protected by \sieveq. 
He can deploy many nodes, whose aim is to delay the communications or bypass \sieveq protection and reach the internal network. 
\sieveq addresses these attacks by discarding unauthenticated or corrupted messages with minimal effort at the \presieve filtering stage.
As with any other firewall, if a \gls{dos} attack completely overloads the incoming channels, \sieveq cannot handle or react to the attack.
The network needs to include other defense mechanisms to deal with this sort of problem~\cite{Mishra:2011}.


As (authenticated) senders might be spread over many (outside) networks, it is advisable to consider a second scenario where an adversary is capable of taking control of some of these nodes. 
In this case, we assume that the adversary gains access to all data stored locally, including the \sender keys. 
Thus, he will be able to generate traffic that is correctly authenticated, allowing these messages to go through the first filtering step.
The messages are however still checked against the application-related rules (namely, the ones defined in the \repsieve), which can cause most malicious traffic to be dropped (e.g., a pre-defined \sender can only send messages to a particular \postsieve accordingly to a specific application protocol).
If the messages follow all the rules, the firewall has to forward them because they are indistinguishable from any other valid messages.


A third scenario occurs when the adversary can cause an intrusion in \sieveq and compromises a few of the \presieves and/or \repsieves. 
When this happens, these components can act in an erroneous (Byzantine) way. 
However, unlike with an intruded \sender, malicious \presieves cannot generate fully authenticated messages, since they lack all the required keys. 
They can still perform \gls{dos} attacks on the \repsieves, e.g., by transmitting many messages, but this strategy creates apparent misbehavior allowing immediate discovery. 
Malicious \presieves are detected with the assistance of correct \sender and \repsieves, eventually leading to their substitution.


\Repsieves modules are much harder to exploit because they do not face the external network. 
However, if they end up being intruded, \repsieves can produce arbitrary traffic to the internal network. 
\Postsieve addresses this issue by carrying out a voting step, which excludes these messages. Moreover, an alarm is generated and sent to the \emph{controller}.


The \sieveq architecture does not attempt to recover from intrusions in the controller and \postsieve.
The first is assumed to be trusted, as it is deployed in a separated administrative domain and is to be used only in a few particular operations.
Moreover, its simplicity allows the audit of its code and ensures correctness with a high level of confidence.
The \postsieve already runs in the internal network, and therefore, \sieveq can not preclude its misbehavior.


%%%%%%%%%%%%%%%%%%%%%%




\section{\sieveq Protocol}
\label{protocol}

This section details the \sieveq protocol and service properties. 
We conclude the section with an analysis of the behavior of the system under different kinds of attacks and component failures and highlight how the countermeasures integrated into our design mitigate such threats.

\subsection{System and Threat Model}
\label{fault_model}

The system is composed of a (potentially) large number of external nodes, called \emph{senders}, some internal nodes, called \emph{receivers}, and \sieveq nodes.
Senders run \sender modules to be able to transmit packets through the \sieveq, while receivers receive validated messages by using \postsieve modules. 

Communications can experience accidental faults or attacks.
Thus, packets might be lost, delayed, reordered or corrupted, but we assume that if messages are retransmitted, eventually they will be correctly received by \sieveq.
The fault model also assumes that \sender, \presieve, and \repsieve nodes can suffer from arbitrary (Byzantine) faults.
When this happens, failed nodes may perform actions that deviate from their specification, including colluding against the system.
However, at most $f_{ps}$ \presieves from a total of $N_{ps} = f_{ps} + k$ (with $k > 1$), and $f_{rs}$ \repsieve from a total of $N_{rs} = 3f_{rs}+1$ may fail.
Redundant components should fail independently by employing diversity techniques such as the ones described in Chapter~\ref{chap:lazarus_design}.
A component that is unable to communicate is also considered faulty because from a practical perspective it is indistinguishable from a crashed module.

The cryptographic operations used in the \sieveq protocol are assumed to be secure, and therefore, they cannot be subverted by an adversary. 
Consequently, traditional properties of digital signatures, \glspl{mac}, and hash functions will hold as long as the associated keys are kept safe.
The deployment of \sieveq requires a key distribution scheme to create shared keys between the \sender and the \presieve and to periodically re-issue private-public key pairs for the \sender.
We assume that the key distribution scheme is similar to solutions that already address this sort of problem (e.g.,~\cite{Harkins:1998}).
If required, the key distribution infrastructure could also be made intrusion-tolerant (e.g.,~\cite{Kreutz:2014,Zhou:2002}).


\subsection{Properties}
\label{properties}

\sieveq protocol guarantees the following three properties for messages transmitted from a sender to a receiver:

\begin{security}
If a message, transmitted by a correct sender, is delivered to a correct receiver then the message is in accordance with the security policy of \sieveq.
\end{security}

\begin{validity}
If a correct receiver delivers a message \msg.\texttt{DATA}, then the message was transmitted by \msg.\texttt{sender}.
\end{validity}

\begin{liveness}
If a correct sender sends a message, then the message eventually will be delivered to the correct receiver.
\end{liveness}

These properties require \sieveq to behave in a way similar to most firewalls while offering a few extra guarantees. 
Only external messages that are approved by the policies defined in \sieveq can reach the receivers, and the rest should be dropped (Compliance). 
\Postsieve can use the message field \msg.\texttt{sender} to find who transmitted the message contents (\msg.\texttt{DATA}), and accordingly decide if the message should be delivered to the receiver application (Validity). 
Progress is also ensured, as correct senders eventually can transmit their messages (Liveness).

Besides these functional properties (related to message filtering), \sieveq also ensures a \emph{resilience} property related to the detection and recovery of components of the system that exhibit faulty behavior:


\begin{resilience}
Every component exhibiting observable faulty behavior will be eventually removed or recovered.
\end{resilience}

In the following, we present the mechanisms for implementing the \sieveq functionalities, i.e., the mechanisms for satisfying the three functional properties stated above.
The mechanisms for ensuring resilience will be detailed after that.

\subsection{Message Transmission}
\label{transmittingmessage}


\paragraph{\Sender processing.}

This module gets a buffer with the \texttt{DATA} to be transmitted to a particular application placed behind \sieveq.
The buffer needs to be encapsulated in a message with some extra information required for protection (see Equation \ref{msg}): we add a \sender identifier $\senderi_i$ and a sequence number \sn that is incremented on each message.
This information is needed to prevent replay attacks, either from the network or from a compromised \presieve.
Some information is also added to protect the integrity and authenticate the message.
A signature $\signature_{Si}$ is performed over the message contents, and a \gls{mac} $\mac_{ski}$ is computed using a shared key established with the \presieve.
This \gls{mac} serves as an optimization to speed up checks~\cite{Clement:2009}.

\begin{equation}
\msg = \langle \senderi_i, \sn, \texttt{DATA}, \signature_{Si} \rangle_{\mac_{ski}}
\label{msg}
\end{equation}

\begin{figure}[t]
\centering
\includegraphics[scale=0.5]{images/images/filtering_steps_1.pdf}
\caption{Filtering stages at the \sieveq.}
\label{fig:filters}
\end{figure}


After constructing the message, it is sent to the \presieve assigned to the \sender, and a timer is started.
If this timer expires before the \sender receives an acknowledgment message, it re-sends the \msg to another \presieve.

\paragraph{\Presieve filtering.}

The \presieve determines if an arriving message \msg should be forwarded or discarded (stages (a)--(d) in Figure~\ref{fig:filters}). It applies the following checks to make this decision:

\begin{enumerate}

\item[(a)] \textbf{White list:} each \presieve maintains a list of the nodes that are allowed to transmit messages (i.e., which were authorized by the system administrator). 
Messages coming from other nodes are dropped.
This check is based on the address of the message sender, and therefore it serves as an efficient first test, but it is vulnerable to spoofing.

\item[(b)]  \textbf{Sequence number:} finds out if a message with sequence number \sn from \sender $\senderi_i$ was seen before.
In the affirmative case, the message is discarded to prevent replay attacks.
Messages are also dropped if their \sn is much higher than the largest sequence number ever observed from that \sender (a sliding window of acceptable sequence numbers is used).

\item[(c)]  \textbf{MAC test:} \gls{mac} $\mac_{ski}$ is verified to authenticate the message contents, including to find out if the expected $\senderi_i$  is the sender. 
If the check is invalid, the message is dropped, and the sequence number information is updated to forget that this message was ever received (the update carried out in the previous step needs to be undone).

\item[(d)] \textbf{Grant check:} each \presieve controls the amount of traffic that a \sender is transmitting.
Messages that fall outside the allocated amount are dropped to ensure that all senders get a fair share of the available bandwidth (and to avoid \gls{dos} attacks by compromised senders).
The amount of traffic that is allowed to each \sender is adjusted dynamically based on the available and consumed resources.


\end{enumerate}


Finally, the \presieve invokes the \gls{tom} primitive to forward the message to every correct \repsieve.

\paragraph{\Repsieve filtering.}
When a correct \repsieve delivers a message to be filtered, the following checks are applied:

\begin{enumerate}
\item[(e)] \textbf{Grant check, sequence number, and signature:} since a \presieve may have been intruded and may collude with malicious \sender, extra checks are required on the amount of forwarded traffic and the integrity of the message.
The grant check and the test on the sequence number are similar to the ones performed by the \presieve, and the signature ensures that all \repsieves reach the same decision regarding the validity of the message content.
\item[(f)] \textbf{Application-level rules:} apply the application-defined filtering rules to determine if the message is compliant with the security policy of the firewall.
\end{enumerate}

Although uncommon, \repsieves may receive messages in a different order from what is defined in their sequence numbers.
As a consequence, the \repsieve's application-level rules may drop some of the out-of-order messages, which later on will have to be re-transmitted by the \sender.
For example, if messages \texttt{A} and \texttt{B} should appear in this sequence but are re-ordered, then the rules may consider \texttt{B} invalid and then accept \texttt{A}.
At some point, the \sender would consider \texttt{B} as lost, and re-transmit it.\footnote{It is important to remark that, independently of any violation on the order of delivery of sender messages, all correct \repsieves receive the messages in the same order.}

To address this issue, each \repsieve enqueues messages with a sequence number greater than the expected for a while, as long as they do not exceed a threshold above the last processed sequence number. These messages are processed when either: 1) the missing messages with smaller sequence numbers arrive, and then they are all tested in order, or 2) the \repsieve gives up on waiting and checks the enqueued messages. This last decision is made after processing a pre-determined number of other messages.

Valid messages are encapsulated in a new format (see Equation \ref{msgl}) and are sent to their receivers.
Basically, \repsieve $\replicak_j$ substitutes the signature with a new \gls{mac}, $\mac_{skj}$.
This \gls{mac} is created with a shared key between the \repsieve and the \postsieve.


\begin{equation}
\msg' = \langle \senderi_i, \sn, \texttt{DATA}, \replicak_j \rangle_{\mac_{skj}}
\label{msgl}
\end{equation}

\paragraph{\Postsieve processing.}

\Postsieve accumulates the messages that arrive from \repsieves until enough evidence is collected to allow their delivery.

\begin{enumerate}
\item[(g)] \textit{vote:} A message can be delivered to the application when it is received from $f_{rs} + 1$ \repsieves.

\end{enumerate}

Algorithm \ref{algo:post_filter} presents the \postsieve voting protocol.
The \postsieve starts by checking if the message contains a valid \gls{mac} (lines 5 and 6).
Then, it finds out if the message carries an acceptable sequence number before storing it in a \texttt{WaitingQuorum} set (lines 7 and 8).
Notice that a different set is used for each sender $S_i$ and $sn$ pair.
Messages with sequence numbers already delivered or higher than a threshold (\texttt{snThreshold}) are discarded.

The expected sequence number is stored in the auxiliary variable $k$ (line 9).
Next, the \postsieve tries to find a message with this sequence number and with at least $f_{rs} + 1$ votes (by searching the corresponding set and using function \texttt{equalMsg} --- line 10).
For each different \texttt{DATA} value that may exist in the set, the \texttt{equalMsg} function counts the number of times its appears, and returns the largest count.\footnote{Notice that all correct \repsieves transmit messages with the same \texttt{DATA} value and that malicious replicas can send at most $f_{rs}$ arbitrary \texttt{DATA} values.
Therefore, eventually there will be a \texttt{DATA} value with at least $f_{rs}+1$ votes because there are at least $2f_{rs}+1$ correct \repsieves.}
Next, while there are \msg's with a $f_{rs} + 1$ quorum, \postsieve delivers \msg's in order (line 10-13).
The function \texttt{mostVotedMsg} returns the message with the \texttt{DATA} value with most votes (line 11).
Then, the \texttt{deliver} function delivers \texttt{DATA} to the application on the receiver and deletes the \texttt{WaitingQuorum} set.
Finally, the \texttt{snExpect} is incremented (line 12) and the $k$ index is updated (line 13).
This allows \postsieve to deliver messages in order while there are messages with the expected sequence number in its buffer.

\SetKwInOut{Input}{input}
\SetKwData{repsievealgo}{\repsieve}

\SetKwData{MAC}{$\mac_{si}$}
\SetKwData{Signature}{\signature}
\SetKwData{SeqNumber}{\texttt{\sn}}
\SetKwData{SeqNumberExp}{\texttt{snExpect}}
\SetKwData{SeqNumberThreshold}{\texttt{snThreshold}}
\SetKwData{MESSAGE}{\msg}
\SetKwData{MESSAGEX}{\texttt{aux}}
\SetKwData{DATA}{\texttt{DATA}}

\SetKwData{client}{\senderi$_{i}$}
\SetKwData{prefirewall}{\presieve\emph{$_{i}$}}
\SetKwData{prefirewalltwo}{\presieve\emph{$_{k}$}}
\SetKwData{RepFirewallID}{\repsieve\emph{$_{i}$}}
\SetKwData{WaitingQuorum}{\texttt{WaitingQuorum}}

\SetKwFunction{seqnumber}{expected\_seq\_numb}
\SetKwFunction{macverification}{verifyMAC}
\SetKwFunction{sendVoter}{send\_\postsieve}
\SetKwFunction{applicationRules}{application\_rules}
\SetKwFunction{discard}{discard}
\SetKwFunction{deliver}{deliver\_to\_Application}
\SetKwFunction{existsQuorom}{existsQuorom}
\SetKwFunction{deliver}{deliver}
\SetKwFunction{return}{return}
\SetKwFunction{equalMsg}{equalMsg}
\SetKwFunction{mostVotedMsg}{mostVotedMsg}

{\centering
\begin{minipage}{.8\linewidth}


\begin{algorithm}[H]
%\SetInd{0.3em}{0.2em}
\caption{\postsieve protocol}\label{algo:post_filter}
{
\small

  $Init:\ executed\ only\ once$\\
	$\SeqNumberExp_{Si}, k \leftarrow 0$\;
	$\WaitingQuorum_{Si,\sn} \leftarrow \perp$\;
  \BlankLine
\Fn{Filter (\MESSAGE')}{
    \If{( \macverification{\MESSAGE'} = FALSE )}{
        \return $errorMAC$\;
    }
	\If{ ($\SeqNumberExp_{Si}$ $\leq$ \MESSAGE'.\SeqNumber $<$ ($\SeqNumberExp_{Si}$ + \SeqNumberThreshold)) }{
		$\WaitingQuorum_{Si,\sn} \leftarrow \WaitingQuorum_{Si,\sn} \cup \{ \MESSAGE' \}$\;
	}
	$k \leftarrow \SeqNumberExp_{Si}$\;



	\While{ ( \equalMsg{$\WaitingQuorum_{Si,k}$} $\geq$ $f_{rs}+1$ ) }{

		  	\deliver{\mostVotedMsg{$\WaitingQuorum_{Si,k}$}} \;
		  	$\SeqNumberExp_{Si} \leftarrow \SeqNumberExp_{Si} + 1$ \;
			$k \leftarrow \SeqNumberExp_{Si}$\;

    }
}
}
\end{algorithm}
\end{minipage}
\par
}





\subsubsection{Correctness Argument}
In the following, we show that \sieveq design satisfies the three functional properties described in Section~\ref{properties}.
The proofs work under the assumptions of our system and threat model (defined in Section~\ref{fault_model}).

\begin{validity}

If a correct receiver delivers a message \msg.\texttt{DATA}, then the message was transmitted by \msg.\texttt{sender}.
\end{validity}

\begin{proof}
Assume that a receiver $\receiverj_j$ gets a message content \msg.\texttt{DATA} with the \msg.\texttt{sender} $=$ \msg.\texttt{$\senderi_i$}.
For this to happen, the \postsieve of $\receiverj_j$ waited for the arrival of at least $f_{rs}+1$ correctly authenticated messages\footnote{Note that \repsieves send $\msg'$ (defined in (\ref{msgl}))  instead of \msg (defined in (\ref{msg})). Both are similar, but \msg' has a \gls{mac} instead of a signature to authenticate its contents with \postsieve. For the sake of simplicity we will only use the $\msg$ notation in the rest of the proof.} with equal \msg.\texttt{$\senderi_i$}, \msg.$\sn$, and \msg.\texttt{DATA}.
Therefore, at least $f_{rs}+1$ \repsieves received $\msg$ signed by $\senderi_i$, and verified the signature $\signature_{Si}$ as valid.
This indicates that \msg.\texttt{DATA} was not modified by the network or by any \presieve.
Only a sender \msg.\texttt{$\senderi_i$} (correct or not) can create and authenticate a $\msg$ with its signature.
Therefore, $\senderi_i$ is the \msg.\texttt{sender}, i.e., the creator of \msg.\texttt{DATA}.
\end{proof}


\begin{security}
If a message, transmitted by a correct sender, is delivered to a correct receiver then the message is in accordance with the security policy of \sieveq.
\end{security}


\begin{proof}
The proof of the \emph{Compliance} property is very similar to the \emph{Validity} proof.
The difference is that, we also know that if $f_{rs}+1$ \repsieves sent $\msg$ to a \postsieve, then $\msg$ was verified against the security policy in at least one correct \repsieve, which approved it.
\end{proof}



\begin{liveness}
If a correct sender sends a message, then the message eventually will be delivered to the correct receiver.
\end{liveness}

\begin{proof}
For the sake of simplicity, our proof only considers nodes that drop or delay messages.
Attacks to the integrity will make the message be discarded.

Assume that a sender $\senderi_i$ transmits a message $\msg$ with the sequence number $\sn$ to a \presieve $\presievei_u$.
Then $\senderi_i$ sets a timer $\mathit{timer_{sn}}$ for \msg.\sn.
If $\presievei_u$ is correct, after it receives a message, it re-sends $\msg$ via \gls{tom} to the \repsieves.
At least $f_{rs}+1$ correct \repsieves will send $\msg$ to the \postsieve, which will deliver  \msg.\texttt{DATA} to the receiver $\receiverj_j$.
If the $\presievei_u$ is faulty, i.e., drops or delays messages, $timer_{sn}$ in $\senderi_i$ will eventually expire.
When this happens, $\senderi_i$ re-transmits $\msg$ to another \presieve.
Eventually, this process will make some correct \presieve forward $\msg$ to the \repsieves using the \gls{tom} primitive, which will cause \msg.\texttt{DATA} to be delivered to $\receiverj_j$.
\end{proof}


\subsection{Addressing Component Failures}
\label{sec:failures}

In this section, we discuss the implications of failures in the different components of the \sieveq, and how they are handled for ensuring the \emph{Resilience} property stated in Section~\ref{properties}.

In the Byzantine model, every failed component can behave arbitrarily, intentionally or accidentally. Therefore, the \sieveq design incorporates mechanisms that are resilient to different failure scenarios. 
Given the architecture of Figure~\ref{fig:arch}, one has to address faults in authenticated \senders, \presieves and \repsieves, as they are the main components subject to Byzantine failures in our model. 
The \postsieve is not considered in this section as it is co-located with the receiver.
Therefore, we can not make further assumptions about it. 
We assume the \emph{controller} as trusted in this chapter.
In Chapter~\ref{chap:lazarus_implementation} we present an intrusion-tolerant solution for a different controller component that can be adapted for this \emph{controller}.


In the following, we try to focus on complex scenarios whereas faulty components send syntactically-valid messages. 
Therefore avoiding cases that could be easily detected and recovered by existing network monitoring and protection tools (e.g., it is easy to discover that a \presieve is sending messages to another \presieve, something our protocol does not allow).

Since \presieves are directly exposed to the external network, there is a higher risk of them being compromised.
Then, to keep the \sieveq operational, it is required that failed \presieves be identified and recovered.
We leverage from the \repsieve setup to perform failure detection, and then use the \emph{controller} to restart erroneous \presieves.
Replacing these components is almost trivial because they are stateless.

\Repsieves execute as a \gls{bft} replicated state machine, processing messages in the same order and producing identical results.
Consequently, \repsieves faults can be tolerated by employing a voting technique on the \postsieve that selects results supported by a sufficiently large quorum (as explained above, an output with at least $f_{rs} + 1$ votes).
Below, we discuss in more detail a few failure scenarios.



\subsubsection{Faulty \Sender}
A faulty \sender is authorized to communicate while suffering from some arbitrary problem (e.g., intrusion). 
Therefore, it can produce correctly authenticated messages to attack the firewall. 
In some scenarios, it is possible to discard these messages. 
For example, if the \sieveq receives a correctly signed message with a sequence number is higher than what was expected, then it can be easily detected and eliminated (checks \emph{(b)} and \emph{(e)} in Figure~\ref{fig:filters}).

A more demanding scenario occurs when a \sender transmits faster than the allowed rate (verification (d) of Figure~\ref{fig:filters}).
In this case, some defense action has to be carried out, as these attacks can lead \sieveq to waste resources.
To be conservative, we decided to follow a simple procedure to protect the firewall: \sieveq maintains a counter per sender that is incremented whenever new evidence of failure can be attributed to it, e.g., the faulty \sender is overloading the system with invalid messages or if it is sending messages to \presieves which it was not assigned.
When the counter reaches a pre-defined value, the \sender is disallowed from communicating with \sieveq by temporarily removing it from the whitelist and adding it to a quarantine list (failing verification (a) of Figure~\ref{fig:filters}) and by giving a warning to the system administrator.
\emph{This ensures that faulty \emph{\senders} are eventually removed from the system.}

Excluded \senders may regain access to the service later on because the counter is periodically decreased (when the counter falls below a certain threshold, the \sender is moved back into the whitelist). 
Additionally, the administrator is free to update the white/quarantine lists. 
For instance, he may choose to manually add a faulty sender to the quarantine list or even deploy policies to do that under certain conditions (e.g., if the same sender is in the quarantine list for a certain number of times).


\subsubsection{Faulty \Presieve}
\label{faultypresieve}
Addressing failures in \presieves is difficult because these components may look as compromised even when they are correct.
Notably, when a \presieve is under a \gls{dos} attack, messages can start to be dropped due to buffer exhaustion, and this is indistinguishable from malicious behavior in which messages are selectively discarded.
In the same way, a failed signature check at a \repsieve indicates that either the \presieve is faulty (it is tampering/generating invalid messages) or that a \sender is misbehaving (recall that a \presieve verifies the message \gls{mac}, but not its signature).
Finally, a \repsieve may also detect problems if it observes a sudden increase in the arrival of messages (above the grant check), which could indicate a \gls{dos} attack by a malicious \presieve (maybe colluding with a compromised \sender).
This kind of ambiguity precludes accurate failure detection, and consequently, we aim to provide a mechanism that allows \sieveq to recover from end-to-end problems and continue to deliver a correct service.

An initial step to deal with these complex failure scenarios is to make \presieves evaluate their own state.
This is done by analyzing the amount of arriving traffic and by observing if it could overload the \presieve.
The analysis can be done by measuring the inter-arrival times of messages over a specified period.
If those intervals are small (on average), there is a chance that the \presieve is working at its full capacity or is even overloaded.
When this happens, the \presieve broadcasts (using the \gls{tom} primitive) a \texttt{WARNREQ} message to the \repsieves, so that they may take some action to solve the problem (see below).


When the \presieve is faulty, the \sender and \repsieves need to detect it together.
\sieveq provides a procedure to find how many messages are being discarded on the \presieve:


\begin{enumerate}

\item Periodically, the \sender sends a special \texttt{ACKREQ} request to \repsieves, in which it indicates the sequence number of the last message that was sent (plus a signature and a \gls{mac}).
This request is first sent to the preferred \presieve, but if no answer is received within some time window, and then it is forwarded to another \presieve.
The waiting period is adjusted in each retransmission by doubling its value.

\item When the \repsieves receive the request, the included sequence number together with local information is used to find how many messages are missing. 
The local information is the set of sequence numbers of the messages that were correctly delivered since the last \texttt{ACKREQ}.

\item Based on the number of missing messages, the \repsieves transmit through the same \presieve a response \texttt{ACKRES} to the sender, where they state the observed failure rate and other control information (plus a signature).
\Repsieves may also perform some recovery action if the failure rate is too high.

\end{enumerate}

Additionally, if a \repsieve detects that a signature is invalid or that it is receiving more messages than the expected (check (e) in Figure \ref{fig:filters}), it suspects the \presieve that sent the messages and takes some action.


Once the problem is detected, the \repsieves should attempt to fix the erroneous behavior by employing one of three possible remediation actions, depending on the extent of the perceived failures:

\begin{itemize}


\item \emph{Redistribute load:} if a \presieve has sent a warning about its load, or a high failure rate related with this component is observed, the first course of action is to move some of the messages flows from the problematic component to other \presieve.
This is achieved by specifying, in the \texttt{ACKRES} response to a \sender, the identifier of a new \presieve that should be contacted.
At that point, the \sender is expected to connect to the indicated \presieve and begin sending its traffic through it.

\item \emph{Increase the \emph{\presieves} capacity:} if the existing \presieves are unable to process the current load, then the \sieveq needs to create more \presieves (depending on the available resources).
To do that, \repsieves contact the \emph{controller} informing that an extra \presieve should be started.
When the controller receives $f_{rs} + 1$ messages, it performs the necessary steps to launch the new \presieve (which are dependent on the deployment environment).
The new \presieve begins with a few startup operations, which include the creation of a communication endpoint, and then it uses the \gls{tom} channel to inform the \repsieve that it is ready to accept messages from \senders.

\item \emph{Kill the \emph{\presieve}:} when there is a significant level of suspicion on a \presieve, the safest course of action is for \repsieves to ask the controller to destroy it.
Moreover, if the load on the firewall is perceived as having decreased substantially, the \repsieves select the oldest \presieve for elimination, allowing future aging problems to be addressed.
The controller carries out the needed actions when it gets $f_{rs} + 1$ of such requests (once again, which depend on how \sieveq is deployed).
The affected \senders will be informed about the \presieve replacement through the \texttt{ACKREQ} mechanism, i.e., they will eventually use another \presieve to send a request, and get the information about their newly assigned \presieve in the response.
Moreover, the new \presieve is informed about the expected sequence number for each \sender.
This information is stored by \repsieves, which contrary to \presieves are stateful.

\end{itemize}

Together, these mechanisms ensure that \emph{a faulty \emph{\presieve} affecting the \sieveq performance will be eventually removed from the system}.


\subsubsection{Faulty \Repsieve}
\label{faultyrepsieve}

A \postsieve only delivers a message if it receives $f_{rs}+1$ matching approvals for this message.
Given the number of \repsieves, this quorum is achievable for a correct message even if up to $f_{rs}$ \repsieves are faulty.
However, a faulty \repsieve can create a large number of messages addressed to other components of the system, effectively causing a \gls{dos} attack.
Therefore, we need countermeasures to disallow a faulty \repsieve to degrade the performance of the system in a similar way as illustrated in Figure~\ref{fig:attack_traditional}.
For example, a faulty \repsieve can send an unexpected amount of messages to a \presieve, making it slower and triggering suspicions that may lead it to be killed.
Similarly, it can attack other \repsieves to make the system slower.
Finally, a faulty \repsieve can also overload the \postsieve with invalid messages.

Each of these attacks requires a different detection and recovery strategy:

\begin{itemize}

\item When a \presieve is being attacked it complains to the controller, which first requests the \presieve replacement (assuming it might be compromised) and increases a suspect counter against the \repsieve.
If $f_{ps}+1$ \presieves also complain about the same \repsieve, the controller starts an \emph{individual recovery} in this \repsieve (see below).

\item If more than $f_{rs}$ other \repsieves are being attacked, they will probably become slower.
In this case, it is possible to detect such attacks if $f_{rs}+1$ \repsieves complain about a single \repsieve.
This would be possible because target \repsieves would observe unjustifiable high traffic coming from a single replica and because such spontaneously generated messages would be deemed invalid.
When this attack is detected, the controller starts an individual \repsieve recovery.

\item If only up to $f_{rs}$ other \repsieves are being attacked it is still possible to make the system slower without being detected by the previous mechanism.
This happens because $N_{rs}-f_{rs}$ replicas must participate in the \gls{tom} protocol~\cite{Bessani:2014}, and the target $f_{rs}$ plus the attacker intersect this quorum in a least one component. 
This intersecting component will define the pace of the messages coming, which typically will be slow.
We can mitigate this attack as each \repsieve periodically informs the \postsieve about its throughput (number of processed messages per second), piggybacking this value in some approved messages submitted for voting.
The \postsieve verifies that there are \repsieves presenting throughputs lower than expected and initiates a \emph{recovery round} on all the \repsieves (as described below).

\item When the \postsieve is being attacked it detects the abnormal behavior.
Then, it requests the individual recovery of the compromised replica.

\end{itemize}

The previous mitigation mechanisms suggest two kinds of recovery actions.
First, an individual recovery, where the machine is rebooted with clean code (we addressed this in Chapter~\ref{chap:lazarus_design}) and then is reintegrated in the system.
Second, a recovery round is used when a performance degradation attack is detected, but there is no certainty about which \repsieve was compromised.
%Therefore, in this case, we recover all replicas, one after another to avoid unavailability periods in the system, just like in proactive recovery systems~\cite{Castro:2002,Sousa:2010,Roeder:2010}.
%There are several works that present these techniques, and most of them are compatible with our architecture, therefore we refrain from describing them in more detail.
%Notice that the use of a recovery round is enough to ensure that \emph{a faulty \emph{\repsieve} will eventually be recovered in \sieveq}.


\section{Implementation}
\label{implementation}


We implemented a prototype of \sieveq following the specification of the previous section to validate our design.
The \sender and \postsieve were developed as libraries that are linked with the sender and receiver applications. 
The libraries offer an interface similar to the TCP sockets to simplify the integration and minimize changes both in the client and server applications. 
A sender can provide a buffer to be transmitted, and the receiver can indicate a buffer where the received data is to be stored.
Internally, the \sender library adds a \gls{mac} and a signature to each message. 
The \gls{mac} is created using \gls{hmac} with the \gls{sha} (256) hash function and a 256-bit key, and the signature employs \gls{rsa} with 512-bit keys. 
Messages are authenticated at the \postsieve also using \gls{hmac}. 
The \gls{rsa} keys were kept relatively small for performance reasons. 
However, since they should be updated periodically (e.g., every few hours), this precludes all practical brute force attacks~\cite{David:2015}.%%%%%%%%%%%%%%%%%%%%%%%%%%%%%%%%%

The \presieves operate as separate processes receiving the messages and forwarding them to the \repsieves.
We resorted to \textsc{BFT-SMaRt}~\cite{Bessani:2014} to implement the \gls{tom} and manage the \repsieves.
\textsc{BFT-SMaRt}, like most \gls{smr} systems (e.g., \textsc{Pbft}~\cite{Castro:2002}, \textsc{Zyzzyva}~\cite{Kotla:2010}, \textsc{Prime}~\cite{Amir:2011}) follows a client-server model, where a client transmits request messages to a group of server replicas and then receives the output messages with the results of some computation.
We had to modify \textsc{BFT-SMaRt} because this model does not fit well with the message flow of \sieveq.
Therefore, we have decoupled the client-server model into a client-server-client model.
The client sends messages (but does not wait for responses as in traditional \gls{smr}), and the server forwards them (after validation) to another client, which is the last receiver.
A reverse procedure is carried out for the traffic originating from the receiver.
Furthermore, in \textsc{BFT-SMaRt} the replicated servers are typically designed with a single-thread to process requests.
To improve performance, we also modified the system to allow CPU-costly operations (like a signature verification) to occur concurrently with the rest of the checks performed by \repsieves.

In the prototype, a \presieve can be replaced on two occasions: first, voluntarily by asking for a substitution to the \repsieves, when it is flooded with unauthorized messages (\gls{dos} attack); and second, when a \repsieve detects message corruptions by a \presieve.
In both situations, the \repsieves make a request to the \emph{controller}, which will replace a \presieve instance.
Notice that the \emph{controller} has to wait for $\mathit{f_{rs}+1}$ messages, requesting a \presieve replacement, to ensure that a faulty \repsieve cannot force the recovery of a correct \presieve.
The \repsieves are recovered when the collected information indicates that some component might be attacking other components.
The information is collected and sent to the trusted controller to evaluate and decide if the replicas are making progress as expected.
Based on this information, the controller can suspect on $\mathit{f}$ \repsieves and recover them avoiding the need to recover all of them at once.


Since \textsc{BFT-SMaRt} is programmed in Java, we decided to use the same language to develop the various \sieveq components. 
If the sender and receiver applications are coded in other languages, they can still be supported by implementing specific \sender and \postsieve libraries.


\section{Evaluation}
\label{evaluation}


In this section, we present the evaluation of \sieveq under different network and attack conditions.
We present the results of four types of experiments.
In the first one, we evaluate the latency for different \sender workloads, assessing the performance of \sieveq in the absence of failures.
The second experiment assesses the effect of filtering rules complexity on the performance of the system.
In the third experiment, we assess the throughput in three scenarios: \emph{i)} the normal case, \emph{ii)} during a DoS attack without countermeasures; and \emph{iii)} during  a DoS with all the resilience mechanisms enabled (in fact we considered two \gls{dos} attacks: external, from a malicious \sender, and internal, from a compromised \repsieve).
The last experiment considers the capability of \sieveq to safeguard a \gls{siem} system under a similar workload as the one observed in the 2012 Summer Olympic Games.

\subsection{Testbed Setup}

Figure~\ref{fig:testbed} illustrates the testbed, showing how the various \sieveq components were deployed in the machines.
We consider one \sender and one \postsieve deployed in different physical nodes, and an additional host acting as a malicious external adversary.
The \emph{controller} and \presieves were located in the same physical machine for convenience but in different \glspl{vm}.
Four \repsieves were placed in distinct physical nodes.
Every machine had two Quad-core Intel Xeon $2.27$ GHz CPUs, with $32$ GB of memory, and a Broadcom NetXtreme II Gigabit network card.
All the machines were connected by a 1Gbps switched network and run Ubuntu $10.04$ $64$-bit LTS (kernel $2.6.32$-server) and Java $7$ ($1.7.0\_67$).

\begin{figure}[t]
\centering
\includegraphics[width=0.7\columnwidth]{images/images/testbed1.pdf}
\caption{The \sieveq testbed architecture used in the experiments.}
\label{fig:testbed}
\end{figure}


\subsection{Methodology}

\sieveq acts as a highly resilient protection device, receiving messages on one side and forwarding them to the other side.
Therefore, the performance of \sieveq is assessed with latency/throughput measurements that can be attained under different network loads. 
In the experiments, the sender transmits data at a constant rate, i.e., 100 to 10000 messages per second, with three different message payload sizes, i.e., 100 bytes, 500 bytes, and 1k bytes.
We used the Guava library~\cite{guava} to control the message sending rate.


\emph{Latency} measures the time it takes to transmit a message from \sender until it is delivered to the application on the \postsieve. 
The following procedure was employed to compute the latency: the \sender obtains the local time before transmitting a message. 
When the message arrives and is ready to be delivered to the receiver application (after voting), the \postsieve returns an acknowledgment over a dedicated UDP channel.
The \Sender gets the current time again when the acknowledgment arrives. 
The latency of a message is the elapsed time calculated at \sender (receive time minus send time) subtracted by the average time it takes to transmit a UDP message from the \postsieve to the \sender.

\emph{Throughput} gives a measure of the number of messages per second that can be processed by \sieveq. 
It was calculated at the \postsieve using a counter. 
This counter is incremented every time a message is delivered to the receiver application, and the counter is reset to zero after one second. 
Consequently, the server can calculate the number of messages delivered every second. 
The throughput is computed as the average value of the individual measurements collected over a period of time (in our case, 5 minutes after the steady state was reached).

In some experiments, we wanted to assess the behavior of \sieveq under a \gls{dos} attack. 
The attack was made using \texttt{PyLoris}~\cite{pyloris}, a tool built to exploit vulnerabilities on TCP connection handling.
The tool implements the Slowloris attack method, which opens many TCP connections and keeps them open.
The tool allows the user to define parameters like group size of attack threads, the maximum number of connections, and the time interval between connections among others.
In our setup, \texttt{PyLoris} was configured to perform an unlimited number of connections, with $0.1$ milliseconds between each connection.

In all experiments, measurements were taken only after the \gls{jvm} was warmed-up, and the disks were not used (all data is kept in memory).


\subsection{Performance in Failure-free Executions}
\label{throughput_latency}

This experiment measures the latency of \sieveq with several message sizes and distinct message transmission rates. 
It demonstrates the overall performance of \sieveq in different scenarios, gradually stressing the \postsieve side as the workload is slowly increased. 
Measurements were collected after the system reached a steady state. 
The experiments were repeated 10 times for every workload, and the average result is reported.

\begin{figure}[t]
\centering
\includegraphics[width=\columnwidth]{images/gnuplot/sieveq/new_plot_latencyvsthroughput/latencyVSthroughput.pdf}
\caption{\sieveq latency for each workload (message size and transmission rate).}
\label{fig:lat_vs_throu}
\end{figure}

Figure~\ref{fig:lat_vs_throu} shows how the latency is affected by the transmission rate and message size.
As expected, when the system becomes increasingly loaded, the latency grows proportionally because resources have to be shared among the various messages.
The latency increase is approximately linear for the messages with 100 and 500 bytes until the throughput reaches $10k$ messages per second.
The messages with $1k$ bytes have a linear increase on latency until the throughput reaches approximately $7k$ messages per second, and then it has a higher increase as more load is put on the system.
This means that very high workloads can only be supported if applications have some tolerance to network delays.
Overall, the performance degrades gracefully when varying the message payload sizes, for rates under $7k$ messages per second.


\begin{table}[t]
\begin{center}
{\small
\begin{tabular}{ | c | c |  c | }\hline
\textbf{Payload size } & \textbf{Non-optimized}  & \textbf{Optimized}   \\
\textbf{$(bytes)$} & \textbf{$(msg/sec)$} & \textbf{$(msg/sec)$}  \\\hline \hline
$100  $ & 1360 & 10682 \\
$500  $ & 1320 & 10618 \\
$1000 $ & 1311 & 10332  \\
\hline
\end{tabular}
}
\vspace{2mm}
\caption{Maximum load induced by the \sender library with various message sizes\label{tab:client_evaluation}.}
\end{center}
\end{table}

We performed a more detailed analysis of the overheads introduced by the various components of \sieveq.
We observed that \sender performs the most expensive operations, which is interesting because it shows that our design offloads part of the effort to the edges, reducing bottlenecks.
The most significant overheads were caused by the tasks associated with securing the message payload (which are the most costly operations in the system). Several optimizations were made to mitigate the performance penalties during the message serialization (e.g., the creation of the signature), including the use of parallelization to take advantage of the multicore architecture (as done, for example, in~\cite{Kirsch:2014}).
Table~\ref{tab:client_evaluation} shows the gain of using the optimized version of the \sender library. 
Similar optimizations were employed in other components.



\subsection{Effect of Filtering Rules Complexity}


The results presented in the previous section do not consider any complex filtering rule set.
This section presents experiments with a fully loaded system and application-level filtering rules with different complexity at the \repsieves.

Given the diverse requirements imposed by application-level firewalls, we decided to approximate the complexity of the filtering rules by considering a variable number of string matchings on processed messages.  It is well recognized that the most expensive aspect of message filtering is exactly finding (or not) specific strings in the packet contents (besides crypto verifications, which were included in all our experiments). 
The high cost of running such algorithms leads for instance to several implementations in \glspl{fpga} and \glspl{gpu} to improve performance in firewalls and \gls{ids} (e.g.,~\cite{Moscola:2003,Lee:2015}).

We used a classical algorithm for string matching (Knuth–Morris–Pratt~\cite{Knuth:1977}) at the \repsieves to implement the filtering rules.
This algorithm is employed in \glspl{ids}~\cite{Prabha:2014} and firewalls like \texttt{iptables}~\cite{iptables}.
The algorithm employs a pre-computed table to execute string matching with $O(n+k)$ comparisons, where $n$ is the string length, and $k$ is the pattern length.


We measured the latency of the system by varying the message size from 100 to 1000 bytes and the string pattern size from 5 to 20 bytes. 
Both the message content and the strings were randomly generated. 
The experiments were performed with the maximum throughput of $10 000$ messages per second, as identified in the previous section. 
Figure~\ref{fig:KMP} shows the latency of \sieveq without message filtering (Baseline) and string matching of different sizes.
In the last group of experiments, where the message size is 1000 bytes, and the pattern is 20 bytes, the latency increases by $50\%$. 
In other cases, sometimes higher overheads were observed, e.g., with 500 bytes messages and 20 bytes pattern the overhead is approximately $200\%$. 
This result is expected as string matching is a slow operation and it needs to be performed in the critical path of message processing. 
Implementations with hardware support (as mentioned before) could be integrated with \sieveq to reduce these delays significantly.


\begin{figure}[t]
\centering
\includegraphics[width=\columnwidth]{images/gnuplot/sieveq/new_plot_fw/fw.pdf}
\caption{Comparison of the \sieveq's latency between the baseline and filtering rules with patterns of different sizes.}
\label{fig:KMP}
\end{figure}


\subsection{\sieveq Under Attack}

Our next set of experiments aims to evaluate the system under different attack scenarios.
Here, the \sender creates a steady load of 1000 500-byte messages per second.
We measured the latency and throughput of the system in three conditions: failure-free operation, a malicious external and internal \gls{dos} attack, and a malicious attack with remediation mechanisms.
Before presenting the results, we need to stress that these experiments must be compared with the results displayed in Figure \ref{fig:attack_traditional}, which were obtained in the same way but with a different architecture (see Figure \ref{fig:traditional}).

Figures~\ref{fig:latency_normal} and~\ref{fig:throughput_normal} show the latency and throughput observed in the failure-free scenario.
One can observe that latency stays (on average) around $3.4$ milliseconds.
The throughput is approximately constant during the whole period.
It is possible to observe some momentary spikes in the latency and throughput, which happens due to Java garbage collector and a queuing effect from the \gls{smr}.

\begin{figure}[t]

\subfloat[Baseline latency.]{\includegraphics[width=0.5\columnwidth]{images/gnuplot/sieveq/plots/latency_normal_t1000_s500.pdf}\label{fig:latency_normal}}
\hspace{-5mm}
\subfloat[Baseline throughput.]{\includegraphics[width=0.5\columnwidth]{images/gnuplot/sieveq/plots/server_throughput_normal.pdf}\label{fig:throughput_normal}}
\caption{Performance of \sieveq in fault-free executions.}
%\label{fig:performance_attacks}
\end{figure}




The behavior of \sieveq during a \gls{dos} attack is displayed in Figures~\ref{fig:latency_attack} and~\ref{fig:throughput_attack}.
In this scenario, we have disabled the \sieveq capability of replacing \presieves at runtime.
The attack consists in stressing the TCP socket interface of the \presieves by creating many TCP connections, which consumes network bandwidth and wastes resources at system and application levels. When the attack is started, it executes for 50 seconds. The latency graph displays a reasonable impact regarding an increase in the delays for message delivery. 
In some cases, the latency is not too affected, but in others, there is a drastic delay, with some messages taking more than 3 seconds to be received.
The attack also has consequences on the throughput as it is possible to observe an oscillation between 0 to 4000 messages per second (which correspond to the situation when the \postsieve processes a batch of messages that have been accumulated).


\begin{figure}[t]
\subfloat[DoS w/ no recovery.]{\includegraphics[width=0.5\columnwidth]{images/gnuplot/sieveq/plots/latency_normal_t1000_s500_under_attack.pdf}\label{fig:latency_attack}}
\hspace{-5mm}
\subfloat[DoS w/ no recovery.]{\includegraphics[width=0.5\columnwidth]{images/gnuplot/sieveq/plots/server_throughput_attack.pdf}\label{fig:throughput_attack}}
\hspace{-5mm}
\caption{Performance of \sieveq under DoS attack conditions.}
%\label{fig:performance_attacks}
\end{figure}


Figures~\ref{fig:latency_attack_replacement} and~\ref{fig:throughput_attack_replacement} show the latency and throughput when a similar \gls{dos} was carried out, but in this case the \sieveq replaced the \presieve under attack with a new \presieve.
When the \presieve finds out that it is being overloaded with messages coming from non-authorized senders, it asks for a replacement.
After that, the controller replaces the faulty \presieve, and the existing \senders are contacted to migrate their connections.
As the figures show, the impact of the attack is minimized, since only a few messages are delayed, and throughput is only affected momentarily while the \presieve is switched.
Once the new \presieve takes over, the messages lost during the switching period are retransmitted and delivered.
In practice, the attack becomes ineffective because, although it continues to consume network bandwidth, there is no longer a \presieve to process the malicious messages.
An adversary could increase the attack sophistication and try to find a new \presieve target.
However, even in this case, the attack has a limited effect because during an interval of time (while there is a search for a fresh target) the system can make progress.

\begin{figure}[t]
\subfloat[DoS w/ recovery.]{\includegraphics[width=0.5\columnwidth]{images/gnuplot/sieveq/plots/latency_t1000_s500_change_with_attack.pdf}\label{fig:latency_attack_replacement}}
\hspace{-5mm}
\subfloat[DoS w/ recovery.]{\includegraphics[width=0.5\columnwidth]{images/gnuplot/sieveq/plots/server_throughput_with_attack_change.pdf}\label{fig:throughput_attack_replacement}}
\hspace{-5mm}
\caption{Performance of \sieveq under DoS attack conditions with recovery.}
%\label{fig:performance_attacks}
\end{figure}


Figures~\ref{fig:replica_dos_no_recovery} and~\ref{fig:replica_dos_recovery} shows the \sieveq throughput when an internal attack is carried out by a compromised \repsieve. 
The experiment was made with a \repsieve ($r1$) launching a \gls{dos} to another \repsieve ($r2$).
The attack consists in overloading a \repsieve with \emph{state transfer} requests, which are the most demanding request a replica can receive in \textsc{BFT-SMaRt}~\cite{Bessani:2013}.
Figure~\ref{fig:replica_dos_no_recovery} shows the impact on the \sieveq throughput during an attack lasting 50 seconds, without any recovery capability on the system.
As can be seen, the performance of the system is severely disrupted during the attack.
Figure~\ref{fig:replica_dos_recovery} shows the same attack but with the detection and recovery mechanism described in Section~\ref{faultyrepsieve}.
The \postsieve detects the problem by noticing that $\mathit{f_{rs}+1}$ \repsieves are sending fewer messages than the others and then requests a recovery.
When the \repsieve is recovered, it requests the state from the other replicas, and then after applying the new state, the replica resumes the normal execution (end line in the figure).
In the experiment of Figure \ref{fig:replica_dos_recovery} we show a case in which the faulty \repsieve is the first to be recovered.
It could happen that \sieveq recovered $\mathit{f_{rs}}$ \repsieves before the faulty one.
This would take  $\mathit{(f_{rs}+1)} \times$ 3 seconds (in our setup) before the system resumes the normal execution.



\begin{figure}[t]
\subfloat[Internal DoS execution without recovery actions.]{\includegraphics[width=.5\columnwidth]{images/gnuplot/sieveq/new_internal_recovery/server_throughput_no_recovery_replicas_dos.pdf}\label{fig:replica_dos_no_recovery}}
\subfloat[Internal DoS execution with recovery actions.]{\includegraphics[width=0.5\columnwidth]{images/gnuplot/sieveq/new_internal_recovery/server_throughput_recovery_replicas_dos.pdf}\label{fig:replica_dos_recovery}}
\caption{Performance of \sieveq under internal DoS attack conditions without and with recovery.}
%\label{fig:performance_attacks}
\end{figure}

\subsection{\sieveq to Protect a SIEM System}
\label{use_case}

\gls{siem} systems offer various capabilities for the collection and analysis of security events and information in networked infrastructures~\cite{Miller:2010}.
Organizations are employing these systems as a way to help with the monitoring and analysis of their infrastructures.
They integrate an extensive range of security and network capabilities, which allow the correlation of thousands of events and the reporting of attacks and intrusions in near real-time.

A \gls{siem} operates by collecting data from the monitored network and applications through a group of sensors, which then forward the events towards a correlation engine at the core facility.
The engine performs an analysis of the stream of events and generates alarms and other information for post-processing by other \gls{siem} components.
Examples of such components are an archival subsystem for the storage of data needed to support forensic investigations, or a communication subsystem to send alarms to the system administrators.

\begin{figure}[t]
\centering
\includegraphics[width=0.70\columnwidth]{images/images/SIEM.pdf}
\caption{Overview of a SIEM architecture, showing some of the core facility subsystems protected by the \sieveq.}
\label{fig:siem}
\end{figure}


As part of the MASSIF European project~\cite{Vianello:2013}, we have implemented a resilient \gls{siem} system where \sieveq was used to protect the access to the core facility (see Figure~\ref{fig:siem}).
In the \sieveq architecture, the sensors integrate the \sender while the \postsieve was placed in the correlation engine.
Additionally, we had access to an anonymized trace with the security events collected during the 2012 Olympic Games.
Each event corresponds basically to a string describing some observed problem by a sensor. The strings of text had lengths varying between a minimum of 551 bytes and a maximum of 2132 bytes, with an average length of 1990 bytes (and a standard deviation of 420 bytes). Based on this log, we built a sensor emulator that generates traffic at a pre-defined rate. Basically, when it is time to produce a new event, the emulator selects an event from the trace and feeds it to the \sender.

During the 2012 Olympic Games, the workload was approximately 11 million events per day, i.e., around 127 events per second.
Figure~\ref{fig:massif} shows the latency imposed by \sieveq for this workload, and when it is scaled up from 2 to 16 times more. 
As can be observed, the latency is in the order of 4 milliseconds for the emulated scenario.
Even with the highest load (16 times), the observed values had a latency below $70$ milliseconds.
This means that \sieveq could potentially deliver 176 million events per day, which is more than enough to accommodate the expected growth in the number of events for the next Olympic Games.\footnote{In 2016 Rio's Olympic Games the number of events per second was approximated 174, source: \url{https://diginomica.com/2016/11/24/securing-the-olympics-lessons-for-enterprise-cyber-security/}}

\begin{figure}[t]
\centering
\includegraphics[width=\columnwidth]{images/gnuplot/sieveq/new_massif/massif.pdf}
\caption{\sieveq latency for the 2012 Summer Olympic Games scenario throughput requirement (127 events per second) and how the \sieveq scale.}
\label{fig:massif}
\end{figure}

\section{Discussion}
\label{discussion}

\sieveq differs from standard firewalls like \texttt{iptables}~\cite{iptables} that do not need client- or server-side code modifications.
However, our system requires these modifications to ensure end-to-end message integrity and tolerance of compromised components.
Complete transparency would be hard to achieve mainly due to the use of voting.
Nonetheless, it is worth stressing that \sieveq is to be used as an additional protection device in critical systems, not as a substitute to regular L3 firewalls.
\sieveq provides a protection similar to an application-level/L7 firewall, as one can implement arbitrary rules on the \repsieve module.


State machine replication is a well-known approach for replication~\cite{Schneider:1990}.
In this technique, every replica is required to process requests in a deterministic way.
This requirement traditionally implies in two limitations: (1) replicas cannot use their local clock during request processing, and (2) all requests are executed sequentially.
The first limitation can affect the capacity of \sieveq to process rules that use time.
We remove this limitation by making use of the timestamps generated by the leader replica and agreed upon on each consensus, as proposed in \textsc{Pbft}~\cite{Castro:2002} and implemented in \textsc{BFT-SMaRt}~\cite{Bessani:2014}.
The second limitation can constraint the performance of the system, especially when CPU-costly operations such as signature verifications are executed.
One of the optimizations we implemented in \sieveq was to add multi-threading support to the \textsc{BFT-SMaRt} replicas.
More precisely, the signature verification is done by a pool of threads that either accept or discard messages.
Once accepted, a message is added to a processing queue following the order established by the total order multicast protocol.
A single thread consumes messages from this queue, verifies them against the security policy, updates the firewall state (if needed) and forwards them to their destinations, without violating the determinism requirement.



\section{Final Remarks}
\label{sec:finalremarkssieveq}

We presented \sieveq, a new intrusion-tolerant protection system for critical services, such as \gls{ics} and \gls{siem} systems.
Our system exports a message queue interface which is used by senders and receivers to interact in a regulated way.
The main improvement of the \sieveq architecture, when compared with previous systems, is the separation of message filtering in several components that carry on verifications progressively more costly and complex.
This allows the proposed system to be more efficient than the state-of-the-art replicated firewalls under attack.
\sieveq also includes several resilience mechanisms that allow the creation, removal, and recovery of components in a dynamic way, to respond to evolving threats against the system effectively. 
Experimental results show that such resilience mechanisms can significantly reduce the effects of \gls{dos} attacks against the system.



\chapter{Conclusion and Future Research Directions}
\label{chap:conclusion}

\section{Conclusions}
The correctness of \gls{bft} systems is tied to the guarantee that replicas fail independtly. 
Otherwise, once a replica is compromised the next $f+1$ replicas would be compromised virtually within the same time or effort.
In this thesis, we addresses the long-standing open problem of many works on the past 20 years of \gls{bft} replication research: evaluate, select, and manage the failure independence (through diversity) of a \gls{bft} system to make it resilient to malicious adversaries.
That said, we developed a control plane that maximizes the failure inpdenpence of \gls{bft} replicated nodes in a practical and effective way.


The first step to achieve this objective was, following the line of the masters thesis, to validate the diversity hypothesis.  
To this end, we levarage on the study using \gls{nvd} data about the shared vulnerabilities among \glspl{os} and conducted and analysis more focused on the deciding which \glspl{os} provide a dependable set for replicated systems.
In particular, we presented several strategies to choose most diverse \glspl{os}.
These strategies comprise different approaches data depending on whether one (i) considers all common vulnerabilities as being of equal importance; (ii) place greater emphasis on more recent common vulnerabilities; and (iii) are primarily interested in the common vulnerabilities being reported less frequently in calendar time.
All strategies delivered the same best conbination of \glspl{os}, this emphasizes the importance of carefully select \glspl{os} that minimize the number of common weaknesses.



Despite the results that we have achieved in the previous contribution, we have identified some problems that leads us to pursue on further analysis.
Namely, we have identified that our (and several other) work(s) that used \gls{nvd} as a source of vulnerability studies were missing important data. 
In particular, not all products vulnerable to a certain vulnerability were reported as affected in \gls{nvd}.
We addressed this issue by using clustering techcniques to group similar vulnerabilities.
Moreover, we add other data sources with other relevant data (that \gls{nvd} does not provide) to enrich our data.
We devise a new metric that uses the vulnerabilities attributes provided by \gls{nvd}, \gls{cvss}, and the the ones that we added. 
This metric is used in an algorithm that builds and reconfigures replica sets on \gls{bft} systems.
The algorithm decides when and which replica should be replaced by estimating the risk of the \gls{bft} being vulnerable and potentially compromised.
The results show that our algorithm supports better decisions when selecting \glspl{os} to run in the replicated system. 
For example, for the evaluated time it \emph{compromised rate} varies between 2\%-5\% while a random selection varies between 98\%-99\%.
 

Besides the lack of proof for supporting diversity, most of works did not address diversity in a practical manner.
We have implemented \system to assess the limitations that diversity could introduce in \gls{bft} systems.
\system uses the algorithm previously described to manage real \gls{bft} systems. 
We conducted an extensive evaluation of the costs of using real \gls{os} diversity in \gls{bft} systems.
In the experiements, we have found a major limitation on the virtualization technology, which forced us to limit our evalution by reconfiguring the bare metal machines to use fewer resources.
Therefore, we manage to make a fair comparison (for some of the cases) between a bare metal configuration with a virtualized (yet limited) solution that accomates \system.
Although most of the results show a non-negligible overhead, we have shown that for \gls{os} that did not suffer virtualization limitations the overhead is minimal.
In these cases, the performance of the replicated system managed by \system is 86\%-94\% of the bare metal, which runs a replicated non-virtualized homogenous \gls{os} configuration.

One of the \gls{bft} systems used in the \system evaluation, is \sieveq. 
It is multi-layer firewall-like application that was designed to absorb most of the external and itnernal attacks with additional resilient mechanisms.
Firewalls represent one of the most critical roles in infrastrucres, as they controll which packets go in and out of the network.
The main improvement of the multi-layered architecture, when compared with previous systems, is the separation of message filtering in several components that carry on verifications progressively more costly and complex.
This allows the proposed system to be more efficient than the state-of-the-art replicated firewalls under attack.
\sieveq also includes several resilience mechanisms that allow the creation, removal and recovery of components in a dynamic way, to effectively respond to evolving threats against the system. Experimental results show that such resilience mechanisms can significantly reduce the effects of \gls{dos} attacks against the system.


\section{Future Work}
This thesis addresses the long-standing open problems of diversity in \gls{bft} system.
The results of this thesis open many avenues for future work on this research topic, which we address in the following topics:


\textbf{Integration of prevention techniques:}
In Chapter~\ref{chap:related_work}, we have briefly described a few prevention techniques.
Although we did not address them in this thesis, we initiate the discussion on how to integrate such techniques in \system.
The \system replicas have two phases were they are dormant, when they are in quarantine or before they are deployed in the execution plane.
Thus, this time could be used to apply a battery of automatic procedures (e.g., vulnerability detectors) that would (1) detect additional weaknesses, which could be reported, and (2) remediate those weakness with automatic mechanisms like automatic patching~\cite{Huang:2016}.
Such mechanisms would decrease the vulnerability surface, especially if common vulnerabilities are detected.

Additionally, use automatic attacks~\cite{Hu:2015} to exploit common vulnerabilities in differnet replicas. 
Although we are most concern with \gls{apts}, using automatic attacks could activate some common vulnerabilities that would trigger the algorithm to adjust the risk associated with such replicas.


\textbf{Extending the \system sources:}


\textbf{Use vulnerable clone decttors on Opens source software to aggravete pairs with more clones} with dependcy graphs and autidting tools~\cite{Kim:2017}


\textbf{Vulnreabilities in the black market as a inditicar of severity}~\cite{Allodi:2014}.


\textbf{Integration with other sensors:}
\system monitors only five security data feeds on the internet looking for vulnerabilities, exploits, and patches in the OSes it manages, but it could be extended to monitor other indicators of compromise (e.g., IP black lists) extracted from a much richer set of sources~\cite{Liao:2016,Sabottke:2015}.
Similarly, \system can be extended to additionally use the outputs of IDSes to assess the BFT system behavior and trigger replica reconfigurations in case of need.



\textbf{Virtualization technology:}
\system paid a performance penalty due to the limitations of the virtualization platform we used (VirtualBox).
VirtualBox was selected because it was the platform we could run more OSes.
Therefore its use enabled \system to support 17 different OSes for running BFT systems.
It would be great to have a VM technology capable of supporting all existing OSes without the resource limitations we experienced.

\textbf{Distributed control plane:}
As in previous works~\cite{Roeder:2010,Platania:2014}, our current design for \system considers a centralized trusted control plane that analyzes OSINT and orchestrates replica set reconfigurations.
It would be desirable to have a distributed version of such control plane, not only for improving its dependability but also to support the existence of multi-domain applications, such as blockchain platforms.

\textbf{Trusted components:}
Our prototype implements the LTU as a trusted component isolated from the rest of the replica in a VM, as many works on hybrid BFT~\cite{Veronese:2013,Roeder:2010,Platania:2014,Sousa:2010,Distler:2011}.
The recent popularization of trusted computing technologies such as Intel SGX~\cite{sgx}, and its use for implementing efficient BFT replication~\cite{Behl:2017}, open interesting possibilities for using novel hardware to support services like \system on bare metal.



\textbf{Diversity-aware replication:}
The evaluation of BFT-SMaRt on top of \system shows that different replica set configurations can impact on the performance of applications, mostly due to the performance heterogeneity of the different OSes.
It would be interesting to consider protocols in which this heterogeneity is taken into account.
For example, the leader could be allocated in the fastest replica, or weighted-replication protocols such as WHEAT~\cite{Sousa:2015} could be used to assign higher weights to the replicas running in faster replicas.




% Fim do conteudo
% ----------------------------------------------------------------------

% Glossario

%
% Para actualizar o glossario, e' preciso correr o script ./fazindice
% e voltar a gerar o PDF
%
\LIMPA
\renewcommand{\glossaryname}{Acronyms}

\newacronym{aslr}{ASLR}{Address Space Layout Randomization}
\newacronym{arff}{ARFF}{Attribute-Relation File Format}



\newacronym{bft}{BFT}{Byzantine Fault Tolerance}
\newacronym{bm}{BM}{Bare Metal}

\newacronym[plural=CVEs,firstplural=Common Vulnerabilities and Exposures (CVEs)]{cve}{CVE}{Common Vulnerabilities and Exposures}
\newacronym{cpe}{CPE}{Common Platform Enumeration}
\newacronym{cvi}{CVI}{Common Vulnerability Indicator}
\newacronym{cvcs}{CVCst}{Common Vulnerability Count Strategy}
\newacronym{cvc}{CVC}{Common Vulnerability Count}
\newacronym{cvis}{CVIst}{Common Vulnerability Indicator Strategy}
\newacronym{irt}{IRT}{Inter-Reporting Times}
\newacronym{irts}{IRTst}{Inter-Reporting Times Strategy}
\newacronym{ip}{IP}{Internet Protocol}

\newacronym[plural=GPUs,firstplural=Graphics Processing Units (GPUs)]{gpu}{GPU}{Graphics Processing Unit}
\newacronym[plural=FPGAs,firstplural=Field-Programmable Gate Arrays (FPGAs)]{fpga}{FPGA}{Field-Programmable Gate Array}


\newacronym{cvss}{CVSS}{Common Vulnerability Scoring System}
\newacronym{xss}{XSS}{Cross-site scripting}


\newacronym{cots}{COTS}{Components Off-the-Shelf}
\newacronym{cis}{CIS}{CRUTIAL Information Switch}
\newacronym{dos}{DoS}{Denial-of-Service}
\newacronym{ddos}{DDoS}{Distributed Denial-of-Service}
\newacronym{dbms}{DBMS}{Databases Management Systems}
\newacronym{dns}{DNS}{Domain Name System}

\newacronym[plural=DBs,firstplural=Databases (DBs)]{db}{DB}{Database}

\newacronym{hmac}{HMAC}{Hash-based Message Authentication Code}


\newacronym{jvm}{JVM}{Java Virtual Machine}
\newacronym{ics}{ICS}{Industrial Control Systems}

\newacronym{ots}{OTS}{Off-the-Shelf}
\newacronym{osi}{OSI}{Open Systems Interconnection}

\newacronym[plural=OSes,firstplural=Operating Systems (OSes)]{os}{OS}{Operating System}
\newacronym[plural=IDSs,firstplural=Intrusion Detection Systems (IDSs)]{ids}{IDS}{Intrusion Detection System}


\newacronym{sql}{SQL}{Structured Query Language}

\newacronym[plural=MACs,firstplural=Message Authentication Codes (MACs)]{mac}{MAC}{Message Authentication Code}

\newacronym{nist}{NIST}{National Institute of Standards and Technology}
\newacronym{nvd}{NVD}{National Vulnerability Database}
\newacronym{nfs}{NFS}{Network File System}


\newacronym{ltu}{LTU}{Logical Trusted Unit}


\newacronym{scada}{SCADA}{Supervisory Control and Data Acquisition}
\newacronym{smr}{SMR}{State Machine Replication}
\newacronym{siem}{SIEM}{Security Information and Event Management}

\newacronym{ycsb}{YCSB}{Yahoo! Cloud Serving Benchmark}



\newacronym{osint}{OSINT}{Open Source Intelligence}
\newacronym{osvdb}{OSVDB}{Open Sourced Vulnerability Database}

\newacronym{tom}{TOM}{Total Ordered Multicast}
\newacronym{tls}{TLS}{Transport Layer Security}


\newacronym{mtd}{MTD}{Moving Target Defense}


\newacronym{po}{PO}{Proactive Obfuscation}

\newacronym{pr}{PR}{Proactive Recovery}
\newacronym{prr}{PRR}{Proactive-Reactive Recovery}
\newacronym{prrw}{PRRW}{Proactive-Reactive Recovery Wormhole}

\newacronym{zda}{ZDA}{Zero-Day Attack}
\newacronym{pzda}{PZDA}{Pseudo Zero-Day Attack}
\newacronym{ppzda}{PPZDA}{Potential Pseudo Zero-Day Attack}
\newacronym{poa}{POA}{Potential for Attack}


\newacronym{rsa}{RSA}{Rivest–Shamir–Adleman}
\newacronym{kvs}{KVS}{Key-Value Store}
%\newacronym{kmp}{KMP}{Knuth–Morris–Pratt}


\newacronym{svm}{SVM}{Support Vector Machines}
\newacronym{scit}{SCIT}{Self-Cleaning Intrusion Tolerance}

\newacronym{sha}{SHA}{Secure Hash Algorithm }

\newacronym{tcp}{TCP}{Transmission Control Protocol}
\newacronym{wine}{WINE}{World Wide Intelligence Network Environment}


\newacronym[plural=VMs,firstplural=Virtual Machines (VMs)]{vm}{VM}{Virtual Machine}
\newacronym{vmm}{VMM}{Virtual Machine Manager}
\newacronym{vpn}{VPN}{Virtual Private Network}

\newacronym{xml}{XML}{Extensible Markup Language}




\printglossaries
\addcontentsline {toc} {chapter} {Acronyms}
% Bibliografia

\LIMPA
\bibliographystyle{abbrv}
\bibliography{chapters/references/references}
%\addcontentsline {toc} {chapter} {Bibliography}

\end{document}
